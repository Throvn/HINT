% This is a sample LaTeX input file.  (Version of 11 April 1994.)
%
% A '%' character causes TeX to ignore all remaining text on the line,
% and is used for comments like this one.

\documentclass{article}      % Specifies the document class
\hsize=320pt
\vsize=400pt
\pretolerance=1000
\tolerance=2000
\emergencystretch=16pt
\overfullrule=0pt
\textwidth=\hsize
\textheight=\vsize
\input ifhint.tex
\ifhint
  \dimen0=\hsize
  \advance\dimen0 by 4em
  \dimen1=\vsize
  \advance\dimen1 by 4em
\setpage2 = latex
  priority 10
  width  \dimen0
  height \dimen1
{ \maxdepth=400pt
  \topskip=10pt
  \hbox{%
   \kern 2em\vbox {
   \kern 2em
   \vbox to \vsize{ 
    \stream0 %main text
    \stream\footins    
    \vskip -2pt % \unskip depth of output
    \kern 2em
   }
   \kern 2em
  \hbox to \hsize {}
 }}
 \setstream\footins =%footnotes
    prefered 255
    ratio 0
  { \hsize=\hsize
    \count\footins=1000 % the magnification factor
    \skip\footins=\bigskipamount % the extra space needed
    \dimen\footins=0.25\vsize % maximum height on the page
    \before = 
     {\vskip 0pt plus 1filll\vskip 9pt plus 4pt minus 2pt
      \kern -3pt
      \hrule width 5em
      \kern 2.6pt}
  }
}% end page template
\fi

                             % The preamble begins here.
\title{An Example Document}  % Declares the document's title.
\author{Leslie Lamport}      % Declares the author's name.
\date{January 21, 1994}      % Deleting this command produces today's date.

\newcommand{\ip}[2]{(#1, #2)}
                             % Defines \ip{arg1}{arg2} to mean
                             % (arg1, arg2).

%\newcommand{\ip}[2]{\langle #1 | #2\rangle}
                             % This is an alternative definition of
                             % \ip that is commented out.
\pagestyle{empty}

\begin{document}             % End of preamble and beginning of text.


\maketitle                   % Produces the title.
\thispagestyle{empty}

This is an example input file.  Comparing it with
the output it generates can show you how to
produce a simple document of your own.


\section{Ordinary Text}      % Produces section heading.  Lower-level
                             % sections are begun with similar 
                             % \subsection and \subsubsection commands.

The ends  of words and sentences are marked 
  by   spaces. It  doesn't matter how many 
spaces    you type; one is as good as 100.  The
end of   a line counts as a space.

One   or more   blank lines denote the  end 
of  a paragraph.  

Since any number of consecutive spaces are treated
like a single one, the formatting of the input
file makes no difference to
      \LaTeX,                % The \LaTeX command generates the LaTeX logo.
but it makes a difference to you.  When you use
\LaTeX, making your input file as easy to read 
as possible will be a great help as you write 
your document and when you change it.  This sample 
file shows how you can add comments to your own input 
file.

Because printing is different from typewriting,
there are a number of things that you have to do
differently when preparing an input file than if
you were just typing the document directly.
Quotation marks like
       ``this'' 
have to be handled specially, as do quotes within
quotes:
       ``\,`this'            % \, separates the double and single quote.
        is what I just 
        wrote, not  `that'\,''.  

Dashes come in three sizes: an 
       intra-word 
dash, a medium dash for number ranges like 
       1--2, 
and a punctuation 
       dash---like 
this.

A sentence-ending space should be larger than the
space between words within a sentence.  You
sometimes have to type special commands in
conjunction with punctuation characters to get
this right, as in the following sentence.
       Gnats, gnus, etc.\ all  % `\ ' makes an inter-word space.
       begin with G\@.         % \@ marks end-of-sentence punctuation.
You should check the spaces after periods when
reading your output to make sure you haven't
forgotten any special cases.  Generating an
ellipsis
       \ldots\               % `\ ' is needed after `\ldots' because TeX 
                             % ignores spaces after command names like \ldots 
                             % made from \ + letters.
                             %
                             % Note how a `%' character causes TeX to ignore 
                             % the end of the input line, so these blank lines 
                             % do not start a new paragraph.
                             %
with the right spacing around the periods requires
a special command.

\LaTeX\ interprets some common characters as
commands, so you must type special commands to
generate them.  These characters include the
following:
       \$ \& \% \# \{ and \}.

In printing, text is usually emphasized with an
       \emph{italic}  
type style.  

\begin{em}
   A long segment of text can also be emphasized 
   in this way.  Text within such a segment can be 
   given \emph{additional} emphasis.
\end{em}

It is sometimes necessary to prevent \LaTeX\ from
breaking a line where it might otherwise do so.
This may be at a space, as between the ``Mr.'' and
``Jones'' in
       ``Mr.~Jones'',        % ~ produces an unbreakable interword space.
or within a word---especially when the word is a
symbol like
       \mbox{\emph{itemnum}} 
that makes little sense when hyphenated across
lines.

Footnotes\footnote{This is an example of a footnote.}
pose no problem.

\LaTeX\ is good at typesetting mathematical formulas
like
       \( x-3y + z = 7 \) 
or
       \( a_{1} > x^{2n} + y^{2n} > x' \)
or  
       \( \ip{A}{B} = \sum_{i} a_{i} b_{i} \).
The spaces you type in a formula are 
ignored.  Remember that a letter like
       $x$                   % $ ... $  and  \( ... \)  are equivalent
is a formula when it denotes a mathematical
symbol, and it should be typed as one.

\section{Displayed Text}

Text is displayed by indenting it from the left
margin.  Quotations are commonly displayed.  There
are short quotations
\begin{quote}
   This is a short a quotation.  It consists of a 
   single paragraph of text.  See how it is formatted.
\end{quote}
and longer ones.
\begin{quotation}
   This is a longer quotation.  It consists of two
   paragraphs of text, neither of which are
   particularly interesting.

   This is the second paragraph of the quotation.  It
   is just as dull as the first paragraph.
\end{quotation}
Another frequently-displayed structure is a list.
The following is an example of an \emph{itemized}
list.
\begin{itemize}
   \item This is the first item of an itemized list.
         Each item in the list is marked with a ``tick''.
         You don't have to worry about what kind of tick
         mark is used.

   \item This is the second item of the list.  It
         contains another list nested inside it.  The inner
         list is an \emph{enumerated} list.
         \begin{enumerate}
            \item This is the first item of an enumerated 
                  list that is nested within the itemized list.

            \item This is the second item of the inner list.  
                  \LaTeX\ allows you to nest lists deeper than 
                  you really should.
         \end{enumerate}
         This is the rest of the second item of the outer
         list.  It is no more interesting than any other
         part of the item.
   \item This is the third item of the list.
\end{itemize}
You can even display poetry.
\begin{verse}
   There is an environment 
    for verse \\             % The \\ command separates lines
   Whose features some poets % within a stanza.
   will curse.   

                             % One or more blank lines separate stanzas.

   For instead of making\\
   Them do \emph{all} line breaking, \\
   It allows them to put too many words on a line when they'd rather be 
   forced to be terse.
\end{verse}

Mathematical formulas may also be displayed.  A
displayed formula 
is 
one-line long; multiline
formulas require special formatting instructions.
   \[  \ip{\Gamma}{\psi'} = x'' + y^{2} + z_{i}^{n}\]
Don't start a paragraph with a displayed equation,
nor make one a paragraph by itself.

\end{document}               % End of document.
