\input cwebmac
% This file is part of HINT
% Copyright 2017 Martin Ruckert
%
% HINT is free software: you can redistribute it and/or modify
% it under the terms of the GNU General Public License as published by
% the Free Software Foundation, either version 3 of the License, or
% (at your option) any later version.
%
% HINT is distributed in the hope that it will be useful,
% but WITHOUT ANY WARRANTY; without even the implied warranty of
% MERCHANTABILITY or FITNESS FOR A PARTICULAR PURPOSE.  See the
% GNU General Public License for more details.
%
% You should have received a copy of the GNU General Public License
% along with HINT.  If not, see <http://www.gnu.org/licenses/>.
%
% Martin Ruckert, Hochschule Muenchen, Lothstrasse 64, 80336 Muenchen
%

\input format.sty

%% defining how to display certain C identifiers



\M{1}

\makeindex
\maketoc
\makecode
%\makefigindex
\titletrue

\def\setrevision$#1: #2 ${\gdef\lastrevision{#2}}
\setrevision$Revision: 2020 $
\def\setdate$#1(#2) ${\gdef\lastdate{#2}}
\setdate$Date: 2020-07-19 19:34:43 +0200 (Sun, 19 Jul 2020) $

\null

\font\largetitlefont=cmssbx10 scaled\magstep4
\font\Largetitlefont=cmssbx10 at 40pt
\font\hugetitlefont=cmssbx10 at 48pt
\font\smalltitlefontit=cmbxti10 scaled\magstep3
\font\smalltitlefont=cmssbx10 scaled\magstep3

%halftitle
\def\raggedleft{\leftskip=0pt plus 0.5em\parfillskip=0pt
\spaceskip=.3333em \xspaceskip=0.5em \emergencystretch=1em\relax
\hyphenpenalty=1000\exhyphenpenalty=1000\pretolerance=10000\linepenalty=5000
}
\hbox{}
\vskip 0pt plus 1fill
{ \baselineskip=60pt
\hugetitlefont\raggedleft HINT:\par
\Largetitlefont\raggedleft The File Format\par
}
\vskip 0pt plus 5fill
\eject
% verso of half title
\titletrue
\null
\vfill
\eject

% title
\titletrue
\hbox{}
\vskip 0pt plus 1fill
{
\baselineskip=1cm\parindent=0pt
\largetitlefont\raggedright HINT: The File Format\par
\vskip 10pt plus 0.5fill
\leftline{\smalltitlefont Reflowable}
\vskip-3pt
\leftline{\smalltitlefont Output}
\vskip-3pt
\leftline{\smalltitlefont for \TeX}
\vskip 10pt plus 0.5fill
\hskip 0pt plus 2fill\it F\"ur meine Mutter\hskip 0pt plus 0.5fill\hbox{}
\vskip 10pt plus 3fill
\leftline{\smalltitlefont Version 1.0}
\bigskip
\raggedright\baselineskip=12pt
\bf MARTIN RUCKERT \ \it Munich University of Applied Sciences\par
%  \leftline{\bf Eigendruck im Selbstverlag}
%  \bigskip
}
\eject

% verso of title
% copyright page (ii)
\titletrue
\begingroup
\figrm
\parindent=0pt
%\null
{\raggedright\advance\rightskip 3.5pc
The author has taken care in the preparation of this book,
but makes no expressed or implied warranty of any kind and assumes no
responsibility for errors or omissions. No liability is assumed for
incidental or consequential damages in connection with or arising out
of the use of the information or programs contained herein.

\bigskip
{\figtt\obeylines\obeyspaces\baselineskip=11pt
Ruckert, Martin.
HINT: The File Format
Includes index.
ISBN 978-1079481594
}
\bigskip

{\raggedright\advance\rightskip 3.5pc
\def\:{\discretionary{}{}{}}
Internet page {\tt https:\://www.\:cs.\:hm.\:edu/\:\TL ruckert}
may contain current information about this book, downloadable software,
and news.

\vfill
Copyright $\copyright$ 2019 by Martin Ruckert
\smallskip
All rights reserved.
Printed using Kindle Direct Publishing.
This publication is protected by copyright, and permission must be
obtained from the publisher prior to any prohibited reproduction, storage in
a~retrieval system, or transmission in any form or by any means, electronic,
mechanical, photocopying, recording, or likewise.
To obtain permission to use material from this work, please submit a written
request to Martin Ruckert,
Hochschule M\"unchen,
Fakult\"at f\"ur Informatik und Mathematik,
Lothstrasse 64,
80335 M\"unchen,
Germany.
\medskip
{\tt ruckert\:@cs.hm.edu}
\medskip
%ISBN-10: 0-000-00000-0

ISBN-13: 978-\:1079481594
\medskip
First printing: August 2019\par
\medskip
Revision: \lastrevision,\quad Date: \lastdate\par
}
}
\endgroup


\frontmatter



\plainsection{Preface}
Late in summer 2017, with my new \CEE/ based {\tt cweb} implementation
of \TeX\cite{Knuth:tex} in hand\cite{MR:webtocweb}\cite{MR:tug38}, I started to write
the first prototype of the \HINT/ viewer. I basically made two copies
of \TeX: In the first copy, I replaced the \index{build page+\\{build\_page}}\\{build\_page} procedure by
an output routine which used more or less the printing routines
already available in \TeX. This was the beginning of the
\HINT/ file format.
In the second copy, I replaced \TeX's main loop by an input routine
that would feed the \HINT/ file more or less directly to \TeX's
\index{build page+\\{build\_page}}\\{build\_page} procedure. And after replacing \TeX's \index{ship out+\\{ship\_out}}\\{ship\_out}
procedure by a modified rendering routine of a dvi viewer that I had
written earlier for my experiments with \TeX's Computer Modern
fonts\cite{MR:tug37}, I had my first running \HINT/ viewer.  My
sabbatical during the following Fall term gave me time for ``rapid
prototyping'' various features that I considered necessary for
reflowable \TeX\ output\cite{MR:tug39}.

The textual output format derived from the original \TeX\ debugging
routines proved to be insufficient when I implemented a ``page up''
button because it did not support reading the page content
``backwards''. As a consequence, I developed a compact binary file
format that could be parsed easily in both directions. The \HINT/
short file format war born. I stopped an initial attempt at
eliminating the old textual format because it was so much nicer when
debugging. Instead, I converted the long textual format into the short
binary format as a preliminary step in the viewer. This was not a long
term solution.  When opening a big file, as produced from a 1000
pages \TeX\ file, the parsing took several seconds before the first
page would appear on screen. This delay, observed on a fast desktop
PC, is barley tolerable, and the delay one would expect on a low-cost,
low-power, mobile device seemed prohibitive.  The consequence is
simple: The viewer will need an input file in the short format; and to
support debugging (or editing), separate programs are needed to
translate the short format into the long format and back again.  But
for the moment, I did not bother to implement any of this but
continued with unrestricted experimentation.

With the beginning of the Spring term 2018, I stopped further
experiments with the \HINT/ viewer and decided that I have to write
down a clean design of the \HINT/ file format. Or of both file
formats?  Professors are supposed to do research, and hence I tried an
experiment: Instead of writing down a traditional language
specification, I decided to stick with the ``literate programming''
paradigm\cite{Knuth:lp} and write the present book.  It describes and implements
the \index{stretch+\.{stretch}}\.{stretch} and \.{shrink} programs translating one file format
into the other.  As a side effect, it contains the underlying language
specification. Whether this experiment is a success as a language
specification remains to be seen, and you should see for yourself. But
the only important measure for the value of a scientific experiment is
how much you can learn form it---and I learned a lot.

The whole project turned out to be much more difficult than I had
expected.  Early on, I decided that I would use a recursive descent
parser for the short format and an LR($k$) parser for the long
format. Of course, I would use {\tt lex}/{\tt flex} and {\tt yacc}/{\tt bison}
to generate the LR($k$) parser, and so I had to extend the {\tt
cweb} tools\cite{Knuth:cweb} to support the corresponding source files.

About in mid May, after writing down about 100 pages, the first
problems emerged that could not be resolved with my current
approach. I had started to describe font definitions containing
definitions of the interword glue and the default hyphen, and the
declarative style of my exposition started to conflict with the
sequential demands of writing an output file. So it was time for a
first complete redesign.  Two more passes over the whole book were
necessary to find the concepts and the structure that would allow me
to go forward and complete the book as you see it now.

While rewriting was on its way, many ``nice ideas'' were pruned from
the book. For example, the initial idea of optimizing the \HINT/ file
while translating it was first reduced to just gathering statistics
and then disappeared completely.  The added code and complexity was
just too distracting.

What you see before you is still a snapshot of the \HINT/ file format
because its development is still under way.  We will know what
features are needed for a reflowable \TeX\ file format only after many
people have started using the format. To use the format, the end-user
will need implementations, and the implementer will need a language
specification.  The present book is the first step in an attempt to
solve this ``chicken or egg'' dilemma.


\vskip 1cm
\noindent {\it M\"unchen\hfil\break
August 20, 2019 \hfill Martin Ruckert}


\tableofcontent
%\thefigindex


\mainmatter

\section{Introduction}\label{intro}
This book defines a file format for reflowable text.
Actually it describes two file formats: a long format
that optimizes readability for human beings, and
a short format that optimizes readability for machines
and storage space. Both formats use the concept of nodes and lists of
nodes to describe the file content. Programs that process these nodes
will likely want to convert the compressed binary representation of a
node---the short format---or the lengthy textual representation of a
node---the long format---into a convenient internal representation.
So most of what follows is just a description of these nodes: their short format,
their long format and sometimes their internal representation.
Where as the description of the long and short external format is part
of the file specification, the description of the internal representation
is just informational. Different internal representations can be chosen
based on the individual needs of the program.

While defining the format, I illustrate the processing of long and short format
files by implementing two utilities: \.{shrink} and \index{stretch+\.{stretch}}\.{stretch}.
\.{shrink} converts the long format into the short format and \index{stretch+\.{stretch}}\.{stretch}
goes the other way.

There is also a prototype viewer for this
file format and a special version of \TeX\cite{DK:texbook} to produce output
in this format. Both are not described here; a survey describing
them can be found in \cite{MR:tug39}.

\subsection{Glyphs}
Let's start with a simple and very common kind of node: a node describing
a character.
Because we describe a format that is used to display text,
we are not so much interested in the
character itself but we are interested in the specific glyph\index{glyph}.
In typography, a glyph is a unique mark to be placed on the page representing
a character. For example the glyph representing the character `a' can have
many forms among them `{\it a\/}', `{\bf a}', or `{\tenss a}'.
Such glyphs come in collections, called fonts, representing every character
of the alphabet in a consistent way.

The long format of a node describing the glyph `a'
might look like this:`` \.{<glyph} \.{97} \.{*1>}''.
Here ``\.{97}'' is the character code which
happens to be the ASCII code of the letter `a' and ``{\tt *1}'' is a font reference
that stands for ``Computer Modern Roman 10pt''.
Reference numbers, as you can see,
start with an asterisk reminiscent of references in the \CEE/ programming language.
The Astrix enables us to distinguish between ordinary numbers like ``\.{1}'' and references like ``{\tt *1}''.

To make this node more readable, we will see in section~\secref{chars} that it is also
possible to write `` \.{<glyph 'a' (cmr10) *1>}''.
The latter form uses a comment ``\.{(cmr10)}'', enclosed in parentheses, to
give an indication of what kind of font happens to be font 1, and it uses ``\.{'a'}'',
the character enclosed in single quotes to denote the ASCII code of `a'.
But let's keep things simple for now and stick with the decimal notation of the character code.

The rest is common for all nodes: a keyword, here ``\index{glyph+\.{glyph}}\.{glyph}'', and a pair of pointed brackets ``\.{<}\dots\.{>}''.

Internally, we represent a glyph by the font number
and the character number or character code.
To store the internal representation of a glyph node,
we define an appropriate structure type, named after the node with a trailing {\dots\bf\_t}.
\Y\B\4\X1:hint types\X${}\E{}$\6
\&{typedef} \&{struct} ${}\{{}$\5
\1\&{uint32\_t} \|c;\5
\&{uint8\_t} \|f;\5
\2${}\}{}$ \index{glyph t+\&{glyph\_t}}\&{glyph\_t};
\As111, 130, 138, 149, 150, 164, 185, 193, 228\ETs251.
\Us434, 438, 439\ETs441.\Y
\fi

\M{2}

Let us now look at the program \.{shrink} and see how it will convert the long format description
to the internal representation of the glyph and finally to a short format description.


\subsection{Scanning the Long Format}
First, \.{shrink} reads the input file and extract a sequence of tokens. This is called ``scanning''\index{scanning}.
We generate the procedure to do the scanning using the program \.{flex}\cite{JL:flexbison}\index{flex+{\tt flex}} which is the
GNU version of the common UNIX tool \.{lex}\cite{JL:lexyacc}\index{lex+{\tt lex}}.

The input to \.{flex} is a list of pattern/\kern -1pt action rules where the pattern is a regular
expression and the action is a piece of \CEE/ code.
Most of the time, the \CEE/ code is very simple: it just returns the right token\index{token} number
to the parser which we consider shortly.

The code that defines the tokens will be marked with a line ending in ``\redsymbol''.
This symbol\index{symbol} stands for ``{\it Reading the long format\/}''.
These code sequences define the syntactical elements of the long format and at the same time
implement the reading process. All sections where that happens are preceded by a similar heading
and for reference they are conveniently listed together starting on page~\pageref{codeindex}.

\codesection{\redsymbol}{Reading the Long Format}\redindex{1}{2}{Glyphs}
\Y\par
\par
\par
\par
\par
\Y\B\4\X2:symbols\X${}\E{}$\6
\8\%\&{token} \index{START+\ts{START}}\ts{START}\5\.{"<"}\6
\8\%\&{token} \index{END+\ts{END}}\ts{END}\5\.{">"}\6
\8\%\&{token} \index{GLYPH+\ts{GLYPH}}\ts{GLYPH}\5\.{"glyph"}\6
\8\%\&{token} $<$ \|u $>$ \index{UNSIGNED+\ts{UNSIGNED}}\ts{UNSIGNED} \6
\8\%\&{token} $<$ \|u $>$ \index{REFERENCE+\ts{REFERENCE}}\ts{REFERENCE}
\As4, 24, 31, 45, 47, 52, 77, 85, 96, 100, 106, 112, 122, 131, 140, 151, 159, 165, 172, 179, 186, 194, 202, 207, 212, 216, 222, 229, 236, 242, 247, 255, 280, 299, 306, 315, 319, 327\ETs355.
\U437.\Y
\fi

\M{3}
You might notice that a small caps font is used for \index{START+\ts{START}}\ts{START}, \index{END+\ts{END}}\ts{END} or \index{GLYPH+\ts{GLYPH}}\ts{GLYPH}.
These are  ``terminal symbols'' or ``tokens''.
Next are the scanning rules which define the connection between tokens and their
textual representation.

\Y\B\4\X3:scanning rules\X${}\E{}$\6
${}\8\re{\vb{"<"}}{}$\ac\index{SCAN START+\.{SCAN\_START}}\.{SCAN\_START};\5
\&{return} \index{START+\ts{START}}\ts{START};\eac\7
${}\8\re{\vb{">"}}{}$\ac\index{SCAN END+\.{SCAN\_END}}\.{SCAN\_END};\5
\&{return} \index{END+\ts{END}}\ts{END};\eac\7
${}\8\re{\vb{glyph}}{}$\ac\&{return} \index{GLYPH+\ts{GLYPH}}\ts{GLYPH};\eac\7
${}\8\re{\vb{0|[1-9][0-9]*}}{}$\ac\index{SCAN UDEC+\.{SCAN\_UDEC}}\.{SCAN\_UDEC}(\index{yytext+\\{yytext}}\\{yytext});\5
\&{return} \index{UNSIGNED+\ts{UNSIGNED}}\ts{UNSIGNED};\eac\7
${}\8\re{\vb{\\*(0|[1-9][0-9]*)}}{}$\ac${}\index{SCAN UDEC+\.{SCAN\_UDEC}}\.{SCAN\_UDEC}(\index{yytext+\\{yytext}}\\{yytext}+\T{1});{}$\5
\&{return} \index{REFERENCE+\ts{REFERENCE}}\ts{REFERENCE};\eac\7
${}\8\re{\vb{[[:space:]]}}{}$\ac ;\eac\7
${}\8\re{\vb{\\([\^()\\n]*[)\\n]}}{}$\ac ;\eac
\As22, 25, 32, 46, 53, 57, 78, 86, 97, 101, 107, 113, 123, 141, 152, 160, 166, 173, 180, 187, 195, 203, 208, 213, 217, 223, 230, 237, 248, 256, 281, 300, 307, 320, 328\ETs356.
\U436.\Y
\fi

\M{4}

As we will see later, the macros starting with \.{SCAN\_}\dots\ are scanning macros.
Here \index{SCAN UDEC+\.{SCAN\_UDEC}}\.{SCAN\_UDEC} is a macro that converts the decimal representation
that did match the given pattern to an unsigned integer value; it is explained in
section~\secref{integers}.
The macros \index{SCAN START+\.{SCAN\_START}}\.{SCAN\_START} and \index{SCAN END+\.{SCAN\_END}}\.{SCAN\_END} are explained in section~\secref{text}.


The action ``{\tt ;}'' is a ``do nothing'' action; here it causes spaces or comments\index{comment}
to be ignored. Comments start with an opening parenthesis and are terminated by a
closing parenthesis or the end of line character.
The pattern ``\.{[\^()\\n]}'' is a negated
character class that matches all characters except parentheses and the newline
character. These are not allowed inside comments. For detailed information about
the patterns used in a \.{flex} program, see the \.{flex} user manual\cite{JL:flexbison}.

\subsection{Parsing the Long Format}
\label{parse_glyph}
Next, the tokens produced by the scanner are assembled into larger entities.
This is called ``parsing''\index{parsing}.
We generate the procedure to do the parsing using the program \.{bison}\cite{JL:flexbison}\index{bison+{\tt bison}} which is
the GNU version of the common UNIX tool \.{yacc}\cite{JL:lexyacc}\index{yacc+{\tt yacc}}.

The input to \.{bison} is a list of parsing rules, called a ``grammar''\index{grammar}.
The rules describe how to build larger entities from smaller entities.
For a simple glyph node like `` \.{<glyph 97 *1>}'', we need just these rules:
\codesection{\redsymbol}{Reading the Long Format}%\redindex{1}{2}{Glyphs}
\Y\par
\par
\par
\par
\par
\Y\B\4\X2:symbols\X${}\mathrel+\E{}$\6
\8\%\index{type+\&{type}}\&{type} $<$ \|u $>$ \index{start+\nts{start}}\nts{start} \6
\8\%\index{type+\&{type}}\&{type} $<$ \|c $>$ \index{glyph+\nts{glyph}}\nts{glyph}
\Y
\fi

\M{5}

\Y\B\4\X5:parsing rules\X${}\E{}$\6
\index{glyph+\nts{glyph}}\nts{glyph}: \1\1\5
\index{UNSIGNED+\ts{UNSIGNED}}\ts{UNSIGNED}\5
\index{REFERENCE+\ts{REFERENCE}}\ts{REFERENCE}\6
${}\{{}$\1\5
${}\.{\$\$}.\|c\K\.{\$1};{}$\5
${}\index{REF+\.{REF}}\.{REF}(\index{font kind+\\{font\_kind}}\\{font\_kind},\39\.{\$2});{}$\5
${}\.{\$\$}.\|f\K\.{\$2};{}$\5
${}\}{}$\2;\2\2\7
\index{content node+\nts{content\_node}}\nts{content\_node}: \1\1\5
\index{start+\nts{start}}\nts{start}\5
\index{GLYPH+\ts{GLYPH}}\ts{GLYPH}\5
\index{glyph+\nts{glyph}}\nts{glyph}\5
\index{END+\ts{END}}\ts{END}\5
${}\{{}$\1\5
${}\index{hput tags+\\{hput\_tags}}\\{hput\_tags}(\.{\$1},\39\index{hput glyph+\\{hput\_glyph}}\\{hput\_glyph}({\AND}(\.{\$3})));{}$\5
${}\}{}$\2;\2\2\7
\index{start+\nts{start}}\nts{start}: \1\1\5
\index{START+\ts{START}}\ts{START}\5
${}\{{}$\1\5
\index{HPUTNODE+\.{HPUTNODE}}\.{HPUTNODE};\5
${}\.{\$\$}\K(\&{uint32\_t})(\index{hpos+\\{hpos}}\\{hpos}\PP-\index{hstart+\\{hstart}}\\{hstart});{}$\5
${}\}{}$\2
\As27, 36, 48, 56, 80, 87, 98, 102, 114, 124, 132, 143, 153, 161, 167, 174, 181, 188, 196, 204, 209, 214, 218, 224, 231, 238, 244, 249, 257, 266, 282, 301, 308, 316, 321, 329, 332\ETs357.
\U437.\Y
\fi

\M{6}

You might notice that a slanted font is used for \index{glyph+\nts{glyph}}\nts{glyph}, \index{content node+\nts{content\_node}}\nts{content\_node}, or \index{start+\nts{start}}\nts{start}.
These are ``nonterminal symbols' and occur on the left hand side of a rule. On the
right hand side of a rule you find nonterminal symbols, as well as terminal\index{terminal symbol} symbols
and \CEE/ code enclosed in braces.

Within the \CEE/ code, the expressions \.{\$1} and \.{\$2} refer to the variables on the parse stack
that are associated with the first and second symbol on the right hand side of the rule.
In the case of our glyph node, these will be the values 97 and 1, respectively, as produced
by the macro \index{SCAN UDEC+\.{SCAN\_UDEC}}\.{SCAN\_UDEC}.
\.{\$\$} refers to the variable associated with the left hand side of the rule.
These variables contain the internal representation of the object in question.
The type of the variable is specified by a mandatory {\bf token} or optional {\bf type} clause
when we define the symbol.
In the above {\bf type} clause for \index{start+\nts{start}}\nts{start} and \index{glyph+\nts{glyph}}\nts{glyph} , the identifiers \|u and \|c refer to
the \&{union} declaration of the parser (see page~\pageref{union})
where we find \&{uint32\_t} \|u and \index{glyph t+\&{glyph\_t}}\&{glyph\_t} \|c. The macro \index{REF+\.{REF}}\.{REF} tests a reference number for
its valid range.


Reading a node is usually split into the following sequence of steps:
\itemize
\item Reading the node specification, here a \index{glyph+\nts{glyph}}\nts{glyph}
consisting of an \index{UNSIGNED+\ts{UNSIGNED}}\ts{UNSIGNED} value and a \index{REFERENCE+\ts{REFERENCE}}\ts{REFERENCE} value.
\item Creating the internal representation in the variable \.{\$\$}
based on the values of \.{\$1}, \.{\$2}, \dots\ Here the character
code field \|c is initialized using  the \index{UNSIGNED+\ts{UNSIGNED}}\ts{UNSIGNED} value
stored in \.{\$1} and the font field \|f is initialized using
\.{\$2} after checking the reference number for the proper range.
\item A \index{content node+\nts{content\_node}}\nts{content\_node} rule explaining that \index{start+\nts{start}}\nts{start} is followed by \index{GLYPH+\ts{GLYPH}}\ts{GLYPH},
the keyword that directs the parser  to \index{glyph+\nts{glyph}}\nts{glyph}, the
node specification, and a final \index{END+\ts{END}}\ts{END}.
\item Parsing \index{start+\nts{start}}\nts{start}, which is defined as the token \index{START+\ts{START}}\ts{START} will assign
to the corresponding variable \|p on the parse stack the current
position \index{hpos+\\{hpos}}\\{hpos} in the output and increments that position
to make room for the start byte, which we will discuss shortly.
\item At the end of the \index{content node+\nts{content\_node}}\nts{content\_node} rule, the \.{shrink} program calls
a {\it hput\_\dots\/} function, here \index{hput glyph+\\{hput\_glyph}}\\{hput\_glyph}, to write the short
format of the node as given by its internal representation to the output
and return the correct tag value.
\item Finally the \index{hput tags+\\{hput\_tags}}\\{hput\_tags} function will add the tag as a start byte and end byte
to the output stream.
\enditemize

Now let's see how writing the short format works in detail.


\subsection{Writing the Short Format}
A content node in short form begins with a start\index{start byte} byte. It tells us what kind of node it is.
To describe the content of a short \HINT/ file, 32 different kinds\index{kind} of nodes are defined.
Hence the kind of a node can be stored in 5 bits and the remaining bits of the start byte
can be used to contain a 3 bit ``info''\index{info} value.

We define an enumeration type to give symbolic names to the kind values.
The exact numerical values are of no specific importance;
we will see in section~\secref{text}, however, that the assignment chosen below,
has certain advantages.

Because the usage of kind values in content nodes is
slightly different from the usage in definition nodes, we define alternative names for some kind values.
To display readable names instead of numerical values when debugging,
we define two arrays of strings as well. Keeping the definitions consistent
is achieved by creating all definitions from the same list
of identifiers using different definitions of the macro \index{DEF KIND+\.{DEF\_KIND}}\.{DEF\_KIND}.

\Y\B\4\X6:hint basic types\X${}\E{}$\6
\8\#\&{define} \index{DEF KIND+\.{DEF\_KIND}}\.{DEF\_KIND} ${}(\|C,\39\|D,\39\|N)\5\|C\#{\#}\\{\_kind}\K\|N{}$\6
\&{typedef} \&{enum} ${}\{{}$\5
\1\X8:kinds\X\5
${},{}$\5
\X9:alternative kind names\X\5
\2${}\}{}$ \index{kind t+\&{kind\_t}}\&{kind\_t};\6
\8\#\&{undef} \index{DEF KIND+\.{DEF\_KIND}}\.{DEF\_KIND}
\As10, 54, 74, 79, 84, 93, 120, 178\ETs325.
\U336.\Y
\fi

\M{7}

\Y\B\4\X7:define \\{content\_name} and \\{definition\_name}\X${}\E{}$\6
\8\#\&{define} \index{DEF KIND+\.{DEF\_KIND}}\.{DEF\_KIND} ${}(\|C,\39\|D,\39\|N)\5{\#}\|C{}$\6
\&{const} \&{char} ${}{*}\index{content name+\\{content\_name}}\\{content\_name}[\T{32}]$ $\K{}$\5
\1${}\{{}$\5
\X8:kinds\X\5
${}\}{}$\5
\2;\6
\8\#\&{undef} \index{DEF KIND+\.{DEF\_KIND}}\.{DEF\_KIND}\7
\index{printf+\\{printf}}\\{printf}(\.{"const\ char\ *content}\)\.{\_name[32]=\{"});\6
\&{for} ${}(\|k\K\T{0};{}$ ${}\|k\Z\T{31};{}$ ${}\|k\PP){}$\5
\1${}\{{}$\5
${}\index{printf+\\{printf}}\\{printf}(\.{"\\"\%s\\""},\39\index{content name+\\{content\_name}}\\{content\_name}[\|k]);{}$\6
\&{if} ${}(\|k<\T{31}){}$\1\5
\index{printf+\\{printf}}\\{printf}(\.{",\ "});\2\6
\4${}\}{}$\2\6
\index{printf+\\{printf}}\\{printf}(\.{"\};\\n\\n"});\6
\8\#\&{define} \index{DEF KIND+\.{DEF\_KIND}}\.{DEF\_KIND} ${}(\|C,\39\|D,\39\|N)\5{\#}\|D{}$\6
\&{const} \&{char} ${}{*}\index{definition name+\\{definition\_name}}\\{definition\_name}[\T{\^20}]$ $\K{}$\5
\1${}\{{}$\5
\X8:kinds\X\5
${}\}{}$\5
\2;\6
\8\#\&{undef} \index{DEF KIND+\.{DEF\_KIND}}\.{DEF\_KIND}\6
\index{printf+\\{printf}}\\{printf}(\.{"const\ char\ *definit}\)\.{ion\_name[32]=\{"});\6
\&{for} ${}(\|k\K\T{0};{}$ ${}\|k\Z\T{31};{}$ ${}\|k\PP){}$\5
\1${}\{{}$\5
${}\index{printf+\\{printf}}\\{printf}(\.{"\\"\%s\\""},\39\index{definition name+\\{definition\_name}}\\{definition\_name}[\|k]);{}$\6
\&{if} ${}(\|k<\T{31}){}$\1\5
\index{printf+\\{printf}}\\{printf}(\.{",\ "});\2\6
\4${}\}{}$\2\6
\index{printf+\\{printf}}\\{printf}(\.{"\};\\n\\n"});
\U337.\Y
\fi

\M{8}

\goodbreak
\index{glyph kind+\\{glyph\_kind}}
\index{font kind+\\{font\_kind}}
\index{penalty kind+\\{penalty\_kind}}
\index{int kind+\\{int\_kind}}
\index{kern kind+\\{kern\_kind}}
\index{xdimen kind+\\{xdimen\_kind}}
\index{ligature kind+\\{ligature\_kind}}
\index{hyphen kind+\\{hyphen\_kind}}
\index{glue kind+\\{glue\_kind}}
\index{language kind+\\{language\_kind}}
\index{rule kind+\\{rule\_kind}}
\index{image kind+\\{image\_kind}}
\index{baseline kind+\\{baseline\_kind}}
\index{dimen kind+\\{dimen\_kind}}
\index{hbox kind+\\{hbox\_kind}}
\index{vbox kind+\\{vbox\_kind}}
\index{par kind+\\{par\_kind}}
\index{math kind+\\{math\_kind}}
\index{table kind+\\{table\_kind}}
\index{item kind+\\{item\_kind}}
\index{hset kind+\\{hset\_kind}}
\index{vset kind+\\{vset\_kind}}
\index{hpack kind+\\{hpack\_kind}}
\index{vpack kind+\\{vpack\_kind}}
\index{stream kind+\\{stream\_kind}}
\index{page kind+\\{page\_kind}}
\index{range kind+\\{range\_kind}}
\index{adjust kind+\\{adjust\_kind}}
\index{param kind+\\{param\_kind}}
\index{text kind+\\{text\_kind}}
\index{list kind+\\{list\_kind}}
\label{kinddef}
\Y\B\4\X8:kinds\X${}\E{}$\6
\index{DEF KIND+\.{DEF\_KIND}}\.{DEF\_KIND}(\|t\J\\{ext}${},\39{}$\|t\J\\{ext}${},\39\T{0}),{}$\6
\index{DEF KIND+\.{DEF\_KIND}}\.{DEF\_KIND}(\|l\J\\{ist}${},\39{}$\|l\J\\{ist}${},\39\T{1}),{}$\6
\index{DEF KIND+\.{DEF\_KIND}}\.{DEF\_KIND}(\|p\J\\{aram}${},\39{}$\|p\J\\{aram}${},\39\T{2}),{}$\6
\index{DEF KIND+\.{DEF\_KIND}}\.{DEF\_KIND}(\|x\J\\{dimen}${},\39{}$\|x\J\\{dimen}${},\39\T{3}),{}$\6
\index{DEF KIND+\.{DEF\_KIND}}\.{DEF\_KIND}(\|a\J\\{djust}${},\39{}$\|a\J\\{djust}${},\39\T{4}),{}$\6
\index{DEF KIND+\.{DEF\_KIND}}\.{DEF\_KIND}(\|g\J\\{lyph}${},\39{}$\|f\J\\{ont}${},\39\T{5}),{}$\6
\index{DEF KIND+\.{DEF\_KIND}}\.{DEF\_KIND}(\|k\J\\{ern}${},\39{}$\|d\J\\{imen}${},\39\T{6}),{}$\6
\index{DEF KIND+\.{DEF\_KIND}}\.{DEF\_KIND}(\|g\J\\{lue}${},\39{}$\|g\J\\{lue}${},\39\T{7}),{}$\6
\index{DEF KIND+\.{DEF\_KIND}}\.{DEF\_KIND}(\|l\J\\{igature}${},\39{}$\|l\J\\{igature}${},\39\T{8}),{}$\6
\index{DEF KIND+\.{DEF\_KIND}}\.{DEF\_KIND}(\|h\J\\{yphen}${},\39{}$\|h\J\\{yphen}${},\39\T{9}),{}$\6
\index{DEF KIND+\.{DEF\_KIND}}\.{DEF\_KIND}(\|l\J\\{anguage}${},\39{}$\|l\J\\{anguage}${},\39\T{10}),{}$\6
\index{DEF KIND+\.{DEF\_KIND}}\.{DEF\_KIND}(\|r\J\\{ule}${},\39{}$\|r\J\\{ule}${},\39\T{11}),{}$\6
\index{DEF KIND+\.{DEF\_KIND}}\.{DEF\_KIND}(\|i\J\\{mage}${},\39{}$\|i\J\\{mage}${},\39\T{12}),{}$\6
\index{DEF KIND+\.{DEF\_KIND}}\.{DEF\_KIND}(\|l\J\\{eaders}${},\39{}$\|l\J\\{eaders}${},\39\T{13}),{}$\6
\index{DEF KIND+\.{DEF\_KIND}}\.{DEF\_KIND}(\|b\J\\{aseline}${},\39{}$\|b\J\\{aseline}${},\39\T{14}),{}$\6
\index{DEF KIND+\.{DEF\_KIND}}\.{DEF\_KIND}(\|h\J\|b\J\\{ox}${},\39{}$\|h\J\|b\J\\{ox}${},\39\T{15}),{}$\6
\index{DEF KIND+\.{DEF\_KIND}}\.{DEF\_KIND}(\|v\J\|b\J\\{ox}${},\39{}$\|v\J\|b\J\\{ox}${},\39\T{16}),{}$\6
\index{DEF KIND+\.{DEF\_KIND}}\.{DEF\_KIND}(\|p\J\\{ar}${},\39{}$\|p\J\\{ar}${},\39\T{17}),{}$\6
\index{DEF KIND+\.{DEF\_KIND}}\.{DEF\_KIND}(\|m\J\\{ath}${},\39{}$\|m\J\\{ath}${},\39\T{18}),{}$\6
\index{DEF KIND+\.{DEF\_KIND}}\.{DEF\_KIND}(\|t\J\\{able}${},\39{}$\|t\J\\{able}${},\39\T{19}),{}$\6
\index{DEF KIND+\.{DEF\_KIND}}\.{DEF\_KIND}(\|i\J\\{tem}${},\39{}$\|i\J\\{tem}${},\39\T{20}),{}$\6
\index{DEF KIND+\.{DEF\_KIND}}\.{DEF\_KIND}(\|h\J\\{set}${},\39{}$\|h\J\\{set}${},\39\T{21}),{}$\6
\index{DEF KIND+\.{DEF\_KIND}}\.{DEF\_KIND}(\|v\J\\{set}${},\39{}$\|v\J\\{set}${},\39\T{22}),{}$\6
\index{DEF KIND+\.{DEF\_KIND}}\.{DEF\_KIND}(\|h\J\\{pack}${},\39{}$\|h\J\\{pack}${},\39\T{23}),{}$\6
\index{DEF KIND+\.{DEF\_KIND}}\.{DEF\_KIND}(\|v\J\\{pack}${},\39{}$\|v\J\\{pack}${},\39\T{24}),{}$\6
\index{DEF KIND+\.{DEF\_KIND}}\.{DEF\_KIND}(\|s\J\\{tream}${},\39{}$\|s\J\\{tream}${},\39\T{25}),{}$\6
\index{DEF KIND+\.{DEF\_KIND}}\.{DEF\_KIND}(\|p\J\\{age}${},\39{}$\|p\J\\{age}${},\39\T{26}),{}$\6
\index{DEF KIND+\.{DEF\_KIND}}\.{DEF\_KIND}(\|r\J\\{ange}${},\39{}$\|r\J\\{ange}${},\39\T{27}),{}$\6
\index{DEF KIND+\.{DEF\_KIND}}\.{DEF\_KIND}(\|u\J\\{ndefined1}${},\39{}$\|u\J\\{ndefined1}${},\39\T{28}),{}$\6
\index{DEF KIND+\.{DEF\_KIND}}\.{DEF\_KIND}(\|u\J\\{ndefined2}${},\39{}$\|u\J\\{ndefined2}${},\39\T{29}),{}$\6
\index{DEF KIND+\.{DEF\_KIND}}\.{DEF\_KIND}(\|u\J\\{ndefined3}${},\39{}$\|u\J\\{ndefined3}${},\39\T{30}),{}$\6
\index{DEF KIND+\.{DEF\_KIND}}\.{DEF\_KIND}(\|p\J\\{enalty}${},\39{}$\|i\J\\{nt}${},\39\T{31}{}$)\hbox{}
\Us6\ET7.\Y
\fi

\M{9}

For a few kind values we have
alternative names; we will use them in the definition section
to express different intentions when using them.
\Y\B\4\X9:alternative kind names\X${}\E{}$\6
$\index{font kind+\\{font\_kind}}\\{font\_kind}\K\index{glyph kind+\\{glyph\_kind}}\\{glyph\_kind},\39\index{int kind+\\{int\_kind}}\\{int\_kind}\K\index{penalty kind+\\{penalty\_kind}}\\{penalty\_kind},\39\index{dimen kind+\\{dimen\_kind}}\\{dimen\_kind}\K\index{kern kind+\\{kern\_kind}}\\{kern\_kind}$ $,{}$\6
\hbox{{}}
\U6.\Y
\fi

\M{10}

The info\index{info value} values can be used to represent numbers in the range 0 to 7; for an example
see the \index{hput glyph+\\{hput\_glyph}}\\{hput\_glyph} function later in this section.
Mostly, however, the individual bits are used as flags indicating the presence
or absence of immediate parameter values. If the info bit is set, it
means the corresponding parameter is present as an immediate value; if it
is zero, it means that there is no immediate parameter value present, and
the node specification will reveal what value to use instead.
In some cases there is a common default value that can be used, in other
cases a one byte reference number is used to select a predefined value.

To make the binary
representation of the info bits more readable, we define an
enumeration type.

\index{b000+\\{b000}}
\index{b001+\\{b001}}
\index{b010+\\{b010}}
\index{b011+\\{b011}}
\index{b100+\\{b100}}
\index{b101+\\{b101}}
\index{b110+\\{b110}}
\index{b111+\\{b111}}
\Y\B\4\X6:hint basic types\X${}\mathrel+\E{}$\6
\&{typedef} \&{enum} ${}\{{}$\5
\1${}\\{b000}\K\T{0},\39\\{b001}\K\T{1},\39\\{b010}\K\T{2},\39\\{b011}\K\T{3},\39\\{b100}\K\T{4},\39\\{b101}\K\T{5},\39\\{b110}\K\T{6},\39\\{b111}\K{}$\T{7}\5
\2${}\}{}$ \index{info t+\&{info\_t}}\&{info\_t};
\Y
\fi

\M{11}


After the start byte follows the node content and it is the purpose of
the start byte to reveal the exact syntax and semantics of the node
content. Because we want to be able to read the short form of a \HINT/
file in forward direction and in backward direction, the start byte is
duplicated after the content as an end\index{end byte} byte.


We store a kind and an info value in one byte and call this a tag.
The following macros are used to assemble and disassemble tags:\index{TAG+\.{TAG}}
\Y\B\4\X11:hint macros\X${}\E{}$\6
\8\#\&{define} \index{KIND+\.{KIND}}\.{KIND}(\|T)\5${}(((\|T)\GG\T{3})\AND\T{\^1F}){}$\6
\8\#\&{define} \index{NAME+\.{NAME}}\.{NAME}(\|T)\5\index{content name+\\{content\_name}}\\{content\_name}[\index{KIND+\.{KIND}}\.{KIND}(\|T)]\6
\8\#\&{define} \index{INFO+\.{INFO}}\.{INFO}(\|T)\5${}((\|T)\AND\T{\^7}){}$\6
\8\#\&{define} ${}\.{TAG}(\|K,\39\|I)\5(((\|K)\LL\T{3})\OR(\|I)){}$
\As75, 110, 121, 254, 261\ETs273.
\Us336\ET434.\Y
\fi

\M{12}

Writing a  short format \HINT/ file is implemented by a collection of {\it hput\_\kern 1pt\dots\/}  functions;
they follow most of the time the same schema:
\itemize
\item First, we define a variable for \index{info+\\{info}}\\{info}.
\item Then follows the main part of the function body, where we
decide on the output format, do the actual output and set the \index{info+\\{info}}\\{info} value accordingly.
\item We combine the info value with the kind value and return the correct tag.
\item The tag value will be passed to \index{hput tags+\\{hput\_tags}}\\{hput\_tags} which generates
debugging information, if requested, and stores the tag before and after the node content.
\enditemize


After these preparations, we turn our attention again to the \index{hput glyph+\\{hput\_glyph}}\\{hput\_glyph} function.

The font number in a glyph node is between 0 and 255 and fits nicely in one byte,
but the character code is more difficult: we want to store the most common character
codes as a single byte and less frequent codes with two, three, or even four byte.
Naturally, we use the \index{info+\\{info}}\\{info} bits to store the number of bytes needed for the character code.

\codesection{\putsymbol}{Writing the Short Format}\putindex{1}{2}{Glyphs}
\Y\B\4\X12:put functions\X${}\E{}$\6
\&{uint8\_t} \index{hput glyph+\\{hput\_glyph}}\\{hput\_glyph}(\index{glyph t+\&{glyph\_t}}\&{glyph\_t} ${}{*}\|g){}$\1\1\2\2\1\6
\4${}\{{}$\5
\index{info t+\&{info\_t}}\&{info\_t} \index{info+\\{info}}\\{info};\7
\&{if} ${}(\|g\MG\|c\Z\T{\^FF}{}$)\5
\1${}\{{}$\5
${}\index{HPUT8+\.{HPUT8}}\.{HPUT8}(\|g\MG\|c){}$;\5
${}\index{info+\\{info}}\\{info}\K\T{1}{}$;\5
${}\}{}$\2\6
\&{else} \&{if} ${}(\|g\MG\|c\Z\T{\^FFFF}{}$)\5
\1${}\{{}$\5
${}\index{HPUT16+\.{HPUT16}}\.{HPUT16}(\|g\MG\|c){}$;\5
${}\index{info+\\{info}}\\{info}\K\T{2}{}$;\5
${}\}{}$\2\6
\&{else} \&{if} ${}(\|g\MG\|c\Z\T{\^FFFFFF}{}$)\5
\1${}\{{}$\5
${}\index{HPUT24+\.{HPUT24}}\.{HPUT24}(\|g\MG\|c){}$;\5
${}\index{info+\\{info}}\\{info}\K\T{3}{}$;\5
${}\}{}$\2\6
\&{else}\5
\1${}\{{}$\5
${}\index{HPUT32+\.{HPUT32}}\.{HPUT32}(\|g\MG\|c){}$;\5
${}\index{info+\\{info}}\\{info}\K\T{4}{}$;\5
${}\}{}$\2\6
${}\index{HPUT8+\.{HPUT8}}\.{HPUT8}(\|g\MG\|f){}$;\6
\&{return} \.{TAG}${}(\index{glyph kind+\\{glyph\_kind}}\\{glyph\_kind},\39\index{info+\\{info}}\\{info});{}$\6
\4${}\}{}$\2
\As13, 34, 51, 72, 83, 92, 94, 105, 109, 119, 129, 137, 148, 158, 171, 184, 192, 201, 227, 235, 260, 267, 270, 277, 297, 298, 304, 312, 331\ETs360.
\Us435\ET438.\Y
\fi

\M{13}
The \index{hput tags+\\{hput\_tags}}\\{hput\_tags} function is called after the node content has been written to the
stream. It gets a the position of the start byte and the tag. With this information
it writes the start byte at the given position and the end byte at the current stream position.
\Y\B\4\X12:put functions\X${}\mathrel+\E{}$\6
\&{void} \index{hput tags+\\{hput\_tags}}\\{hput\_tags}(\&{uint32\_t} \index{pos+\\{pos}}\\{pos}${},\39{}$\&{uint8\_t} \index{tag+\\{tag}}\\{tag})\1\1\2\2\1\6
\4${}\{{}$\5
${}\.{DBGTAG}(\index{tag+\\{tag}}\\{tag},\39\index{hstart+\\{hstart}}\\{hstart}+\index{pos+\\{pos}}\\{pos});{}$\5
${}\.{DBGTAG}(\index{tag+\\{tag}}\\{tag},\39\index{hpos+\\{hpos}}\\{hpos});{}$\5
\index{HPUTX+\.{HPUTX}}\.{HPUTX}(\T{1});\5
${}{*}(\index{hstart+\\{hstart}}\\{hstart}+\index{pos+\\{pos}}\\{pos})\K{*}(\index{hpos+\\{hpos}}\\{hpos}\PP)\K\index{tag+\\{tag}}\\{tag}{}$;\5
${}\}{}$\2
\Y
\fi

\M{14}



The variables \index{hpos+\\{hpos}}\\{hpos} and \index{hstart+\\{hstart}}\\{hstart}, the
macros  \index{HPUT8+\.{HPUT8}}\.{HPUT8}, \index{HPUT16+\.{HPUT16}}\.{HPUT16}, \index{HPUT24+\.{HPUT24}}\.{HPUT24}, \index{HPUT32+\.{HPUT32}}\.{HPUT32}, and  \index{HPUTX+\.{HPUTX}}\.{HPUTX} are all defined in section~\secref{HPUT};
they put 8, 16, 24, or 32 bits into the output stream and check for sufficient space in the output buffer.
The macro \.{DBGTAG} writes debugging output; its definition is found in section~\secref{error_section}.

Now that we have seen the general outline of the \.{shrink} program, starting with a long format file
and ending with a short format file, we will look at the program \index{stretch+\.{stretch}}\.{stretch} that reverses this transformation.


\subsection{Parsing the Short Format}
The inverse of writing the short format with a {\it hput\_\kern 1pt\dots\/}  function
is reading the short format with a {\it hget\_\kern 1pt\dots\/}  function.

The schema of  {\it hget\_\kern 1pt\dots\/}  functions reverse the schema of  {\it hput\_\kern 1pt\dots\/}  functions.
Here is the code for the initial and final part of a get function:

\Y\B\4\X14:read the start byte \|a\X${}\E{}$\6
\&{uint8\_t} \|a${},{}$ \|z;\C{ the start and the end byte}\6
\&{uint32\_t} \\{node\_pos}${}\K\index{hpos+\\{hpos}}\\{hpos}-\index{hstart+\\{hstart}}\\{hstart};{}$\7
\&{if} ${}(\index{hpos+\\{hpos}}\\{hpos}\G\index{hend+\\{hend}}\\{hend}){}$\1\5
\.{QUIT}(\.{"Attempt\ to\ read\ a\ s}\)\.{tart\ byte\ at\ the\ end}\)\.{\ of\ the\ section"});\2\6
\index{HGETTAG+\.{HGETTAG}}\.{HGETTAG}(\|a);
\Us16, 91, 118, 128, 135, 147, 157, 200, 240, 294, 311, 317\ETs330.\Y
\fi

\M{15}

\Y\B\4\X15:read and check the end byte \|z\X${}\E{}$\6
\index{HGETTAG+\.{HGETTAG}}\.{HGETTAG}(\|z);\5
\&{if} ${}(\|a\I\|z){}$\1\5
${}\.{QUIT}(\.{"Tag\ mismatch\ [\%s,\%d}\)\.{]!=[\%s,\%d]\ at\ 0x\%x\ t}\)\.{o\ "}\.{SIZE\_F}\.{"\\n"},\30\39\index{NAME+\.{NAME}}\.{NAME}(\|a),\39\index{INFO+\.{INFO}}\.{INFO}(\|a),\39\index{NAME+\.{NAME}}\.{NAME}(\|z),\39\index{INFO+\.{INFO}}\.{INFO}(\|z),\30\39\\{node\_pos},\39\index{hpos+\\{hpos}}\\{hpos}-\index{hstart+\\{hstart}}\\{hstart}-\T{1}){}$;\2
\Us16, 91, 118, 128, 135, 147, 157, 200, 240, 294, 311, 317\ETs330.\Y
\fi

\M{16}


The central routine to parse\index{parsing} the content section of a short format
file is the function \index{hget content node+\\{hget\_content\_node}}\\{hget\_content\_node} which calls \index{hget content+\\{hget\_content}}\\{hget\_content} to
do most of the processing.

\index{hget content node+\\{hget\_content\_node}}\\{hget\_content\_node} will read a content node in short format and write
it out in long format: It reads the start\index{start byte} byte \|a, writes the \index{START+\ts{START}}\ts{START}
token using the function \index{hwrite start+\\{hwrite\_start}}\\{hwrite\_start}, and based on \index{KIND+\.{KIND}}\.{KIND}(\|a), it
writes the nodes' keyword found in the \index{content name+\\{content\_name}}\\{content\_name} array.  Then it
calls \index{hget content+\\{hget\_content}}\\{hget\_content} to read the nodes content and write it out.
Finally it reads the end\index{end byte} byte, checks it against the start byte, and
finishes up the content node by writing the \index{END+\ts{END}}\ts{END} token using the
\index{hwrite end+\\{hwrite\_end}}\\{hwrite\_end} function. The function returns the tag byte so that
the calling function might check that the content node meets its requirements.

\index{hget content+\\{hget\_content}}\\{hget\_content} uses the start byte \|a, passed as a parameter, to
branch directly to the reading routine for the given combination of
kind and info value.  The reading routine will read the data and store
its internal representation in a variable.  All that the \index{stretch+\.{stretch}}\.{stretch}
program needs to do with this internal representation is writing it in
the long format. As we will see, the call to the proper
{\it hwrite\_\kern 1pt\dots} function is included as final part of the the
reading routine (avoiding another switch statement).


\codesection{\getsymbol}{Reading the Short Format}\getindex{1}{2}{Content Nodes}
\Y\B\4\X16:get functions\X${}\E{}$\6
\&{uint8\_t} \index{hget content node+\\{hget\_content\_node}}\\{hget\_content\_node}(\&{void})\1\1\2\2\1\6
\4${}\{{}$\5
\X14:read the start byte \|a\X\5
\index{hwrite start+\\{hwrite\_start}}\\{hwrite\_start}(\,);\5
${}\index{hwritef+\\{hwritef}}\\{hwritef}(\.{"\%s"},\39\index{content name+\\{content\_name}}\\{content\_name}[\index{KIND+\.{KIND}}\.{KIND}(\|a)]);{}$\5
\index{hget content+\\{hget\_content}}\\{hget\_content}(\|a);\5
\X15:read and check the end byte \|z\X\5
\index{hwrite end+\\{hwrite\_end}}\\{hwrite\_end}(\,);\5
\&{return} \|a;\6
\4${}\}{}$\2\7
\&{void} \index{hget content+\\{hget\_content}}\\{hget\_content}(\&{uint8\_t} \|a)\1\1\2\2\1\6
\4${}\{{}$\5
\&{uint32\_t} \\{node\_pos}${}\K\index{hpos+\\{hpos}}\\{hpos}-\index{hstart+\\{hstart}}\\{hstart};{}$\7
\&{switch} (\|a)\6
\1${}\{{}$\5
\X18:cases to get content\X\hbox{\1}\6
\4\&{default}:\5
\.{TAGERR}(\|a);\5
\&{break};\hbox{\2}\6
\4${}\}{}$\2\6
\4${}\}{}$\2
\As50, 73, 82, 90, 91, 118, 128, 135, 147, 157, 200, 240, 250, 259, 303, 317, 323, 330\ETs359.
\U439.\Y
\fi

\M{17}

We implement the code to read a glyph node in two stages.
First we define a general reading macro $\index{HGET GLYPH+\.{HGET\_GLYPH}}\.{HGET\_GLYPH}(\|I,\|G)$ that reads a glyph node with info value \|I into
a \index{glyph t+\&{glyph\_t}}\&{glyph\_t} variable \|G; then we insert this macro
in the above switch statement for all cases where it applies.
Knowing the function \index{hput glyph+\\{hput\_glyph}}\\{hput\_glyph}, the macro \index{HGET GLYPH+\.{HGET\_GLYPH}}\.{HGET\_GLYPH} should not be a surprise.
It reverses \index{hput glyph+\\{hput\_glyph}}\\{hput\_glyph}, storing the glyph node in its internal representation.
After that, the \index{stretch+\.{stretch}}\.{stretch} program calls \index{hwrite glyph+\\{hwrite\_glyph}}\\{hwrite\_glyph} to produce the glyph
node in long format.

\codesection{\getsymbol}{Reading the Short Format}\getindex{1}{2}{Glyphs}
\Y\B\4\X17:get macros\X${}\E{}$\6
\8\#\&{define} $\index{HGET GLYPH+\.{HGET\_GLYPH}}\.{HGET\_GLYPH}(\|I,\39\|G)$ \6
\&{if} ${}(\|I\E\T{1}){}$\1\5
${}(\|G).\|c\K\index{HGET8+\.{HGET8}}\.{HGET8};{}$\2\6
\&{else} \&{if} ${}(\|I\E\T{2}){}$\1\5
${}\index{HGET16+\.{HGET16}}\.{HGET16}((\|G).\|c);{}$\2\6
\&{else} \&{if} ${}(\|I\E\T{3}){}$\1\5
${}\index{HGET24+\.{HGET24}}\.{HGET24}((\|G).\|c);{}$\2\6
\&{else} \&{if} ${}(\|I\E\T{4}){}$\1\5
${}\index{HGET32+\.{HGET32}}\.{HGET32}((\|G).\|c);{}$\2\6
${}(\|G).\|f\K\index{HGET8+\.{HGET8}}\.{HGET8}{}$;\5
${}\index{REF RNG+\.{REF\_RNG}}\.{REF\_RNG}(\index{font kind+\\{font\_kind}}\\{font\_kind},\39(\|G).\|f){}$;\6
${}\index{hwrite glyph+\\{hwrite\_glyph}}\\{hwrite\_glyph}({\AND}(\|G)){}$;
\As89, 95, 104, 117, 127, 136, 146, 156, 163, 170, 177, 183, 191, 199, 206, 211, 226, 234, 246\ETs334.
\U439.\Y
\fi

\M{18}

Note that we allow aglyph to reference a font even before that font is defined.
This is necessary because fonts usually contain definitions---for example
the fonts hyphen character---that reference this or other fonts.


\Y\B\4\X18:cases to get content\X${}\E{}$\6
\4\&{case} \.{TAG}${}(\index{glyph kind+\\{glyph\_kind}}\\{glyph\_kind},\39\T{1}){}$:\5
\1${}\{{}$\5
\index{glyph t+\&{glyph\_t}}\&{glyph\_t} \|g;\5
${}\index{HGET GLYPH+\.{HGET\_GLYPH}}\.{HGET\_GLYPH}(\T{1},\39\|g){}$;\5
${}\}{}$\5
\2\&{break};\6
\4\&{case} \.{TAG}${}(\index{glyph kind+\\{glyph\_kind}}\\{glyph\_kind},\39\T{2}){}$:\5
\1${}\{{}$\5
\index{glyph t+\&{glyph\_t}}\&{glyph\_t} \|g;\5
${}\index{HGET GLYPH+\.{HGET\_GLYPH}}\.{HGET\_GLYPH}(\T{2},\39\|g){}$;\5
${}\}{}$\5
\2\&{break};\6
\4\&{case} \.{TAG}${}(\index{glyph kind+\\{glyph\_kind}}\\{glyph\_kind},\39\T{3}){}$:\5
\1${}\{{}$\5
\index{glyph t+\&{glyph\_t}}\&{glyph\_t} \|g;\5
${}\index{HGET GLYPH+\.{HGET\_GLYPH}}\.{HGET\_GLYPH}(\T{3},\39\|g){}$;\5
${}\}{}$\5
\2\&{break};\6
\4\&{case} \.{TAG}${}(\index{glyph kind+\\{glyph\_kind}}\\{glyph\_kind},\39\T{4}){}$:\5
\1${}\{{}$\5
\index{glyph t+\&{glyph\_t}}\&{glyph\_t} \|g;\5
${}\index{HGET GLYPH+\.{HGET\_GLYPH}}\.{HGET\_GLYPH}(\T{4},\39\|g){}$;\5
${}\}{}$\5
\2\&{break};
\As103, 108, 116, 126, 155, 162, 169, 176, 182, 190, 198, 205, 210, 215, 219, 225, 233, 241, 245\ETs333.
\U16.\Y
\fi

\M{19}

If this two stage method seems strange to you, consider what the \CEE/ compiler will
do with it. It will expand the \index{HGET GLYPH+\.{HGET\_GLYPH}}\.{HGET\_GLYPH} macro four times inside the switch
statement. The macro is, however, expanded with a constant \|I value, so the expansion
of the \&{if} statement in $\index{HGET GLYPH+\.{HGET\_GLYPH}}\.{HGET\_GLYPH}(\T{1},\|g)$, for example,
will become ``\&{if} ${}(\T{1}\E\T{1})$ \dots\ \&{else} \&{if} ${}(\T{1}\E\T{2})$ \dots''
and the compiler will have no difficulties eliminating the constant tests and
the dead branches altogether. This is the most effective use of the switch statement:
a single jump takes you to a specialized code to handle just the given combination
of kind and info value.

Last not least, we implement the function \index{hwrite glyph+\\{hwrite\_glyph}}\\{hwrite\_glyph} to write a
glyph node in long form---that is: in a form that is as readable as possible.

\subsection{Writing the Long Format}

The \index{hwrite glyph+\\{hwrite\_glyph}}\\{hwrite\_glyph} function inverts the scanning and parsing process we have described
at the very beginning of this chapter.
To implement the \index{hwrite glyph+\\{hwrite\_glyph}}\\{hwrite\_glyph} function, we use the function \index{hwrite charcode+\\{hwrite\_charcode}}\\{hwrite\_charcode}
to write the character code.
Besides writing the character code as a decimal number, this function can handle also other
representations of character codes as fully explained in section~\secref{chars}.
We split off the writing of the opening and the closing pointed bracket, because
we will need this function very often and because it will keep track of the \index{nesting+\\{nesting}}\\{nesting}
of nodes and indent them accordingly. The \index{hwrite range+\\{hwrite\_range}}\\{hwrite\_range} function used in \index{hwrite end+\\{hwrite\_end}}\\{hwrite\_end}
is discussed in section~\secref{range}.

\codesection{\wrtsymbol}{Writing the Long Format}\wrtindex{1}{2}{Glyphs}
\Y\B\4\X19:write functions\X${}\E{}$\6
\&{int} \index{nesting+\\{nesting}}\\{nesting}${}\K\T{0};{}$\7
\&{void} \index{hwrite nesting+\\{hwrite\_nesting}}\\{hwrite\_nesting}(\&{void})\1\1\2\2\1\6
\4${}\{{}$\5
\&{int} \|i;\7
\index{hwritec+\\{hwritec}}\\{hwritec}(\.{'\\n'});\6
\&{for} ${}(\|i\K\T{0};{}$ ${}\|i<\index{nesting+\\{nesting}}\\{nesting};{}$ ${}\|i\PP){}$\1\5
\index{hwritec+\\{hwritec}}\\{hwritec}(\.{'\ '});\2\6
\4${}\}{}$\2\7
\&{void} \index{hwrite start+\\{hwrite\_start}}\\{hwrite\_start}(\&{void})\1\1\2\2\1\6
\4${}\{{}$\5
\index{hwrite nesting+\\{hwrite\_nesting}}\\{hwrite\_nesting}(\,);\5
\index{hwritec+\\{hwritec}}\\{hwritec}(\.{'<'});\5
${}\index{nesting+\\{nesting}}\\{nesting}\PP;{}$\6
\4${}\}{}$\2\7
\&{void} \index{hwrite end+\\{hwrite\_end}}\\{hwrite\_end}(\&{void})\1\1\2\2\1\6
\4${}\{{}$\5
${}\index{nesting+\\{nesting}}\\{nesting}\MM;{}$\5
\index{hwritec+\\{hwritec}}\\{hwritec}(\.{'>'});\6
\&{if} ${}(\index{nesting+\\{nesting}}\\{nesting}\E\T{0}\W\index{section no+\\{section\_no}}\\{section\_no}\E\T{2}){}$\1\5
\index{hwrite range+\\{hwrite\_range}}\\{hwrite\_range}(\,);\2\6
\4${}\}{}$\2\7
\&{void} \index{hwrite comment+\\{hwrite\_comment}}\\{hwrite\_comment}(\&{char} ${}{*}\index{str+\\{str}}\\{str}){}$\1\1\2\2\1\6
\4${}\{{}$\5
\&{char} \|c;\7
\&{if} ${}(\index{str+\\{str}}\\{str}\E\NULL){}$\1\5
\&{return};\2\6
\index{hwritef+\\{hwritef}}\\{hwritef}(\.{"\ ("});\6
\&{while} ${}((\|c\K{*}\index{str+\\{str}}\\{str}\PP)\I\T{0}){}$\1\6
\&{if} ${}(\|c\E\.{'('}\V\|c\E\.{')'}){}$\1\5
\index{hwritec+\\{hwritec}}\\{hwritec}(\.{'\_'});\2\6
\&{else} \&{if} ${}(\|c\E\.{'\\n'}){}$\1\5
\index{hwritef+\\{hwritef}}\\{hwritef}(\.{"\\n("});\2\6
\&{else}\1\5
\index{hwritec+\\{hwritec}}\\{hwritec}(\|c);\2\2\6
\index{hwritec+\\{hwritec}}\\{hwritec}(\.{')'});\6
\4${}\}{}$\2\7
\&{void} \index{hwrite glyph+\\{hwrite\_glyph}}\\{hwrite\_glyph}(\index{glyph t+\&{glyph\_t}}\&{glyph\_t} ${}{*}\|g){}$\1\1\2\2\1\6
\4${}\{{}$\5
\&{char} ${}{*}\|n\K\index{hfont name+\\{hfont\_name}}\\{hfont\_name}[\|g\MG\|f];{}$\7
${}\index{hwrite charcode+\\{hwrite\_charcode}}\\{hwrite\_charcode}(\|g\MG\|c);{}$\5
${}\index{hwrite ref+\\{hwrite\_ref}}\\{hwrite\_ref}(\|g\MG\|f);{}$\6
\&{if} ${}(\|n\I\NULL){}$\1\5
\index{hwrite comment+\\{hwrite\_comment}}\\{hwrite\_comment}(\|n);\2\6
\4${}\}{}$\2
\As28, 33, 49, 65, 76, 81, 88, 99, 115, 125, 133, 145, 154, 168, 175, 189, 197, 220, 232, 258, 291, 292, 302, 310, 322, 335\ETs358.
\U439.\Y
\fi

\M{20}

Now that we have completed the round trip of shrinking and stretching
glyph nodes, we continue the description of the \HINT/ file formats
in a more systematic way.


\section{Data Types}\hascode
\subsection{Integers}
\label{integers}
We have already seen the pattern/\kern -1pt action rule for unsigned decimal\index{decimal number} numbers. It remains
to define the macro \index{SCAN UDEC+\.{SCAN\_UDEC}}\.{SCAN\_UDEC} which converts a string containing an unsigned\index{unsigned} decimal
number into an unsigned integer\index{integer}.
We use the \CEE/ library function \index{strtoul+\\{strtoul}}\\{strtoul}:

\readcode
\Y\B\4\X20:scanning macros\X${}\E{}$\6
\8\#\&{define} \index{SCAN UDEC+\.{SCAN\_UDEC}}\.{SCAN\_UDEC}(\|S)\5${}\index{yylval+\\{yylval}}\\{yylval}.\|u\K\index{strtoul+\\{strtoul}}\\{strtoul}(\|S,\39\NULL,\39\T{10}){}$
\As23, 26, 29, 38, 40, 42, 44, 55, 58\ETs142.
\U436.\Y
\fi

\M{21}
Unsigned integers can be given in hexadecimal\index{hexadecimal} notation as well.
\Y\B\4\X21:scanning definitions\X${}\E{}$\6
${}\8\re{\vb{HEX}}{}$\ac\vb{[0-9A-F]}\eac
\As30, 37, 39, 41, 43\ETs139.
\U436.\Y
\fi

\M{22}

\Y\B\4\X3:scanning rules\X${}\mathrel+\E{}$\6
${}\8\re{\vb{0x\{HEX\}+}}{}$\ac${}\index{SCAN HEX+\.{SCAN\_HEX}}\.{SCAN\_HEX}(\index{yytext+\\{yytext}}\\{yytext}+\T{2});{}$\5
\&{return} \index{UNSIGNED+\ts{UNSIGNED}}\ts{UNSIGNED};\eac
\Y
\fi

\M{23}

Note that the pattern above allows only upper case letters in the
hexadecimal notation for integers.

\Y\B\4\X20:scanning macros\X${}\mathrel+\E{}$\6
\8\#\&{define} \index{SCAN HEX+\.{SCAN\_HEX}}\.{SCAN\_HEX}(\|S)\5${}\index{yylval+\\{yylval}}\\{yylval}.\|u\K\index{strtoul+\\{strtoul}}\\{strtoul}(\|S,\39\NULL,\39\T{16}){}$
\Y
\fi

\M{24}

Last not least, we add rules for signed\index{signed integer} integers.
\Y\par
\par
\par
\Y\B\4\X2:symbols\X${}\mathrel+\E{}$\6
\8\%\&{token} $<$ \|i $>$ \index{SIGNED+\ts{SIGNED}}\ts{SIGNED} \6
\8\%\index{type+\&{type}}\&{type} $<$ \|i $>$ \index{integer+\nts{integer}}\nts{integer}
\Y
\fi

\M{25}

\Y\B\4\X3:scanning rules\X${}\mathrel+\E{}$\6
${}\8\re{\vb{[+-](0|[1-9][0-9]*)}}{}$\ac\index{SCAN DEC+\.{SCAN\_DEC}}\.{SCAN\_DEC}(\index{yytext+\\{yytext}}\\{yytext});\5
\&{return} \index{SIGNED+\ts{SIGNED}}\ts{SIGNED};\eac
\Y
\fi

\M{26}

\Y\B\4\X20:scanning macros\X${}\mathrel+\E{}$\6
\8\#\&{define} \index{SCAN DEC+\.{SCAN\_DEC}}\.{SCAN\_DEC}(\|S)\5${}\index{yylval+\\{yylval}}\\{yylval}.\|i\K\index{strtol+\\{strtol}}\\{strtol}(\|S,\39\NULL,\39\T{10}){}$
\Y
\fi

\M{27}

\Y\B\4\X5:parsing rules\X${}\mathrel+\E{}$\6
\index{integer+\nts{integer}}\nts{integer}: \1\1\5
\index{SIGNED+\ts{SIGNED}}\ts{SIGNED}\5
\hbox to 0.5em{\hss${}\OR{}$}\5
\index{UNSIGNED+\ts{UNSIGNED}}\ts{UNSIGNED}\5
${}\{{}$\1\5
${}\.{RNG}(\.{"number"},\39\.{\$1},\39\T{0},\39\index{INT32 MAX+\.{INT32\_MAX}}\.{INT32\_MAX});{}$\5
${}\}{}$\2;\2\2
\Y
\fi

\M{28}

To preserve the ``signedness'' of an integer also for positive signed integers
in the long format, we implement the function \index{hwrite signed+\\{hwrite\_signed}}\\{hwrite\_signed}.

\writecode
\Y\B\4\X19:write functions\X${}\mathrel+\E{}$\6
\&{void} \index{hwrite signed+\\{hwrite\_signed}}\\{hwrite\_signed}(\&{int32\_t} \|i)\1\1\2\2\1\6
\4${}\{{}$\6
\&{if} ${}(\|i<\T{0}){}$\1\5
${}\index{hwritef+\\{hwritef}}\\{hwritef}(\.{"\ -\%d"},\39{-}\|i);{}$\2\6
\&{else}\1\5
${}\index{hwritef+\\{hwritef}}\\{hwritef}(\.{"\ +\%d"},\39{+}\|i);{}$\2\6
\4${}\}{}$\2
\Y
\fi

\M{29}

Reading and writing integers in the short format is done directly with the {\tt HPUT} and {\tt HGET}
macros.


\subsection{Strings}
\label{strings}
Strings\index{string} are needed in the definition part of a \HINT/
file to specify names of objects, and in the long file format, we also use them for file\index{file name} names.
In the long format, strings are sequences of characters delimited by single quote\index{single quote} characters;
for example: ``\.{'Hello'}'' or ``\.{'cmr10-600dpi.tfm'}''; in the short format, strings are
byte sequences terminated by a zero byte.
Because file names are system dependent, we no not allow arbitrary characters in strings
but only printable ASCII codes which we can reasonably expect to be available on most operating systems.
If your file names in a long format \HINT/ file are supposed to be portable,
you should probably be even more restrictive. For example you should avoid characters like
``\.{\\}'' or ``\.{/}'' which are used in different ways for directories.

The internal representation of a string is a simple zero terminated \CEE/ string.
When scanning a string, we copy it to the \index{str buffer+\\{str\_buffer}}\\{str\_buffer} keeping track
of its length in \index{str length+\\{str\_length}}\\{str\_length}. When done,
we make a copy for permanent storage and return the pointer to the parser.
To operate on the \index{str buffer+\\{str\_buffer}}\\{str\_buffer}, we define a few macros.
The constant \index{MAX STR+\.{MAX\_STR}}\.{MAX\_STR} determines the maximum size of a string (including the zero byte) to be $2^{10}$ byte.
This restriction is part of the \HINT/ file format specification.

\Y\B\4\X20:scanning macros\X${}\mathrel+\E{}$\6
\8\#\&{define} \index{MAX STR+\.{MAX\_STR}}\.{MAX\_STR}\5${}(\T{1}\LL\T{10}{}$)\C{ $2^{10}$ Byte or 1kByte }\6
\&{static} \&{char} \index{str buffer+\\{str\_buffer}}\\{str\_buffer}[\index{MAX STR+\.{MAX\_STR}}\.{MAX\_STR}];\6
\&{static} \&{int} \index{str length+\\{str\_length}}\\{str\_length};\6
\8\#\&{define} \index{STR START+\.{STR\_START}}\.{STR\_START}\5${}(\index{str length+\\{str\_length}}\\{str\_length}\K\T{0}){}$\6
\8\#\&{define} \index{STR PUT+\.{STR\_PUT}}\.{STR\_PUT}(\|C)\5${}(\index{str buffer+\\{str\_buffer}}\\{str\_buffer}[\index{str length+\\{str\_length}}\\{str\_length}\PP]\K(\|C)){}$\6
\8\#\&{define} \index{STR ADD+\.{STR\_ADD}}\.{STR\_ADD}(\|C)\5${}\index{STR PUT+\.{STR\_PUT}}\.{STR\_PUT}(\|C);\.{RNG}(\.{"String\ length"},\39\index{str length+\\{str\_length}}\\{str\_length},\39\T{0},\39\index{MAX STR+\.{MAX\_STR}}\.{MAX\_STR}-\T{1}){}$\6
\8\#\&{define} \index{STR END+\.{STR\_END}}\.{STR\_END}\5${}\index{str buffer+\\{str\_buffer}}\\{str\_buffer}[\index{str length+\\{str\_length}}\\{str\_length}]\K\T{0}{}$\6
\8\#\&{define} \index{SCAN STR+\.{SCAN\_STR}}\.{SCAN\_STR}\5${}\index{yylval+\\{yylval}}\\{yylval}.\|s\K\index{str buffer+\\{str\_buffer}}\\{str\_buffer}{}$
\Y
\fi

\M{30}


To scan a string, we switch the scanner to \index{STR+\ts{STR}}\ts{STR} mode when we find a quote character,
then we scan bytes in the range \T{\^20} to \T{\^7E}, which is the range of printable ASCII
characters, until we find the closing single\index{single quote} quote.
Quote characters inside the string are written as two consecutive single quote characters.

\readcode
\Y\par
\par
\par
\Y\B\4\X21:scanning definitions\X${}\mathrel+\E{}$\6
\8\%\&{x} \index{STR+\ts{STR}}\ts{STR}
\Y
\fi

\M{31}

\Y\B\4\X2:symbols\X${}\mathrel+\E{}$\6
\8\%\&{token} $<$ \|s $>$ \index{STRING+\ts{STRING}}\ts{STRING}
\Y
\fi

\M{32}

\Y\B\4\X3:scanning rules\X${}\mathrel+\E{}$\6
${}\8\re{\vb{'}}{}$\ac\index{STR START+\.{STR\_START}}\.{STR\_START};\5
\.{BEGIN}(\index{STR+\ts{STR}}\ts{STR});\eac\7
\8\re{${}<{}$\ts{STR}${}>{}\{$}\6
${}\8\re{\vb{'}}{}$\ac\.{STR\_END};\5
\.{SCAN\_STR};\5
\.{BEGIN}(\ts{INITIAL});\5
\&{return} \ts{STRING};\eac\7
${}\8\re{\vb{''}}{}$\ac\.{STR\_ADD}(\.{'\\''});\eac\7
${}\8\re{\vb{[\\x20-\\x7E]}}{}$\ac\.{STR\_ADD}(\\{yytext}[\T{0}]);\eac\7
${}\8\re{\vb{.|\\n}}{}$\ac${}\.{RNG}(\.{"String\ character"},\39\\{yytext}[\T{0}],\39\T{\^20},\39\T{\^7E});\eac{}$\7
${}\}{}$
\Y
\fi

\M{33}
The function \index{hwrite string+\\{hwrite\_string}}\\{hwrite\_string} reverses this process; it must take care of the quote symbols.
\writecode
\Y\B\4\X19:write functions\X${}\mathrel+\E{}$\6
\&{void} \index{hwrite string+\\{hwrite\_string}}\\{hwrite\_string}(\&{char} ${}{*}\index{str+\\{str}}\\{str}){}$\1\1\2\2\1\6
\4${}\{{}$\5
\index{hwritec+\\{hwritec}}\\{hwritec}(\.{'\ '});\6
\&{if} ${}(\index{str+\\{str}}\\{str}\E\NULL){}$\1\5
\index{hwritef+\\{hwritef}}\\{hwritef}(\.{"''"});\2\6
\&{else}\6
\1${}\{{}$\5
\index{hwritec+\\{hwritec}}\\{hwritec}(\.{'\\''});\6
\&{while} ${}({*}\index{str+\\{str}}\\{str}\I\T{0}{}$)\6
\1${}\{{}$\5
\&{if} ${}({*}\index{str+\\{str}}\\{str}\E\.{'\\''}){}$\1\5
\index{hwritec+\\{hwritec}}\\{hwritec}(\.{'\\''});\2\6
${}\index{hwritec+\\{hwritec}}\\{hwritec}({*}\index{str+\\{str}}\\{str}\PP);{}$\6
\4${}\}{}$\2\6
\index{hwritec+\\{hwritec}}\\{hwritec}(\.{'\\''});\6
\4${}\}{}$\2\6
\4${}\}{}$\2
\Y
\fi

\M{34}
In the short format, a string is just a byte sequence terminated by a zero byte.
This makes the function \index{hput string+\\{hput\_string}}\\{hput\_string}, to write a string, and the macro \index{HGET STRING+\.{HGET\_STRING}}\.{HGET\_STRING},
to read a string in short format, very simple. Note that after writing an unbounded
string to the output buffer, the macro \index{HPUTNODE+\.{HPUTNODE}}\.{HPUTNODE} will make sure that there is enough
space left to write the remainder of the node.

\putcode
\Y\B\4\X12:put functions\X${}\mathrel+\E{}$\6
\&{void} \index{hput string+\\{hput\_string}}\\{hput\_string}(\&{char} ${}{*}\index{str+\\{str}}\\{str}){}$\1\1\2\2\1\6
\4${}\{{}$\5
\&{char} ${}{*}\|s\K\index{str+\\{str}}\\{str};{}$\7
\&{if} ${}(\|s\I\NULL){}$\5
\1${}\{{}$\5
\&{do}\5
\1${}\{{}$\5
\index{HPUTX+\.{HPUTX}}\.{HPUTX}(\T{1});\6
${}\index{HPUT8+\.{HPUT8}}\.{HPUT8}({*}\|s);{}$\6
\4${}\}{}$\2\5
\&{while} ${}({*}\|s\PP\I\T{0});{}$\6
\index{HPUTNODE+\.{HPUTNODE}}\.{HPUTNODE};\6
\4${}\}{}$\2\6
\&{else}\1\5
\index{HPUT8+\.{HPUT8}}\.{HPUT8}(\T{0});\2\6
\4${}\}{}$\2
\Y
\fi

\M{35}

\getcode
\Y\B\4\X35:get file macros\X${}\E{}$\6
\8\#\&{define} \index{HGET STRING+\.{HGET\_STRING}}\.{HGET\_STRING}(\|S)  $\|S\K{}$(\&{char} ${}{*}){}$ \index{hpos+\\{hpos}}\\{hpos};\6
\&{while} ${}(\index{hpos+\\{hpos}}\\{hpos}<\index{hend+\\{hend}}\\{hend}\W{*}\index{hpos+\\{hpos}}\\{hpos}\I\T{0}){}$\5
\1${}\{{}$\5
${}\.{RNG}(\.{"String\ character"},\39{*}\index{hpos+\\{hpos}}\\{hpos},\39\T{\^20},\39\T{\^7E});{}$\6
${}\index{hpos+\\{hpos}}\\{hpos}\PP;{}$\6
\4${}\}{}$\2\6
${}\index{hpos+\\{hpos}}\\{hpos}\PP;$
\As269\ET293.
\Us432, 433, 439\ETs441.\Y
\fi

\M{36}

\subsection{Character Codes}
\label{chars}
We have already seen in the introduction that character\index{character code} codes can be written as decimal numbers
and section~\secref{integers} adds the possibility to use hexadecimal numbers as well.

It is, however, in most cases more readable if we represent character codes directly
using the characters themselves. Writing ``\.{a}'' is just so much better than writing ``\.{97}''.
To distinguish the character ``\.{9}'' from the number ``\.{9}'', we use the common technique
of enclosing characters within single\index{single quote} quotes. So ``\.{'9'}'' is the character code and
``\.{9}'' is the number.
Therefore we will define \index{CHARCODE+\ts{CHARCODE}}\ts{CHARCODE} tokens and complement the parsing rules of section~\secref{parse_glyph}
with the following rule:
\readcode
\Y\B\4\X5:parsing rules\X${}\mathrel+\E{}$\6
\index{glyph+\nts{glyph}}\nts{glyph}: \1\1\5
\index{CHARCODE+\ts{CHARCODE}}\ts{CHARCODE}\5
\index{REFERENCE+\ts{REFERENCE}}\ts{REFERENCE}\3{-1}\5
${}\{{}$\1\5
${}\.{\$\$}.\|c\K\.{\$1};{}$\5
${}\index{REF+\.{REF}}\.{REF}(\index{font kind+\\{font\_kind}}\\{font\_kind},\39\.{\$2});{}$\5
${}\.{\$\$}.\|f\K\.{\$2};{}$\5
${}\}{}$\2;\2\2
\Y
\fi

\M{37}



If the character codes are small, we can represent them using
ASCII character codes. We do not offer a special notation for very small
character codes that map to the non-printable ASCII control codes; for them, the decimal
or hexadecimal notation will suffice.
For larger character codes, we use the multibyte encoding scheme known from UTF8\index{UTF8} as
follows. Given a character code~\|c:

\itemize
\item
Values in the range \T{\^00} to \T{\^7f} are encoded as a single byte with a leading bit of 0.

\Y\B\4\X21:scanning definitions\X${}\mathrel+\E{}$\6
${}\8\re{\vb{UTF8\_1}}{}$\ac\vb{[\\x00-\\x7F]}\eac
\Y
\fi

\M{38}
\Y\B\4\X20:scanning macros\X${}\mathrel+\E{}$\6
\8\#\&{define} \index{SCAN UTF8 1+\.{SCAN\_UTF8\_1}}\.{SCAN\_UTF8\_1}(\|S)\5${}\index{yylval+\\{yylval}}\\{yylval}.\|u\K((\|S)[\T{0}]\AND\T{\^7F}){}$
\Y
\fi

\M{39}


\item
Values in the range \T{\^80} to \T{\^7ff} are encoded in two byte with the first byte
having three high bits \T{110}, indicating a two byte sequence, and the lower five bits equal
to the five high bits of \|c. It is followed by a continuation byte having two high bits \T{10}
and the lower six bits
equal to the lower six bits of \|c.

\Y\B\4\X21:scanning definitions\X${}\mathrel+\E{}$\6
${}\8\re{\vb{UTF8\_2}}{}$\ac\vb{[\\xC0-\\xDF][\\x80-\\xBF]}\eac
\Y
\fi

\M{40}

\Y\B\4\X20:scanning macros\X${}\mathrel+\E{}$\6
\8\#\&{define} \index{SCAN UTF8 2+\.{SCAN\_UTF8\_2}}\.{SCAN\_UTF8\_2}(\|S)\5${}\index{yylval+\\{yylval}}\\{yylval}.\|u\K(((\|S)[\T{0}]\AND\T{\^1F})\LL\T{6})+((\|S)[\T{1}]\AND\T{\^3F}){}$
\Y
\fi

\M{41}

\item
Values in the range \T{\^800} to \T{\^FFFF} are encoded in three byte with the first byte
having the high bits \T{1110} indicating a three byte sequence followed by two continuation bytes.

\Y\B\4\X21:scanning definitions\X${}\mathrel+\E{}$\6
${}\8\re{\vb{UTF8\_3}}{}$\ac\vb{[\\xE0-\\xEF][\\x80-\\xBF][\\x80-\\xBF]}\eac
\Y
\fi

\M{42}

\Y\B\4\X20:scanning macros\X${}\mathrel+\E{}$\6
\8\#\&{define} \index{SCAN UTF8 3+\.{SCAN\_UTF8\_3}}\.{SCAN\_UTF8\_3}(\|S)\5${}\index{yylval+\\{yylval}}\\{yylval}.\|u\K(((\|S)[\T{0}]\AND\T{\^0F})\LL\T{12})+(((\|S)[\T{1}]\AND\T{\^3F})\LL\T{6})+((\|S)[\T{2}]\AND\T{\^3F}){}$
\Y
\fi

\M{43}

\item
Values in the range \T{\^1000} to \T{\^1FFFFF} are encoded in four byte with the first byte
having the high bits \T{11110} indicating a four byte sequence followed by three continuation bytes.

\Y\B\4\X21:scanning definitions\X${}\mathrel+\E{}$\6
${}\8\re{\vb{UTF8\_4}}{}$\ac\vb{[\\xF0-\\xF7][\\x80-\\xBF][\\x80-\\xBF][\\x80-\\xBF]}\eac
\Y
\fi

\M{44}

\Y\B\4\X20:scanning macros\X${}\mathrel+\E{}$\6
\8\#\&{define} \index{SCAN UTF8 4+\.{SCAN\_UTF8\_4}}\.{SCAN\_UTF8\_4}(\|S)\5${}\index{yylval+\\{yylval}}\\{yylval}.\|u\K(((\|S)[\T{0}]\AND\T{\^03})\LL\T{18})+(((\|S)[\T{1}]\AND\T{\^3F})\LL\T{12})+\3{-1}(((\|S)[\T{2}]\AND\T{\^3F})\LL\T{6})+((\|S)[\T{3}]\AND\T{\^3F}){}$
\Y
\fi

\M{45}

\enditemize

In the long format file, we enclose a character code in single\index{single quote} quotes, just as we do for strings.
This is convenient but it has the downside that we must exercise special care when giving the
scanning rules in order
not to confuse character codes with strings. Further we must convert character codes back into strings
in the rare case where the parser expects a string and gets a character code because the string
was only a single character long.

Let's start with the first problem:
The scanner might confuse a string\index{string} and a character code if the first or second
character of the string is a quote character which is written as two consecutive quotes.
For example \.{'a''b'} is a string with three characters, ``\.{a}'',
``\.{'}'', and ``\.{b}''. Two character codes would need a space to separate
them like this: \.{'a' 'b'}.


\Y\par
\Y\B\4\X2:symbols\X${}\mathrel+\E{}$\6
\8\%\&{token} $<$ \|u $>$ \index{CHARCODE+\ts{CHARCODE}}\ts{CHARCODE}
\Y
\fi

\M{46}

\Y\B\4\X3:scanning rules\X${}\mathrel+\E{}$\6
${}\8\re{\vb{'''}}{}$\ac\index{STR START+\.{STR\_START}}\.{STR\_START};\5
\index{STR PUT+\.{STR\_PUT}}\.{STR\_PUT}(\.{'\\''});\5
\.{BEGIN}(\index{STR+\ts{STR}}\ts{STR});\eac\7
${}\8\re{\vb{''''}}{}$\ac${}\index{SCAN UTF8 1+\.{SCAN\_UTF8\_1}}\.{SCAN\_UTF8\_1}(\index{yytext+\\{yytext}}\\{yytext}+\T{1});{}$\5
\&{return} \index{CHARCODE+\ts{CHARCODE}}\ts{CHARCODE};\eac\7
${}\8\re{\vb{'[\\x20-\\x7E]''}}{}$\ac\index{STR START+\.{STR\_START}}\.{STR\_START};\5
\index{STR PUT+\.{STR\_PUT}}\.{STR\_PUT}(\index{yytext+\\{yytext}}\\{yytext}[\T{1}]);\5
\index{STR PUT+\.{STR\_PUT}}\.{STR\_PUT}(\.{'\\''});\5
\.{BEGIN}(\index{STR+\ts{STR}}\ts{STR});\eac\7
${}\8\re{\vb{'''''}}{}$\ac\index{STR START+\.{STR\_START}}\.{STR\_START};\5
\index{STR PUT+\.{STR\_PUT}}\.{STR\_PUT}(\.{'\\''});\5
\index{STR PUT+\.{STR\_PUT}}\.{STR\_PUT}(\.{'\\''});\5
\.{BEGIN}(\index{STR+\ts{STR}}\ts{STR});\eac\7
${}\8\re{\vb{'\{UTF8\_1\}'}}{}$\ac${}\index{SCAN UTF8 1+\.{SCAN\_UTF8\_1}}\.{SCAN\_UTF8\_1}(\index{yytext+\\{yytext}}\\{yytext}+\T{1});{}$\5
\&{return} \index{CHARCODE+\ts{CHARCODE}}\ts{CHARCODE};\eac\7
${}\8\re{\vb{'\{UTF8\_2\}'}}{}$\ac${}\index{SCAN UTF8 2+\.{SCAN\_UTF8\_2}}\.{SCAN\_UTF8\_2}(\index{yytext+\\{yytext}}\\{yytext}+\T{1});{}$\5
\&{return} \index{CHARCODE+\ts{CHARCODE}}\ts{CHARCODE};\eac\7
${}\8\re{\vb{'\{UTF8\_3\}'}}{}$\ac${}\index{SCAN UTF8 3+\.{SCAN\_UTF8\_3}}\.{SCAN\_UTF8\_3}(\index{yytext+\\{yytext}}\\{yytext}+\T{1});{}$\5
\&{return} \index{CHARCODE+\ts{CHARCODE}}\ts{CHARCODE};\eac\7
${}\8\re{\vb{'\{UTF8\_4\}'}}{}$\ac${}\index{SCAN UTF8 4+\.{SCAN\_UTF8\_4}}\.{SCAN\_UTF8\_4}(\index{yytext+\\{yytext}}\\{yytext}+\T{1});{}$\5
\&{return} \index{CHARCODE+\ts{CHARCODE}}\ts{CHARCODE};\eac
\Y
\fi

\M{47}

If needed, the parser can convert character codes back to single character strings.

\Y\par
\Y\B\4\X2:symbols\X${}\mathrel+\E{}$\6
\8\%\index{type+\&{type}}\&{type} $<$ \|s $>$ \index{string+\nts{string}}\nts{string}
\Y
\fi

\M{48}

\Y\B\4\X5:parsing rules\X${}\mathrel+\E{}$\6
\index{string+\nts{string}}\nts{string}: \1\1\5
\index{STRING+\ts{STRING}}\ts{STRING}\5
\hbox to 0.5em{\hss${}\OR{}$}\5
\index{CHARCODE+\ts{CHARCODE}}\ts{CHARCODE}\5
${}\{{}$\1\5
\&{static} \&{char} \|s[\T{2}];\7
${}\.{RNG}(\.{"String\ element"},\39\.{\$1},\39\T{\^20},\39\T{\^7E});{}$\5
${}\|s[\T{0}]\K\.{\$1};{}$\5
${}\|s[\T{1}]\K\T{0};{}$\5
${}\.{\$\$}\K\|s;{}$\5
${}\}{}$\2;\2\2
\Y
\fi

\M{49}


The function \index{hwrite charcode+\\{hwrite\_charcode}}\\{hwrite\_charcode} will write a character code. While ASCII codes are handled directly,
larger character codes are passed to the function \index{hwrite utf8+\\{hwrite\_utf8}}\\{hwrite\_utf8}.
It returns the number of characters written.

\writecode
\Y\B\4\X19:write functions\X${}\mathrel+\E{}$\6
\&{int} \index{hwrite utf8+\\{hwrite\_utf8}}\\{hwrite\_utf8}(\&{uint32\_t} \|c)\1\1\2\2\1\6
\4${}\{{}$\5
\&{if} ${}(\|c<\T{\^80}){}$\5
\1${}\{{}$\5
\index{hwritec+\\{hwritec}}\\{hwritec}(\|c);\6
\&{return} \T{1};\6
\4${}\}{}$\2\6
\&{else} \&{if} ${}(\|c<\T{\^800}){}$\5
\1${}\{{}$\5
${}\index{hwritec+\\{hwritec}}\\{hwritec}(\T{\^C0}\OR(\|c\GG\T{6})){}$;\5
${}\index{hwritec+\\{hwritec}}\\{hwritec}(\T{\^80}\OR(\|c\AND\T{\^3F})){}$;\5
\&{return} \T{2};\6
\4${}\}{}$\2\6
\&{else} \&{if} ${}(\|c<\T{\^10000}{}$)\6
\1${}\{{}$\5
${}\index{hwritec+\\{hwritec}}\\{hwritec}(\T{\^E0}\OR(\|c\GG\T{12}));{}$\6
${}\index{hwritec+\\{hwritec}}\\{hwritec}(\T{\^80}\OR((\|c\GG\T{6})\AND\T{\^3F})){}$;\5
${}\index{hwritec+\\{hwritec}}\\{hwritec}(\T{\^80}\OR(\|c\AND\T{\^3F}));{}$\6
\&{return} \T{3};\6
\4${}\}{}$\2\6
\&{else} \&{if} ${}(\|c<\T{\^200000}{}$)\6
\1${}\{{}$\5
${}\index{hwritec+\\{hwritec}}\\{hwritec}(\T{\^F0}\OR(\|c\GG\T{18})){}$;\5
${}\index{hwritec+\\{hwritec}}\\{hwritec}(\T{\^80}\OR((\|c\GG\T{12})\AND\T{\^3F}));{}$\6
${}\index{hwritec+\\{hwritec}}\\{hwritec}(\T{\^80}\OR((\|c\GG\T{6})\AND\T{\^3F})){}$;\5
${}\index{hwritec+\\{hwritec}}\\{hwritec}(\T{\^80}\OR(\|c\AND\T{\^3F}));{}$\6
\&{return} \T{4};\6
\4${}\}{}$\2\6
\&{else}\1\5
${}\.{RNG}(\.{"character\ code"},\39\|c,\39\T{0},\39\T{\^1FFFFF});{}$\2\6
\&{return} \T{0};\6
\4${}\}{}$\2\7
\&{void} \index{hwrite charcode+\\{hwrite\_charcode}}\\{hwrite\_charcode}(\&{uint32\_t} \|c)\1\1\2\2\1\6
\4${}\{{}$\5
\&{if} ${}(\|c<\T{\^20}){}$\5
\1${}\{{}$\6
\&{if} (\index{option hex+\\{option\_hex}}\\{option\_hex})\1\5
${}\index{hwritef+\\{hwritef}}\\{hwritef}(\.{"\ 0x\%02X"},\39\|c){}$;\C{ non printable ASCII }\2\6
\&{else}\1\5
${}\index{hwritef+\\{hwritef}}\\{hwritef}(\.{"\ \%u"},\39\|c);{}$\2\6
\4${}\}{}$\2\6
\&{else} \&{if} ${}(\|c\E\.{'\\''}){}$\1\5
\index{hwritef+\\{hwritef}}\\{hwritef}(\.{"\ ''''"});\2\6
\&{else} \&{if} ${}(\|c\Z\T{\^7E}){}$\1\5
${}\index{hwritef+\\{hwritef}}\\{hwritef}(\.{"\ \\'\%c\\'"},\39\|c){}$;\C{ printable ASCII }\2\6
\&{else} \&{if} (\index{option utf8+\\{option\_utf8}}\\{option\_utf8})\5
\1${}\{{}$\5
\index{hwritef+\\{hwritef}}\\{hwritef}(\.{"\ \\'"});\5
\index{hwrite utf8+\\{hwrite\_utf8}}\\{hwrite\_utf8}(\|c);\5
\index{hwritec+\\{hwritec}}\\{hwritec}(\.{'\\''});\5
${}\}{}$\2\6
\&{else} \&{if} (\index{option hex+\\{option\_hex}}\\{option\_hex})\1\5
${}\index{hwritef+\\{hwritef}}\\{hwritef}(\.{"\ 0x\%04X"},\39\|c);{}$\2\6
\&{else}\1\5
${}\index{hwritef+\\{hwritef}}\\{hwritef}(\.{"\ \%u"},\39\|c);{}$\2\6
\4${}\}{}$\2
\Y
\fi

\M{50}

\getcode
\Y\B\4\X16:get functions\X${}\mathrel+\E{}$\6
\8\#\&{define} \index{HGET UTF8C+\.{HGET\_UTF8C}}\.{HGET\_UTF8C}(\|X) (\|X)${}\K\index{HGET8+\.{HGET8}}\.{HGET8}{}$;\5
\&{if} ${}((\|X\AND\T{\^C0})\I\T{\^80}){}$\1\5
${}\.{QUIT}(\.{"UTF8\ continuation\ b}\)\.{yte\ expected\ at\ "}\.{SIZE\_F}\.{"\ got\ 0x\%02X\\n"},\39\index{hpos+\\{hpos}}\\{hpos}-\index{hstart+\\{hstart}}\\{hstart}-\T{1},\39\|X){}$\2\6
\&{uint32\_t} \index{hget utf8+\\{hget\_utf8}}\\{hget\_utf8}(\&{void})\1\1\2\2\1\6
\4${}\{{}$\5
\&{uint8\_t} \|a;\7
${}\|a\K\index{HGET8+\.{HGET8}}\.{HGET8};{}$\6
\&{if} ${}(\|a<\T{\^80}){}$\1\5
\&{return} \|a;\2\6
\&{else}\5
\1${}\{{}$\6
\&{if} ${}((\|a\AND\T{\^E0})\E\T{\^C0}{}$)\6
\1${}\{{}$\5
\&{uint8\_t} \|b;\5
\index{HGET UTF8C+\.{HGET\_UTF8C}}\.{HGET\_UTF8C}(\|b);\6
\&{return} ${}((\|a\AND\CM\T{\^E0})\LL\T{6})+(\|b\AND\CM\T{\^C0});{}$\6
\4${}\}{}$\2\6
\&{else} \&{if} ${}((\|a\AND\T{\^F0})\E\T{\^E0}{}$)\6
\1${}\{{}$\5
\&{uint8\_t} \|b${},{}$ \|c;\5
\index{HGET UTF8C+\.{HGET\_UTF8C}}\.{HGET\_UTF8C}(\|b);\5
\index{HGET UTF8C+\.{HGET\_UTF8C}}\.{HGET\_UTF8C}(\|c);\6
\&{return} ${}((\|a\AND\CM\T{\^F0})\LL\T{12})+((\|b\AND\CM\T{\^C0})\LL\T{6})+(\|c\AND\CM\T{\^C0});{}$\6
\4${}\}{}$\2\6
\&{else} \&{if} ${}((\|a\AND\T{\^F8})\E\T{\^F0}{}$)\6
\1${}\{{}$\5
\&{uint8\_t} \|b${},{}$ \|c${},{}$ \|d;\5
\index{HGET UTF8C+\.{HGET\_UTF8C}}\.{HGET\_UTF8C}(\|b);\5
\index{HGET UTF8C+\.{HGET\_UTF8C}}\.{HGET\_UTF8C}(\|c);\5
\index{HGET UTF8C+\.{HGET\_UTF8C}}\.{HGET\_UTF8C}(\|d);\6
\&{return} ${}((\|a\AND\CM\T{\^F8})\LL\T{18})\3{-1}+((\|b\AND\CM\T{\^C0})\LL\T{12})+((\|c\AND\CM\T{\^C0})\LL\T{6})+(\|d\AND\CM\T{\^C0});{}$\6
\4${}\}{}$\2\6
\&{else}\1\5
\.{QUIT}(\.{"UTF8\ byte\ sequence\ }\)\.{expected"});\2\6
\4${}\}{}$\2\6
\4${}\}{}$\2
\Y
\fi

\M{51}
\putcode
\Y\B\4\X12:put functions\X${}\mathrel+\E{}$\6
\&{void} \index{hput utf8+\\{hput\_utf8}}\\{hput\_utf8}(\&{uint32\_t} \|c)\1\1\2\2\1\6
\4${}\{{}$\5
\index{HPUTX+\.{HPUTX}}\.{HPUTX}(\T{4});\6
\&{if} ${}(\|c<\T{\^80}){}$\1\5
\index{HPUT8+\.{HPUT8}}\.{HPUT8}(\|c);\2\6
\&{else} \&{if} ${}(\|c<\T{\^800}){}$\5
\1${}\{{}$\5
${}\index{HPUT8+\.{HPUT8}}\.{HPUT8}(\T{\^C0}\OR(\|c\GG\T{6})){}$;\5
${}\index{HPUT8+\.{HPUT8}}\.{HPUT8}(\T{\^80}\OR(\|c\AND\T{\^3F})){}$;\5
${}\}{}$\2\6
\&{else} \&{if} ${}(\|c<\T{\^10000}{}$)\6
\1${}\{{}$\5
${}\index{HPUT8+\.{HPUT8}}\.{HPUT8}(\T{\^E0}\OR(\|c\GG\T{12}));{}$\6
${}\index{HPUT8+\.{HPUT8}}\.{HPUT8}(\T{\^80}\OR((\|c\GG\T{6})\AND\T{\^3F})){}$;\5
${}\index{HPUT8+\.{HPUT8}}\.{HPUT8}(\T{\^80}\OR(\|c\AND\T{\^3F}));{}$\6
\4${}\}{}$\2\6
\&{else} \&{if} ${}(\|c<\T{\^200000}{}$)\6
\1${}\{{}$\5
${}\index{HPUT8+\.{HPUT8}}\.{HPUT8}(\T{\^F0}\OR(\|c\GG\T{18})){}$;\5
${}\index{HPUT8+\.{HPUT8}}\.{HPUT8}(\T{\^80}\OR((\|c\GG\T{12})\AND\T{\^3F}));{}$\6
${}\index{HPUT8+\.{HPUT8}}\.{HPUT8}(\T{\^80}\OR((\|c\GG\T{6})\AND\T{\^3F})){}$;\5
${}\index{HPUT8+\.{HPUT8}}\.{HPUT8}(\T{\^80}\OR(\|c\AND\T{\^3F}));{}$\6
\4${}\}{}$\2\6
\&{else}\1\5
${}\.{RNG}(\.{"character\ code"},\39\|c,\39\T{0},\39\T{\^1FFFFF});{}$\2\6
\4${}\}{}$\2
\Y
\fi

\M{52}

\subsection{Floating Point Numbers}
You know a floating point numbers\index{floating point number} when you see it because it features a radix\index{radix point} point.
The optional exponent\index{exponent} allows you to ``float'' the point.

\readcode
\Y\par
\par
\Y\B\4\X2:symbols\X${}\mathrel+\E{}$\6
\8\%\&{token} $<$ \|f $>$ \index{FPNUM+\ts{FPNUM}}\ts{FPNUM} \6
\8\%\index{type+\&{type}}\&{type} $<$ \|f $>$ \index{number+\nts{number}}\nts{number}
\Y
\fi

\M{53}
\Y\B\4\X3:scanning rules\X${}\mathrel+\E{}$\6
${}\8\re{\vb{[+-]?[0-9]+\\.[0-9]+(e[+-]?[0-9])?}}{}$\ac\index{SCAN DECFLOAT+\.{SCAN\_DECFLOAT}}\.{SCAN\_DECFLOAT};\5
\&{return} \index{FPNUM+\ts{FPNUM}}\ts{FPNUM};\eac
\Y
\fi

\M{54}

The layout of floating point variables of type \&{double}
or \&{float} typically follows the IEEE754\index{IEEE754} standard\cite{IEEE754-1985}\cite{IEEE754-2008}.
We use the following definitions:

\index{float32 t+\&{float32\_t}}
\index{float64 t+\&{float64\_t}}

\Y\B\4\X6:hint basic types\X${}\mathrel+\E{}$\6
\8\#\&{define} \index{FLT M BITS+\.{FLT\_M\_BITS}}\.{FLT\_M\_BITS}\5\T{23}\6
\8\#\&{define} \index{FLT E BITS+\.{FLT\_E\_BITS}}\.{FLT\_E\_BITS}\5\T{8}\6
\8\#\&{define} \index{FLT EXCESS+\.{FLT\_EXCESS}}\.{FLT\_EXCESS}\5\T{127}\6
\8\#\&{define} \.{DBL\_M\_BITS}\5\T{52}\6
\8\#\&{define} \index{DBL E BITS+\.{DBL\_E\_BITS}}\.{DBL\_E\_BITS}\5\T{11}\6
\8\#\&{define} \index{DBL EXCESS+\.{DBL\_EXCESS}}\.{DBL\_EXCESS}\5\T{1023}
\Y
\fi

\M{55}

\Y\par
\par
\Y\B\4\X20:scanning macros\X${}\mathrel+\E{}$\6
\8\#\&{define} \index{SCAN DECFLOAT+\.{SCAN\_DECFLOAT}}\.{SCAN\_DECFLOAT}\5${}\index{yylval+\\{yylval}}\\{yylval}.\|f\K\index{atof+\\{atof}}\\{atof}(\index{yytext+\\{yytext}}\\{yytext}){}$
\Y
\fi

\M{56}

When the parser expects a floating point number and gets an integer number,
it converts it. So whenever in the long format a floating point number is expected,
an integer number will do as well.

\Y\B\4\X5:parsing rules\X${}\mathrel+\E{}$\6
\index{number+\nts{number}}\nts{number}: \1\1\5
\index{UNSIGNED+\ts{UNSIGNED}}\ts{UNSIGNED}\5
${}\{{}$\1\5
${}\.{\$\$}\K{}$(\&{float64\_t}) \.{\$1};\5
${}\}{}$\2\6
\4\hbox to 0.5em{\hss${}\OR{}$}\5
\index{SIGNED+\ts{SIGNED}}\ts{SIGNED}\5
${}\{{}$\1\5
${}\.{\$\$}\K{}$(\&{float64\_t}) \.{\$1};\5
${}\}{}$\2\6
\4\hbox to 0.5em{\hss${}\OR{}$}\5
\index{FPNUM+\ts{FPNUM}}\ts{FPNUM};\2\2
\Y
\fi

\M{57}

Unfortunately the decimal representation is not optimal for floating point numbers
since even simple numbers in decimal notation like $0.1$ do not have an exact
representation as a binary floating point number.
So if we want a notation that allows an exact representation
of binary floating point numbers, we must use a hexadecimal\index{hexadecimal} representation.
Hexadecimal floating point numbers start with an optional sign, then as usual the two characters ``{\tt 0x}'',
then follows a sequence of hex digits, a radix point, more hex digits, and an optional exponent.
The optional exponent starts with the character ``{\tt x}'', followed by an optional sign, and some more
hex digits. The hexadecimal exponent is given as a base 16 number and it is interpreted as an exponent
with the base 16. As an example an exponent of ``{\tt x10}'', would multiply the mantissa by $16^{16}$.
In other words it would shift any mantissa \T{\^10} hexadecimal digits to the left. Here are the exact rules:

\Y\B\4\X3:scanning rules\X${}\mathrel+\E{}$\6
${}\8\re{\vb{[+-]?0x\{HEX\}+\\.\{HEX\}+(x[+-]?\{HEX\}+)?}}{}$\ac\index{SCAN HEXFLOAT+\.{SCAN\_HEXFLOAT}}\.{SCAN\_HEXFLOAT};\5
\&{return} \index{FPNUM+\ts{FPNUM}}\ts{FPNUM};\eac
\Y
\fi

\M{58}

\Y\B\4\X20:scanning macros\X${}\mathrel+\E{}$\6
\8\#\&{define} \index{SCAN HEXFLOAT+\.{SCAN\_HEXFLOAT}}\.{SCAN\_HEXFLOAT}\5${}\index{yylval+\\{yylval}}\\{yylval}.\|f\K\index{xtof+\\{xtof}}\\{xtof}(\index{yytext+\\{yytext}}\\{yytext}){}$
\Y
\fi

\M{59}
There is no function in the \CEE/ library for hexadecimal floating point notation
so we have to write our own conversion routine.
The function \index{xtof+\\{xtof}}\\{xtof} converts a string matching the above regular expression to
its binary representation. Its outline is very simple:

\Y\B\4\X59:scanning functions\X${}\E{}$\6
\&{float64\_t} \index{xtof+\\{xtof}}\\{xtof}(\&{char} ${}{*}\|x){}$\1\1\2\2\1\6
\4${}\{{}$\5
\&{int} \\{sign}${},{}$ \index{digits+\\{digits}}\\{digits}${},{}$ \index{exp+\\{exp}}\\{exp};\6
\&{uint64\_t} \\{mantissa}${}\K\T{0};{}$\7
${}\.{DBG}(\index{DBGFLOAT+\.{DBGFLOAT}}\.{DBGFLOAT},\39\.{"converting\ \%s:\\n"},\39\|x);{}$\6
\X60:read the optional sign\X\6
${}\|x\K\|x+\T{2}{}$;\C{ skip ``\.{0x}'' }\6
\X61:read the mantissa\X\6
\X62:normalize the mantissa\X\6
\X63:read the optional exponent\X\6
\X64:return the binary representation\X\6
\4${}\}{}$\2
\U436.\Y
\fi

\M{60}

Now the pieces:

\Y\B\4\X60:read the optional sign\X${}\E{}$\6
\&{if} ${}({*}\|x\E\.{'-'}){}$\5
\1${}\{{}$\5
${}\\{sign}\K{-}\T{1}{}$;\5
${}\|x\PP{}$;\5
${}\}{}$\2\6
\&{else} \&{if} ${}({*}\|x\E\.{'+'}){}$\5
\1${}\{{}$\5
${}\\{sign}\K{+}\T{1}{}$;\5
${}\|x\PP{}$;\5
${}\}{}$\2\6
\&{else}\5
\1${}\\{sign}\K{+}\T{1};{}$\2\6
${}\.{DBG}(\index{DBGFLOAT+\.{DBGFLOAT}}\.{DBGFLOAT},\39\.{"\\tsign=\%d\\n"},\39\\{sign}){}$;
\U59.\Y
\fi

\M{61}

When we read the mantissa, we use the temporary variable \\{mantissa}, keep track
of the number of digits, and adjust the exponent while reading the fractional part.
\Y\B\4\X61:read the mantissa\X${}\E{}$\6
$\index{digits+\\{digits}}\\{digits}\K\T{0};{}$\6
\&{while} ${}({*}\|x\E\.{'0'}){}$\1\5
${}\|x\PP{}$;\C{ignore leading zeros}\2\6
\&{while} ${}({*}\|x\I\.{'.'}{}$)\6
\1${}\{{}$\5
${}\\{mantissa}\K\\{mantissa}\LL\T{4};{}$\6
\&{if} ${}({*}\|x<\.{'A'}){}$\1\5
${}\\{mantissa}\K\\{mantissa}+{*}\|x-\.{'0'};{}$\2\6
\&{else}\1\5
${}\\{mantissa}\K\\{mantissa}+{*}\|x-\.{'A'}+\T{10};{}$\2\6
${}\|x\PP;{}$\6
${}\index{digits+\\{digits}}\\{digits}\PP;{}$\6
\4${}\}{}$\2\6
${}\|x\PP{}$;\C{ skip ``\.{.}'' }\6
${}\index{exp+\\{exp}}\\{exp}\K\T{0};{}$\6
\&{while} ${}({*}\|x\I\T{0}\W{*}\|x\I\.{'x'}{}$)\6
\1${}\{{}$\5
${}\\{mantissa}\K\\{mantissa}\LL\T{4};{}$\6
${}\index{exp+\\{exp}}\\{exp}\K\index{exp+\\{exp}}\\{exp}-\T{4};{}$\6
\&{if} ${}({*}\|x<\.{'A'}){}$\1\5
${}\\{mantissa}\K\\{mantissa}+{*}\|x-\.{'0'};{}$\2\6
\&{else}\1\5
${}\\{mantissa}\K\\{mantissa}+{*}\|x-\.{'A'}+\T{10};{}$\2\6
${}\|x\PP;{}$\6
${}\index{digits+\\{digits}}\\{digits}\PP;{}$\6
\4${}\}{}$\2\6
${}\.{DBG}(\index{DBGFLOAT+\.{DBGFLOAT}}\.{DBGFLOAT},\39\.{"\\tdigits=\%d\ mantiss}\)\.{a=0x\%"}\index{PRIx64+\\{PRIx64}}\\{PRIx64}\.{",\ exp=\%d\\n"},\3{-1}\39\index{digits+\\{digits}}\\{digits},\39\\{mantissa},\39\index{exp+\\{exp}}\\{exp}){}$;
\U59.\Y
\fi

\M{62}

To normalize the mantissa, first we shift it to place exactly one nonzero hexadecimal
digit to the left of the radix point. Then we shift it right bit-wise until there is
just a single 1 bit to the left of the radix point.
To compensate for the shifting, we adjust the exponent accordingly.
Finally we remove the most significant bit because it is
not stored.

\Y\B\4\X62:normalize the mantissa\X${}\E{}$\6
\&{if} ${}(\\{mantissa}\E\T{0}){}$\1\5
\&{return} \T{0.0};\2\1\6
\4${}\{{}$\5
\&{int} \|s;\7
${}\|s\K\index{digits+\\{digits}}\\{digits}-\.{DBL\_M\_BITS}/\T{4};{}$\6
\&{if} ${}(\|s>\T{1}){}$\1\5
${}\\{mantissa}\K\\{mantissa}\GG(\T{4}*(\|s-\T{1}));{}$\2\6
\&{else} \&{if} ${}(\|s<\T{1}){}$\1\5
${}\\{mantissa}\K\\{mantissa}\LL(\T{4}*(\T{1}-\|s));{}$\2\6
${}\index{exp+\\{exp}}\\{exp}\K\index{exp+\\{exp}}\\{exp}+\T{4}*(\index{digits+\\{digits}}\\{digits}-\T{1});{}$\6
${}\.{DBG}(\index{DBGFLOAT+\.{DBGFLOAT}}\.{DBGFLOAT},\39\.{"\\tdigits=\%d\ mantiss}\)\.{a=0x\%"}\index{PRIx64+\\{PRIx64}}\\{PRIx64}\.{",\ exp=\%d\\n"},\3{-1}\39\index{digits+\\{digits}}\\{digits},\39\\{mantissa},\39\index{exp+\\{exp}}\\{exp});{}$\6
\&{while} ${}((\\{mantissa}\GG\.{DBL\_M\_BITS})>\T{1}{}$)\6
\1${}\{{}$\5
${}\\{mantissa}\K\\{mantissa}\GG\T{1}{}$;\5
${}\index{exp+\\{exp}}\\{exp}\PP{}$;\5
${}\}{}$\2\6
${}\.{DBG}(\index{DBGFLOAT+\.{DBGFLOAT}}\.{DBGFLOAT},\39\.{"\\tdigits=\%d\ mantiss}\)\.{a=0x\%"}\index{PRIx64+\\{PRIx64}}\\{PRIx64}\.{",\ exp=\%d\\n"},\3{-1}\39\index{digits+\\{digits}}\\{digits},\39\\{mantissa},\39\index{exp+\\{exp}}\\{exp});{}$\6
${}\\{mantissa}\K\\{mantissa}\AND\CM{}$((\&{uint64\_t}) \T{1}${}\LL\.{DBL\_M\_BITS});{}$\6
${}\.{DBG}(\index{DBGFLOAT+\.{DBGFLOAT}}\.{DBGFLOAT},\39\.{"\\tdigits=\%d\ mantiss}\)\.{a=0x\%"}\index{PRIx64+\\{PRIx64}}\\{PRIx64}\.{",\ exp=\%d\\n"},\3{-1}\39\index{digits+\\{digits}}\\{digits},\39\\{mantissa},\39\index{exp+\\{exp}}\\{exp});{}$\6
\4${}\}{}$\2
\U59.\Y
\fi

\M{63}

In the printed representation,
the exponent is an exponent with base 16. For example, an exponent of 2 shifts
the hexadecimal mantissa two hexadecimal digits to the left, which corresponds to a
multiplication by ${16}^2$.

\Y\B\4\X63:read the optional exponent\X${}\E{}$\6
\&{if} ${}({*}\|x\E\.{'x'}{}$)\6
\1${}\{{}$\5
\&{int} \|s;\7
${}\|x\PP{}$;\C{ skip the ``\.{x}'' }\6
\&{if} ${}({*}\|x\E\.{'-'}){}$\5
\1${}\{{}$\5
${}\|s\K{-}\T{1}{}$;\5
${}\|x\PP{}$;\5
${}\}{}$\2\6
\&{else} \&{if} ${}({*}\|x\E\.{'+'}){}$\5
\1${}\{{}$\5
${}\|s\K{+}\T{1}{}$;\5
${}\|x\PP{}$;\5
${}\}{}$\2\6
\&{else}\1\5
${}\|s\K{+}\T{1};{}$\2\6
${}\.{DBG}(\index{DBGFLOAT+\.{DBGFLOAT}}\.{DBGFLOAT},\39\.{"\\texpsign=\%d\\n"},\39\|s);{}$\6
${}\.{DBG}(\index{DBGFLOAT+\.{DBGFLOAT}}\.{DBGFLOAT},\39\.{"\\texp=\%d\\n"},\39\index{exp+\\{exp}}\\{exp});{}$\6
\&{while} ${}({*}\|x\I\T{0}){}$\5
\1${}\{{}$\6
\&{if} ${}({*}\|x<\.{'A'}){}$\1\5
${}\index{exp+\\{exp}}\\{exp}\K\index{exp+\\{exp}}\\{exp}+\T{4}*\|s*({*}\|x-\.{'0'});{}$\2\6
\&{else}\1\5
${}\index{exp+\\{exp}}\\{exp}\K\index{exp+\\{exp}}\\{exp}+\T{4}*\|s*({*}\|x-\.{'A'}+\T{10});{}$\2\6
${}\|x\PP;{}$\6
${}\.{DBG}(\index{DBGFLOAT+\.{DBGFLOAT}}\.{DBGFLOAT},\39\.{"\\texp=\%d\\n"},\39\index{exp+\\{exp}}\\{exp});{}$\6
\4${}\}{}$\2\6
\4${}\}{}$\2\6
${}\.{RNG}(\.{"Floating\ point\ expo}\)\.{nent"},\3{-1}\39\index{exp+\\{exp}}\\{exp},\39{-}\index{DBL EXCESS+\.{DBL\_EXCESS}}\.{DBL\_EXCESS},\39\index{DBL EXCESS+\.{DBL\_EXCESS}}\.{DBL\_EXCESS}){}$;
\U59.\Y
\fi

\M{64}

To assemble the binary representation, we use a \&{union} of a \&{float64\_t} and \&{uint64\_t}.


\Y\B\4\X64:return the binary representation\X${}\E{}$\1\6
\4${}\{{}$\5
\&{union} ${}\{{}$\5
\1\&{float64\_t} \|d;\5
\&{uint64\_t} \index{bits+\\{bits}}\\{bits};\5
\2${}\}{}$ \|u;\7
\&{if} ${}(\\{sign}<\T{0}){}$\1\5
${}\\{sign}\K\T{1}{}$;\5
\2\&{else}\5
\1${}\\{sign}\K\T{0}{}$;\C{ the sign bit }\2\6
${}\index{exp+\\{exp}}\\{exp}\K\index{exp+\\{exp}}\\{exp}+\index{DBL EXCESS+\.{DBL\_EXCESS}}\.{DBL\_EXCESS}{}$;\C{ the exponent bits }\6
${}\|u.\index{bits+\\{bits}}\\{bits}\K{}$((\&{uint64\_t}) \\{sign}${}\LL\T{63}{}$)\6
${}\OR{}$((\&{uint64\_t}) \index{exp+\\{exp}}\\{exp}${}\LL\.{DBL\_M\_BITS})\OR\\{mantissa};{}$\6
${}\.{DBG}(\index{DBGFLOAT+\.{DBGFLOAT}}\.{DBGFLOAT},\39\.{"\ return\ \%f\\n"},\39\|u.\|d);{}$\6
\&{return} \|u${}.\|d;{}$\6
\4${}\}{}$\2
\U59.\Y
\fi

\M{65}

The inverse function is \index{hwrite float64+\\{hwrite\_float64}}\\{hwrite\_float64}. It strives to print floating point numbers
as readable as possible. So numbers without fractional part are written as integers.
Numbers that can be represented exactly in decimal notation are represented in
decimal notation. All other values are written as hexadecimal floating point numbers.
We avoid an exponent if it can be avoided by using up to \index{MAX HEX DIGITS+\.{MAX\_HEX\_DIGITS}}\.{MAX\_HEX\_DIGITS}

\writecode
\Y\B\4\X19:write functions\X${}\mathrel+\E{}$\6
\8\#\&{define} \index{MAX HEX DIGITS+\.{MAX\_HEX\_DIGITS}}\.{MAX\_HEX\_DIGITS}\5\T{12}\6
\&{void} \index{hwrite float64+\\{hwrite\_float64}}\\{hwrite\_float64}(\&{float64\_t} \|d)\1\1\2\2\1\6
\4${}\{{}$\5
\&{uint64\_t} \index{bits+\\{bits}}\\{bits}${},{}$ \\{mantissa};\6
\&{int} \index{exp+\\{exp}}\\{exp}${},{}$ \index{digits+\\{digits}}\\{digits};\7
\index{hwritec+\\{hwritec}}\\{hwritec}(\.{'\ '});\6
\&{if} ${}(\index{floor+\\{floor}}\\{floor}(\|d)\E\|d){}$\5
\1${}\{{}$\5
${}\index{hwritef+\\{hwritef}}\\{hwritef}(\.{"\%d"},\39{}$(\&{int}) \|d);\5
\&{return};\5
${}\}{}$\2\6
\&{if} ${}(\index{floor+\\{floor}}\\{floor}(\T{10000.0}*\|d)\E\T{10000.0}*\|d){}$\5
\1${}\{{}$\5
${}\index{hwritef+\\{hwritef}}\\{hwritef}(\.{"\%g"},\39\|d){}$;\5
\&{return};\5
${}\}{}$\2\6
${}\.{DBG}(\index{DBGFLOAT+\.{DBGFLOAT}}\.{DBGFLOAT},\39\.{"Writing\ hexadecimal}\)\.{\ float\ \%f\\n"},\39\|d);{}$\6
\&{if} ${}(\|d<\T{0}){}$\5
\1${}\{{}$\5
\index{hwritec+\\{hwritec}}\\{hwritec}(\.{'-'});\5
${}\|d\K{-}\|d{}$;\5
${}\}{}$\2\6
\index{hwritef+\\{hwritef}}\\{hwritef}(\.{"0x"});\6
\X66:extract mantissa and exponent\X\6
\&{if} ${}(\index{exp+\\{exp}}\\{exp}>\index{MAX HEX DIGITS+\.{MAX\_HEX\_DIGITS}}\.{MAX\_HEX\_DIGITS}){}$\1\5
\X69:write large numbers\X\2\6
\&{else} \&{if} ${}(\index{exp+\\{exp}}\\{exp}\G\T{0}){}$\1\5
\X70:write medium numbers\X\2\6
\&{else}\1\5
\X71:write small numbers\X\2\6
\4${}\}{}$\2
\Y
\fi

\M{66}

The extraction just reverses the creation of the binary representation.

\Y\B\4\X66:extract mantissa and exponent\X${}\E{}$\1\6
\4${}\{{}$\5
\&{union} ${}\{{}$\5
\1\&{float64\_t} \|d;\5
\&{uint64\_t} \index{bits+\\{bits}}\\{bits};\5
\2${}\}{}$ \|u;\7
${}\|u.\|d\K\|d{}$;\5
${}\index{bits+\\{bits}}\\{bits}\K\|u.\index{bits+\\{bits}}\\{bits};{}$\6
\4${}\}{}$\2\6
${}\\{mantissa}\K\index{bits+\\{bits}}\\{bits}\AND{}$(((\&{uint64\_t}) \T{1}${}\LL\.{DBL\_M\_BITS})-\T{1});{}$\6
${}\\{mantissa}\K\\{mantissa}+{}$((\&{uint64\_t}) \T{1}${}\LL\.{DBL\_M\_BITS});{}$\6
${}\index{exp+\\{exp}}\\{exp}\K((\index{bits+\\{bits}}\\{bits}\GG\.{DBL\_M\_BITS})\AND((\T{1}\LL\index{DBL E BITS+\.{DBL\_E\_BITS}}\.{DBL\_E\_BITS})-\T{1}))-\index{DBL EXCESS+\.{DBL\_EXCESS}}\.{DBL\_EXCESS};{}$\6
${}\index{digits+\\{digits}}\\{digits}\K\.{DBL\_M\_BITS}+\T{1};{}$\6
${}\.{DBG}(\index{DBGFLOAT+\.{DBGFLOAT}}\.{DBGFLOAT},\39\.{"\\tdigits=\%d\ mantiss}\)\.{a=0x\%"}\index{PRIx64+\\{PRIx64}}\\{PRIx64}\.{"\ binary\ exp=\%d\\n"},\3{-1}\39\index{digits+\\{digits}}\\{digits},\39\\{mantissa},\39\index{exp+\\{exp}}\\{exp}){}$;
\As67\ET68.
\U65.\Y
\fi

\M{67}

After we have obtained the binary exponent,
we round it down, and convert it to a hexadecimal
exponent.
\Y\B\4\X66:extract mantissa and exponent\X${}\mathrel+\E{}$\1\6
\4${}\{{}$\5
\&{int} \|r;\7
\&{if} ${}(\index{exp+\\{exp}}\\{exp}\G\T{0}){}$\5
\1${}\{{}$\5
${}\|r\K\index{exp+\\{exp}}\\{exp}\MOD\T{4};{}$\6
\&{if} ${}(\|r>\T{0}){}$\5
\1${}\{{}$\5
${}\\{mantissa}\K\\{mantissa}\LL\|r{}$;\5
${}\index{exp+\\{exp}}\\{exp}\K\index{exp+\\{exp}}\\{exp}-\|r{}$;\5
${}\index{digits+\\{digits}}\\{digits}\K\index{digits+\\{digits}}\\{digits}+\|r{}$;\5
${}\}{}$\2\6
\4${}\}{}$\2\6
\&{else}\5
\1${}\{{}$\5
${}\|r\K({-}\index{exp+\\{exp}}\\{exp})\MOD\T{4};{}$\6
\&{if} ${}(\|r>\T{0}){}$\5
\1${}\{{}$\5
${}\\{mantissa}\K\\{mantissa}\GG\|r{}$;\5
${}\index{exp+\\{exp}}\\{exp}\K\index{exp+\\{exp}}\\{exp}+\|r{}$;\5
${}\index{digits+\\{digits}}\\{digits}\K\index{digits+\\{digits}}\\{digits}-\|r{}$;\5
${}\}{}$\2\6
\4${}\}{}$\2\6
\4${}\}{}$\2\6
${}\index{exp+\\{exp}}\\{exp}\K\index{exp+\\{exp}}\\{exp}/\T{4};{}$\6
${}\.{DBG}(\index{DBGFLOAT+\.{DBGFLOAT}}\.{DBGFLOAT},\39\.{"\\tdigits=\%d\ mantiss}\)\.{a=0x\%"}\index{PRIx64+\\{PRIx64}}\\{PRIx64}\.{"\ hex\ exp=\%d\\n"},\3{-1}\39\index{digits+\\{digits}}\\{digits},\39\\{mantissa},\39\index{exp+\\{exp}}\\{exp}){}$;
\Y
\fi

\M{68}

In preparation for writing,
we shift the mantissa to the left so that the leftmost hexadecimal
digit of it will occupy the 4 leftmost bits of the variable \index{bits+\\{bits}}\\{bits} .

\Y\B\4\X66:extract mantissa and exponent\X${}\mathrel+\E{}$\6
$\\{mantissa}\K\\{mantissa}\LL(\T{64}-\.{DBL\_M\_BITS}-\T{4}){}$;\C{ move leading digit to leftmost nibble }
\Y
\fi

\M{69}

If the exponent is larger than \index{MAX HEX DIGITS+\.{MAX\_HEX\_DIGITS}}\.{MAX\_HEX\_DIGITS}, we need to
use an exponent even if the mantissa uses only a few digits.
When we use an exponent, we always write exactly one digit preceding the radix point.

\Y\B\4\X69:write large numbers\X${}\E{}$\1\6
\4${}\{{}$\5
${}\.{DBG}(\index{DBGFLOAT+\.{DBGFLOAT}}\.{DBGFLOAT},\39\.{"writing\ large\ numbe}\)\.{r\\n"});{}$\6
${}\index{hwritef+\\{hwritef}}\\{hwritef}(\.{"\%X."},\39(\&{uint8\_t})(\\{mantissa}\GG\T{60}));{}$\6
${}\\{mantissa}\K\\{mantissa}\LL\T{4};{}$\6
\&{do}\5
\1${}\{{}$\5
${}\index{hwritef+\\{hwritef}}\\{hwritef}(\.{"\%X"},\39(\&{uint8\_t})(\\{mantissa}\GG\.{DBL\_M\_BITS})\AND\T{\^F});{}$\6
${}\\{mantissa}\K\\{mantissa}\LL\T{4};{}$\6
\4${}\}{}$\2\5
\&{while} ${}(\\{mantissa}\I\T{0});{}$\6
${}\index{hwritef+\\{hwritef}}\\{hwritef}(\.{"x+\%X"},\39\index{exp+\\{exp}}\\{exp});{}$\6
\4${}\}{}$\2
\U65.\Y
\fi

\M{70}
If the exponent is small and non negative, we can write the
number without an exponent by writing the radix point at the
appropriate place.
\Y\B\4\X70:write medium numbers\X${}\E{}$\1\6
\4${}\{{}$\5
${}\.{DBG}(\index{DBGFLOAT+\.{DBGFLOAT}}\.{DBGFLOAT},\39\.{"writing\ medium\ numb}\)\.{er\\n"});{}$\6
\&{do}\5
\1${}\{{}$\5
${}\index{hwritef+\\{hwritef}}\\{hwritef}(\.{"\%X"},\39(\&{uint8\_t})(\\{mantissa}\GG\T{60}));{}$\6
${}\\{mantissa}\K\\{mantissa}\LL\T{4};{}$\6
\&{if} ${}(\index{exp+\\{exp}}\\{exp}\MM\E\T{0}){}$\1\5
\index{hwritec+\\{hwritec}}\\{hwritec}(\.{'.'});\2\6
\4${}\}{}$\2\5
\&{while} ${}(\\{mantissa}\I\T{0}\V\index{exp+\\{exp}}\\{exp}\G{-}\T{1});{}$\6
\4${}\}{}$\2
\U65.\Y
\fi

\M{71}
Last non least, we write numbers that would require additional zeros after the
radix point with an exponent, because it keeps the mantissa shorter.
\Y\B\4\X71:write small numbers\X${}\E{}$\1\6
\4${}\{{}$\5
${}\.{DBG}(\index{DBGFLOAT+\.{DBGFLOAT}}\.{DBGFLOAT},\39\.{"writing\ small\ numbe}\)\.{r\\n"});{}$\6
${}\index{hwritef+\\{hwritef}}\\{hwritef}(\.{"\%X."},\39(\&{uint8\_t})(\\{mantissa}\GG\T{60}));{}$\6
${}\\{mantissa}\K\\{mantissa}\LL\T{4};{}$\6
\&{do}\5
\1${}\{{}$\5
${}\index{hwritef+\\{hwritef}}\\{hwritef}(\.{"\%X"},\39(\&{uint8\_t})(\\{mantissa}\GG\T{60}));{}$\6
${}\\{mantissa}\K\\{mantissa}\LL\T{4};{}$\6
\4${}\}{}$\2\5
\&{while} ${}(\\{mantissa}\I\T{0});{}$\6
${}\index{hwritef+\\{hwritef}}\\{hwritef}(\.{"x-\%X"},\39{-}\index{exp+\\{exp}}\\{exp});{}$\6
\4${}\}{}$\2
\U65.\Y
\fi

\M{72}

Compared to the complications of long format floating point numbers,
the short format is very simple because we just use the binary representation.
Since 32 bit floating point numbers offer sufficient precision we use only
the \&{float32\_t} type.
It is however not possible to just write \index{HPUT32+\.{HPUT32}}\.{HPUT32}(\|d) for a \&{float32\_t} variable \|d
or \index{HPUT32+\.{HPUT32}}\.{HPUT32}((\&{uint32\_t}) \|d) because in the \CEE/ language this would imply
rounding the floating point number to the nearest integer.
But we have seen how to convert floating point values to bit pattern before.

\Y\B\4\X12:put functions\X${}\mathrel+\E{}$\6
\&{void} \index{hput float32+\\{hput\_float32}}\\{hput\_float32}(\&{float32\_t} \|d)\1\1\2\2\1\6
\4${}\{{}$\5
\&{union} ${}\{{}$\5
\1\&{float32\_t} \|d;\5
\&{uint32\_t} \index{bits+\\{bits}}\\{bits};\5
\2${}\}{}$ \|u;\7
${}\|u.\|d\K\|d{}$;\5
${}\index{HPUT32+\.{HPUT32}}\.{HPUT32}(\|u.\index{bits+\\{bits}}\\{bits});{}$\6
\4${}\}{}$\2
\Y
\fi

\M{73}

\Y\B\4\X16:get functions\X${}\mathrel+\E{}$\6
\&{float32\_t} \index{hget float32+\\{hget\_float32}}\\{hget\_float32}(\&{void})\1\1\2\2\1\6
\4${}\{{}$\5
\&{union} ${}\{{}$\5
\1\&{float32\_t} \|d;\5
\&{uint32\_t} \index{bits+\\{bits}}\\{bits};\5
\2${}\}{}$ \|u;\7
${}\index{HGET32+\.{HGET32}}\.{HGET32}(\|u.\index{bits+\\{bits}}\\{bits});{}$\6
\&{return} \|u${}.\|d;{}$\6
\4${}\}{}$\2
\Y
\fi

\M{74}

\subsection{Fixed Point Numbers}
\TeX\ internally represents most real numbers as fixed\index{fixed point number} point numbers or ``scaled integers''\index{scaled integer}.
The type {\bf scaled\_t} is defined as a signed 32 bit integer, but we consider it as a fixed point number
with the binary radix point just in the middle with sixteen bits before and sixteen bits after it.
To convert an integer into a scaled number, we multiply it by \index{ONE+\.{ONE}}\.{ONE}; to convert a floating point number
into a scaled number, we multiply it by \index{ONE+\.{ONE}}\.{ONE} and \index{ROUND+\.{ROUND}}\.{ROUND} the result to the nearest integer;
to convert a scaled number to a floating point number we divide it by (\&{float64\_t}) \index{ONE+\.{ONE}}\.{ONE}.

\noindent
\Y\B\4\X6:hint basic types\X${}\mathrel+\E{}$\6
\&{typedef} \&{int32\_t} \index{scaled t+\&{scaled\_t}}\&{scaled\_t};\6
\8\#\&{define} \index{ONE+\.{ONE}}\.{ONE}\5${}((\index{scaled t+\&{scaled\_t}}\&{scaled\_t})(\T{1}\LL\T{16})){}$
\Y
\fi

\M{75}

\Y\B\4\X11:hint macros\X${}\mathrel+\E{}$\6
\8\#\&{define} \index{ROUND+\.{ROUND}}\.{ROUND} (\|X)\5${}((\&{int})((\|X)\G\T{0.0}\?\index{floor+\\{floor}}\\{floor}((\|X)+\T{0.5}):\index{ceil+\\{ceil}}\\{ceil}((\|X)-\T{0.5}))){}$
\Y
\fi

\M{76}

\writecode
\Y\B\4\X19:write functions\X${}\mathrel+\E{}$\6
\&{void} \index{hwrite scaled+\\{hwrite\_scaled}}\\{hwrite\_scaled}(\index{scaled t+\&{scaled\_t}}\&{scaled\_t} \|x)\1\1\2\2\1\6
\4${}\{{}$\5
${}\index{hwrite float64+\\{hwrite\_float64}}\\{hwrite\_float64}(\|x/{}$(\&{float64\_t}) \index{ONE+\.{ONE}}\.{ONE});\6
\4${}\}{}$\2
\Y
\fi

\M{77}

\subsection{Dimensions}
In the long format,
the dimensions\index{dimension} of characters, boxes, and other things can be given
in three units:  \.{pt}, \index{in+\.{in}}\.{in}, and \.{mm}.

\readcode
\Y\par
\par
\par
\par
\par
\Y\B\4\X2:symbols\X${}\mathrel+\E{}$\6
\8\%\&{token} \index{DIMEN+\ts{DIMEN}}\ts{DIMEN}\5\.{"dimen"}\6
\8\%\&{token} \index{PT+\ts{PT}}\ts{PT}\5\.{"pt"}\6
\8\%\&{token} \index{MM+\ts{MM}}\ts{MM}\5\.{"mm"}\6
\8\%\&{token} \index{INCH+\ts{INCH}}\ts{INCH}\5\.{"in"}\6
\8\%\index{type+\&{type}}\&{type} $<$ \|d $>$ \index{dimension+\nts{dimension}}\nts{dimension}
\Y
\fi

\M{78}

\Y\B\4\X3:scanning rules\X${}\mathrel+\E{}$\6
${}\8\re{\vb{dimen}}{}$\ac\&{return} \index{DIMEN+\ts{DIMEN}}\ts{DIMEN};\eac\7
${}\8\re{\vb{pt}}{}$\ac\&{return} \index{PT+\ts{PT}}\ts{PT};\eac\7
${}\8\re{\vb{mm}}{}$\ac\&{return} \index{MM+\ts{MM}}\ts{MM};\eac\7
${}\8\re{\vb{in}}{}$\ac\&{return} \index{INCH+\ts{INCH}}\ts{INCH};\eac
\Y
\fi

\M{79}

The unit \.{pt} is a printers point\index{point}\index{pt+{\tt pt}}.
The unit ``\index{in+\.{in}}\.{in}'' stands for inches\index{inch}\index{in+{\tt in}} and we have $1\index{in+\.{in}}\.{in}= 72.27\,\.{pt}$.
The unit ``\.{mm}'' stands for millimeter\index{millimeter}\index{mm+{\tt mm}} and we have $1\index{in+\.{in}}\.{in}= 25.4\,\.{mm}$.

The definition of a printers\index{printers point} point given above follows the definition used in
\TeX\ which is slightly larger than the official definition of a printer's
point which was defined to equal exactly 0.013837\index{in+\.{in}}\.{in} by the American Typefounders
Association in~1886\cite{DK:texbook}.

We follow the tradition of \TeX\ and
store dimensions as ``scaled points''\index{scaled point} that is a dimension of $d$ points is
stored as $d\cdot2^{16}$ rounded to the nearest integer.
The maximum absolute value of a dimension is $(2^{30}-1)$ scaled points.

\Y\B\4\X6:hint basic types\X${}\mathrel+\E{}$\6
\&{typedef} \index{scaled t+\&{scaled\_t}}\&{scaled\_t} \index{dimen t+\&{dimen\_t}}\&{dimen\_t};\6
\8\#\&{define} \index{MAX DIMEN+\.{MAX\_DIMEN}}\.{MAX\_DIMEN}\5((\index{dimen t+\&{dimen\_t}}\&{dimen\_t})(\T{\^3FFFFFFF}))
\Y
\fi

\M{80}

\Y\B\4\X5:parsing rules\X${}\mathrel+\E{}$\6
\index{dimension+\nts{dimension}}\nts{dimension}: \1\1\5
\index{number+\nts{number}}\nts{number}\5
\index{PT+\ts{PT}}\ts{PT}\3{-1}\5
${}\{{}$\1\5
${}\.{\$\$}\K\index{ROUND+\.{ROUND}}\.{ROUND}(\.{\$1}*\index{ONE+\.{ONE}}\.{ONE});{}$\5
${}\.{RNG}(\.{"Dimension"},\39\.{\$\$},\39{-}\index{MAX DIMEN+\.{MAX\_DIMEN}}\.{MAX\_DIMEN},\39\index{MAX DIMEN+\.{MAX\_DIMEN}}\.{MAX\_DIMEN});{}$\5
${}\}{}$\2\6
\4\hbox to 0.5em{\hss${}\OR{}$}\5
\index{number+\nts{number}}\nts{number}\5
\index{INCH+\ts{INCH}}\ts{INCH}\3{-1}\5
${}\{{}$\1\5
${}\.{\$\$}\K\index{ROUND+\.{ROUND}}\.{ROUND}(\.{\$1}*\index{ONE+\.{ONE}}\.{ONE}*\T{72.27});{}$\5
${}\.{RNG}(\.{"Dimension"},\39\.{\$\$},\39{-}\index{MAX DIMEN+\.{MAX\_DIMEN}}\.{MAX\_DIMEN},\39\index{MAX DIMEN+\.{MAX\_DIMEN}}\.{MAX\_DIMEN}){}$;\5
${}\}{}$\2\6
\4\hbox to 0.5em{\hss${}\OR{}$}\5
\index{number+\nts{number}}\nts{number}\5
\index{MM+\ts{MM}}\ts{MM}\3{-1}\5
${}\{{}$\1\5
${}\.{\$\$}\K\index{ROUND+\.{ROUND}}\.{ROUND}(\.{\$1}*\index{ONE+\.{ONE}}\.{ONE}*(\T{72.27}/\T{25.4}));{}$\5
${}\.{RNG}(\.{"Dimension"},\39\.{\$\$},\39{-}\index{MAX DIMEN+\.{MAX\_DIMEN}}\.{MAX\_DIMEN},\39\index{MAX DIMEN+\.{MAX\_DIMEN}}\.{MAX\_DIMEN}){}$;\5
${}\}{}$\2;\2\2
\Y
\fi

\M{81}

When \index{stretch+\.{stretch}}\.{stretch} is writing dimensions in the long format,
for simplicity it always uses the unit ``\.{pt}''.
\writecode
\Y\B\4\X19:write functions\X${}\mathrel+\E{}$\6
\&{void} \index{hwrite dimension+\\{hwrite\_dimension}}\\{hwrite\_dimension}(\index{dimen t+\&{dimen\_t}}\&{dimen\_t} \|x)\1\1\2\2\1\6
\4${}\{{}$\5
\index{hwrite scaled+\\{hwrite\_scaled}}\\{hwrite\_scaled}(\|x);\6
\index{hwritef+\\{hwritef}}\\{hwritef}(\.{"pt"});\6
\4${}\}{}$\2
\Y
\fi

\M{82}

In the short format, dimensions are stored as 32 bit scaled point values without conversion.
\getcode
\Y\B\4\X16:get functions\X${}\mathrel+\E{}$\6
\&{void} \index{hget dimen+\\{hget\_dimen}}\\{hget\_dimen}(\&{void})\1\1\2\2\1\6
\4${}\{{}$\5
\&{uint32\_t} \|d;\7
\index{HGET32+\.{HGET32}}\.{HGET32}(\|d);\6
\index{hwrite dimension+\\{hwrite\_dimension}}\\{hwrite\_dimension}(\|d);\6
\4${}\}{}$\2
\Y
\fi

\M{83}

\putcode
\Y\B\4\X12:put functions\X${}\mathrel+\E{}$\6
\&{uint8\_t} \index{hput dimen+\\{hput\_dimen}}\\{hput\_dimen}(\index{dimen t+\&{dimen\_t}}\&{dimen\_t} \|d)\1\1\2\2\1\6
\4${}\{{}$\5
\index{HPUT32+\.{HPUT32}}\.{HPUT32}(\|d);\6
\&{return} \.{TAG}${}(\index{dimen kind+\\{dimen\_kind}}\\{dimen\_kind},\39\\{b001});{}$\6
\4${}\}{}$\2
\Y
\fi

\M{84}



\subsection{Extended Dimensions}\index{extended dimension}\index{hsize+{\tt hsize}}\index{vsize+{\tt vsize}}
The dimension that is probably used most frequently in a \TeX\ file is {\tt hsize}:
the ho\-ri\-zon\-tal size of a line of text. Common are also assignments
like \.{\\hsize=0.5\\hsize} \.{\\advance\\hsize by -10pt}, for example to
get two columns with lines almost half as wide as usual, leaving a small gap
between left and right column. Similar considerations apply to {\tt vsize}.

Because we aim at a reflowable format for \TeX\ output, we have to postpone
such computations until the values of \.{hsize} and \.{vsize} are known in the viewer.
Until then, we do symbolic computations on linear functions\index{linear function} of \.{hsize} and \.{vsize}.
We call such a linear function $w+h\cdot\.{hsize}+v\cdot\.{vsize}$
an extended dimension and represent it by the three numbers $w$, $h$, and $v$.

\Y\B\4\X6:hint basic types\X${}\mathrel+\E{}$\6
\&{typedef} \&{struct} ${}\{{}$\5
\1\index{dimen t+\&{dimen\_t}}\&{dimen\_t} \|w;\5
\&{float32\_t} \|h${},{}$ \|v;\5
\2${}\}{}$ \index{xdimen t+\&{xdimen\_t}}\&{xdimen\_t};
\Y
\fi

\M{85}
Since very often a component of an extended dimension is zero, we
store in the short format only the nonzero components and use the
info bits to mark them: \\{b100} implies $\|w\ne0$,
\\{b010} implies $\|h\ne 0$, and \\{b001} implies  $\|v\ne 0$.

\readcode
\Y\par
\par
\par
\par
\par
\Y\B\4\X2:symbols\X${}\mathrel+\E{}$\6
\8\%\&{token} \index{XDIMEN+\ts{XDIMEN}}\ts{XDIMEN}\5\.{"xdimen"}\6
\8\%\&{token} \ts{H}\5\.{"h"}\6
\8\%\&{token} \ts{V}\5\.{"v"}\6
\8\%\index{type+\&{type}}\&{type} $<$ \index{xd+\\{xd}}\\{xd} $>$ \index{xdimen+\nts{xdimen}}\nts{xdimen}
\Y
\fi

\M{86}
\Y\B\4\X3:scanning rules\X${}\mathrel+\E{}$\6
${}\8\re{\vb{xdimen}}{}$\ac\&{return} \index{XDIMEN+\ts{XDIMEN}}\ts{XDIMEN};\eac\7
${}\8\re{\vb{h}}{}$\ac\&{return} \ts{H};\eac\7
${}\8\re{\vb{v}}{}$\ac\&{return} \ts{V};\eac
\Y
\fi

\M{87}


\Y\B\4\X5:parsing rules\X${}\mathrel+\E{}$\6
\index{xdimen+\nts{xdimen}}\nts{xdimen}: \1\1\5
\index{dimension+\nts{dimension}}\nts{dimension}\5
\index{number+\nts{number}}\nts{number}\5
\ts{H}\5
\index{number+\nts{number}}\nts{number}\5
\ts{V}\5
${}\{{}$\1\5
${}\.{\$\$}.\|w\K\.{\$1}{}$;\5
${}\.{\$\$}.\|h\K\.{\$2}{}$;\5
${}\.{\$\$}.\|v\K\.{\$4};{}$\5
${}\}{}$\2\6
\4\hbox to 0.5em{\hss${}\OR{}$}\5
\index{dimension+\nts{dimension}}\nts{dimension}\5
\index{number+\nts{number}}\nts{number}\5
\ts{H}\5
${}\{{}$\1\5
${}\.{\$\$}.\|w\K\.{\$1}{}$;\5
${}\.{\$\$}.\|h\K\.{\$2}{}$;\5
${}\.{\$\$}.\|v\K\T{0.0};{}$\5
${}\}{}$\2\6
\4\hbox to 0.5em{\hss${}\OR{}$}\5
\index{dimension+\nts{dimension}}\nts{dimension}\5
\index{number+\nts{number}}\nts{number}\5
\ts{V}\5
${}\{{}$\1\5
${}\.{\$\$}.\|w\K\.{\$1}{}$;\5
${}\.{\$\$}.\|h\K\T{0.0}{}$;\5
${}\.{\$\$}.\|v\K\.{\$2};{}$\5
${}\}{}$\2\6
\4\hbox to 0.5em{\hss${}\OR{}$}\5
\index{dimension+\nts{dimension}}\nts{dimension}\5
${}\{{}$\1\5
${}\.{\$\$}.\|w\K\.{\$1}{}$;\5
${}\.{\$\$}.\|h\K\T{0.0}{}$;\5
${}\.{\$\$}.\|v\K\T{0.0};{}$\5
${}\}{}$\2;\2\2\7
\index{xdimen node+\nts{xdimen\_node}}\nts{xdimen\_node}: \1\1\5
\index{start+\nts{start}}\nts{start}\5
\index{XDIMEN+\ts{XDIMEN}}\ts{XDIMEN}\5
\index{xdimen+\nts{xdimen}}\nts{xdimen}\5
\index{END+\ts{END}}\ts{END}\5
${}\{{}$\1\5
${}\index{hput tags+\\{hput\_tags}}\\{hput\_tags}(\.{\$1},\39\index{hput xdimen+\\{hput\_xdimen}}\\{hput\_xdimen}({\AND}(\.{\$3})));{}$\5
${}\}{}$\2;\2\2
\Y
\fi

\M{88}

\writecode
\Y\B\4\X19:write functions\X${}\mathrel+\E{}$\6
\&{void} \index{hwrite xdimen+\\{hwrite\_xdimen}}\\{hwrite\_xdimen}(\index{xdimen t+\&{xdimen\_t}}\&{xdimen\_t} ${}{*}\|x){}$\1\1\2\2\1\6
\4${}\{{}$\5
${}\index{hwrite dimension+\\{hwrite\_dimension}}\\{hwrite\_dimension}(\|x\MG\|w);{}$\6
\&{if} ${}(\|x\MG\|h\I\T{0.0}){}$\5
\1${}\{{}$\5
${}\index{hwrite float64+\\{hwrite\_float64}}\\{hwrite\_float64}(\|x\MG\|h){}$;\5
\index{hwritec+\\{hwritec}}\\{hwritec}(\.{'h'});\5
${}\}{}$\2\6
\&{if} ${}(\|x\MG\|v\I\T{0.0}){}$\5
\1${}\{{}$\5
${}\index{hwrite float64+\\{hwrite\_float64}}\\{hwrite\_float64}(\|x\MG\|v){}$;\5
\index{hwritec+\\{hwritec}}\\{hwritec}(\.{'v'});\5
${}\}{}$\2\6
\4${}\}{}$\2\7
\&{void} \index{hwrite xdimen node+\\{hwrite\_xdimen\_node}}\\{hwrite\_xdimen\_node}(\index{xdimen t+\&{xdimen\_t}}\&{xdimen\_t} ${}{*}\|x){}$\1\1\2\2\1\6
\4${}\{{}$\5
\index{hwrite start+\\{hwrite\_start}}\\{hwrite\_start}(\,);\6
\index{hwritef+\\{hwritef}}\\{hwritef}(\.{"xdimen"});\6
\index{hwrite xdimen+\\{hwrite\_xdimen}}\\{hwrite\_xdimen}(\|x);\6
\index{hwrite end+\\{hwrite\_end}}\\{hwrite\_end}(\,);\6
\4${}\}{}$\2
\Y
\fi

\M{89}

\getcode

\Y\B\4\X17:get macros\X${}\mathrel+\E{}$\6
\8\#\&{define} $\index{HGET XDIMEN+\.{HGET\_XDIMEN}}\.{HGET\_XDIMEN}(\|I,\39\|X)$ \6
\&{if} ${}((\|I)\AND\\{b100}){}$\1\5
${}\index{HGET32+\.{HGET32}}\.{HGET32}((\|X).\|w){}$;\5
\2\&{else}\1\5
${}(\|X).\|w\K\T{0};{}$\2\6
\&{if} ${}((\|I)\AND\\{b010}){}$\1\5
${}(\|X).\|h\K\index{hget float32+\\{hget\_float32}}\\{hget\_float32}(\,){}$;\5
\2\&{else}\1\5
${}(\|X).\|h\K\T{0.0};{}$\2\6
\&{if} ${}((\|I)\AND\\{b001}){}$\1\5
${}(\|X).\|v\K\index{hget float32+\\{hget\_float32}}\\{hget\_float32}(\,){}$;\5
\2\&{else}\1\5
${}(\|X).\|v\K\T{0.0};{}$\2
\Y
\fi

\M{90}

\Y\B\4\X16:get functions\X${}\mathrel+\E{}$\6
\&{void} \index{hget xdimen+\\{hget\_xdimen}}\\{hget\_xdimen}(\&{uint8\_t} \|a${},\39{}$\index{xdimen t+\&{xdimen\_t}}\&{xdimen\_t} ${}{*}\|x){}$\1\1\2\2\1\6
\4${}\{{}$\6
\&{switch} (\|a)\5
\1${}\{{}$\6
\4\&{case} \.{TAG}${}(\index{xdimen kind+\\{xdimen\_kind}}\\{xdimen\_kind},\39\\{b001}){}$:\5
${}\index{HGET XDIMEN+\.{HGET\_XDIMEN}}\.{HGET\_XDIMEN}(\\{b001},\39{*}\|x){}$;\5
\&{break};\6
\4\&{case} \.{TAG}${}(\index{xdimen kind+\\{xdimen\_kind}}\\{xdimen\_kind},\39\\{b010}){}$:\5
${}\index{HGET XDIMEN+\.{HGET\_XDIMEN}}\.{HGET\_XDIMEN}(\\{b010},\39{*}\|x){}$;\5
\&{break};\6
\4\&{case} \.{TAG}${}(\index{xdimen kind+\\{xdimen\_kind}}\\{xdimen\_kind},\39\\{b011}){}$:\5
${}\index{HGET XDIMEN+\.{HGET\_XDIMEN}}\.{HGET\_XDIMEN}(\\{b011},\39{*}\|x){}$;\5
\&{break};\6
\4\&{case} \.{TAG}${}(\index{xdimen kind+\\{xdimen\_kind}}\\{xdimen\_kind},\39\\{b100}){}$:\5
${}\index{HGET XDIMEN+\.{HGET\_XDIMEN}}\.{HGET\_XDIMEN}(\\{b100},\39{*}\|x){}$;\5
\&{break};\6
\4\&{case} \.{TAG}${}(\index{xdimen kind+\\{xdimen\_kind}}\\{xdimen\_kind},\39\\{b101}){}$:\5
${}\index{HGET XDIMEN+\.{HGET\_XDIMEN}}\.{HGET\_XDIMEN}(\\{b101},\39{*}\|x){}$;\5
\&{break};\6
\4\&{case} \.{TAG}${}(\index{xdimen kind+\\{xdimen\_kind}}\\{xdimen\_kind},\39\\{b110}){}$:\5
${}\index{HGET XDIMEN+\.{HGET\_XDIMEN}}\.{HGET\_XDIMEN}(\\{b110},\39{*}\|x){}$;\5
\&{break};\6
\4\&{case} \.{TAG}${}(\index{xdimen kind+\\{xdimen\_kind}}\\{xdimen\_kind},\39\\{b111}){}$:\5
${}\index{HGET XDIMEN+\.{HGET\_XDIMEN}}\.{HGET\_XDIMEN}(\\{b111},\39{*}\|x){}$;\5
\&{break};\6
\4\&{default}:\5
${}\.{QUIT}(\.{"Extent\ expected\ got}\)\.{\ [\%s,\%d]"},\39\index{NAME+\.{NAME}}\.{NAME}(\|a),\39\index{INFO+\.{INFO}}\.{INFO}(\|a));{}$\6
\4${}\}{}$\2\6
\4${}\}{}$\2
\Y
\fi

\M{91}

Note that the info value \\{b000}, usually indicating a reference,
is not supported for extended dimensions.
Most nodes that need an extended dimension offer the opportunity to give
a reference directly without the start and end byte. An exception is the glue node,
but glue nodes that need an extended width are rare.

\Y\B\4\X16:get functions\X${}\mathrel+\E{}$\6
\&{void} \index{hget xdimen node+\\{hget\_xdimen\_node}}\\{hget\_xdimen\_node}(\index{xdimen t+\&{xdimen\_t}}\&{xdimen\_t} ${}{*}\|x){}$\1\1\2\2\1\6
\4${}\{{}$\5
\X14:read the start byte \|a\X\6
\&{if} ${}(\index{KIND+\.{KIND}}\.{KIND}(\|a)\E\index{xdimen kind+\\{xdimen\_kind}}\\{xdimen\_kind}){}$\1\5
${}\index{hget xdimen+\\{hget\_xdimen}}\\{hget\_xdimen}(\|a,\39\|x);{}$\2\6
\&{else}\1\5
${}\.{QUIT}(\.{"Extent\ expected\ at\ }\)\.{0x\%x\ got\ \%s"},\39\\{node\_pos},\39\index{NAME+\.{NAME}}\.{NAME}(\|a));{}$\2\6
\X15:read and check the end byte \|z\X\6
\4${}\}{}$\2
\Y
\fi

\M{92}



\putcode
\Y\B\4\X12:put functions\X${}\mathrel+\E{}$\6
\&{uint8\_t} \index{hput xdimen+\\{hput\_xdimen}}\\{hput\_xdimen}(\index{xdimen t+\&{xdimen\_t}}\&{xdimen\_t} ${}{*}\|x){}$\1\1\2\2\1\6
\4${}\{{}$\5
\index{info t+\&{info\_t}}\&{info\_t} \index{info+\\{info}}\\{info}${}\K\\{b000};{}$\7
\&{if} ${}(\|x\MG\|w\E\T{0}\W\|x\MG\|h\E\T{0.0}\W\|x\MG\|v\E\T{0.0}){}$\5
\1${}\{{}$\5
\index{HPUT32+\.{HPUT32}}\.{HPUT32}(\T{0});\5
${}\index{info+\\{info}}\\{info}\MRL{{\OR}{\K}}\\{b100}{}$;\5
${}\}{}$\2\6
\&{else}\5
\1${}\{{}$\6
\&{if} ${}(\|x\MG\|w\I\T{0}){}$\5
\1${}\{{}$\5
${}\index{HPUT32+\.{HPUT32}}\.{HPUT32}(\|x\MG\|w){}$;\5
${}\index{info+\\{info}}\\{info}\MRL{{\OR}{\K}}\\{b100}{}$;\5
${}\}{}$\2\6
\&{if} ${}(\|x\MG\|h\I\T{0.0}){}$\5
\1${}\{{}$\5
${}\index{hput float32+\\{hput\_float32}}\\{hput\_float32}(\|x\MG\|h){}$;\5
${}\index{info+\\{info}}\\{info}\MRL{{\OR}{\K}}\\{b010}{}$;\5
${}\}{}$\2\6
\&{if} ${}(\|x\MG\|v\I\T{0.0}){}$\5
\1${}\{{}$\5
${}\index{hput float32+\\{hput\_float32}}\\{hput\_float32}(\|x\MG\|v){}$;\5
${}\index{info+\\{info}}\\{info}\MRL{{\OR}{\K}}\\{b001}{}$;\5
${}\}{}$\2\6
\4${}\}{}$\2\6
\&{return} \.{TAG}${}(\index{xdimen kind+\\{xdimen\_kind}}\\{xdimen\_kind},\39\index{info+\\{info}}\\{info});{}$\6
\4${}\}{}$\2\7
\&{void} \index{hput xdimen node+\\{hput\_xdimen\_node}}\\{hput\_xdimen\_node}(\index{xdimen t+\&{xdimen\_t}}\&{xdimen\_t} ${}{*}\|x){}$\1\1\2\2\1\6
\4${}\{{}$\5
\&{uint32\_t} \|p${}\K\index{hpos+\\{hpos}}\\{hpos}\PP-\index{hstart+\\{hstart}}\\{hstart};{}$\7
${}\index{hput tags+\\{hput\_tags}}\\{hput\_tags}(\|p,\39\index{hput xdimen+\\{hput\_xdimen}}\\{hput\_xdimen}(\|x));{}$\6
\4${}\}{}$\2
\Y
\fi

\M{93}



\subsection{Stretch and Shrink}\label{stretch}
In section~\secref{glue}, we will consider glue\index{glue} which
is something that can stretch  and  shrink.
The stretchability\index{stretchability} and shrinkability\index{shrinkability} of the
glue can be given in ``\.{pt}'' like a dimension,
but there are three more units: \.{fil}, \.{fill}, and \.{filll}.
A glue with a stretchability of $1\,\hbox{\tt fil}$ will stretch infinitely more
than a glue with a stretchability of $1\,\hbox{\tt pt}$. So if you stretch both glues
together, the first glue will do all the stretching and the latter will not stretch
at all. The ``\.{fil}'' glue has simply a higher order of infinity.
You might guess that ``\.{fill}'' glue and ``\.{filll}'' glue have even higher
orders of infinite stretchability.
The order of infinity is 0 for \.{pt}, 1 for \.{fil}, 2 for \.{fill}, and 3 for \.{filll}.

The internal representation of a stretch is a variable of type \index{stretch t+\&{stretch\_t}}\&{stretch\_t}.
It stores the floating point value and the order of infinity separate as a \&{float64\_t} and a \&{uint8\_t}.


The short format tries to be space efficient and because it is not necessary to give the
stretchability with a precision exceeding about six decimal digits,
we use a single 32 bit floating point value.
To write a \&{float32\_t} value and an order value as one 32 bit value,
we round the two lowest bit of the \&{float32\_t} variable to zero
using ``round to even'' and store the order of infinity in these bits.
We define a union type \index{stch t+\&{stch\_t}}\&{stch\_t} to simplify conversion.

\Y\B\4\X6:hint basic types\X${}\mathrel+\E{}$\6
\&{typedef} \&{enum} ${}\{{}$\5
\1${}\index{normal o+\\{normal\_o}}\\{normal\_o}\K\T{0},\39\index{fil o+\\{fil\_o}}\\{fil\_o}\K\T{1},\39\index{fill o+\\{fill\_o}}\\{fill\_o}\K\T{2},\39\index{filll o+\\{filll\_o}}\\{filll\_o}\K{}$\T{3}\5
\2${}\}{}$ \index{order t+\&{order\_t}}\&{order\_t};\6
\&{typedef} \&{struct} ${}\{{}$\5
\1\&{float64\_t} \|f;\5
\index{order t+\&{order\_t}}\&{order\_t} \|o;\5
\2${}\}{}$ \index{stretch t+\&{stretch\_t}}\&{stretch\_t};\6
\&{typedef} \&{union} ${}\{{}$\5
\1\&{float32\_t} \|f;\5
\&{uint32\_t} \|u;\5
\2${}\}{}$ \index{stch t+\&{stch\_t}}\&{stch\_t};
\Y
\fi

\M{94}

\putcode
\Y\B\4\X12:put functions\X${}\mathrel+\E{}$\6
\&{void} \index{hput stretch+\\{hput\_stretch}}\\{hput\_stretch}(\index{stretch t+\&{stretch\_t}}\&{stretch\_t} ${}{*}\|s){}$\1\1\2\2\1\6
\4${}\{{}$\5
\&{uint32\_t} \\{mantissa}${},{}$ \\{lowbits}${},{}$ \\{sign}${},{}$ \\{exponent};\6
\index{stch t+\&{stch\_t}}\&{stch\_t} \index{st+\\{st}}\\{st};\7
${}\index{st+\\{st}}\\{st}.\|f\K\|s\MG\|f;{}$\6
${}\.{DBG}(\index{DBGFLOAT+\.{DBGFLOAT}}\.{DBGFLOAT},\39\.{"joining\ \%f->\%f(0x\%X}\)\.{),\%d:"},\39\|s\MG\|f,\39\index{st+\\{st}}\\{st}.\|f,\39\index{st+\\{st}}\\{st}.\|u,\39\|s\MG\|o);{}$\6
${}\\{mantissa}\K\index{st+\\{st}}\\{st}.\|u\AND{}$(((\&{uint32\_t}) \T{1}${}\LL\index{FLT M BITS+\.{FLT\_M\_BITS}}\.{FLT\_M\_BITS})-\T{1});{}$\6
${}\\{lowbits}\K\\{mantissa}\AND\T{\^7}{}$;\C{ lowest 3 bits }\6
${}\\{exponent}\K(\index{st+\\{st}}\\{st}.\|u\GG\index{FLT M BITS+\.{FLT\_M\_BITS}}\.{FLT\_M\_BITS})\AND{}$(((\&{uint32\_t}) \T{1}${}\LL\index{FLT E BITS+\.{FLT\_E\_BITS}}\.{FLT\_E\_BITS})-\T{1});{}$\6
${}\\{sign}\K\index{st+\\{st}}\\{st}.\|u\AND{}$((\&{uint32\_t}) \T{1}${}\LL(\index{FLT E BITS+\.{FLT\_E\_BITS}}\.{FLT\_E\_BITS}+\index{FLT M BITS+\.{FLT\_M\_BITS}}\.{FLT\_M\_BITS}));{}$\6
${}\.{DBG}(\index{DBGFLOAT+\.{DBGFLOAT}}\.{DBGFLOAT},\39\.{"s=\%d\ e=0x\%x\ m=0x\%x"},\39\\{sign},\39\\{exponent},\39\\{mantissa});{}$\6
\&{switch} (\\{lowbits})\C{ round to even }\6
\1${}\{{}$\5
\&{case} \T{0}:\5
\&{break};\C{ no change }\6
\4\&{case} \T{1}:\5
${}\\{mantissa}\K\\{mantissa}-\T{1}{}$;\5
\&{break};\C{ round down }\6
\4\&{case} \T{2}:\5
${}\\{mantissa}\K\\{mantissa}-\T{2}{}$;\5
\&{break};\C{ round down to even }\6
\4\&{case} \T{3}:\5
${}\\{mantissa}\K\\{mantissa}+\T{1}{}$;\5
\&{break};\C{ round up }\6
\4\&{case} \T{4}:\5
\&{break};\C{ no change }\6
\4\&{case} \T{5}:\5
${}\\{mantissa}\K\\{mantissa}-\T{1}{}$;\5
\&{break};\C{ round down }\6
\4\&{case} \T{6}:\5
${}\\{mantissa}\K\\{mantissa}+\T{1}{}$;\C{ round up to even, fall through }\6
\4\&{case} \T{7}:\5
${}\\{mantissa}\K\\{mantissa}+\T{1}{}$;\C{ round up to even }\6
\&{if} ${}(\\{mantissa}\G{}$((\&{uint32\_t}) \T{1}${}\LL\index{FLT M BITS+\.{FLT\_M\_BITS}}\.{FLT\_M\_BITS}){}$)\6
\1${}\{{}$\5
${}\\{exponent}\PP{}$;\C{ adjust exponent }\6
${}\.{RNG}(\.{"Float32\ exponent"},\39\\{exponent},\39\T{1},\39\T{2}*\index{FLT EXCESS+\.{FLT\_EXCESS}}\.{FLT\_EXCESS}){}$;\5
${}\\{mantissa}\K\\{mantissa}\GG\T{1};{}$\6
\4${}\}{}$\2\6
\&{break};\6
\4${}\}{}$\2\6
${}\.{DBG}(\index{DBGFLOAT+\.{DBGFLOAT}}\.{DBGFLOAT},\39\.{"\ round\ s=\%d\ e=0x\%x\ }\)\.{m=0x\%x"},\39\\{sign},\39\\{exponent},\39\\{mantissa});{}$\6
${}\index{st+\\{st}}\\{st}.\|u\K\\{sign}\OR(\\{exponent}\LL\index{FLT M BITS+\.{FLT\_M\_BITS}}\.{FLT\_M\_BITS})\OR\\{mantissa}\OR\|s\MG\|o;{}$\6
${}\.{DBG}(\index{DBGFLOAT+\.{DBGFLOAT}}\.{DBGFLOAT},\39\.{"float\ \%f\ hex\ 0x\%x\\n}\)\.{"},\39\index{st+\\{st}}\\{st}.\|f,\39\index{st+\\{st}}\\{st}.\|u);{}$\6
${}\index{HPUT32+\.{HPUT32}}\.{HPUT32}(\index{st+\\{st}}\\{st}.\|u);{}$\6
\4${}\}{}$\2
\Y
\fi

\M{95}

\getcode
\Y\B\4\X17:get macros\X${}\mathrel+\E{}$\6
\8\#\&{define} \index{HGET STRETCH+\.{HGET\_STRETCH}}\.{HGET\_STRETCH}(\|S)\1\1\2\2\1\6
\4${}\{{}$\5
\index{stch t+\&{stch\_t}}\&{stch\_t} \index{st+\\{st}}\\{st};\5
${}\index{HGET32+\.{HGET32}}\.{HGET32}(\index{st+\\{st}}\\{st}.\|u){}$;\5
${}\|S.\|o\K\index{st+\\{st}}\\{st}.\|u\AND\T{3};{}$\6
${}\index{st+\\{st}}\\{st}.\|u\MRL{\AND{\K}}\CM\T{3};{}$\6
${}\|S.\|f\K\index{st+\\{st}}\\{st}.\|f{}$;\5
${}\}{}$\2
\Y
\fi

\M{96}

\readcode
\Y\par
\par
\par
\par
\Y\B\4\X2:symbols\X${}\mathrel+\E{}$\6
\8\%\&{token} \index{FIL+\ts{FIL}}\ts{FIL}\5\.{"fil"}\6
\8\%\&{token} \index{FILL+\ts{FILL}}\ts{FILL}\5\.{"fill"}\6
\8\%\&{token} \index{FILLL+\ts{FILLL}}\ts{FILLL}\5\.{"filll"}\6
\8\%\index{type+\&{type}}\&{type} $<$ \index{st+\\{st}}\\{st} $>$ \index{stretch+\nts{stretch}}\nts{stretch} \6
\8\%\index{type+\&{type}}\&{type} $<$ \|o $>$ \index{order+\nts{order}}\nts{order}
\Y
\fi

\M{97}

\Y\B\4\X3:scanning rules\X${}\mathrel+\E{}$\6
${}\8\re{\vb{fil}}{}$\ac\&{return} \index{FIL+\ts{FIL}}\ts{FIL};\eac\7
${}\8\re{\vb{fill}}{}$\ac\&{return} \index{FILL+\ts{FILL}}\ts{FILL};\eac\7
${}\8\re{\vb{filll}}{}$\ac\&{return} \index{FILLL+\ts{FILLL}}\ts{FILLL};\eac
\Y
\fi

\M{98}

\Y\par
\par
\Y\B\4\X5:parsing rules\X${}\mathrel+\E{}$\6
\index{order+\nts{order}}\nts{order}: \1\1\5
\index{PT+\ts{PT}}\ts{PT}\5
${}\{{}$\1\5
${}\.{\$\$}\K\index{normal o+\\{normal\_o}}\\{normal\_o};{}$\5
${}\}{}$\2\6
\4\hbox to 0.5em{\hss${}\OR{}$}\5
\index{FIL+\ts{FIL}}\ts{FIL}\5
${}\{{}$\1\5
${}\.{\$\$}\K\index{fil o+\\{fil\_o}}\\{fil\_o};{}$\5
${}\}{}$\5
\2\hbox to 0.5em{\hss${}\OR{}$}\5
\index{FILL+\ts{FILL}}\ts{FILL}\5
${}\{{}$\1\5
${}\.{\$\$}\K\index{fill o+\\{fill\_o}}\\{fill\_o};{}$\5
${}\}{}$\5
\2\hbox to 0.5em{\hss${}\OR{}$}\5
\index{FILLL+\ts{FILLL}}\ts{FILLL}\5
${}\{{}$\1\5
${}\.{\$\$}\K\index{filll o+\\{filll\_o}}\\{filll\_o};{}$\5
${}\}{}$\2;\2\2\7
\index{stretch+\nts{stretch}}\nts{stretch}: \1\1\5
\index{number+\nts{number}}\nts{number}\5
\index{order+\nts{order}}\nts{order}\5
${}\{{}$\1\5
${}\.{\$\$}.\|f\K\.{\$1};{}$\5
${}\.{\$\$}.\|o\K\.{\$2};{}$\5
${}\}{}$\2;\2\2
\Y
\fi

\M{99}

\writecode

\Y\B\4\X19:write functions\X${}\mathrel+\E{}$\6
\&{void} \index{hwrite order+\\{hwrite\_order}}\\{hwrite\_order}(\index{order t+\&{order\_t}}\&{order\_t} \|o)\1\1\2\2\1\6
\4${}\{{}$\6
\&{switch} (\|o)\5
\1${}\{{}$\6
\4\&{case} \index{normal o+\\{normal\_o}}\\{normal\_o}:\5
\index{hwritef+\\{hwritef}}\\{hwritef}(\.{"pt"});\5
\&{break};\6
\4\&{case} \index{fil o+\\{fil\_o}}\\{fil\_o}:\5
\index{hwritef+\\{hwritef}}\\{hwritef}(\.{"fil"});\5
\&{break};\6
\4\&{case} \index{fill o+\\{fill\_o}}\\{fill\_o}:\5
\index{hwritef+\\{hwritef}}\\{hwritef}(\.{"fill"});\5
\&{break};\6
\4\&{case} \index{filll o+\\{filll\_o}}\\{filll\_o}:\5
\index{hwritef+\\{hwritef}}\\{hwritef}(\.{"filll"});\5
\&{break};\6
\4\&{default}:\5
${}\.{QUIT}(\.{"Illegal\ order\ \%d"},\39\|o){}$;\5
\&{break};\6
\4${}\}{}$\2\6
\4${}\}{}$\2\7
\&{void} \index{hwrite stretch+\\{hwrite\_stretch}}\\{hwrite\_stretch}(\index{stretch t+\&{stretch\_t}}\&{stretch\_t} ${}{*}\|s){}$\1\1\2\2\1\6
\4${}\{{}$\5
${}\index{hwrite float64+\\{hwrite\_float64}}\\{hwrite\_float64}(\|s\MG\|f);{}$\6
${}\index{hwrite order+\\{hwrite\_order}}\\{hwrite\_order}(\|s\MG\|o);{}$\6
\4${}\}{}$\2
\Y
\fi

\M{100}

\section{Simple Nodes}\hascode
\label{simple}
\subsection{Penalties}
Penalties\index{penalty} are very simple nodes. They specify the cost of breaking a
line or page at the present position. For the internal representation
we use an \&{int32\_t}. The full range of integers is, however, not
used. Instead penalties must be between -20000 and +20000.
(\TeX\ specifies a range of -10000 to +10000, but plain \TeX\ uses the value -20000
when it defines the supereject control sequence.)
The more general node is called an integer node;
it shares the same kind value $\index{int kind+\\{int\_kind}}\\{int\_kind}\K\index{penalty kind+\\{penalty\_kind}}\\{penalty\_kind}$
but allows the full range of values.
The info value of a penalty node is 1 or 2 and indicates the number of bytes
used to store the integer. The info value 4 can be used for general
integers (see section~\secref{definitions}) that need four byte of storage.

\readcode
\Y\par
\par
\par
\Y\B\4\X2:symbols\X${}\mathrel+\E{}$\6
\8\%\&{token} \index{PENALTY+\ts{PENALTY}}\ts{PENALTY}\5\.{"penalty"}\6
\8\%\&{token} \index{INTEGER+\ts{INTEGER}}\ts{INTEGER}\5\.{"int"}\6
\8\%\index{type+\&{type}}\&{type} $<$ \|i $>$ \index{penalty+\nts{penalty}}\nts{penalty}
\Y
\fi

\M{101}

\Y\B\4\X3:scanning rules\X${}\mathrel+\E{}$\6
${}\8\re{\vb{penalty}}{}$\ac\&{return} \index{PENALTY+\ts{PENALTY}}\ts{PENALTY};\eac\7
${}\8\re{\vb{int}}{}$\ac\&{return} \index{INTEGER+\ts{INTEGER}}\ts{INTEGER};\eac
\Y
\fi

\M{102}

\Y\B\4\X5:parsing rules\X${}\mathrel+\E{}$\6
\index{penalty+\nts{penalty}}\nts{penalty}: \1\1\5
\index{integer+\nts{integer}}\nts{integer}\5
${}\{{}$\1\5
${}\.{RNG}(\.{"Penalty"},\39\.{\$1},\39{-}\T{20000},\39{+}\T{20000});{}$\5
${}\.{\$\$}\K\.{\$1};{}$\5
${}\}{}$\2;\2\2\7
\index{content node+\nts{content\_node}}\nts{content\_node}: \1\1\5
\index{start+\nts{start}}\nts{start}\5
\index{PENALTY+\ts{PENALTY}}\ts{PENALTY}\5
\index{penalty+\nts{penalty}}\nts{penalty}\5
\index{END+\ts{END}}\ts{END}\5
${}\{{}$\1\5
${}\index{hput tags+\\{hput\_tags}}\\{hput\_tags}(\.{\$1},\39\index{hput int+\\{hput\_int}}\\{hput\_int}(\.{\$3})){}$;\5
${}\}{}$\2;\2\2
\Y
\fi

\M{103}

\getcode
\Y\B\4\X18:cases to get content\X${}\mathrel+\E{}$\6
\4\&{case} \.{TAG}${}(\index{penalty kind+\\{penalty\_kind}}\\{penalty\_kind},\39\T{1}){}$:\5
\1${}\{{}$\5
\&{int32\_t} \|p;\5
${}\index{HGET PENALTY+\.{HGET\_PENALTY}}\.{HGET\_PENALTY}(\T{1},\39\|p){}$;\5
${}\}{}$\5
\2\&{break};\6
\4\&{case} \.{TAG}${}(\index{penalty kind+\\{penalty\_kind}}\\{penalty\_kind},\39\T{2}){}$:\5
\1${}\{{}$\5
\&{int32\_t} \|p;\5
${}\index{HGET PENALTY+\.{HGET\_PENALTY}}\.{HGET\_PENALTY}(\T{2},\39\|p){}$;\5
${}\}{}$\5
\2\&{break};
\Y
\fi

\M{104}

\Y\B\4\X17:get macros\X${}\mathrel+\E{}$\6
\8\#\&{define} $\index{HGET PENALTY+\.{HGET\_PENALTY}}\.{HGET\_PENALTY}(\|I,\39\|P)$ \6
\&{if} ${}(\|I\E\T{1}){}$\5
\1${}\{{}$\5
\&{int8\_t} \|n${}\K\index{HGET8+\.{HGET8}}\.{HGET8}{}$;\5
${}\|P\K\|n{}$;\5
${}\}{}$\2\6
\&{else}\5
\1${}\{{}$\5
\&{int16\_t} \|n;\5
\index{HGET16+\.{HGET16}}\.{HGET16}(\|n);\5
${}\.{RNG}(\.{"Penalty"},\39\|n,\39{-}\T{20000},\39{+}\T{20000}){}$;\5
${}\|P\K\|n{}$;\5
${}\}{}$\2\6
\index{hwrite signed+\\{hwrite\_signed}}\\{hwrite\_signed}(\|P);
\Y
\fi

\M{105}

\putcode
\Y\B\4\X12:put functions\X${}\mathrel+\E{}$\6
\&{uint8\_t} \index{hput int+\\{hput\_int}}\\{hput\_int}(\&{int32\_t} \|n)\1\1\2\2\1\6
\4${}\{{}$\5
\index{info t+\&{info\_t}}\&{info\_t} \index{info+\\{info}}\\{info};\7
\&{if} ${}(\|n\G\T{0}{}$)\6
\1${}\{{}$\5
\&{if} ${}(\|n<\T{\^80}){}$\5
\1${}\{{}$\5
\index{HPUT8+\.{HPUT8}}\.{HPUT8}(\|n);\5
${}\index{info+\\{info}}\\{info}\K\T{1}{}$;\5
${}\}{}$\2\6
\&{else} \&{if} ${}(\|n<\T{\^8000}){}$\5
\1${}\{{}$\5
\index{HPUT16+\.{HPUT16}}\.{HPUT16}(\|n);\5
${}\index{info+\\{info}}\\{info}\K\T{2}{}$;\5
${}\}{}$\2\6
\&{else}\5
\1${}\{{}$\5
\index{HPUT32+\.{HPUT32}}\.{HPUT32}(\|n);\5
${}\index{info+\\{info}}\\{info}\K\T{4}{}$;\5
${}\}{}$\2\6
\4${}\}{}$\2\6
\&{else}\6
\1${}\{{}$\5
\&{if} ${}(\|n\G{-}\T{\^80}){}$\5
\1${}\{{}$\5
\index{HPUT8+\.{HPUT8}}\.{HPUT8}(\|n);\5
${}\index{info+\\{info}}\\{info}\K\T{1}{}$;\5
${}\}{}$\2\6
\&{else} \&{if} ${}(\|n\G{-}\T{\^8000}){}$\5
\1${}\{{}$\5
\index{HPUT16+\.{HPUT16}}\.{HPUT16}(\|n);\5
${}\index{info+\\{info}}\\{info}\K\T{2}{}$;\5
${}\}{}$\2\6
\&{else}\5
\1${}\{{}$\5
\index{HPUT32+\.{HPUT32}}\.{HPUT32}(\|n);\5
${}\index{info+\\{info}}\\{info}\K\T{4}{}$;\5
${}\}{}$\2\6
\4${}\}{}$\2\6
\&{return} \.{TAG}${}(\index{int kind+\\{int\_kind}}\\{int\_kind},\39\index{info+\\{info}}\\{info});{}$\6
\4${}\}{}$\2
\Y
\fi

\M{106}




\subsection{Languages}
To render a \HINT/ file on screen, information about the language is not necessary.
Knowing the language is, however, very important for language translation and
text to speech conversion which makes texts accessible to the visually-impaired.
For this reason, \HINT/ offers the opportunity to add this information
and encourages authors to supply this information.

Language information by itself is not sufficient to decode text. It must be supplemented
by information about the character encoding (see section~\secref{fonts}).

To represent language information, the world wide web has set universaly
accepted standards. The Internet Engineering Task Force IETF has defined tags for identifying
languages\cite{ietf:language}: short strings like ``en'' for English
or ``de'' for Deutsch, and longer ones like ``sl-IT-nedis'', for the specific variant of
the Nadiza dialect of Slovenian that is spoken in Italy.
We assume that any \HINT/ file
will contain only a small number of different languages and all language nodes can be
encoded using a reference to a predefined node from the
definition section (see section~\secref{reference}).
In the definition section, a language node will just
contain the language tag as given in~\cite{iana:language} (see section~\secref{definitions}).

\readcode
\Y\par
\par
\Y\B\4\X2:symbols\X${}\mathrel+\E{}$\6
\8\%\&{token} \index{LANGUAGE+\ts{LANGUAGE}}\ts{LANGUAGE}\5\.{"language"}
\Y
\fi

\M{107}

\Y\B\4\X3:scanning rules\X${}\mathrel+\E{}$\6
${}\8\re{\vb{language}}{}$\ac\&{return} \index{LANGUAGE+\ts{LANGUAGE}}\ts{LANGUAGE};\eac
\Y
\fi

\M{108}

When encoding language nodes in the short format,
We use the info value \\{b000} for language nodes in the definition section
and for language nodes in the content section that contain just a one-byte
reference (see section~\secref{reference}).
We use the info value \T{1} to \T{7} as a shorthand for
the references {\tt *0} and {\tt *6} to the predefined language nodes.


\goodbreak
\vbox{\getcode\vskip -\baselineskip\writecode}
\Y\B\4\X18:cases to get content\X${}\mathrel+\E{}$\6
\4\&{case} \.{TAG}${}(\index{language kind+\\{language\_kind}}\\{language\_kind},\39\T{1}){}$:\5
${}\index{REF+\.{REF}}\.{REF}(\index{language kind+\\{language\_kind}}\\{language\_kind},\39\T{0}){}$;\5
\index{hwrite ref+\\{hwrite\_ref}}\\{hwrite\_ref}(\T{0});\5
\&{break};\6
\4\&{case} \.{TAG}${}(\index{language kind+\\{language\_kind}}\\{language\_kind},\39\T{2}){}$:\5
${}\index{REF+\.{REF}}\.{REF}(\index{language kind+\\{language\_kind}}\\{language\_kind},\39\T{1}){}$;\5
\index{hwrite ref+\\{hwrite\_ref}}\\{hwrite\_ref}(\T{1});\5
\&{break};\6
\4\&{case} \.{TAG}${}(\index{language kind+\\{language\_kind}}\\{language\_kind},\39\T{3}){}$:\5
${}\index{REF+\.{REF}}\.{REF}(\index{language kind+\\{language\_kind}}\\{language\_kind},\39\T{2}){}$;\5
\index{hwrite ref+\\{hwrite\_ref}}\\{hwrite\_ref}(\T{2});\5
\&{break};\6
\4\&{case} \.{TAG}${}(\index{language kind+\\{language\_kind}}\\{language\_kind},\39\T{4}){}$:\5
${}\index{REF+\.{REF}}\.{REF}(\index{language kind+\\{language\_kind}}\\{language\_kind},\39\T{3}){}$;\5
\index{hwrite ref+\\{hwrite\_ref}}\\{hwrite\_ref}(\T{3});\5
\&{break};\6
\4\&{case} \.{TAG}${}(\index{language kind+\\{language\_kind}}\\{language\_kind},\39\T{5}){}$:\5
${}\index{REF+\.{REF}}\.{REF}(\index{language kind+\\{language\_kind}}\\{language\_kind},\39\T{4}){}$;\5
\index{hwrite ref+\\{hwrite\_ref}}\\{hwrite\_ref}(\T{4});\5
\&{break};\6
\4\&{case} \.{TAG}${}(\index{language kind+\\{language\_kind}}\\{language\_kind},\39\T{6}){}$:\5
${}\index{REF+\.{REF}}\.{REF}(\index{language kind+\\{language\_kind}}\\{language\_kind},\39\T{5}){}$;\5
\index{hwrite ref+\\{hwrite\_ref}}\\{hwrite\_ref}(\T{5});\5
\&{break};\6
\4\&{case} \.{TAG}${}(\index{language kind+\\{language\_kind}}\\{language\_kind},\39\T{7}){}$:\5
${}\index{REF+\.{REF}}\.{REF}(\index{language kind+\\{language\_kind}}\\{language\_kind},\39\T{6}){}$;\5
\index{hwrite ref+\\{hwrite\_ref}}\\{hwrite\_ref}(\T{6});\5
\&{break};
\Y
\fi

\M{109}

\putcode
\Y\B\4\X12:put functions\X${}\mathrel+\E{}$\6
\&{uint8\_t} \index{hput language+\\{hput\_language}}\\{hput\_language}(\&{uint8\_t} \|n)\1\1\2\2\1\6
\4${}\{{}$\6
\&{if} ${}(\|n<\T{7}){}$\1\5
\&{return} \.{TAG}${}(\index{language kind+\\{language\_kind}}\\{language\_kind},\39\|n+\T{1});{}$\2\6
\index{HPUT8+\.{HPUT8}}\.{HPUT8}(\|n);\6
\&{return} \.{TAG}${}(\index{language kind+\\{language\_kind}}\\{language\_kind},\39\T{0});{}$\6
\4${}\}{}$\2
\Y
\fi

\M{110}



\subsection{Rules}
Rules\index{rule} are simply black rectangles having a height, a depth, and a
width.  All of these dimensions can also be negative but a rule will
not be visible unless its width is positive and its height plus depth
is positive.

As a specialty, rules can have ``running dimensions''\index{running dimension}. If any of the
three dimensions is a running dimension, its actual value will be
determined by running the rule up to the boundary of the innermost
enclosing box.  The width is never running in an horizontal\index{horizontal list} list; the
height and depth are never running in a vertical\index{vertical list} list.  In the long
format, we use a vertical bar ``{\tt \VB}'' or a horizontal bar
``{\tt \_}'' (underscore character) to indicate a running
dimension. Of course the vertical bar is meant to indicate a running
height or depth while the horizontal bar stands for a running
width. The parser, however, makes no distinction between the two and
you can use either of them.  In the short format, we follow \TeX\ and
implement a running dimension by using the special value
$-2^{30}=\T{\^C0000000}$.


\Y\B\4\X11:hint macros\X${}\mathrel+\E{}$\6
\8\#\&{define} \index{RUNNING DIMEN+\.{RUNNING\_DIMEN}}\.{RUNNING\_DIMEN}\5\T{\^C0000000}
\Y
\fi

\M{111}

It could have been possible to allow extended dimensions in a rule node,
but in most circumstances, the mechanism of running dimensions is sufficient
and simpler to use. If a rule is needed that requires an extended dimension as
its length, it is always possible to put it inside a suitable box and use a
running dimension.


To make the short format encoding more compact, the first info bit
\\{b100} will be zero to indicate a running height, bit \\{b010} will be
zero to indicate a running depth, and bit \\{b001} will be zero to
indicate a running width.

Because leaders\index{leaders} (see section~\secref{leaders}) may contain a rule
node, we also provide functions to read and write a complete rule
node. While parsing the symbol ``{\sl rule\/}'' will just initialize a variable of type
\index{rule t+\&{rule\_t}}\&{rule\_t} (the writing is done with a separate routine),
parsing a {\sl rule\_node\/} will always include writing it.

% Currently no predefined rules.
%Further, a {\sl rule\_node} will permit the
%use of a predefined rule (see section~\secref{reference}),


\Y\B\4\X1:hint types\X${}\mathrel+\E{}$\6
\&{typedef} \&{struct} ${}\{{}$\5
\1\index{dimen t+\&{dimen\_t}}\&{dimen\_t} \|h${},{}$ \|d${},{}$ \|w;\5
\2${}\}{}$ \index{rule t+\&{rule\_t}}\&{rule\_t};
\Y
\fi

\M{112}

\readcode
\Y\par
\par
\par
\par
\par
\Y\B\4\X2:symbols\X${}\mathrel+\E{}$\6
\8\%\&{token} \index{RULE+\ts{RULE}}\ts{RULE}\5\.{"rule"}\6
\8\%\&{token} \index{RUNNING+\ts{RUNNING}}\ts{RUNNING}\5\.{"|"}\6
\8\%\index{type+\&{type}}\&{type} $<$ \|d $>$ \index{rule dimension+\nts{rule\_dimension}}\nts{rule\_dimension} \6
\8\%\index{type+\&{type}}\&{type} $<$ \|r $>$ \index{rule+\nts{rule}}\nts{rule}
\Y
\fi

\M{113}

\Y\B\4\X3:scanning rules\X${}\mathrel+\E{}$\6
${}\8\re{\vb{rule}}{}$\ac\&{return} \index{RULE+\ts{RULE}}\ts{RULE};\eac\7
${}\8\re{\vb{"|"}}{}$\ac\&{return} \index{RUNNING+\ts{RUNNING}}\ts{RUNNING};\eac\7
${}\8\re{\vb{"\_"}}{}$\ac\&{return} \index{RUNNING+\ts{RUNNING}}\ts{RUNNING};\eac
\Y
\fi

\M{114}

\Y\B\4\X5:parsing rules\X${}\mathrel+\E{}$\6
\index{rule dimension+\nts{rule\_dimension}}\nts{rule\_dimension}: \1\1\5
\index{dimension+\nts{dimension}}\nts{dimension}\5
\hbox to 0.5em{\hss${}\OR{}$}\5
\index{RUNNING+\ts{RUNNING}}\ts{RUNNING}\5
${}\{{}$\1\5
${}\.{\$\$}\K\index{RUNNING DIMEN+\.{RUNNING\_DIMEN}}\.{RUNNING\_DIMEN};{}$\5
${}\}{}$\2;\2\2\7
\index{rule+\nts{rule}}\nts{rule}: \1\1\5
\index{rule dimension+\nts{rule\_dimension}}\nts{rule\_dimension}\5
\index{rule dimension+\nts{rule\_dimension}}\nts{rule\_dimension}\5
\index{rule dimension+\nts{rule\_dimension}}\nts{rule\_dimension}\6
${}\{{}$\1\5
${}\.{\$\$}.\|h\K\.{\$1}{}$;\5
${}\.{\$\$}.\|d\K\.{\$2}{}$;\5
${}\.{\$\$}.\|w\K\.{\$3};{}$\6
\&{if} ${}(\.{\$3}\E\index{RUNNING DIMEN+\.{RUNNING\_DIMEN}}\.{RUNNING\_DIMEN}\W(\.{\$1}\E\index{RUNNING DIMEN+\.{RUNNING\_DIMEN}}\.{RUNNING\_DIMEN}\V\.{\$2}\E\index{RUNNING DIMEN+\.{RUNNING\_DIMEN}}\.{RUNNING\_DIMEN})){}$\1\5
${}\.{QUIT}(\.{"Incompatible\ runnin}\)\.{g\ dimensions\ 0x\%x\ 0x}\)\.{\%x\ 0x\%x"},\3{-1}\39\.{\$1},\39\.{\$2},\39\.{\$3});{}$\2\6
${}\}{}$\2;\2\2\7
\index{rule node+\nts{rule\_node}}\nts{rule\_node}: \1\1\5
\index{start+\nts{start}}\nts{start}\5
\index{RULE+\ts{RULE}}\ts{RULE}\5
\index{rule+\nts{rule}}\nts{rule}\5
\index{END+\ts{END}}\ts{END}\5
${}\{{}$\1\5
${}\index{hput tags+\\{hput\_tags}}\\{hput\_tags}(\.{\$1},\39\index{hput rule+\\{hput\_rule}}\\{hput\_rule}({\AND}(\.{\$3})));{}$\5
${}\}{}$\2;\2\2\7
\index{content node+\nts{content\_node}}\nts{content\_node}: \1\1\5
\index{rule node+\nts{rule\_node}}\nts{rule\_node};\2\2
\Y
\fi

\M{115}

\writecode
\Y\B\4\X19:write functions\X${}\mathrel+\E{}$\6
\&{static} \&{void} \index{hwrite rule dimension+\\{hwrite\_rule\_dimension}}\\{hwrite\_rule\_dimension}(\index{dimen t+\&{dimen\_t}}\&{dimen\_t} \|d${},\39{}$\&{char} \|c)\1\1\2\2\1\6
\4${}\{{}$\5
\&{if} ${}(\|d\E\index{RUNNING DIMEN+\.{RUNNING\_DIMEN}}\.{RUNNING\_DIMEN}){}$\1\5
${}\index{hwritef+\\{hwritef}}\\{hwritef}(\.{"\ \%c"},\39\|c);{}$\2\6
\&{else}\1\5
\index{hwrite dimension+\\{hwrite\_dimension}}\\{hwrite\_dimension}(\|d);\2\6
\4${}\}{}$\2\7
\&{void} \index{hwrite rule+\\{hwrite\_rule}}\\{hwrite\_rule}(\index{rule t+\&{rule\_t}}\&{rule\_t} ${}{*}\|r){}$\1\1\2\2\1\6
\4${}\{{}$\5
${}\index{hwrite rule dimension+\\{hwrite\_rule\_dimension}}\\{hwrite\_rule\_dimension}(\|r\MG\|h,\39\.{'|'});{}$\6
${}\index{hwrite rule dimension+\\{hwrite\_rule\_dimension}}\\{hwrite\_rule\_dimension}(\|r\MG\|d,\39\.{'|'});{}$\6
${}\index{hwrite rule dimension+\\{hwrite\_rule\_dimension}}\\{hwrite\_rule\_dimension}(\|r\MG\|w,\39\.{'\_'});{}$\6
\4${}\}{}$\2
\Y
\fi

\M{116}
\getcode
\Y\B\4\X18:cases to get content\X${}\mathrel+\E{}$\6
\4\&{case} \.{TAG}${}(\index{rule kind+\\{rule\_kind}}\\{rule\_kind},\39\\{b011}){}$:\5
\1${}\{{}$\5
\index{rule t+\&{rule\_t}}\&{rule\_t} \|r;\5
${}\index{HGET RULE+\.{HGET\_RULE}}\.{HGET\_RULE}(\\{b011},\39\|r){}$;\5
${}\index{hwrite rule+\\{hwrite\_rule}}\\{hwrite\_rule}({\AND}(\|r)){}$;\5
${}\}{}$\2\6
\&{break};\6
\4\&{case} \.{TAG}${}(\index{rule kind+\\{rule\_kind}}\\{rule\_kind},\39\\{b101}){}$:\5
\1${}\{{}$\5
\index{rule t+\&{rule\_t}}\&{rule\_t} \|r;\5
${}\index{HGET RULE+\.{HGET\_RULE}}\.{HGET\_RULE}(\\{b101},\39\|r){}$;\5
${}\index{hwrite rule+\\{hwrite\_rule}}\\{hwrite\_rule}({\AND}(\|r)){}$;\5
${}\}{}$\2\6
\&{break};\6
\4\&{case} \.{TAG}${}(\index{rule kind+\\{rule\_kind}}\\{rule\_kind},\39\\{b001}){}$:\5
\1${}\{{}$\5
\index{rule t+\&{rule\_t}}\&{rule\_t} \|r;\5
${}\index{HGET RULE+\.{HGET\_RULE}}\.{HGET\_RULE}(\\{b001},\39\|r){}$;\5
${}\index{hwrite rule+\\{hwrite\_rule}}\\{hwrite\_rule}({\AND}(\|r)){}$;\5
${}\}{}$\2\6
\&{break};\6
\4\&{case} \.{TAG}${}(\index{rule kind+\\{rule\_kind}}\\{rule\_kind},\39\\{b110}){}$:\5
\1${}\{{}$\5
\index{rule t+\&{rule\_t}}\&{rule\_t} \|r;\5
${}\index{HGET RULE+\.{HGET\_RULE}}\.{HGET\_RULE}(\\{b110},\39\|r){}$;\5
${}\index{hwrite rule+\\{hwrite\_rule}}\\{hwrite\_rule}({\AND}(\|r)){}$;\5
${}\}{}$\2\6
\&{break};\6
\4\&{case} \.{TAG}${}(\index{rule kind+\\{rule\_kind}}\\{rule\_kind},\39\\{b111}){}$:\5
\1${}\{{}$\5
\index{rule t+\&{rule\_t}}\&{rule\_t} \|r;\5
${}\index{HGET RULE+\.{HGET\_RULE}}\.{HGET\_RULE}(\\{b111},\39\|r){}$;\5
${}\index{hwrite rule+\\{hwrite\_rule}}\\{hwrite\_rule}({\AND}(\|r)){}$;\5
${}\}{}$\2\6
\&{break};
\Y
\fi

\M{117}

\Y\B\4\X17:get macros\X${}\mathrel+\E{}$\6
\8\#\&{define} $\index{HGET RULE+\.{HGET\_RULE}}\.{HGET\_RULE}(\|I,\39\|R){}$\6
\&{if} ${}((\|I)\AND\\{b100}){}$\1\5
${}\index{HGET32+\.{HGET32}}\.{HGET32}((\|R).\|h){}$;\5
\2\&{else}\1\5
${}(\|R).\|h\K\index{RUNNING DIMEN+\.{RUNNING\_DIMEN}}\.{RUNNING\_DIMEN};{}$\2\6
\&{if} ${}((\|I)\AND\\{b010}){}$\1\5
${}\index{HGET32+\.{HGET32}}\.{HGET32}((\|R).\|d){}$;\5
\2\&{else}\1\5
${}(\|R).\|d\K\index{RUNNING DIMEN+\.{RUNNING\_DIMEN}}\.{RUNNING\_DIMEN};{}$\2\6
\&{if} ${}((\|I)\AND\\{b001}){}$\1\5
${}\index{HGET32+\.{HGET32}}\.{HGET32}((\|R).\|w){}$;\5
\2\&{else}\1\5
${}(\|R).\|w\K\index{RUNNING DIMEN+\.{RUNNING\_DIMEN}}\.{RUNNING\_DIMEN};{}$\2
\Y
\fi

\M{118}

\Y\B\4\X16:get functions\X${}\mathrel+\E{}$\6
\&{void} \index{hget rule node+\\{hget\_rule\_node}}\\{hget\_rule\_node}(\&{void})\1\1\2\2\1\6
\4${}\{{}$\5
\X14:read the start byte \|a\X\6
\&{if} ${}(\index{KIND+\.{KIND}}\.{KIND}(\|a)\E\index{rule kind+\\{rule\_kind}}\\{rule\_kind}{}$)\6
\1${}\{{}$\5
\index{rule t+\&{rule\_t}}\&{rule\_t} \|r;\5
${}\index{HGET RULE+\.{HGET\_RULE}}\.{HGET\_RULE}(\index{INFO+\.{INFO}}\.{INFO}(\|a),\39\|r){}$;\6
\index{hwrite start+\\{hwrite\_start}}\\{hwrite\_start}(\,);\5
\index{hwritef+\\{hwritef}}\\{hwritef}(\.{"rule"});\5
${}\index{hwrite rule+\\{hwrite\_rule}}\\{hwrite\_rule}({\AND}\|r){}$;\5
\index{hwrite end+\\{hwrite\_end}}\\{hwrite\_end}(\,);\6
\4${}\}{}$\2\6
\&{else}\1\5
${}\.{QUIT}(\.{"Rule\ expected\ at\ 0x}\)\.{\%x\ got\ \%s"},\39\\{node\_pos},\39\index{NAME+\.{NAME}}\.{NAME}(\|a));{}$\2\6
\X15:read and check the end byte \|z\X\6
\4${}\}{}$\2
\Y
\fi

\M{119}

\putcode
\Y\B\4\X12:put functions\X${}\mathrel+\E{}$\6
\&{uint8\_t} \index{hput rule+\\{hput\_rule}}\\{hput\_rule}(\index{rule t+\&{rule\_t}}\&{rule\_t} ${}{*}\|r){}$\1\1\2\2\1\6
\4${}\{{}$\5
\index{info t+\&{info\_t}}\&{info\_t} \index{info+\\{info}}\\{info}${}\K\\{b000};{}$\7
\&{if} ${}(\|r\MG\|h\I\index{RUNNING DIMEN+\.{RUNNING\_DIMEN}}\.{RUNNING\_DIMEN}){}$\5
\1${}\{{}$\5
${}\index{HPUT32+\.{HPUT32}}\.{HPUT32}(\|r\MG\|h){}$;\5
${}\index{info+\\{info}}\\{info}\MRL{{\OR}{\K}}\\{b100}{}$;\5
${}\}{}$\2\6
\&{if} ${}(\|r\MG\|d\I\index{RUNNING DIMEN+\.{RUNNING\_DIMEN}}\.{RUNNING\_DIMEN}){}$\5
\1${}\{{}$\5
${}\index{HPUT32+\.{HPUT32}}\.{HPUT32}(\|r\MG\|d){}$;\5
${}\index{info+\\{info}}\\{info}\MRL{{\OR}{\K}}\\{b010}{}$;\5
${}\}{}$\2\6
\&{if} ${}(\|r\MG\|w\I\index{RUNNING DIMEN+\.{RUNNING\_DIMEN}}\.{RUNNING\_DIMEN}){}$\5
\1${}\{{}$\5
${}\index{HPUT32+\.{HPUT32}}\.{HPUT32}(\|r\MG\|w){}$;\5
${}\index{info+\\{info}}\\{info}\MRL{{\OR}{\K}}\\{b001}{}$;\5
${}\}{}$\2\6
\&{return} \.{TAG}${}(\index{rule kind+\\{rule\_kind}}\\{rule\_kind},\39\index{info+\\{info}}\\{info});{}$\6
\4${}\}{}$\2
\Y
\fi

\M{120}

\subsection{Glue}\label{glue}

%Glue considerations

%So what are the cases:
%\itemize
%\item reference to a dimen (common)
%\item reference to a xdimen
%\item reference to a dimen plus and minus
%\item reference to a xdimen plus and minus
%\item reference to a dimen plus
%\item reference to a xdimen plus
%\item reference to a dimen  minus
%\item reference to a xdimen minus
%\item dimen
%\item xdimen
%\item dimen plus and minus
%\item xdimen plus and minus (covers all other cases)
%\item dimen plus
%\item xdimen plus
%\item dimen  minus
%\item xdimen minus
%\item plus and minus
%\item plus
%\item minus
%\item zero glue (rare, can be replaced by a reference to the zero glue)
%\item reference to a predefined glue (common)
%\enditemize
%This is a total of 21 cases. Can we use the info bits to specify 7 common
%cases and one catch all? First the use of an extended dimension in a glue
%is probably not very common. More typically is the use of a fill glue
%that extends to the boundaries of the enclosing box.

%Here is the statistics for ctex:
%total 58937 glue entries
%total 49 defined glues (so 200 still available)
%There are three font specific glues defined for each font used in texts.
%The explicit glue nodes are the following:
%\itemize
%\item 35\% is predefined zero glue
%\item 30\% are 39 other predefined glue most of them less than 1%
%\item 8\% (4839) is one glue with 25pt pure stretch with order 0
%\item 25\% (14746) is one glue with 100pt stretch and 10pt shrink with order 0
%\item 2\% (1096) is one glue with 10pt no stretch and shrink
%\item 0\% (13) are 7 different glues with no stretch and shrink
%\item 0\% (3) different glues with width!=0 and some stretch of order 0
%\item 0\% (27) 20 different glues with stretch and shrink
%\enditemize

%Some more glue with 1fil is insider 55  leaders
%one vset has an extent 1 no stretch and shrink
%56 hset specify an extent 2 and 1 fil stretch


We have seen in section~\secref{stretch} how to deal with
stretchability\index{stretchability} and
shrinkability\index{shrinkability} and we will need this now.
Glue\index{glue} has a natural width---which in general can be an
extended dimension---and in addition it can stretch and shrink.  It
might have been possible to allow an extended dimension also for the
stretch\-ability or shrink\-ability of a glue, but this seems of
little practical relevance and so simplicity won over generality.
Even with that restriction, it is an understatement to regard glue
nodes as "simple" nodes, and we could equally well list them in
section~\secref{composite} as composite nodes.

To use the info bits in the short format wisely, I collected some
statistical data using the \TeX book as an example. It turns out that
about 99\% of all the 58937 glue nodes (not counting the interword
glues used inside texts) could be covered with only 43 predefined
glues.  So this is by far the most common case; we reserve the info
value \\{b000} to cover it and postpone the description of such glue
nodes until we describe references in section~\secref{reference}.

We expect the remaining cases to contribute not too much to the file
size, and hence, simplicity is a more important aspect than efficiency
when allocating the remaining info values.

Looking at the glues in more detail, we find that the most common
cases are those where either one, two, or all three glue components
are zero. We use the two lowest bits to indicate the presence of a
nonzero stretchability or shrinkability and reserve the info values
\\{b001}, \\{b010}, and \\{b011} for those cases where the width of the glue
is zero.  The zero glue, where all components are zero, is defined as
a fixed, predefined glue instead of reserving a special info value for
it.  The cost of one extra byte when encoding it seems not too high a
price to pay.  After reserving the info value \\{b111} for the most
general case of a glue, we have only three more info values left:
\\{b100}, \\{b101}, and \\{b110}.  Keeping things simple implies using the
two lowest info bits---as before---to indicate a nonzero
stretchability or shrinkability. For the width, three choices remain:
using a reference to a dimension, using a reference to an extended
dimension, or using an immediate value.  Since references to glues are
already supported, an immediate width seems best for glues that are
not frequently reused, avoiding the overhead of references.

% It also makes parsing simpler because we avoid the confusion
% between references to dimensions
% and references to glues and references to extended dimensions.

Here is a summary of the info bits and the implied layout
of glue nodes in the short format:
\itemize
\item \\{b000}: reference to a predefined glue
\item \\{b001}: zero width and nonzero shrinkability
\item \\{b010}: zero width and nonzero stretchability
\item \\{b011}: zero width and nonzero stretchability and  shrinkability
\item \\{b100}: nonzero width
\item \\{b101}: nonzero width and nonzero shrinkability
\item \\{b110}: nonzero width and nonzero stretchability
\item \\{b111}: extended dimension and nonzero stretchability and  shrinkability
\enditemize


\Y\B\4\X6:hint basic types\X${}\mathrel+\E{}$\6
\&{typedef} \&{struct} ${}\{{}$\5
\1\index{xdimen t+\&{xdimen\_t}}\&{xdimen\_t} \|w;\5
\index{stretch t+\&{stretch\_t}}\&{stretch\_t} \|p${},{}$ \|m;\5
\2${}\}{}$ \index{glue t+\&{glue\_t}}\&{glue\_t};
\Y
\fi

\M{121}


To test for a zero glue,
we implement a macro:
\Y\B\4\X11:hint macros\X${}\mathrel+\E{}$\6
\8\#\&{define} \index{ZERO GLUE+\.{ZERO\_GLUE}}\.{ZERO\_GLUE}(\|G)\5${}((\|G).\|w.\|w\E\T{0}\W(\|G).\|w.\|h\E\T{0.0}\W(\|G).\|w.\|v\E\T{0.0}\W(\|G).\|p.\|f\E\T{0.0}\W(\|G).\|m.\|f\E\T{0.0}){}$
\Y
\fi

\M{122}

Because other nodes (leaders, baselines, and fonts)
contain glue nodes as parameters, we provide functions
to read and write a complete glue node in the same way as we did
for rule nodes.
Further, such an internal {\sl glue\_node\/} has the special property that
in the short format a node for the zero glue might be omitted entirely.

\readcode
\Y\par
\par
\par
\par
\par
\par
\par
\Y\B\4\X2:symbols\X${}\mathrel+\E{}$\6
\8\%\&{token} \index{GLUE+\ts{GLUE}}\ts{GLUE}\5\.{"glue"}\6
\8\%\&{token} \index{PLUS+\ts{PLUS}}\ts{PLUS}\5\.{"plus"}\6
\8\%\&{token} \index{MINUS+\ts{MINUS}}\ts{MINUS}\5\.{"minus"}\6
\8\%\index{type+\&{type}}\&{type} $<$ \|g $>$ \index{glue+\nts{glue}}\nts{glue} \6
\8\%\index{type+\&{type}}\&{type} $<$ \|b $>$ \index{glue node+\nts{glue\_node}}\nts{glue\_node} \6
\8\%\index{type+\&{type}}\&{type} $<$ \index{st+\\{st}}\\{st} $>$ \index{plus+\nts{plus}}\nts{plus}\5
\index{minus+\nts{minus}}\nts{minus}
\Y
\fi

\M{123}

\Y\B\4\X3:scanning rules\X${}\mathrel+\E{}$\6
${}\8\re{\vb{glue}}{}$\ac\&{return} \index{GLUE+\ts{GLUE}}\ts{GLUE};\eac\7
${}\8\re{\vb{plus}}{}$\ac\&{return} \index{PLUS+\ts{PLUS}}\ts{PLUS};\eac\7
${}\8\re{\vb{minus}}{}$\ac\&{return} \index{MINUS+\ts{MINUS}}\ts{MINUS};\eac
\Y
\fi

\M{124}

\Y\B\4\X5:parsing rules\X${}\mathrel+\E{}$\6
\index{plus+\nts{plus}}\nts{plus}: \1\1\5
${}\{{}$\1\5
${}\.{\$\$}.\|f\K\T{0.0};{}$\5
${}\.{\$\$}.\|o\K\T{0};{}$\5
${}\}{}$\2\6
\4\hbox to 0.5em{\hss${}\OR{}$}\5
\index{PLUS+\ts{PLUS}}\ts{PLUS}\5
\index{stretch+\nts{stretch}}\nts{stretch}\5
${}\{{}$\1\5
${}\.{\$\$}\K\.{\$2};{}$\5
${}\}{}$\2;\2\2\7
\index{minus+\nts{minus}}\nts{minus}: \1\1\5
${}\{{}$\1\5
${}\.{\$\$}.\|f\K\T{0.0};{}$\5
${}\.{\$\$}.\|o\K\T{0};{}$\5
${}\}{}$\2\6
\4\hbox to 0.5em{\hss${}\OR{}$}\5
\index{MINUS+\ts{MINUS}}\ts{MINUS}\5
\index{stretch+\nts{stretch}}\nts{stretch}\5
${}\{{}$\1\5
${}\.{\$\$}\K\.{\$2};{}$\5
${}\}{}$\2;\2\2\7
\index{glue+\nts{glue}}\nts{glue}: \1\1\5
\index{xdimen+\nts{xdimen}}\nts{xdimen}\5
\index{plus+\nts{plus}}\nts{plus}\5
\index{minus+\nts{minus}}\nts{minus}\5
${}\{{}$\1\5
${}\.{\$\$}.\|w\K\.{\$1};{}$\5
${}\.{\$\$}.\|p\K\.{\$2};{}$\5
${}\.{\$\$}.\|m\K\.{\$3};{}$\5
${}\}{}$\2;\2\2\7
\index{content node+\nts{content\_node}}\nts{content\_node}: \1\1\5
\index{start+\nts{start}}\nts{start}\5
\index{GLUE+\ts{GLUE}}\ts{GLUE}\5
\index{glue+\nts{glue}}\nts{glue}\5
\index{END+\ts{END}}\ts{END}\5
${}\{{}$\1\6
\&{if} (\index{ZERO GLUE+\.{ZERO\_GLUE}}\.{ZERO\_GLUE}(\.{\$3}))\5
\1${}\{{}$\5
\index{HPUT8+\.{HPUT8}}\.{HPUT8}(\index{zero skip no+\\{zero\_skip\_no}}\\{zero\_skip\_no});\5
${}\index{hput tags+\\{hput\_tags}}\\{hput\_tags}(\.{\$1},\39\.{TAG}(\index{glue kind+\\{glue\_kind}}\\{glue\_kind},\39\T{0}));{}$\6
\4${}\}{}$\2\6
\&{else}\1\5
${}\index{hput tags+\\{hput\_tags}}\\{hput\_tags}(\.{\$1},\39\index{hput glue+\\{hput\_glue}}\\{hput\_glue}({\AND}(\.{\$3})));{}$\2\6
${}\}{}$\2;\2\2\7
\index{glue node+\nts{glue\_node}}\nts{glue\_node}: \1\1\5
\index{start+\nts{start}}\nts{start}\5
\index{GLUE+\ts{GLUE}}\ts{GLUE}\5
\index{glue+\nts{glue}}\nts{glue}\5
\index{END+\ts{END}}\ts{END}\6
${}\{{}$\5
\1\&{if} (\index{ZERO GLUE+\.{ZERO\_GLUE}}\.{ZERO\_GLUE}(\.{\$3}))\5
\1${}\{{}$\5
${}\index{hpos+\\{hpos}}\\{hpos}\MM;{}$\5
${}\.{\$\$}\K\\{false}{}$;\5
${}\}{}$\2\6
\&{else}\5
\1${}\{{}$\5
${}\index{hput tags+\\{hput\_tags}}\\{hput\_tags}(\.{\$1},\39\index{hput glue+\\{hput\_glue}}\\{hput\_glue}({\AND}(\.{\$3})));{}$\5
${}\.{\$\$}\K\\{true}{}$;\5
${}\}{}$\5
\2${}\}{}$\2;\2\2
\Y
\fi

\M{125}

\writecode
\Y\B\4\X19:write functions\X${}\mathrel+\E{}$\6
\&{void} \index{hwrite plus+\\{hwrite\_plus}}\\{hwrite\_plus}(\index{stretch t+\&{stretch\_t}}\&{stretch\_t} ${}{*}\|p){}$\1\1\2\2\1\6
\4${}\{{}$\5
\&{if} ${}(\|p\MG\|f\I\T{0.0}){}$\5
\1${}\{{}$\5
\index{hwritef+\\{hwritef}}\\{hwritef}(\.{"\ plus"});\5
\index{hwrite stretch+\\{hwrite\_stretch}}\\{hwrite\_stretch}(\|p);\5
${}\}{}$\2\6
\4${}\}{}$\2\7
\&{void} \index{hwrite minus+\\{hwrite\_minus}}\\{hwrite\_minus}(\index{stretch t+\&{stretch\_t}}\&{stretch\_t} ${}{*}\|m){}$\1\1\2\2\1\6
\4${}\{{}$\5
\&{if} ${}(\|m\MG\|f\I\T{0.0}){}$\5
\1${}\{{}$\5
\index{hwritef+\\{hwritef}}\\{hwritef}(\.{"\ minus"});\5
\index{hwrite stretch+\\{hwrite\_stretch}}\\{hwrite\_stretch}(\|m);\5
${}\}{}$\2\6
\4${}\}{}$\2\7
\&{void} \index{hwrite glue+\\{hwrite\_glue}}\\{hwrite\_glue}(\index{glue t+\&{glue\_t}}\&{glue\_t} ${}{*}\|g){}$\1\1\2\2\1\6
\4${}\{{}$\5
${}\index{hwrite xdimen+\\{hwrite\_xdimen}}\\{hwrite\_xdimen}({\AND}(\|g\MG\|w)){}$;\5
${}\index{hwrite plus+\\{hwrite\_plus}}\\{hwrite\_plus}({\AND}\|g\MG\|p){}$;\5
${}\index{hwrite minus+\\{hwrite\_minus}}\\{hwrite\_minus}({\AND}\|g\MG\|m);{}$\6
\4${}\}{}$\2\7
\&{void} \index{hwrite glue node+\\{hwrite\_glue\_node}}\\{hwrite\_glue\_node}(\index{glue t+\&{glue\_t}}\&{glue\_t} ${}{*}\|g){}$\1\1\2\2\1\6
\4${}\{{}$\5
\&{if} ${}(\index{ZERO GLUE+\.{ZERO\_GLUE}}\.{ZERO\_GLUE}({*}\|g)){}$\1\5
${}\index{hwrite ref node+\\{hwrite\_ref\_node}}\\{hwrite\_ref\_node}(\index{glue kind+\\{glue\_kind}}\\{glue\_kind},\39\index{zero skip no+\\{zero\_skip\_no}}\\{zero\_skip\_no});{}$\2\6
\&{else}\5
\1${}\{{}$\5
\index{hwrite start+\\{hwrite\_start}}\\{hwrite\_start}(\,);\5
\index{hwritef+\\{hwritef}}\\{hwritef}(\.{"glue"});\5
\index{hwrite glue+\\{hwrite\_glue}}\\{hwrite\_glue}(\|g);\5
\index{hwrite end+\\{hwrite\_end}}\\{hwrite\_end}(\,);\5
${}\}{}$\2\6
\4${}\}{}$\2
\Y
\fi

\M{126}

\getcode
\Y\B\4\X18:cases to get content\X${}\mathrel+\E{}$\6
\4\&{case} \.{TAG}${}(\index{glue kind+\\{glue\_kind}}\\{glue\_kind},\39\\{b001}){}$:\1\6
\4${}\{{}$\5
\index{glue t+\&{glue\_t}}\&{glue\_t} \|g;\5
${}\index{HGET GLUE+\.{HGET\_GLUE}}\.{HGET\_GLUE}(\\{b001},\39\|g){}$;\5
${}\index{hwrite glue+\\{hwrite\_glue}}\\{hwrite\_glue}({\AND}\|g){}$;\5
${}\}{}$\5
\2\&{break};\6
\4\&{case} \.{TAG}${}(\index{glue kind+\\{glue\_kind}}\\{glue\_kind},\39\\{b010}){}$:\1\6
\4${}\{{}$\5
\index{glue t+\&{glue\_t}}\&{glue\_t} \|g;\5
${}\index{HGET GLUE+\.{HGET\_GLUE}}\.{HGET\_GLUE}(\\{b010},\39\|g){}$;\5
${}\index{hwrite glue+\\{hwrite\_glue}}\\{hwrite\_glue}({\AND}\|g){}$;\5
${}\}{}$\5
\2\&{break};\6
\4\&{case} \.{TAG}${}(\index{glue kind+\\{glue\_kind}}\\{glue\_kind},\39\\{b011}){}$:\1\6
\4${}\{{}$\5
\index{glue t+\&{glue\_t}}\&{glue\_t} \|g;\5
${}\index{HGET GLUE+\.{HGET\_GLUE}}\.{HGET\_GLUE}(\\{b011},\39\|g){}$;\5
${}\index{hwrite glue+\\{hwrite\_glue}}\\{hwrite\_glue}({\AND}\|g){}$;\5
${}\}{}$\5
\2\&{break};\6
\4\&{case} \.{TAG}${}(\index{glue kind+\\{glue\_kind}}\\{glue\_kind},\39\\{b100}){}$:\1\6
\4${}\{{}$\5
\index{glue t+\&{glue\_t}}\&{glue\_t} \|g;\5
${}\index{HGET GLUE+\.{HGET\_GLUE}}\.{HGET\_GLUE}(\\{b100},\39\|g){}$;\5
${}\index{hwrite glue+\\{hwrite\_glue}}\\{hwrite\_glue}({\AND}\|g){}$;\5
${}\}{}$\5
\2\&{break};\6
\4\&{case} \.{TAG}${}(\index{glue kind+\\{glue\_kind}}\\{glue\_kind},\39\\{b101}){}$:\1\6
\4${}\{{}$\5
\index{glue t+\&{glue\_t}}\&{glue\_t} \|g;\5
${}\index{HGET GLUE+\.{HGET\_GLUE}}\.{HGET\_GLUE}(\\{b101},\39\|g){}$;\5
${}\index{hwrite glue+\\{hwrite\_glue}}\\{hwrite\_glue}({\AND}\|g){}$;\5
${}\}{}$\5
\2\&{break};\6
\4\&{case} \.{TAG}${}(\index{glue kind+\\{glue\_kind}}\\{glue\_kind},\39\\{b110}){}$:\1\6
\4${}\{{}$\5
\index{glue t+\&{glue\_t}}\&{glue\_t} \|g;\5
${}\index{HGET GLUE+\.{HGET\_GLUE}}\.{HGET\_GLUE}(\\{b110},\39\|g){}$;\5
${}\index{hwrite glue+\\{hwrite\_glue}}\\{hwrite\_glue}({\AND}\|g){}$;\5
${}\}{}$\5
\2\&{break};\6
\4\&{case} \.{TAG}${}(\index{glue kind+\\{glue\_kind}}\\{glue\_kind},\39\\{b111}){}$:\1\6
\4${}\{{}$\5
\index{glue t+\&{glue\_t}}\&{glue\_t} \|g;\5
${}\index{HGET GLUE+\.{HGET\_GLUE}}\.{HGET\_GLUE}(\\{b111},\39\|g){}$;\5
${}\index{hwrite glue+\\{hwrite\_glue}}\\{hwrite\_glue}({\AND}\|g){}$;\5
${}\}{}$\5
\2\&{break};
\Y
\fi

\M{127}

\Y\B\4\X17:get macros\X${}\mathrel+\E{}$\6
\8\#\&{define} $\index{HGET GLUE+\.{HGET\_GLUE}}\.{HGET\_GLUE}(\|I,\39\|G){}$\1\1 $\{$ \6
\&{if} ${}(\|I\E\\{b111}){}$\1\5
${}\index{hget xdimen node+\\{hget\_xdimen\_node}}\\{hget\_xdimen\_node}({\AND}((\|G).\|w));{}$\2\6
\&{else}\6
\1${}\{{}$\5
${}(\|G).\|w.\|h\K\T{0.0}{}$;\5
${}(\|G).\|w.\|v\K\T{0.0};{}$\6
\&{if} ${}((\|I)\AND\\{b100}){}$\1\5
${}\index{HGET32+\.{HGET32}}\.{HGET32}((\|G).\|w.\|w){}$;\5
\2\&{else}\1\5
${}(\|G).\|w.\|w\K\T{0}{}$;\5
\2${}\}{}$\2\6
\&{if} ${}((\|I)\AND\\{b010})$ $\index{HGET STRETCH+\.{HGET\_STRETCH}}\.{HGET\_STRETCH}((\|G).\|p{}$)\5
\&{else}\1\5
${}(\|G).\|p.\|f\K\T{0.0},\39(\|G).\|p.\|o\K\T{0};{}$\2\6
\&{if} ${}((\|I)\AND\\{b001})$ $\index{HGET STRETCH+\.{HGET\_STRETCH}}\.{HGET\_STRETCH}((\|G).\|m{}$)\5
\&{else}\1\5
${}(\|G).\|m.\|f\K\T{0.0},\39(\|G).\|m.\|o\K\T{0}{}$;\5
\2${}\}{}$
\Y
\fi

\M{128}

\Y\B\4\X16:get functions\X${}\mathrel+\E{}$\6
\&{void} \index{hget glue node+\\{hget\_glue\_node}}\\{hget\_glue\_node}(\&{void})\1\1\2\2\1\6
\4${}\{{}$\5
\X14:read the start byte \|a\X\6
\&{if} ${}(\index{KIND+\.{KIND}}\.{KIND}(\|a)\I\index{glue kind+\\{glue\_kind}}\\{glue\_kind}){}$\5
\1${}\{{}$\5
${}\index{hpos+\\{hpos}}\\{hpos}\MM;{}$\6
${}\index{hwrite ref node+\\{hwrite\_ref\_node}}\\{hwrite\_ref\_node}(\index{glue kind+\\{glue\_kind}}\\{glue\_kind},\39\index{zero skip no+\\{zero\_skip\_no}}\\{zero\_skip\_no}){}$;\5
\&{return};\5
${}\}{}$\2\6
\&{if} ${}(\index{INFO+\.{INFO}}\.{INFO}(\|a)\E\\{b000}){}$\5
\1${}\{{}$\5
\&{uint8\_t} \|n${}\K\index{HGET8+\.{HGET8}}\.{HGET8}{}$;\5
${}\index{REF+\.{REF}}\.{REF}(\index{glue kind+\\{glue\_kind}}\\{glue\_kind},\39\|n){}$;\5
${}\index{hwrite ref node+\\{hwrite\_ref\_node}}\\{hwrite\_ref\_node}(\index{glue kind+\\{glue\_kind}}\\{glue\_kind},\39\|n){}$;\5
${}\}{}$\2\6
\&{else}\5
\1${}\{{}$\5
\index{glue t+\&{glue\_t}}\&{glue\_t} \|g;\5
${}\index{HGET GLUE+\.{HGET\_GLUE}}\.{HGET\_GLUE}(\index{INFO+\.{INFO}}\.{INFO}(\|a),\39\|g){}$;\5
${}\index{hwrite glue node+\\{hwrite\_glue\_node}}\\{hwrite\_glue\_node}({\AND}\|g){}$;\5
${}\}{}$\2\6
\X15:read and check the end byte \|z\X\6
\4${}\}{}$\2
\Y
\fi

\M{129}


\putcode
\Y\B\4\X12:put functions\X${}\mathrel+\E{}$\6
\&{uint8\_t} \index{hput glue+\\{hput\_glue}}\\{hput\_glue}(\index{glue t+\&{glue\_t}}\&{glue\_t} ${}{*}\|g){}$\1\1\2\2\1\6
\4${}\{{}$\5
\index{info t+\&{info\_t}}\&{info\_t} \index{info+\\{info}}\\{info}${}\K\\{b000};{}$\7
\&{if} ${}(\index{ZERO GLUE+\.{ZERO\_GLUE}}\.{ZERO\_GLUE}({*}\|g)){}$\5
\1${}\{{}$\5
\index{HPUT8+\.{HPUT8}}\.{HPUT8}(\index{zero skip no+\\{zero\_skip\_no}}\\{zero\_skip\_no});\5
${}\index{info+\\{info}}\\{info}\K\\{b000};{}$\6
\4${}\}{}$\2\6
\&{else} \&{if} ${}((\|g\MG\|w.\|w\E\T{0}\W\|g\MG\|w.\|h\E\T{0.0}\W\|g\MG\|w.\|v\E\T{0.0})){}$\5
\1${}\{{}$\6
\&{if} ${}(\|g\MG\|p.\|f\I\T{0.0}){}$\5
\1${}\{{}$\5
${}\index{hput stretch+\\{hput\_stretch}}\\{hput\_stretch}({\AND}\|g\MG\|p){}$;\5
${}\index{info+\\{info}}\\{info}\MRL{{\OR}{\K}}\\{b010}{}$;\5
${}\}{}$\2\6
\&{if} ${}(\|g\MG\|m.\|f\I\T{0.0}){}$\5
\1${}\{{}$\5
${}\index{hput stretch+\\{hput\_stretch}}\\{hput\_stretch}({\AND}\|g\MG\|m){}$;\5
${}\index{info+\\{info}}\\{info}\MRL{{\OR}{\K}}\\{b001}{}$;\5
${}\}{}$\2\6
\4${}\}{}$\2\6
\&{else} \&{if} ${}(\|g\MG\|w.\|h\E\T{0.0}\W\|g\MG\|w.\|v\E\T{0.0}\W(\|g\MG\|p.\|f\E\T{0.0}\V\|g\MG\|m.\|f\E\T{0.0})){}$\5
\1${}\{{}$\5
${}\index{HPUT32+\.{HPUT32}}\.{HPUT32}(\|g\MG\|w.\|w){}$;\5
${}\index{info+\\{info}}\\{info}\K\\{b100};{}$\6
\&{if} ${}(\|g\MG\|p.\|f\I\T{0.0}){}$\5
\1${}\{{}$\5
${}\index{hput stretch+\\{hput\_stretch}}\\{hput\_stretch}({\AND}\|g\MG\|p){}$;\5
${}\index{info+\\{info}}\\{info}\MRL{{\OR}{\K}}\\{b010}{}$;\5
${}\}{}$\2\6
\&{if} ${}(\|g\MG\|m.\|f\I\T{0.0}){}$\5
\1${}\{{}$\5
${}\index{hput stretch+\\{hput\_stretch}}\\{hput\_stretch}({\AND}\|g\MG\|m){}$;\5
${}\index{info+\\{info}}\\{info}\MRL{{\OR}{\K}}\\{b001}{}$;\5
${}\}{}$\2\6
\4${}\}{}$\2\6
\&{else}\6
\1${}\{{}$\5
${}\index{hput xdimen node+\\{hput\_xdimen\_node}}\\{hput\_xdimen\_node}({\AND}(\|g\MG\|w));{}$\6
${}\index{hput stretch+\\{hput\_stretch}}\\{hput\_stretch}({\AND}\|g\MG\|p){}$;\5
${}\index{hput stretch+\\{hput\_stretch}}\\{hput\_stretch}({\AND}\|g\MG\|m);{}$\6
${}\index{info+\\{info}}\\{info}\K\\{b111};{}$\6
\4${}\}{}$\2\6
\&{return} \.{TAG}${}(\index{glue kind+\\{glue\_kind}}\\{glue\_kind},\39\index{info+\\{info}}\\{info});{}$\6
\4${}\}{}$\2
\Y
\fi

\M{130}

\section{Lists}\hascode\label{lists}
When a node contains multiple other nodes, we package these nodes into a list\index{list} node.
It is important to note that list nodes never occur as individual nodes,
they only occur as parts of other nodes.
In total, we have four different types of lists: plain lists that use the
kind value \index{list kind+\\{list\_kind}}\\{list\_kind}, text\index{text} lists that use the kind value \index{text kind+\\{text\_kind}}\\{text\_kind},
adjustments that use the kind value \\{adjust\_kind},
and parameter\index{parameter} lists that use the kind value \index{param kind+\\{param\_kind}}\\{param\_kind}.
A description of the first two types of lists follows here.
Adjustments\index{adjustment} are just plain lists of vertical material described in section~\secref{adjust}, and
parameter lists are described in section~\secref{paramlist}.

Because lists are of variable size, it is not possible in the short format
to tell from the kind and info bits of a tag byte the size of the list node.
So advancing from the beginning of a list node to the next node after the list is not as
simple as usual.
To solve this problem, we  store the size of the list immediately after
the start byte and before the end byte.
Alternatively we could require programs to traverse the
entire list.
The latter solution is more compact but inefficient for list with many
nodes; our solution will cost some extra bytes, but the amount
of extra bytes will only grow logarithmically with the size of the \HINT/ file.
It would be possible to allow both methods so that a \HINT/ file could balance size
and time tradeoffs by making small lists---where the size can be
determined easily by reading the entire list---without size information and making large lists
with size information so that they can be skipped easily without
reading them. But the added complexity seems too high a price to pay.


Now consider the problem of reading a content stream starting at an arbitrary
position $i$ in the middle of the stream. This situation occurs
naturally when resynchronizing\index{resynchronization} a content stream after
an error has been detected, but implementing links poses a similar problem.
We can inspect the byte at position $i$ and see
if it is a valid tag. If yes, we are faced with the problem of
verifying that this is not a mere coincidence.
So we determine the size $s$ of the node. If the byte in question is a start byte,
we should find a matching byte $s$ bytes later in the stream; if it is an end byte,
we should find the matching byte $s$ bytes earlier in the stream; if we
find no matching byte, this was neither a start nor an end byte.
If we find exactly one matching byte, we can be quite confident (error
probability 1/256 if assuming equal probability of all byte values)
that we have found a tag, and we know whether
it is the beginning or the end tag. If we find two matching byte, we
have most likely the start or the end of a node, but we do not know which
of the two. To find out which of the two possibilities is true
or to reduce the probability of an error, we can
check the start and end byte of the node immediately preceding a start byte or
immediately following an end byte in a similar way.
By testing two more byte, this additional check will reduce the error
probability further to $1/2^{24}$ (under the same assumption as before). So
checking more nodes is rarely necessary.  This whole schema
would, however, not work if we happen to find a tag byte that indicated
either the begin or the end of a list without specifying the size
of the list. Sure, we can verify the bytes before and after it to
find out whether the byte following it is the begin of a node and the
byte preceding it is the end of a node, but we still don't know if the
byte itself starts a node list or ends a node list. Even reading along
in either direction until finding a matching tag will not answer the
question. The situation is better if we specify a
size: we can read the suspected size after or before the tag and check if we
find a matching tag and size at the position indicated.
In the short format, we use the \index{info+\\{info}}\\{info} value to indicate the number of
byte used to store the list size: A list with $0<\index{info+\\{info}}\\{info}\le 5$
uses $\index{info+\\{info}}\\{info}-1$ byte to store the size.
The info value zero is reserved for references to predefined lists
(which are currently not implemented).

Storing the list size immediately preceding the end tag creates a new
problem: If we try to recover from an error, we might not know the
size of the list and searching for the end of a list, we might be
unable to tell the difference between the bytes that encode the list
size and the start tag of a possible next node.  If we parse the
content backward, the problem is completely symmetric.

To solve the problem, we insert an additional byte immediately before
the final size and after the initial size marking the size boundary.
We choose the byte values \T{\^FF}, \T{\^FE}, \T{\^FD}, and \T{\^FC} which can
not be confused with valid tag bytes and indicate that the size is
stored using 1, 2, 3, or 4 byte respectively.  Under regular
circumstances, these bytes are simply skipped.  When searching for the
list end (or start) these bytes would correspond to
$\.{TAG}(\index{penalty kind+\\{penalty\_kind}}\\{penalty\_kind},\|i)$ with $7 \ge \hbox{\|i} \ge 4$ and can not be
confused with valid penalty nodes which use only the info values 0, 1,
and~2.


We are a bit lazy when it comes to the internal representation of a list.
Since we need the representation as a short format byte sequence anyway,
it consists of the position \|p of the start of the byte sequence
combined with an integer \|s giving the size of the byte sequence.
If the list is empty, \|s is zero.

\Y\B\4\X1:hint types\X${}\mathrel+\E{}$\6
\&{typedef} \&{struct} ${}\{{}$\5
\1\index{kind t+\&{kind\_t}}\&{kind\_t} \|k;\5
\&{uint32\_t} \|p;\5
\&{uint32\_t} \|s;\5
\2${}\}{}$ \index{list t+\&{list\_t}}\&{list\_t};
\Y
\fi

\M{131}

The major drawback of his choice of representation is that it ties
together the reading of the long format and the writing of the short
format; these are no longer independent.
So starting with the present section, we have to take the short format
representation of a node into account already when we parse the long
format representation.


In the long format, we may start a list node with an
estimate\index{estimate} of the size needed to store the list in the
short format. We do not want to require the exact size because this
would make editing of long format \HINT/ files almost impossible. Of
course this makes it also impossible to derive the exact \|s value of
the internal representation from the long format
representation. Therefore we start by parsing the estimate of the list
size and use it to reserve the necessary number of byte to store the
size.  Then we parse the \index{content list+\nts{content\_list}}\nts{content\_list}. As a side effect---and this
is an important point---this will write the list content in short
format into the output buffer.  As mentioned above, whenever a node
contains a list, we need to consider this side effect when we give the
parsing rules.  We will see examples for this in
section~\secref{composite}.

The function \index{hput list+\\{hput\_list}}\\{hput\_list} will be called {\it after} the short format
of the list is written to the output.  Before we pass the internal
representation of the list to the \index{hput list+\\{hput\_list}}\\{hput\_list}
function, we update \|s and \|p. Further, we pass the position in the stream where the
list size and its boundary mark is supposed to be.
Before \index{hput list+\\{hput\_list}}\\{hput\_list} is called, space for the tag, the size, and the boundary mark
is allocated based on the estimate. The function
\index{hsize bytes+\\{hsize\_bytes}}\\{hsize\_bytes} computes the number of byte required to store the list
size, and the function \index{hput list size+\\{hput\_list\_size}}\\{hput\_list\_size} will later write the list
size.  If the estimate turns out to be wrong, the list data can be moved
to make room for a larger or smaller size field.


If the long format does not specify a size estimate, a suitable default must be chosen.
A statistical analysis shows
%
%statistics about list sizes using my old prototype
%
%name        type size_byte list_count total_size
%hello.hnt  text 1         6          748
%            text 2         2          1967
%            list 1         65         3245
%            list 2         1          352
%web2w.hnt  text 1         1043       121925
%            text 2         1344       859070
%            list 1         19780      725514
%            list 2         487        199243
%ctex.hnt   text 1         9121       4241128
%            text 2         12329      7872687
%            text 3         1          75010
%            list 1         121557     4600743
%            list 2         222        147358
%
that most plain lists need only a single byte to store the size; and even the
total amount of data contained in these lists exceeds the amount of data stored
in longer lists by a factor of about 3. Hence if we do not have an estimate,
we reserve only a single byte to store the size of a list.
The statistics looks different for lists stored as a text: The number of texts
that require two byte for the size is slightly larger than the number of texts that
need only one byte, and the total amount of data stored in these texts is larger by a factor of 2 to 7
than the total amount of data found in all other texts.
Hence as a default, we reserve two byte to store the size for texts.


\subsection{Plain Lists}\label{plainlists}
Plain list nodes start and end with a tag of kind \index{list kind+\\{list\_kind}}\\{list\_kind} or \\{adjust\_kind}.

Not uncommon are empty\index{empty list} lists; these are the only lists that can be
stored using $\index{info+\\{info}}\\{info}=1$; such a list has zero bytes of size
information, and implicitly its size is zero. The \index{info+\\{info}}\\{info} value 0
is not used since we do not use predefined plain lists.

Writing the long format uses the fact that the function
\index{hget content node+\\{hget\_content\_node}}\\{hget\_content\_node}, as implemented in the \index{stretch+\.{stretch}}\.{stretch} program, will
output the node in the long format.

\readcode
\Y\par
\par
\par
\par
\Y\B\4\X2:symbols\X${}\mathrel+\E{}$\6
\8\%\index{type+\&{type}}\&{type} $<$ \|l $>$ \index{list+\nts{list}}\nts{list} \6
\8\%\index{type+\&{type}}\&{type} $<$ \|u $>$ \index{position+\nts{position}}\nts{position}\5
\index{content list+\nts{content\_list}}\nts{content\_list}
\Y
\fi

\M{132}

\Y\B\4\X5:parsing rules\X${}\mathrel+\E{}$\6
\index{position+\nts{position}}\nts{position}: \1\1\5
${}\{{}$\1\5
${}\.{\$\$}\K\index{hpos+\\{hpos}}\\{hpos}-\index{hstart+\\{hstart}}\\{hstart};{}$\5
${}\}{}$\2;\2\2\7
\index{content list+\nts{content\_list}}\nts{content\_list}:\5
\1\1\index{position+\nts{position}}\nts{position}\5
\hbox to 0.5em{\hss${}\OR{}$}\5
\index{content list+\nts{content\_list}}\nts{content\_list}\5
\index{content node+\nts{content\_node}}\nts{content\_node};\2\2\7
\index{estimate+\nts{estimate}}\nts{estimate}: \1\1\5
${}\{{}$\1\5
${}\index{hpos+\\{hpos}}\\{hpos}\MRL{+{\K}}\T{2};{}$\5
${}\}{}$\5
\2\hbox to 0.5em{\hss${}\OR{}$}\5
\index{UNSIGNED+\ts{UNSIGNED}}\ts{UNSIGNED}\5
${}\{{}$\1\5
${}\index{hpos+\\{hpos}}\\{hpos}\MRL{+{\K}}\index{hsize bytes+\\{hsize\_bytes}}\\{hsize\_bytes}(\.{\$1})+\T{1};{}$\5
${}\}{}$\2;\2\2\7
\index{list+\nts{list}}\nts{list}: \1\1\5
\index{start+\nts{start}}\nts{start}\5
\index{estimate+\nts{estimate}}\nts{estimate}\5
\index{content list+\nts{content\_list}}\nts{content\_list}\5
\index{END+\ts{END}}\ts{END}\6
${}\{{}$\5
\1${}\.{\$\$}.\|k\K\index{list kind+\\{list\_kind}}\\{list\_kind}{}$;\5
${}\.{\$\$}.\|p\K\.{\$3}{}$;\5
${}\.{\$\$}.\|s\K(\index{hpos+\\{hpos}}\\{hpos}-\index{hstart+\\{hstart}}\\{hstart})-\.{\$3};{}$\5
${}\index{hput tags+\\{hput\_tags}}\\{hput\_tags}(\.{\$1},\39\index{hput list+\\{hput\_list}}\\{hput\_list}(\.{\$1}+\T{1},\39{\AND}(\.{\$\$}))){}$;\5
${}\}{}$\2;\2\2
\Y
\fi

\M{133}

\writecode
\Y\B\4\X19:write functions\X${}\mathrel+\E{}$\6
\&{void} \index{hwrite list+\\{hwrite\_list}}\\{hwrite\_list}(\index{list t+\&{list\_t}}\&{list\_t} ${}{*}\|l){}$\1\1\2\2\1\6
\4${}\{{}$\5
\&{uint32\_t} \|h${}\K\index{hpos+\\{hpos}}\\{hpos}-\index{hstart+\\{hstart}}\\{hstart},{}$ \|e${}\K\index{hend+\\{hend}}\\{hend}-\index{hstart+\\{hstart}}\\{hstart}{}$;\C{ save \\{hpos} and \\{hend} }\7
${}\index{hpos+\\{hpos}}\\{hpos}\K\|l\MG\|p+\index{hstart+\\{hstart}}\\{hstart}{}$;\5
${}\index{hend+\\{hend}}\\{hend}\K\index{hpos+\\{hpos}}\\{hpos}+\|l\MG\|s;{}$\6
\&{if} ${}(\|l\MG\|k\E\index{list kind+\\{list\_kind}}\\{list\_kind}){}$\1\5
\X134:write a list\X\2\6
\&{else} \&{if} ${}(\|l\MG\|k\E\index{text kind+\\{text\_kind}}\\{text\_kind}){}$\1\5
\X144:write a text\X\2\6
\&{else}\1\5
${}\.{QUIT}(\.{"List\ expected\ got\ \%}\)\.{s"},\39\index{content name+\\{content\_name}}\\{content\_name}[\|l\MG\|k]);{}$\2\6
${}\index{hpos+\\{hpos}}\\{hpos}\K\index{hstart+\\{hstart}}\\{hstart}+\|h{}$;\5
${}\index{hend+\\{hend}}\\{hend}\K\index{hstart+\\{hstart}}\\{hstart}+\|e{}$;\C{ restore  \\{hpos} and \\{hend} }\6
\4${}\}{}$\2
\Y
\fi

\M{134}

\Y\B\4\X134:write a list\X${}\E{}$\1\6
\4${}\{{}$\5
\&{if} ${}(\|l\MG\|s\E\T{0}){}$\1\5
\index{hwritef+\\{hwritef}}\\{hwritef}(\.{"\ <>"});\2\6
\&{else}\6
\1${}\{{}$\5
${}\.{DBG}(\index{DBGNODE+\.{DBGNODE}}\.{DBGNODE},\39\.{"Write\ list\ at\ 0x\%x\ }\)\.{size=\%u\\n"},\39\|l\MG\|p,\39\|l\MG\|s){}$;\5
\index{hwrite start+\\{hwrite\_start}}\\{hwrite\_start}(\,);\5
\&{if} ${}(\|l\MG\|s>\T{\^FF}){}$\1\5
${}\index{hwritef+\\{hwritef}}\\{hwritef}(\.{"\%d"},\39\|l\MG\|s);{}$\2\6
\&{while} ${}(\index{hpos+\\{hpos}}\\{hpos}<\index{hend+\\{hend}}\\{hend}){}$\1\5
\index{hget content node+\\{hget\_content\_node}}\\{hget\_content\_node}(\,);\2\6
\index{hwrite end+\\{hwrite\_end}}\\{hwrite\_end}(\,);\6
\4${}\}{}$\2\6
\4${}\}{}$\2
\U133.\Y
\fi

\M{135}
\getcode
\Y\B\4\X16:get functions\X${}\mathrel+\E{}$\6
\&{void} \index{hget size boundary+\\{hget\_size\_boundary}}\\{hget\_size\_boundary}(\index{info t+\&{info\_t}}\&{info\_t} \index{info+\\{info}}\\{info})\1\1\2\2\1\6
\4${}\{{}$\5
\&{uint32\_t} \|n;\7
\&{if} ${}(\index{info+\\{info}}\\{info}<\T{2}){}$\1\5
\&{return};\2\6
${}\|n\K\index{HGET8+\.{HGET8}}\.{HGET8};{}$\6
\&{if} ${}(\|n-\T{1}\I\T{\^100}-\index{info+\\{info}}\\{info}){}$\1\5
${}\.{QUIT}(\.{"Size\ boundary\ byte\ }\)\.{0x\%x\ with\ info\ value}\)\.{\ \%d\ at\ "}\.{SIZE\_F},\39\|n,\39\index{info+\\{info}}\\{info},\39\index{hpos+\\{hpos}}\\{hpos}-\index{hstart+\\{hstart}}\\{hstart}-\T{1});{}$\2\6
\4${}\}{}$\2\7
\&{uint32\_t} \index{hget list size+\\{hget\_list\_size}}\\{hget\_list\_size}(\index{info t+\&{info\_t}}\&{info\_t} \index{info+\\{info}}\\{info})\1\1\2\2\1\6
\4${}\{{}$\5
\&{uint32\_t} \|n;\7
\&{if} ${}(\index{info+\\{info}}\\{info}\E\T{1}){}$\1\5
\&{return} \T{0};\2\6
\&{else} \&{if} ${}(\index{info+\\{info}}\\{info}\E\T{2}){}$\1\5
${}\|n\K\index{HGET8+\.{HGET8}}\.{HGET8};{}$\2\6
\&{else} \&{if} ${}(\index{info+\\{info}}\\{info}\E\T{3}){}$\1\5
\index{HGET16+\.{HGET16}}\.{HGET16}(\|n);\2\6
\&{else} \&{if} ${}(\index{info+\\{info}}\\{info}\E\T{4}){}$\1\5
\index{HGET24+\.{HGET24}}\.{HGET24}(\|n);\2\6
\&{else} \&{if} ${}(\index{info+\\{info}}\\{info}\E\T{5}){}$\1\5
\index{HGET32+\.{HGET32}}\.{HGET32}(\|n);\2\6
\&{else}\1\5
${}\.{QUIT}(\.{"List\ info\ \%d\ must\ b}\)\.{e\ 1,\ 2,\ 3,\ 4,\ or\ 5"},\39\index{info+\\{info}}\\{info});{}$\2\6
\&{return} \|n;\6
\4${}\}{}$\2\7
\&{void} \index{hget list+\\{hget\_list}}\\{hget\_list}(\index{list t+\&{list\_t}}\&{list\_t} ${}{*}\|l){}$\1\1\2\2\1\6
\4${}\{{}$\5
\&{if} ${}(\index{KIND+\.{KIND}}\.{KIND}({*}\index{hpos+\\{hpos}}\\{hpos})\I\index{list kind+\\{list\_kind}}\\{list\_kind}\W\index{KIND+\.{KIND}}\.{KIND}({*}\index{hpos+\\{hpos}}\\{hpos})\I\\{adjust\_kind}\W{}$\6
\index{KIND+\.{KIND}}\.{KIND}${}({*}\index{hpos+\\{hpos}}\\{hpos})\I\index{text kind+\\{text\_kind}}\\{text\_kind}\W\3{-1}\index{KIND+\.{KIND}}\.{KIND}({*}\index{hpos+\\{hpos}}\\{hpos})\I\index{param kind+\\{param\_kind}}\\{param\_kind}{}$)\6
\1${}\{{}$\5
${}\|l\MG\|p\K\index{hpos+\\{hpos}}\\{hpos}-\index{hstart+\\{hstart}}\\{hstart}{}$;\5
${}\|l\MG\|s\K\T{0}{}$;\5
${}\|l\MG\|k\K\index{list kind+\\{list\_kind}}\\{list\_kind}{}$;\5
${}\}{}$\2\6
\&{else}\5
\1${}\{{}$\5
\X14:read the start byte \|a\X\6
${}\|l\MG\|k\K\index{KIND+\.{KIND}}\.{KIND}(\|a);{}$\6
${}\index{HGET LIST+\.{HGET\_LIST}}\.{HGET\_LIST}(\index{INFO+\.{INFO}}\.{INFO}(\|a),\39{*}\|l);{}$\6
\X15:read and check the end byte \|z\X\6
${}\.{DBG}(\index{DBGNODE+\.{DBGNODE}}\.{DBGNODE},\39\.{"Get\ list\ at\ 0x\%x\ si}\)\.{ze=\%u\\n"},\39\|l\MG\|p,\39\|l\MG\|s);{}$\6
\4${}\}{}$\2\6
\4${}\}{}$\2
\Y
\fi

\M{136}

\Y\B\4\X17:get macros\X${}\mathrel+\E{}$\6
\8\#\&{define} $\index{HGET LIST+\.{HGET\_LIST}}\.{HGET\_LIST}(\|I,\39\|L){}$ (\|L)${}.\|s\K\index{hget list size+\\{hget\_list\_size}}\\{hget\_list\_size}{}$ (\|I);\6
\index{hget size boundary+\\{hget\_size\_boundary}}\\{hget\_size\_boundary}(\|I);\6
${}(\|L).\|p\K\index{hpos+\\{hpos}}\\{hpos}-\index{hstart+\\{hstart}}\\{hstart};{}$\6
${}\index{hpos+\\{hpos}}\\{hpos}\K\index{hpos+\\{hpos}}\\{hpos}+(\|L).\|s;{}$\6
\index{hget size boundary+\\{hget\_size\_boundary}}\\{hget\_size\_boundary}(\|I);\1\6
\4${}\{{}$\5
\&{uint32\_t} \|s${}\K\index{hget list size+\\{hget\_list\_size}}\\{hget\_list\_size}(\|I);{}$\7
\&{if} ${}(\|s\I(\|L).\|s){}$\1\5
${}\.{QUIT}(\.{"List\ sizes\ at\ 0x\%x\ }\)\.{and\ "}\.{SIZE\_F}\.{"\ do\ not\ match\ 0x\%x\ }\)\.{!=\ 0x\%x"},\39\\{node\_pos}+\T{1},\39\index{hpos+\\{hpos}}\\{hpos}-\index{hstart+\\{hstart}}\\{hstart}-\|I-\T{1},\39(\|L).\|s,\39\|s);{}$\2\6
\4${}\}{}$\2
\Y
\fi

\M{137}

\putcode

\Y\B\4\X12:put functions\X${}\mathrel+\E{}$\6
\&{uint8\_t} \index{hsize bytes+\\{hsize\_bytes}}\\{hsize\_bytes}(\&{uint32\_t} \|n)\1\1\2\2\1\6
\4${}\{{}$\5
\&{if} ${}(\|n\E\T{0}){}$\1\5
\&{return} \T{0};\2\6
\&{else} \&{if} ${}(\|n<\T{\^100}){}$\1\5
\&{return} \T{1};\2\6
\&{else} \&{if} ${}(\|n<\T{\^10000}){}$\1\5
\&{return} \T{2};\2\6
\&{else} \&{if} ${}(\|n<\T{\^1000000}){}$\1\5
\&{return} \T{3};\2\6
\&{else}\1\5
\&{return} \T{4};\2\6
\4${}\}{}$\2\7
\&{void} \index{hput list size+\\{hput\_list\_size}}\\{hput\_list\_size}(\&{uint32\_t} \|n${},\39{}$\&{int} \|i)\1\1\2\2\1\6
\4${}\{{}$\5
\&{if} ${}(\|i\E\T{0}){}$\1\5
;\2\6
\&{else} \&{if} ${}(\|i\E\T{1}){}$\1\5
\index{HPUT8+\.{HPUT8}}\.{HPUT8}(\|n);\2\6
\&{else} \&{if} ${}(\|i\E\T{2}){}$\1\5
\index{HPUT16+\.{HPUT16}}\.{HPUT16}(\|n);\2\6
\&{else} \&{if} ${}(\|i\E\T{3}){}$\1\5
\index{HPUT24+\.{HPUT24}}\.{HPUT24}(\|n);\2\6
\&{else}\1\5
\index{HPUT32+\.{HPUT32}}\.{HPUT32}(\|n);\2\6
\4${}\}{}$\2\7
\&{uint8\_t} \index{hput list+\\{hput\_list}}\\{hput\_list}(\&{uint32\_t} \index{start pos+\\{start\_pos}}\\{start\_pos}${},\39{}$\index{list t+\&{list\_t}}\&{list\_t} ${}{*}\|l){}$\1\1\2\2\1\6
\4${}\{{}$\5
\&{if} ${}(\|l\MG\|s\E\T{0}){}$\5
\1${}\{{}$\5
${}\index{hpos+\\{hpos}}\\{hpos}\K\index{hstart+\\{hstart}}\\{hstart}+\index{start pos+\\{start\_pos}}\\{start\_pos};{}$\6
\&{return} \.{TAG}${}(\|l\MG\|k,\39\T{1}){}$;\5
${}\}{}$\2\6
\&{else}\6
\1${}\{{}$\5
\&{uint32\_t} \index{list end+\\{list\_end}}\\{list\_end}${}\K\index{hpos+\\{hpos}}\\{hpos}-\index{hstart+\\{hstart}}\\{hstart};{}$\6
\index{info t+\&{info\_t}}\&{info\_t} \|i${}\K\|l\MG\|p-\index{start pos+\\{start\_pos}}\\{start\_pos}-\T{1}{}$;\C{ number of byte allocated for size }\6
\index{info t+\&{info\_t}}\&{info\_t} \|j${}\K\index{hsize bytes+\\{hsize\_bytes}}\\{hsize\_bytes}(\|l\MG\|s){}$;\C{ number of byte needed for size }\7
${}\.{DBG}(\index{DBGNODE+\.{DBGNODE}}\.{DBGNODE},\39\.{"Put\ list\ at\ 0x\%x\ si}\)\.{ze=\%u\\n"},\39\|l\MG\|p,\39\|l\MG\|s);{}$\6
\&{if} ${}(\|i>\|j\W\|l\MG\|s>\T{\^100}){}$\1\5
${}\|j\K\|i{}$;\C{ avoid moving large lists }\2\6
\&{if} ${}(\|i\I\|j{}$)\6
\1${}\{{}$\5
\&{int} \|d${}\K\|j-\|i;{}$\7
${}\.{DBG}(\index{DBGNODE+\.{DBGNODE}}\.{DBGNODE},\39\.{"Moving\ \%u\ byte\ by\ \%}\)\.{d\\n"},\39\|l\MG\|s,\39\|d);{}$\6
${}\index{memmove+\\{memmove}}\\{memmove}(\index{hstart+\\{hstart}}\\{hstart}+\|l\MG\|p+\|d,\39\index{hstart+\\{hstart}}\\{hstart}+\|l\MG\|p,\39\|l\MG\|s);{}$\6
${}\|l\MG\|p\K\|l\MG\|p+\|d{}$;\5
${}\index{list end+\\{list\_end}}\\{list\_end}\K\index{list end+\\{list\_end}}\\{list\_end}+\|d;{}$\6
\4${}\}{}$\2\6
${}\index{hpos+\\{hpos}}\\{hpos}\K\index{hstart+\\{hstart}}\\{hstart}+\index{start pos+\\{start\_pos}}\\{start\_pos}{}$;\5
${}\index{hput list size+\\{hput\_list\_size}}\\{hput\_list\_size}(\|l\MG\|s,\39\|j){}$;\5
${}\index{HPUT8+\.{HPUT8}}\.{HPUT8}(\T{\^100}-\|j);{}$\6
${}\index{hpos+\\{hpos}}\\{hpos}\K\index{hstart+\\{hstart}}\\{hstart}+\index{list end+\\{list\_end}}\\{list\_end}{}$;\5
${}\index{HPUT8+\.{HPUT8}}\.{HPUT8}(\T{\^100}-\|j){}$;\5
${}\index{hput list size+\\{hput\_list\_size}}\\{hput\_list\_size}(\|l\MG\|s,\39\|j);{}$\6
\&{return} \.{TAG}${}(\|l\MG\|k,\39\|j+\T{1});{}$\6
\4${}\}{}$\2\6
\4${}\}{}$\2
\Y
\fi

\M{138}



\subsection{Texts}\label{text}
A Text\index{text} is a list of nodes with a representation optimized for character nodes.
In the long format, a sequence of characters like \.{Hello} is written
\.{<glyph 'H'} \.{*0>} \.{<glyph} \.{'e'} \.{*0>} \.{<glyph 'l' *0>}
\.{<glyph 'l' *0>} \.{<glyph 'o' *0>}, and even in the short
format it requires 4 byte per character! As a text, the same sequence is written  \.{"Hello"} in the
long format and the short format requires usually just 1 byte per character.
Indeed except the bytes with values from \T{\^00} to \T{\^20}, which are considered
control\index{control code} codes, all bytes and all \hbox{UTF-8}\index{UTF8} multibyte sequences
are simply considered character\index{character code} codes. They are equivalent to a glyph node
in the ``current font''. The current\index{current font} font\index{font}
is font number 0 at the beginning of a text
and it can be changed using the control codes. We introduce the concept of a ``current font''
because we do not expect the font to change too often, and it allows for a more compact
representation if we do not store the font with every character code. It has an
important disadvantage though: storing only font changes prevents us from parsing
a text backwards; we always have to start at the beginning of the text, where the
font is known to be font number 0.

Defining a second format for encoding lists of nodes adds another difficulty to the problem we had
discussed at the beginning of section~\secref{lists}. When we try to recover from an error and
start reading a content stream
at an arbitrary position, the first thing we need to find out is whether at this position we have
the tag byte of an ordinary node or whether we have a position inside a text.

Inside a text, character nodes start with a byte in the range \T{\^21}--\T{\^F7}. This is a wide range
and it overlaps considerably with the range of valid tag bytes. It is however possible to choose
the kind values in such a way that the control codes do not overlap with the valid tag bytes that
start a node. For this reason, the values $\index{text kind+\\{text\_kind}}\\{text\_kind}\E\T{0}$, $\index{list kind+\\{list\_kind}}\\{list\_kind}\E\T{1}$, $\index{param kind+\\{param\_kind}}\\{param\_kind}\E\T{2}$,
$\index{xdimen kind+\\{xdimen\_kind}}\\{xdimen\_kind}\E\T{3}$, and $\\{adjust\_kind}\E\T{4}$ were chosen on page~\pageref{kinddef}.
Texts, lists, parameter lists, and extended dimensions occur only {\it inside} of content nodes, but
are not content nodes in their own right; so the values \T{\^00} to \T{\^1F} are not used as tag bytes
of content nodes. The value \T{\^20} would, as a tag byte, indicate an adjust node ($\\{adjust\_kind}\E\T{4}$)
with info value zero. Because there are no predefined adjustments,  \T{\^20} is not used as a tag byte either.
(An alternative choice would be to use the kind value 4 for paragraph nodes because there are no
predefined paragraphs.)

The largest byte that starts an UTF8 code is \T{\^F7}; hence, there are eight possible control codes,
from \T{\^F8} to \T{\^FF}, available.
The first three values  \T{\^F8}, \T{\^F9}, and \T{\^FA} are actually used for penalty nodes
with info values, 0, 1, and 2. The last four  \T{\^FC},  \T{\^FD},  \T{\^FE}, and  \T{\^FF} are
used as boundary marks for the text size and therefore we use only \T{\^FB} as control code.

In the long format, we do not provide a syntax for specifying a size estimate\index{estimate} as we
did for plain lists, because we expect text to be quite short. We allocate two byte
for the size and hope that this will prove to be sufficient most of the time.
Further, we will disallow the use of non-printable
ASCII codes, because these are---by definition---not very readable, and we will
give special meaning to some of the printable ASCII codes because we will need
a notation for the beginning and ending of a text, for nodes inside a text,
and the control codes.

Here are the details:
\itemize

\item In the long format, a text starts and ends with a double\index{double quote} quote character ``\.{"}''.
In the short format, texts are encoded similar to lists using the kind value \index{text kind+\\{text\_kind}}\\{text\_kind}.

\item Arbitrary nodes can be embedded inside a text. In the long format, they are enclosed
in pointed brackets  \.{<} \dots \.{>} as usual. In the short format, an arbitrary node
can follow the control code $\index{txt node+\\{txt\_node}}\\{txt\_node}=\T{\^1E}$. Because text may occur in nodes, the scanner needs
to be able to parse texts nested inside nodes nested inside nodes nested inside texts \dots\ To
accomplish this, we use the ``stack'' option of \.{flex} and
include the popping and pushing the
stack in the macros \index{SCAN START+\.{SCAN\_START}}\.{SCAN\_START} and \index{SCAN END+\.{SCAN\_END}}\.{SCAN\_END}.

\item The space\index{space character} character ``\.{\ }'' with ASCII value \T{\^20} stands in both formats for the
font specific interword glue node (control code \index{txt glue+\\{txt\_glue}}\\{txt\_glue}).

\item The hyphen\index{hyphen character} character ``\.{-}'' in the long format
and the control code $\index{txt hyphen+\\{txt\_hyphen}}\\{txt\_hyphen}=\T{\^1F}$ in the short format
stand for the font specific hyphen node.

\item In the long format, the backslash\index{backslash} character ``\.{\\}'' is used as an escape character.
It is used to introduce notations for control codes, as described below, and to access
the character codes of those ASCII characters that otherwise carry a special meaning.
For example ``\.{\\"}'' denotes the character code of the double quote character ``\.{"}'';
and similarly ``\.{\\\\}'',  ``\.{\\<}'', ``\.{\\>}'', ``\.{\\\ }'', and ``\.{\\-}''
denote the character codes of ``\.{\\}'', ``\.{<}'', ``\.{>}'', ``\.{\ }'',  and ``\.{-}'' respectively.


\item In the long format, a TAB-character (ASCII code \T{\^09})\index{tab character}
is silently converted to a space\index{space character} character (ASCII code \T{\^20});
a NL-character\index{newline character} (ASCII code \T{\^0A}), together with surrounding
spaces, TAB-characters, and CR-characters\index{carriage return character}  (ASCII code \T{\^0D}), is silently converted
to a single space character.
All other ASCII characters in the range \T{\^00} to \T{\^1F}
are not allowed inside a text. This rule avoids the problems arising from ``invisible''
characters embedded in a text and it allows to break texts into lines, even with indentation\index{indentation},
at word boundaries.

To allow breaking a text into lines without inserting spaces,
a NL-character together with surrounding
spaces, TAB-characters, and CR-characters is completely ignored
if the whole group of spaces, TAB-characters, CR-characters, and the NL-character is
preceded by a backslash character.

For example, the text ``\.{"There\ is\ no\ more\ gas\ in\ the\ tank."}''\hfil\break
can be written as
\medskip

\qquad\vbox{\hsize=0.5\hsize\noindent
\.{"There\ is\ }\hfil\break
\.{\hbox to 2em {$\rightarrow$\hfill}no more g\\\ \ }\hfil\break
\.{\hbox to 2em {$\rightarrow$\hfill}as in the tank."}
}\hss

To break long lines when writing a long format file, we use the variable \index{txt length+\\{txt\_length}}\\{txt\_length}
to keep track of the approximate length of the current line.

\item The control codes  $\index{txt font+\\{txt\_font}}\\{txt\_font}=\T{\^00}$, \T{\^01}, \T{\^02}, \dots, and \T{\^07} are used to
change the current font to
font number 0, 1, 2, \dots, and 7. In the long format these control codes are written
\.{\\0}, \.{\\1}, \.{\\2}, \dots, and \.{\\7}.

\item The control code $\index{txt global+\\{txt\_global}}\\{txt\_global}=\T{\^08}$ is followed by a second parameter byte. If the value
of the parameter byte is $n$, it will set the current font to font number $n$.
In the long format, the two byte sequence is written
``\.{\\F}$n$\.{\\}''  where $n$ is the decimal representation of the font number.


\item The control codes \T{\^09}, \T{\^0A}, \T{\^0B}, \T{\^0C}, \T{\^0E}, \T{\^0E}, \T{\^0F}, and \T{\^10}
are also followed by a second parameter byte. They are used to reference
the global definitions of penalty\index{penalty}, kern\index{kern}, ligature\index{ligature}, hyphen\index{hyphen}, glue\index{glue}, language\index{language}, rule\index{rule}, and image\index{image} nodes.
The parameter byte contains the reference number.
For example, the byte sequence  \T{\^09} \T{\^03}  is equivalent to the node \.{<penalty *3>}.
In the long format these two-byte sequences are written,
``\.{\\P}$n$\.{\\}'' (penalty),
``\.{\\K}$n$\.{\\}'' (kern),
``\.{\\L}$n$\.{\\}'' (ligature),
``\.{\\H}$n$\.{\\}'' (hyphen),
``\.{\\G}$n$\.{\\}'' (glue),
``\.{\\S}$n$\.{\\}'' (speak or german Sprache),
``\.{\\R}$n$\.{\\}'' (rule), and
``\.{\\I}$n$\.{\\}'' (image), where $n$ is the decimal representation of the parameter value.


\item The control codes from  $\index{txt local+\\{txt\_local}}\\{txt\_local}=\T{\^11}$ to \T{\^1C} are used to reference
one of the 12 font specific parameters\index{font parameter}. In the long format they are
written ``\.{\\a}'', ``\.{\\b}'', ``\.{\\c}'', \dots,  ``\.{\\j}'', ``\.{\\k}'', ``\.{\\l}''.


\item  The control code $\index{txt cc+\\{txt\_cc}}\\{txt\_cc}=\T{\^1D}$ is used as a prefix for an arbitrary
character code represented as an UTF-8 multibyte sequence.
Its main purpose is providing a method for including character codes
less or equal to \T{\^20} which otherwise would be considered control
codes.  In the long format, the byte sequence is written
``\.{\\C}$n$\.{\\}'' where $n$ is the decimal representation of the
character code.


\item The control code $\index{txt node+\\{txt\_node}}\\{txt\_node}=\T{\^1E}$ is used as a prefix for an arbitrary node in short format.
In the long format, it is  written ``\.{<}'' and is followed by the node content
in long format terminated by  ``\.{>}''.

\item The control code $\index{txt hyphen+\\{txt\_hyphen}}\\{txt\_hyphen}=\T{\^1F}$  is used to access the font specific discretionary hyphen\index{hyphen}.
In the long format it is simply written as ``\.{-}''.

\item The control code $\index{txt glue+\\{txt\_glue}}\\{txt\_glue}=\T{\^20}$ is the space character, it is used to access the font specific
interword\index{interword glue} glue. In the long format, we use the space character\index{space character} ``\.{\ }'' as well.

\item The control code $\index{txt ignore+\\{txt\_ignore}}\\{txt\_ignore}=\T{\^FB}$ is ignored, its position can be used in a link to specify a position
between two characters. In the long format it is written as ``\.{\\@}''.

\enditemize
For the control codes, we define an enumeration type and for references, a reference type.
\Y\B\4\X1:hint types\X${}\mathrel+\E{}$\6
\&{typedef} \&{enum} ${}\{{}$\5
\1${}\index{txt font+\\{txt\_font}}\\{txt\_font}\K\T{\^00},\39\index{txt global+\\{txt\_global}}\\{txt\_global}\K\T{\^08},\39\index{txt local+\\{txt\_local}}\\{txt\_local}\K\T{\^11},\39\index{txt cc+\\{txt\_cc}}\\{txt\_cc}\K\T{\^1D},\39\index{txt node+\\{txt\_node}}\\{txt\_node}\K\T{\^1E},\39\index{txt hyphen+\\{txt\_hyphen}}\\{txt\_hyphen}\K\T{\^1F},\39\index{txt glue+\\{txt\_glue}}\\{txt\_glue}\K\T{\^20},\39\index{txt ignore+\\{txt\_ignore}}\\{txt\_ignore}\K\T{\^FB}{}$\2\6
${}\}{}$ \index{txt t+\&{txt\_t}}\&{txt\_t};
\Y
\fi

\M{139}

\readcode
\Y\par
\par
\par
\par
\par
\par
\par
\par
\par
\par
\par
\Y\B\4\X21:scanning definitions\X${}\mathrel+\E{}$\6
\8\%\&{x} \index{TXT+\ts{TXT}}\ts{TXT}
\Y
\fi

\M{140}

\Y\B\4\X2:symbols\X${}\mathrel+\E{}$\6
\8\%\&{token} \index{TXT START+\ts{TXT\_START}}\ts{TXT\_START} \index{TXT END+\ts{TXT\_END}}\ts{TXT\_END}\5\index{TXT IGNORE+\ts{TXT\_IGNORE}}\ts{TXT\_IGNORE}\6
\8\%\&{token} \index{TXT FONT GLUE+\ts{TXT\_FONT\_GLUE}}\ts{TXT\_FONT\_GLUE}\5\index{TXT FONT HYPHEN+\ts{TXT\_FONT\_HYPHEN}}\ts{TXT\_FONT\_HYPHEN}\6
\8\%\&{token} $<$ \|u $>$ \index{TXT FONT+\ts{TXT\_FONT}}\ts{TXT\_FONT}\5
\index{TXT LOCAL+\ts{TXT\_LOCAL}}\ts{TXT\_LOCAL} \6
\8\%\&{token} $<$ \index{rf+\\{rf}}\\{rf} $>$ \index{TXT GLOBAL+\ts{TXT\_GLOBAL}}\ts{TXT\_GLOBAL} \6
\8\%\&{token} $<$ \|u $>$ \index{TXT CC+\ts{TXT\_CC}}\ts{TXT\_CC} \6
\8\%\index{type+\&{type}}\&{type} $<$ \|u $>$ \index{text+\nts{text}}\nts{text}
\Y
\fi

\M{141}

\Y\B\4\X3:scanning rules\X${}\mathrel+\E{}$\6
${}\8\re{\vb{\\"}}{}$\ac\index{SCAN TXT START+\.{SCAN\_TXT\_START}}\.{SCAN\_TXT\_START};\5
\&{return} \index{TXT START+\ts{TXT\_START}}\ts{TXT\_START};\eac\7
\8\re{${}<{}$\ts{TXT}${}>{}\{$}\6
${}\8\re{\vb{\\"}}{}$\ac\.{SCAN\_TXT\_END};\5
\&{return} \ts{TXT\_END};\eac\7
${}\8\re{\vb{"<"}}{}$\ac\.{SCAN\_START};\5
\&{return} \ts{START};\eac\7
${}\8\re{\vb{">"}}{}$\ac\.{QUIT}(\.{">\ not\ allowed\ in\ te}\)\.{xt\ mode"});\eac\7
${}\8\re{\vb{\\\\\\\\}}{}$\ac${}\\{yylval}.\|u\K\.{'\\\\'};{}$\5
\&{return} \ts{TXT\_CC};\eac\7
${}\8\re{\vb{\\\\\\"}}{}$\ac${}\\{yylval}.\|u\K\.{'"'};{}$\5
\&{return} \ts{TXT\_CC};\eac\7
${}\8\re{\vb{\\\\"<"}}{}$\ac${}\\{yylval}.\|u\K\.{'<'};{}$\5
\&{return} \ts{TXT\_CC};\eac\7
${}\8\re{\vb{\\\\">"}}{}$\ac${}\\{yylval}.\|u\K\.{'>'};{}$\5
\&{return} \ts{TXT\_CC};\eac\7
${}\8\re{\vb{\\\\"\ "}}{}$\ac${}\\{yylval}.\|u\K\.{'\ '};{}$\5
\&{return} \ts{TXT\_CC};\eac\7
${}\8\re{\vb{\\\\"-"}}{}$\ac${}\\{yylval}.\|u\K\.{'-'};{}$\5
\&{return} \ts{TXT\_CC};\eac\7
${}\8\re{\vb{\\\\"@"}}{}$\ac\&{return} \ts{TXT\_IGNORE};\eac\7
${}\8\re{\vb{[\ \\t\\r]*(\\n[\ \\t\\r]*)+}}{}$\ac\&{return} \ts{TXT\_FONT\_GLUE};\eac\7
${}\8\re{\vb{\\\\[\ \\t\\r]*\\n[\ \\t\\r]*}}{}$\ac ;\eac\7
${}\8\re{\vb{\\\\[0-7]}}{}$\ac $\\{yylval}.\|u\K\\{yytext}[\T{1}]-\.{'0'};{}$\5
\&{return} \ts{TXT\_FONT};\6
${}\8\re{\vb{\\\\F[0-9]+\\\\}}{}$\ac\.{SCAN\_REF}(\\{font\_kind});\5
\&{return} \ts{TXT\_GLOBAL};\eac\7
${}\8\re{\vb{\\\\P[0-9]+\\\\}}{}$\ac\.{SCAN\_REF}(\\{penalty\_kind});\5
\&{return} \ts{TXT\_GLOBAL};\eac\7
${}\8\re{\vb{\\\\K[0-9]+\\\\}}{}$\ac\.{SCAN\_REF}(\\{kern\_kind});\5
\&{return} \ts{TXT\_GLOBAL};\eac\7
${}\8\re{\vb{\\\\L[0-9]+\\\\}}{}$\ac\.{SCAN\_REF}(\\{ligature\_kind});\5
\&{return} \ts{TXT\_GLOBAL};\eac\7
${}\8\re{\vb{\\\\H[0-9]+\\\\}}{}$\ac\.{SCAN\_REF}(\\{hyphen\_kind});\5
\&{return} \ts{TXT\_GLOBAL};\eac\7
${}\8\re{\vb{\\\\G[0-9]+\\\\}}{}$\ac\.{SCAN\_REF}(\\{glue\_kind});\5
\&{return} \ts{TXT\_GLOBAL};\eac\7
${}\8\re{\vb{\\\\S[0-9]+\\\\}}{}$\ac\.{SCAN\_REF}(\\{language\_kind});\5
\&{return} \ts{TXT\_GLOBAL};\eac\7
${}\8\re{\vb{\\\\R[0-9]+\\\\}}{}$\ac\.{SCAN\_REF}(\\{rule\_kind});\5
\&{return} \ts{TXT\_GLOBAL};\eac\7
${}\8\re{\vb{\\\\I[0-9]+\\\\}}{}$\ac\.{SCAN\_REF}(\\{image\_kind});\5
\&{return} \ts{TXT\_GLOBAL};\eac\7
${}\8\re{\vb{\\\\C[0-9]+\\\\}}{}$\ac${}\.{SCAN\_UDEC}(\\{yytext}+\T{2});{}$\5
\&{return} \ts{TXT\_CC};\eac\7
${}\8\re{\vb{\\\\[a-l]}}{}$\ac${}\\{yylval}.\|u\K\\{yytext}[\T{1}]-\.{'a'};{}$\5
\&{return} \ts{TXT\_LOCAL};\eac\7
${}\8\re{\vb{"\ "}}{}$\ac\&{return} \ts{TXT\_FONT\_GLUE};\eac\7
${}\8\re{\vb{"-"}}{}$\ac\&{return} \ts{TXT\_FONT\_HYPHEN};\eac\7
${}\8\re{\vb{\{UTF8\_1\}}}{}$\ac\.{SCAN\_UTF8\_1}(\\{yytext});\5
\&{return} \ts{TXT\_CC};\eac\7
${}\8\re{\vb{\{UTF8\_2\}}}{}$\ac\.{SCAN\_UTF8\_2}(\\{yytext});\5
\&{return} \ts{TXT\_CC};\eac\7
${}\8\re{\vb{\{UTF8\_3\}}}{}$\ac\.{SCAN\_UTF8\_3}(\\{yytext});\5
\&{return} \ts{TXT\_CC};\eac\7
${}\8\re{\vb{\{UTF8\_4\}}}{}$\ac\.{SCAN\_UTF8\_4}(\\{yytext});\5
\&{return} \ts{TXT\_CC};\eac\7
${}\}{}$
\Y
\fi

\M{142}

\Y\B\4\X20:scanning macros\X${}\mathrel+\E{}$\6
\8\#\&{define} \index{SCAN REF+\.{SCAN\_REF}}\.{SCAN\_REF}(\|K)\5${}\index{yylval+\\{yylval}}\\{yylval}.\index{rf+\\{rf}}\\{rf}.\|k\K\|K{}$;\5
${}\index{yylval+\\{yylval}}\\{yylval}.\index{rf+\\{rf}}\\{rf}.\|n\K\index{atoi+\\{atoi}}\\{atoi}(\index{yytext+\\{yytext}}\\{yytext}+\T{2}){}$\6
\&{static} \&{int} \index{scan level+\\{scan\_level}}\\{scan\_level}${}\K\T{0};{}$\6
\8\#\&{define} \index{SCAN START+\.{SCAN\_START}}\.{SCAN\_START}\5\index{yy push state+\\{yy\_push\_state}}\\{yy\_push\_state}(\index{INITIAL+\ts{INITIAL}}\ts{INITIAL});\5
${}\index{scan level+\\{scan\_level}}\\{scan\_level}\PP;{}$\6
\8\#\&{define} \index{SCAN END+\.{SCAN\_END}}\.{SCAN\_END}\5\6
\&{if} ${}(\index{scan level+\\{scan\_level}}\\{scan\_level}\MM){}$\1\5
\index{yy pop state+\\{yy\_pop\_state}}\\{yy\_pop\_state}(\,);\2\6
\&{else}${}\.{QUIT}(\.{"Too\ many\ '>'\ in\ lin}\)\.{e\ \%d"},\39\index{yylineno+\\{yylineno}}\\{yylineno}){}$\6
\8\#\&{define} \index{SCAN TXT START+\.{SCAN\_TXT\_START}}\.{SCAN\_TXT\_START}\5\.{BEGIN}(\index{TXT+\ts{TXT}}\ts{TXT})\6
\8\#\&{define} \index{SCAN TXT END+\.{SCAN\_TXT\_END}}\.{SCAN\_TXT\_END}\5\.{BEGIN}(\index{INITIAL+\ts{INITIAL}}\ts{INITIAL})
\Y
\fi

\M{143}
\Y\par
\Y\B\4\X5:parsing rules\X${}\mathrel+\E{}$\6
\index{list+\nts{list}}\nts{list}: \1\1\5
\index{TXT START+\ts{TXT\_START}}\ts{TXT\_START}\5
\index{position+\nts{position}}\nts{position}\3{-1}\5
${}\{{}$\1\5
${}\index{hpos+\\{hpos}}\\{hpos}\MRL{+{\K}}\T{4}{}$;\C{ start byte, two size byte, and boundary byte }\6
${}\}{}$\2\5
\index{text+\nts{text}}\nts{text}\5
\index{TXT END+\ts{TXT\_END}}\ts{TXT\_END}\3{-1}\5
${}\{{}$\1\5
${}\.{\$\$}.\|k\K\index{text kind+\\{text\_kind}}\\{text\_kind};{}$\5
${}\.{\$\$}.\|p\K\.{\$4};{}$\5
${}\.{\$\$}.\|s\K(\index{hpos+\\{hpos}}\\{hpos}-\index{hstart+\\{hstart}}\\{hstart})-\.{\$4};{}$\5
${}\index{hput tags+\\{hput\_tags}}\\{hput\_tags}(\.{\$2},\39\index{hput list+\\{hput\_list}}\\{hput\_list}(\.{\$2}+\T{1},\39{\AND}(\.{\$\$}))){}$;\5
${}\}{}$\2;\2\2\7
\index{text+\nts{text}}\nts{text}: \1\1\5
\index{position+\nts{position}}\nts{position}\5
\hbox to 0.5em{\hss${}\OR{}$}\5
\index{text+\nts{text}}\nts{text}\5
\index{txt+\nts{txt}}\nts{txt};\2\2\7
\index{txt+\nts{txt}}\nts{txt}: \1\1\5
\index{TXT CC+\ts{TXT\_CC}}\ts{TXT\_CC}\5
${}\{{}$\1\5
\index{hput txt cc+\\{hput\_txt\_cc}}\\{hput\_txt\_cc}(\.{\$1});\5
${}\}{}$\2\6
\4\hbox to 0.5em{\hss${}\OR{}$}\5
\index{TXT FONT+\ts{TXT\_FONT}}\ts{TXT\_FONT}\5
${}\{{}$\1\5
${}\index{REF+\.{REF}}\.{REF}(\index{font kind+\\{font\_kind}}\\{font\_kind},\39\.{\$1});{}$\5
\index{hput txt font+\\{hput\_txt\_font}}\\{hput\_txt\_font}(\.{\$1});\5
${}\}{}$\2\6
\4\hbox to 0.5em{\hss${}\OR{}$}\5
\index{TXT GLOBAL+\ts{TXT\_GLOBAL}}\ts{TXT\_GLOBAL}\5
${}\{{}$\1\5
${}\index{REF+\.{REF}}\.{REF}(\.{\$1}.\|k,\39\.{\$1}.\|n);{}$\5
${}\index{hput txt global+\\{hput\_txt\_global}}\\{hput\_txt\_global}({\AND}(\.{\$1}));{}$\5
${}\}{}$\2\6
\4\hbox to 0.5em{\hss${}\OR{}$}\5
\index{TXT LOCAL+\ts{TXT\_LOCAL}}\ts{TXT\_LOCAL}\5
${}\{{}$\1\5
${}\.{RNG}(\.{"Font\ parameter"},\39\.{\$1},\39\T{0},\39\T{11});{}$\5
\index{hput txt local+\\{hput\_txt\_local}}\\{hput\_txt\_local}(\.{\$1});\5
${}\}{}$\2\6
\4\hbox to 0.5em{\hss${}\OR{}$}\5
\index{TXT FONT GLUE+\ts{TXT\_FONT\_GLUE}}\ts{TXT\_FONT\_GLUE}\5
${}\{{}$\1\5
\index{HPUTX+\.{HPUTX}}\.{HPUTX}(\T{1});\5
\index{HPUT8+\.{HPUT8}}\.{HPUT8}(\index{txt glue+\\{txt\_glue}}\\{txt\_glue});\5
${}\}{}$\2\6
\4\hbox to 0.5em{\hss${}\OR{}$}\5
\index{TXT FONT HYPHEN+\ts{TXT\_FONT\_HYPHEN}}\ts{TXT\_FONT\_HYPHEN}\5
${}\{{}$\1\5
\index{HPUTX+\.{HPUTX}}\.{HPUTX}(\T{1});\5
\index{HPUT8+\.{HPUT8}}\.{HPUT8}(\index{txt hyphen+\\{txt\_hyphen}}\\{txt\_hyphen});\5
${}\}{}$\2\6
\4\hbox to 0.5em{\hss${}\OR{}$}\5
\index{TXT IGNORE+\ts{TXT\_IGNORE}}\ts{TXT\_IGNORE}\5
${}\{{}$\1\5
\index{HPUTX+\.{HPUTX}}\.{HPUTX}(\T{1});\5
\index{HPUT8+\.{HPUT8}}\.{HPUT8}(\index{txt ignore+\\{txt\_ignore}}\\{txt\_ignore});\5
${}\}{}$\2\6
\4\hbox to 0.5em{\hss${}\OR{}$}\5
${}\{{}$\1\5
\index{HPUTX+\.{HPUTX}}\.{HPUTX}(\T{1});\5
\index{HPUT8+\.{HPUT8}}\.{HPUT8}(\index{txt node+\\{txt\_node}}\\{txt\_node});\5
${}\}{}$\2\5
\index{content node+\nts{content\_node}}\nts{content\_node};\2\2
\Y
\fi

\M{144}

The following function keeps track of the position in the current line.
It the line gets too long it will break the text at the next space
character. If no suitable space character comes along,
the line will be broken after any regular character.

\writecode
\Y\B\4\X144:write a text\X${}\E{}$\1\6
\4${}\{{}$\5
\&{if} ${}(\|l\MG\|s\E\T{0}){}$\1\5
\index{hwritef+\\{hwritef}}\\{hwritef}(\.{"\ \\"\\""});\2\6
\&{else}\6
\1${}\{{}$\5
\&{int} \index{pos+\\{pos}}\\{pos}${}\K\index{nesting+\\{nesting}}\\{nesting}+\T{20}{}$;\C{ estimate }\7
\index{hwritef+\\{hwritef}}\\{hwritef}(\.{"\ \\""});\6
\&{while} ${}(\index{hpos+\\{hpos}}\\{hpos}<\index{hend+\\{hend}}\\{hend}{}$)\6
\1${}\{{}$\5
\&{int} \|i${}\K\index{hget txt+\\{hget\_txt}}\\{hget\_txt}(\,);{}$\7
\&{if} ${}(\|i<\T{0}){}$\5
\1${}\{{}$\6
\&{if} ${}(\index{pos+\\{pos}}\\{pos}\PP<\T{70}){}$\1\5
\index{hwritec+\\{hwritec}}\\{hwritec}(\.{'\ '});\2\6
\&{else}\1\5
${}\index{hwrite nesting+\\{hwrite\_nesting}}\\{hwrite\_nesting}(\,),\39\index{pos+\\{pos}}\\{pos}\K\index{nesting+\\{nesting}}\\{nesting};{}$\2\6
\4${}\}{}$\2\6
\&{else} \&{if} ${}(\|i\E\T{1}\W\index{pos+\\{pos}}\\{pos}\G\T{100}{}$)\6
\1${}\{{}$\5
\index{hwritec+\\{hwritec}}\\{hwritec}(\.{'\\\\'});\5
\index{hwrite nesting+\\{hwrite\_nesting}}\\{hwrite\_nesting}(\,);\5
${}\index{pos+\\{pos}}\\{pos}\K\index{nesting+\\{nesting}}\\{nesting}{}$;\5
${}\}{}$\2\6
\&{else}\1\5
${}\index{pos+\\{pos}}\\{pos}\MRL{+{\K}}\|i;{}$\2\6
\4${}\}{}$\2\6
\index{hwritec+\\{hwritec}}\\{hwritec}(\.{'"'});\6
\4${}\}{}$\2\6
\4${}\}{}$\2
\U133.\Y
\fi

\M{145}


The function returns the number of characters written
because this information is needed in \index{hget txt+\\{hget\_txt}}\\{hget\_txt} below.

\Y\B\4\X19:write functions\X${}\mathrel+\E{}$\6
\&{int} \index{hwrite txt cc+\\{hwrite\_txt\_cc}}\\{hwrite\_txt\_cc}(\&{uint32\_t} \|c)\1\1\2\2\1\6
\4${}\{{}$\5
\&{if} ${}(\|c<\T{\^20}){}$\1\5
\&{return} \index{hwritef+\\{hwritef}}\\{hwritef}${}(\.{"\\\\C\%d\\\\"},\39\|c);{}$\2\6
\&{else}\5
\1\&{switch} (\|c)\5
\1${}\{{}$\6
\4\&{case} \.{'\\\\'}:\5
\&{return} \index{hwritef+\\{hwritef}}\\{hwritef}(\.{"\\\\\\\\"});\6
\4\&{case} \.{'"'}:\5
\&{return} \index{hwritef+\\{hwritef}}\\{hwritef}(\.{"\\\\\\""});\6
\4\&{case} \.{'<'}:\5
\&{return} \index{hwritef+\\{hwritef}}\\{hwritef}(\.{"\\\\<"});\6
\4\&{case} \.{'>'}:\5
\&{return} \index{hwritef+\\{hwritef}}\\{hwritef}(\.{"\\\\>"});\6
\4\&{case} \.{'\ '}:\5
\&{return} \index{hwritef+\\{hwritef}}\\{hwritef}(\.{"\\\\\ "});\6
\4\&{case} \.{'-'}:\5
\&{return} \index{hwritef+\\{hwritef}}\\{hwritef}(\.{"\\\\-"});\6
\4\&{default}:\6
\&{if} (\index{option utf8+\\{option\_utf8}}\\{option\_utf8})\1\5
\&{return} \index{hwrite utf8+\\{hwrite\_utf8}}\\{hwrite\_utf8}(\|c);\2\6
\&{else}\1\5
\&{return} \index{hwritef+\\{hwritef}}\\{hwritef}${}(\.{"\\\\C\%d\\\\"},\39\|c);{}$\2\6
\4${}\}{}$\2\2\6
\4${}\}{}$\2
\Y
\fi

\M{146}

\getcode
\Y\B\4\X17:get macros\X${}\mathrel+\E{}$\6
\8\#\&{define} ${}\index{HGET GREF+\.{HGET\_GREF}}\.{HGET\_GREF}(\|K,\39\|S){}$\1\1\2\2\1\6
\4${}\{{}$\5
\&{uint8\_t} \|n${}\K\index{HGET8+\.{HGET8}}\.{HGET8}{}$;\5
${}\index{REF+\.{REF}}\.{REF}(\|K,\39\|n){}$;\5
\&{return} \index{hwritef+\\{hwritef}}\\{hwritef}${}(\.{"\\\\"}\|S\.{"\%d\\\\"},\39\|n){}$;\5
${}\}{}$\2
\Y
\fi

\M{147}

The function \index{hget txt+\\{hget\_txt}}\\{hget\_txt} reads a text element and writes it immediately.
To enable the insertion of line breaks when writing a text, we need to keep track
of the number of characters in the current line. For this purpose
the function \index{hget txt+\\{hget\_txt}}\\{hget\_txt} returns the number of characters written.
It returns $-1$ if a space character needs to be written
providing a good opportunity for a break.

\Y\B\4\X16:get functions\X${}\mathrel+\E{}$\6
\&{int} \index{hget txt+\\{hget\_txt}}\\{hget\_txt}(\&{void})\1\1\2\2\1\6
\4${}\{{}$\5
\&{if} ${}({*}\index{hpos+\\{hpos}}\\{hpos}\G\T{\^80}\W{*}\index{hpos+\\{hpos}}\\{hpos}\Z\T{\^F7}){}$\5
\1${}\{{}$\6
\&{if} (\index{option utf8+\\{option\_utf8}}\\{option\_utf8})\1\5
\&{return} \index{hwrite utf8+\\{hwrite\_utf8}}\\{hwrite\_utf8}(\index{hget utf8+\\{hget\_utf8}}\\{hget\_utf8}(\,));\2\6
\&{else}\1\5
\&{return} \index{hwritef+\\{hwritef}}\\{hwritef}${}(\.{"\\\\C\%d\\\\"},\39\index{hget utf8+\\{hget\_utf8}}\\{hget\_utf8}(\,));{}$\2\6
\4${}\}{}$\2\6
\&{else}\6
\1${}\{{}$\5
\&{uint8\_t} \|a;\7
${}\|a\K\index{HGET8+\.{HGET8}}\.{HGET8};{}$\6
\&{switch} (\|a)\5
\1${}\{{}$\6
\4\&{case} \index{txt font+\\{txt\_font}}\\{txt\_font}${}+\T{0}{}$:\5
\&{return} \index{hwritef+\\{hwritef}}\\{hwritef}(\.{"\\\\0"});\6
\4\&{case} \index{txt font+\\{txt\_font}}\\{txt\_font}${}+\T{1}{}$:\5
\&{return} \index{hwritef+\\{hwritef}}\\{hwritef}(\.{"\\\\1"});\6
\4\&{case} \index{txt font+\\{txt\_font}}\\{txt\_font}${}+\T{2}{}$:\5
\&{return} \index{hwritef+\\{hwritef}}\\{hwritef}(\.{"\\\\2"});\6
\4\&{case} \index{txt font+\\{txt\_font}}\\{txt\_font}${}+\T{3}{}$:\5
\&{return} \index{hwritef+\\{hwritef}}\\{hwritef}(\.{"\\\\3"});\6
\4\&{case} \index{txt font+\\{txt\_font}}\\{txt\_font}${}+\T{4}{}$:\5
\&{return} \index{hwritef+\\{hwritef}}\\{hwritef}(\.{"\\\\4"});\6
\4\&{case} \index{txt font+\\{txt\_font}}\\{txt\_font}${}+\T{5}{}$:\5
\&{return} \index{hwritef+\\{hwritef}}\\{hwritef}(\.{"\\\\5"});\6
\4\&{case} \index{txt font+\\{txt\_font}}\\{txt\_font}${}+\T{6}{}$:\5
\&{return} \index{hwritef+\\{hwritef}}\\{hwritef}(\.{"\\\\6"});\6
\4\&{case} \index{txt font+\\{txt\_font}}\\{txt\_font}${}+\T{7}{}$:\5
\&{return} \index{hwritef+\\{hwritef}}\\{hwritef}(\.{"\\\\7"});\6
\4\&{case} \index{txt global+\\{txt\_global}}\\{txt\_global}${}+\T{0}{}$:\5
${}\index{HGET GREF+\.{HGET\_GREF}}\.{HGET\_GREF}(\index{font kind+\\{font\_kind}}\\{font\_kind},\39\.{"F"});{}$\6
\4\&{case} \index{txt global+\\{txt\_global}}\\{txt\_global}${}+\T{1}{}$:\5
${}\index{HGET GREF+\.{HGET\_GREF}}\.{HGET\_GREF}(\index{penalty kind+\\{penalty\_kind}}\\{penalty\_kind},\39\.{"P"});{}$\6
\4\&{case} \index{txt global+\\{txt\_global}}\\{txt\_global}${}+\T{2}{}$:\5
${}\index{HGET GREF+\.{HGET\_GREF}}\.{HGET\_GREF}(\index{kern kind+\\{kern\_kind}}\\{kern\_kind},\39\.{"K"});{}$\6
\4\&{case} \index{txt global+\\{txt\_global}}\\{txt\_global}${}+\T{3}{}$:\5
${}\index{HGET GREF+\.{HGET\_GREF}}\.{HGET\_GREF}(\index{ligature kind+\\{ligature\_kind}}\\{ligature\_kind},\39\.{"L"});{}$\6
\4\&{case} \index{txt global+\\{txt\_global}}\\{txt\_global}${}+\T{4}{}$:\5
${}\index{HGET GREF+\.{HGET\_GREF}}\.{HGET\_GREF}(\index{hyphen kind+\\{hyphen\_kind}}\\{hyphen\_kind},\39\.{"H"});{}$\6
\4\&{case} \index{txt global+\\{txt\_global}}\\{txt\_global}${}+\T{5}{}$:\5
${}\index{HGET GREF+\.{HGET\_GREF}}\.{HGET\_GREF}(\index{glue kind+\\{glue\_kind}}\\{glue\_kind},\39\.{"G"});{}$\6
\4\&{case} \index{txt global+\\{txt\_global}}\\{txt\_global}${}+\T{6}{}$:\5
${}\index{HGET GREF+\.{HGET\_GREF}}\.{HGET\_GREF}(\index{language kind+\\{language\_kind}}\\{language\_kind},\39\.{"S"});{}$\6
\4\&{case} \index{txt global+\\{txt\_global}}\\{txt\_global}${}+\T{7}{}$:\5
${}\index{HGET GREF+\.{HGET\_GREF}}\.{HGET\_GREF}(\index{rule kind+\\{rule\_kind}}\\{rule\_kind},\39\.{"R"});{}$\6
\4\&{case} \index{txt global+\\{txt\_global}}\\{txt\_global}${}+\T{8}{}$:\5
${}\index{HGET GREF+\.{HGET\_GREF}}\.{HGET\_GREF}(\index{image kind+\\{image\_kind}}\\{image\_kind},\39\.{"I"});{}$\6
\4\&{case} \index{txt local+\\{txt\_local}}\\{txt\_local}${}+\T{0}{}$:\5
\&{return} \index{hwritef+\\{hwritef}}\\{hwritef}(\.{"\\\\a"});\6
\4\&{case} \index{txt local+\\{txt\_local}}\\{txt\_local}${}+\T{1}{}$:\5
\&{return} \index{hwritef+\\{hwritef}}\\{hwritef}(\.{"\\\\b"});\6
\4\&{case} \index{txt local+\\{txt\_local}}\\{txt\_local}${}+\T{2}{}$:\5
\&{return} \index{hwritef+\\{hwritef}}\\{hwritef}(\.{"\\\\c"});\6
\4\&{case} \index{txt local+\\{txt\_local}}\\{txt\_local}${}+\T{3}{}$:\5
\&{return} \index{hwritef+\\{hwritef}}\\{hwritef}(\.{"\\\\d"});\6
\4\&{case} \index{txt local+\\{txt\_local}}\\{txt\_local}${}+\T{4}{}$:\5
\&{return} \index{hwritef+\\{hwritef}}\\{hwritef}(\.{"\\\\e"});\6
\4\&{case} \index{txt local+\\{txt\_local}}\\{txt\_local}${}+\T{5}{}$:\5
\&{return} \index{hwritef+\\{hwritef}}\\{hwritef}(\.{"\\\\f"});\6
\4\&{case} \index{txt local+\\{txt\_local}}\\{txt\_local}${}+\T{6}{}$:\5
\&{return} \index{hwritef+\\{hwritef}}\\{hwritef}(\.{"\\\\g"});\6
\4\&{case} \index{txt local+\\{txt\_local}}\\{txt\_local}${}+\T{7}{}$:\5
\&{return} \index{hwritef+\\{hwritef}}\\{hwritef}(\.{"\\\\h"});\6
\4\&{case} \index{txt local+\\{txt\_local}}\\{txt\_local}${}+\T{8}{}$:\5
\&{return} \index{hwritef+\\{hwritef}}\\{hwritef}(\.{"\\\\i"});\6
\4\&{case} \index{txt local+\\{txt\_local}}\\{txt\_local}${}+\T{9}{}$:\5
\&{return} \index{hwritef+\\{hwritef}}\\{hwritef}(\.{"\\\\j"});\6
\4\&{case} \index{txt local+\\{txt\_local}}\\{txt\_local}${}+\T{10}{}$:\5
\&{return} \index{hwritef+\\{hwritef}}\\{hwritef}(\.{"\\\\k"});\6
\4\&{case} \index{txt local+\\{txt\_local}}\\{txt\_local}${}+\T{11}{}$:\5
\&{return} \index{hwritef+\\{hwritef}}\\{hwritef}(\.{"\\\\l"});\6
\4\&{case} \index{txt cc+\\{txt\_cc}}\\{txt\_cc}:\5
\&{return} \index{hwrite txt cc+\\{hwrite\_txt\_cc}}\\{hwrite\_txt\_cc}(\index{hget utf8+\\{hget\_utf8}}\\{hget\_utf8}(\,));\6
\4\&{case} \index{txt node+\\{txt\_node}}\\{txt\_node}:\1\6
\4${}\{{}$\5
\&{int} \|i;\7
\X14:read the start byte \|a\X\6
${}\|i\K\index{hwritef+\\{hwritef}}\\{hwritef}(\.{"<"});{}$\6
${}\|i\MRL{+{\K}}\index{hwritef+\\{hwritef}}\\{hwritef}(\.{"\%s"},\39\index{content name+\\{content\_name}}\\{content\_name}[\index{KIND+\.{KIND}}\.{KIND}(\|a)]){}$;\5
\index{hget content+\\{hget\_content}}\\{hget\_content}(\|a);\6
\X15:read and check the end byte \|z\X\6
\index{hwritec+\\{hwritec}}\\{hwritec}(\.{'>'});\5
\&{return} \|i${}+\T{10}{}$;\C{ just an estimate }\6
\4${}\}{}$\2\6
\4\&{case} \index{txt hyphen+\\{txt\_hyphen}}\\{txt\_hyphen}:\5
\index{hwritec+\\{hwritec}}\\{hwritec}(\.{'-'});\5
\&{return} \T{1};\6
\4\&{case} \index{txt glue+\\{txt\_glue}}\\{txt\_glue}:\5
\&{return} ${}{-}\T{1};{}$\6
\4\&{case} \.{'<'}:\5
\&{return} \index{hwritef+\\{hwritef}}\\{hwritef}(\.{"\\\\<"});\6
\4\&{case} \.{'>'}:\5
\&{return} \index{hwritef+\\{hwritef}}\\{hwritef}(\.{"\\\\>"});\6
\4\&{case} \.{'"'}:\5
\&{return} \index{hwritef+\\{hwritef}}\\{hwritef}(\.{"\\\\\\""});\6
\4\&{case} \.{'-'}:\5
\&{return} \index{hwritef+\\{hwritef}}\\{hwritef}(\.{"\\\\-"});\6
\4\&{case} \index{txt ignore+\\{txt\_ignore}}\\{txt\_ignore}:\5
\&{return} \index{hwritef+\\{hwritef}}\\{hwritef}(\.{"\\\\@"});\6
\4\&{default}:\5
\index{hwritec+\\{hwritec}}\\{hwritec}(\|a);\5
\&{return} \T{1};\6
\4${}\}{}$\2\6
\4${}\}{}$\2\6
\4${}\}{}$\2
\Y
\fi

\M{148}


\putcode
\Y\B\4\X12:put functions\X${}\mathrel+\E{}$\6
\&{void} \index{hput txt cc+\\{hput\_txt\_cc}}\\{hput\_txt\_cc}(\&{uint32\_t} \|c)\1\1\2\2\1\6
\4${}\{{}$\5
\&{if} ${}(\|c\Z\T{\^20}){}$\5
\1${}\{{}$\5
\index{HPUTX+\.{HPUTX}}\.{HPUTX}(\T{2});\6
\index{HPUT8+\.{HPUT8}}\.{HPUT8}(\index{txt cc+\\{txt\_cc}}\\{txt\_cc});\5
\index{HPUT8+\.{HPUT8}}\.{HPUT8}(\|c);\5
${}\}{}$\2\6
\&{else}\1\5
\index{hput utf8+\\{hput\_utf8}}\\{hput\_utf8}(\|c);\2\6
\4${}\}{}$\2\7
\&{void} \index{hput txt font+\\{hput\_txt\_font}}\\{hput\_txt\_font}(\&{uint8\_t} \|f)\1\1\2\2\1\6
\4${}\{{}$\5
\&{if} ${}(\|f<\T{8}){}$\1\5
${}\index{HPUTX+\.{HPUTX}}\.{HPUTX}(\T{1}),\39\index{HPUT8+\.{HPUT8}}\.{HPUT8}(\index{txt font+\\{txt\_font}}\\{txt\_font}+\|f);{}$\2\6
\&{else}\1\5
${}\.{QUIT}(\.{"Use\ \\\\F\%d\\\\\ instead}\)\.{\ of\ \\\\\%d\ for\ font\ \%d}\)\.{\ in\ a\ text"},\39\|f,\39\|f,\39\|f);{}$\2\6
\4${}\}{}$\2\7
\&{void} ${}\index{hput txt global+\\{hput\_txt\_global}}\\{hput\_txt\_global}(\index{ref t+\\{ref\_t}}\\{ref\_t}*\|d){}$\1\1\2\2\1\6
\4${}\{{}$\5
\index{HPUTX+\.{HPUTX}}\.{HPUTX}(\T{2});\6
\&{switch} ${}(\|d\MG\|k){}$\5
\1${}\{{}$\6
\4\&{case} \index{font kind+\\{font\_kind}}\\{font\_kind}:\5
${}\index{HPUT8+\.{HPUT8}}\.{HPUT8}(\index{txt global+\\{txt\_global}}\\{txt\_global}+\T{0}){}$;\5
\&{break};\6
\4\&{case} \index{penalty kind+\\{penalty\_kind}}\\{penalty\_kind}:\5
${}\index{HPUT8+\.{HPUT8}}\.{HPUT8}(\index{txt global+\\{txt\_global}}\\{txt\_global}+\T{1}){}$;\5
\&{break};\6
\4\&{case} \index{kern kind+\\{kern\_kind}}\\{kern\_kind}:\5
${}\index{HPUT8+\.{HPUT8}}\.{HPUT8}(\index{txt global+\\{txt\_global}}\\{txt\_global}+\T{2}){}$;\5
\&{break};\6
\4\&{case} \index{ligature kind+\\{ligature\_kind}}\\{ligature\_kind}:\5
${}\index{HPUT8+\.{HPUT8}}\.{HPUT8}(\index{txt global+\\{txt\_global}}\\{txt\_global}+\T{3}){}$;\5
\&{break};\6
\4\&{case} \index{hyphen kind+\\{hyphen\_kind}}\\{hyphen\_kind}:\5
${}\index{HPUT8+\.{HPUT8}}\.{HPUT8}(\index{txt global+\\{txt\_global}}\\{txt\_global}+\T{4}){}$;\5
\&{break};\6
\4\&{case} \index{glue kind+\\{glue\_kind}}\\{glue\_kind}:\5
${}\index{HPUT8+\.{HPUT8}}\.{HPUT8}(\index{txt global+\\{txt\_global}}\\{txt\_global}+\T{5}){}$;\5
\&{break};\6
\4\&{case} \index{language kind+\\{language\_kind}}\\{language\_kind}:\5
${}\index{HPUT8+\.{HPUT8}}\.{HPUT8}(\index{txt global+\\{txt\_global}}\\{txt\_global}+\T{6}){}$;\5
\&{break};\6
\4\&{case} \index{rule kind+\\{rule\_kind}}\\{rule\_kind}:\5
${}\index{HPUT8+\.{HPUT8}}\.{HPUT8}(\index{txt global+\\{txt\_global}}\\{txt\_global}+\T{7}){}$;\5
\&{break};\6
\4\&{case} \index{image kind+\\{image\_kind}}\\{image\_kind}:\5
${}\index{HPUT8+\.{HPUT8}}\.{HPUT8}(\index{txt global+\\{txt\_global}}\\{txt\_global}+\T{8}){}$;\5
\&{break};\6
\4\&{default}:\5
${}\.{QUIT}(\.{"Kind\ \%s\ not\ allowed}\)\.{\ as\ a\ global\ referen}\)\.{ce\ in\ a\ text"},\39\index{NAME+\.{NAME}}\.{NAME}(\|d\MG\|k));{}$\6
\4${}\}{}$\2\6
${}\index{HPUT8+\.{HPUT8}}\.{HPUT8}(\|d\MG\|n);{}$\6
\4${}\}{}$\2\7
\&{void} \index{hput txt local+\\{hput\_txt\_local}}\\{hput\_txt\_local}(\&{uint8\_t} \|n)\1\1\2\2\1\6
\4${}\{{}$\5
\index{HPUTX+\.{HPUTX}}\.{HPUTX}(\T{1});\6
${}\index{HPUT8+\.{HPUT8}}\.{HPUT8}(\index{txt local+\\{txt\_local}}\\{txt\_local}+\|n);{}$\6
\4${}\}{}$\2
\Y
\fi

\M{149}


\Y\B\4\X1:hint types\X${}\mathrel+\E{}$\6
\&{typedef} \&{struct} ${}\{{}$\5
\1\index{kind t+\&{kind\_t}}\&{kind\_t} \|k;\5
\&{uint8\_t} \|n;\5
\2${}\}{}$ \index{ref t+\&{ref\_t}}\&{ref\_t};
\Y
\fi

\M{150}


\section{Composite Nodes}\hascode
\label{composite}
The nodes that we consider in this section contain other nodes for example
a glue node or a list of node. When we implement the parsing\index{parsing} routines
for composite nodes in the long format, we have to take into account
that parsing such a glue node or list node  will already write the glue or list node
to the output. So we split the parsing of composite nodes into several parts
and store the parts immediately after parsing them. On the parse stack we will only
keep track of the info value.
This new strategy is not as transparent as  our previous strategy used for
simple nodes where we had a clean separation of reading and writing:
reading would store the internal representation of a node and writing the internal
representation to output would start only after reading is completed.
The new strategy, however, makes it easier to reuse
the grammar\index{grammar} rules for the component nodes.

\subsection{Boxes}
The central structuring elements of \TeX\ are boxes\index{box}.
Boxes have a height \|h, a depth \|d, and a width \|w.
The shift amount \|a shifts the contents of the box, the glue ratio\index{glue ratio} \|r is a factor
applied to the glue inside the box, the glue order \|o is its order of stretchability\index{stretchability},
and the glue sign \|s is $-1$ for shrinking\index{shrinkability}, 0 for rigid, and $+1$ for stretching.
Most importantly, a box contains a list \|l of elements inside the box.


\Y\B\4\X1:hint types\X${}\mathrel+\E{}$\6
\&{typedef} \&{struct}\6
${}\{{}$\5
\1\index{dimen t+\&{dimen\_t}}\&{dimen\_t} \|h${},{}$ \|d${},{}$ \|w${},{}$ \|a;\5
\&{float32\_t} \|r;\5
\&{int8\_t} \|s${},{}$ \|o;\5
\index{list t+\&{list\_t}}\&{list\_t} \|l;\5
\2${}\}{}$ \index{box t+\&{box\_t}}\&{box\_t};
\Y
\fi

\M{151}

There are two types of boxes: horizontal\index{horizontal box} boxes
and vertical\index{vertical box} boxes.
The difference between the two is simple:
a horizontal box aligns the reference\index{reference point}
points of its elements horizontally, the shift amount\index{shift amount} \|a
shifts the box down;
a vertical box aligns\index{alignment} the reference\index{reference point}
points vertically,  the shift amount \|a shifts the box right.

Not all box parameters are used frequently. In the short format, we use the info bits
to indicated which of the parameters are used.
Where as the width of a horizontal box is most of the time (80\%) nonzero,
other parameters are most of the time zero,
like the shift amount (99\%) or the glue settings (99.8\%).
The depth is zero in about 77\%, the height in about 53\%,
and both together are zero in about 47\%. The results for vertical boxes,
which constitute about 20\% of all boxes, are similar,
except that the depth is zero in about 89\%,
but the height and width are almost never zero.
For this reason we use bit \\{b001} to indicate a nonzero depth,
bit \\{b010}  for a nonzero shift amount, and \\{b100} for nonzero glue settings.
Glue sign and glue order can be packed as two nibbles in a single byte.
% A different use of the info bits for vertical and horizontal boxes is possible,
% but does not warrant the added complexity.



\goodbreak
\readcode
\Y\par
\par
\par
\par
\par
\par
\par
\par
\par
\Y\B\4\X2:symbols\X${}\mathrel+\E{}$\6
\8\%\&{token} \index{HBOX+\ts{HBOX}}\ts{HBOX}\5\.{"hbox"}\6
\8\%\&{token} \index{VBOX+\ts{VBOX}}\ts{VBOX}\5\.{"vbox"}\6
\8\%\&{token} \index{SHIFTED+\.{SHIFTED}}\.{SHIFTED}\5\.{"shifted"}\6
\8\%\index{type+\&{type}}\&{type} $<$ \index{info+\\{info}}\\{info} $>$ \index{box+\nts{box}}\nts{box}\5
\index{box dimen+\nts{box\_dimen}}\nts{box\_dimen}\5
\index{box shift+\nts{box\_shift}}\nts{box\_shift}\5
\index{box glue set+\nts{box\_glue\_set}}\nts{box\_glue\_set}
\Y
\fi

\M{152}
\Y\B\4\X3:scanning rules\X${}\mathrel+\E{}$\6
${}\8\re{\vb{hbox}}{}$\ac\&{return} \index{HBOX+\ts{HBOX}}\ts{HBOX};\eac\7
${}\8\re{\vb{vbox}}{}$\ac\&{return} \index{VBOX+\ts{VBOX}}\ts{VBOX};\eac\7
${}\8\re{\vb{shifted}}{}$\ac\&{return} \index{SHIFTED+\.{SHIFTED}}\.{SHIFTED};\eac
\Y
\fi

\M{153}

\Y\B\4\X5:parsing rules\X${}\mathrel+\E{}$\6
\index{box dimen+\nts{box\_dimen}}\nts{box\_dimen}: \1\1\5
\index{dimension+\nts{dimension}}\nts{dimension}\5
\index{dimension+\nts{dimension}}\nts{dimension}\5
\index{dimension+\nts{dimension}}\nts{dimension}\6
${}\{{}$\1\5
${}\.{\$\$}\K\index{hput box dimen+\\{hput\_box\_dimen}}\\{hput\_box\_dimen}(\.{\$1},\39\.{\$2},\39\.{\$3});{}$\5
${}\}{}$\2;\2\2\7
\index{box shift+\nts{box\_shift}}\nts{box\_shift}: \1\1\5
${}\{{}$\1\5
${}\.{\$\$}\K\\{b000};{}$\5
${}\}{}$\5
\2\hbox to 0.5em{\hss${}\OR{}$} \index{SHIFTED+\.{SHIFTED}}\.{SHIFTED} \index{dimension+\nts{dimension}}\nts{dimension}\5
${}\{{}$\1\5
${}\.{\$\$}\K\index{hput box shift+\\{hput\_box\_shift}}\\{hput\_box\_shift}(\.{\$2});{}$\5
${}\}{}$\2;\2\2\7
\index{box glue set+\nts{box\_glue\_set}}\nts{box\_glue\_set}: \1\1\5
${}\{{}$\1\5
${}\.{\$\$}\K\\{b000};{}$\5
${}\}{}$\2\6
\4\hbox to 0.5em{\hss${}\OR{}$}\5
\index{PLUS+\ts{PLUS}}\ts{PLUS}\5
\index{stretch+\nts{stretch}}\nts{stretch}\5
${}\{{}$\1\5
${}\.{\$\$}\K\index{hput box glue set+\\{hput\_box\_glue\_set}}\\{hput\_box\_glue\_set}({+}\T{1},\39\.{\$2}.\|f,\39\.{\$2}.\|o);{}$\5
${}\}{}$\2\6
\4\hbox to 0.5em{\hss${}\OR{}$}\5
\index{MINUS+\ts{MINUS}}\ts{MINUS}\5
\index{stretch+\nts{stretch}}\nts{stretch}\5
${}\{{}$\1\5
${}\.{\$\$}\K\index{hput box glue set+\\{hput\_box\_glue\_set}}\\{hput\_box\_glue\_set}({-}\T{1},\39\.{\$2}.\|f,\39\.{\$2}.\|o);{}$\5
${}\}{}$\2;\2\2\7
\index{box+\nts{box}}\nts{box}: \1\1\5
\index{box dimen+\nts{box\_dimen}}\nts{box\_dimen}\5
\index{box shift+\nts{box\_shift}}\nts{box\_shift}\5
\index{box glue set+\nts{box\_glue\_set}}\nts{box\_glue\_set}\5
\index{list+\nts{list}}\nts{list}\5
${}\{{}$\1\5
${}\.{\$\$}\K\.{\$1}\OR\.{\$2}\OR\.{\$3};{}$\5
${}\}{}$\2;\2\2\7
\index{hbox node+\nts{hbox\_node}}\nts{hbox\_node}: \1\1\5
\index{start+\nts{start}}\nts{start}\5
\index{HBOX+\ts{HBOX}}\ts{HBOX}\5
\index{box+\nts{box}}\nts{box}\5
\index{END+\ts{END}}\ts{END}\5
${}\{{}$\1\5
${}\index{hput tags+\\{hput\_tags}}\\{hput\_tags}(\.{\$1},\39\.{TAG}(\index{hbox kind+\\{hbox\_kind}}\\{hbox\_kind},\39\.{\$3}));{}$\5
${}\}{}$\2;\2\2\7
\index{vbox node+\nts{vbox\_node}}\nts{vbox\_node}: \1\1\5
\index{start+\nts{start}}\nts{start}\5
\index{VBOX+\ts{VBOX}}\ts{VBOX}\5
\index{box+\nts{box}}\nts{box}\5
\index{END+\ts{END}}\ts{END}\5
${}\{{}$\1\5
${}\index{hput tags+\\{hput\_tags}}\\{hput\_tags}(\.{\$1},\39\.{TAG}(\index{vbox kind+\\{vbox\_kind}}\\{vbox\_kind},\39\.{\$3}));{}$\5
${}\}{}$\2;\2\2\7
\index{content node+\nts{content\_node}}\nts{content\_node}: \1\1\5
\index{hbox node+\nts{hbox\_node}}\nts{hbox\_node}\5
\hbox to 0.5em{\hss${}\OR{}$}\5
\index{vbox node+\nts{vbox\_node}}\nts{vbox\_node};\2\2
\Y
\fi

\M{154}

\writecode
\Y\B\4\X19:write functions\X${}\mathrel+\E{}$\6
\&{void} \index{hwrite box+\\{hwrite\_box}}\\{hwrite\_box}(\index{box t+\&{box\_t}}\&{box\_t} ${}{*}\|b){}$\1\1\2\2\1\6
\4${}\{{}$\5
${}\index{hwrite dimension+\\{hwrite\_dimension}}\\{hwrite\_dimension}(\|b\MG\|h);{}$\6
${}\index{hwrite dimension+\\{hwrite\_dimension}}\\{hwrite\_dimension}(\|b\MG\|d);{}$\6
${}\index{hwrite dimension+\\{hwrite\_dimension}}\\{hwrite\_dimension}(\|b\MG\|w);{}$\6
\&{if} ${}(\|b\MG\|a\I\T{0}){}$\5
\1${}\{{}$\5
\index{hwritef+\\{hwritef}}\\{hwritef}(\.{"\ shifted"});\5
${}\index{hwrite dimension+\\{hwrite\_dimension}}\\{hwrite\_dimension}(\|b\MG\|a){}$;\5
${}\}{}$\2\6
\&{if} ${}(\|b\MG\|r\I\T{0.0}\W\|b\MG\|s\I\T{0}{}$)\6
\1${}\{{}$\5
\&{if} ${}(\|b\MG\|s>\T{0}{}$)\5
\1\index{hwritef+\\{hwritef}}\\{hwritef}(\.{"\ plus"});\5
\2\&{else}\5
\1\index{hwritef+\\{hwritef}}\\{hwritef}(\.{"\ minus"});\5
\2${}\index{hwrite float64+\\{hwrite\_float64}}\\{hwrite\_float64}(\|b\MG\|r){}$;\5
${}\index{hwrite order+\\{hwrite\_order}}\\{hwrite\_order}(\|b\MG\|o);{}$\6
\4${}\}{}$\2\6
${}\index{hwrite list+\\{hwrite\_list}}\\{hwrite\_list}({\AND}(\|b\MG\|l));{}$\6
\4${}\}{}$\2
\Y
\fi

\M{155}

\getcode
\Y\B\4\X18:cases to get content\X${}\mathrel+\E{}$\6
\4\&{case} \.{TAG}${}(\index{hbox kind+\\{hbox\_kind}}\\{hbox\_kind},\39\\{b000}){}$:\5
\1${}\{{}$\5
\index{box t+\&{box\_t}}\&{box\_t} \|b;\5
${}\index{HGET BOX+\.{HGET\_BOX}}\.{HGET\_BOX}(\\{b000},\39\|b){}$;\5
${}\index{hwrite box+\\{hwrite\_box}}\\{hwrite\_box}({\AND}\|b){}$;\5
${}\}{}$\5
\2\&{break};\6
\4\&{case} \.{TAG}${}(\index{hbox kind+\\{hbox\_kind}}\\{hbox\_kind},\39\\{b001}){}$:\5
\1${}\{{}$\5
\index{box t+\&{box\_t}}\&{box\_t} \|b;\5
${}\index{HGET BOX+\.{HGET\_BOX}}\.{HGET\_BOX}(\\{b001},\39\|b){}$;\5
${}\index{hwrite box+\\{hwrite\_box}}\\{hwrite\_box}({\AND}\|b){}$;\5
${}\}{}$\5
\2\&{break};\6
\4\&{case} \.{TAG}${}(\index{hbox kind+\\{hbox\_kind}}\\{hbox\_kind},\39\\{b010}){}$:\5
\1${}\{{}$\5
\index{box t+\&{box\_t}}\&{box\_t} \|b;\5
${}\index{HGET BOX+\.{HGET\_BOX}}\.{HGET\_BOX}(\\{b010},\39\|b){}$;\5
${}\index{hwrite box+\\{hwrite\_box}}\\{hwrite\_box}({\AND}\|b){}$;\5
${}\}{}$\5
\2\&{break};\6
\4\&{case} \.{TAG}${}(\index{hbox kind+\\{hbox\_kind}}\\{hbox\_kind},\39\\{b011}){}$:\5
\1${}\{{}$\5
\index{box t+\&{box\_t}}\&{box\_t} \|b;\5
${}\index{HGET BOX+\.{HGET\_BOX}}\.{HGET\_BOX}(\\{b011},\39\|b){}$;\5
${}\index{hwrite box+\\{hwrite\_box}}\\{hwrite\_box}({\AND}\|b){}$;\5
${}\}{}$\5
\2\&{break};\6
\4\&{case} \.{TAG}${}(\index{hbox kind+\\{hbox\_kind}}\\{hbox\_kind},\39\\{b100}){}$:\5
\1${}\{{}$\5
\index{box t+\&{box\_t}}\&{box\_t} \|b;\5
${}\index{HGET BOX+\.{HGET\_BOX}}\.{HGET\_BOX}(\\{b100},\39\|b){}$;\5
${}\index{hwrite box+\\{hwrite\_box}}\\{hwrite\_box}({\AND}\|b){}$;\5
${}\}{}$\5
\2\&{break};\6
\4\&{case} \.{TAG}${}(\index{hbox kind+\\{hbox\_kind}}\\{hbox\_kind},\39\\{b101}){}$:\5
\1${}\{{}$\5
\index{box t+\&{box\_t}}\&{box\_t} \|b;\5
${}\index{HGET BOX+\.{HGET\_BOX}}\.{HGET\_BOX}(\\{b101},\39\|b){}$;\5
${}\index{hwrite box+\\{hwrite\_box}}\\{hwrite\_box}({\AND}\|b){}$;\5
${}\}{}$\5
\2\&{break};\6
\4\&{case} \.{TAG}${}(\index{hbox kind+\\{hbox\_kind}}\\{hbox\_kind},\39\\{b110}){}$:\5
\1${}\{{}$\5
\index{box t+\&{box\_t}}\&{box\_t} \|b;\5
${}\index{HGET BOX+\.{HGET\_BOX}}\.{HGET\_BOX}(\\{b110},\39\|b){}$;\5
${}\index{hwrite box+\\{hwrite\_box}}\\{hwrite\_box}({\AND}\|b){}$;\5
${}\}{}$\5
\2\&{break};\6
\4\&{case} \.{TAG}${}(\index{hbox kind+\\{hbox\_kind}}\\{hbox\_kind},\39\\{b111}){}$:\5
\1${}\{{}$\5
\index{box t+\&{box\_t}}\&{box\_t} \|b;\5
${}\index{HGET BOX+\.{HGET\_BOX}}\.{HGET\_BOX}(\\{b111},\39\|b){}$;\5
${}\index{hwrite box+\\{hwrite\_box}}\\{hwrite\_box}({\AND}\|b){}$;\5
${}\}{}$\5
\2\&{break};\6
\4\&{case} \.{TAG}${}(\index{vbox kind+\\{vbox\_kind}}\\{vbox\_kind},\39\\{b000}){}$:\5
\1${}\{{}$\5
\index{box t+\&{box\_t}}\&{box\_t} \|b;\5
${}\index{HGET BOX+\.{HGET\_BOX}}\.{HGET\_BOX}(\\{b000},\39\|b){}$;\5
${}\index{hwrite box+\\{hwrite\_box}}\\{hwrite\_box}({\AND}\|b){}$;\5
${}\}{}$\5
\2\&{break};\6
\4\&{case} \.{TAG}${}(\index{vbox kind+\\{vbox\_kind}}\\{vbox\_kind},\39\\{b001}){}$:\5
\1${}\{{}$\5
\index{box t+\&{box\_t}}\&{box\_t} \|b;\5
${}\index{HGET BOX+\.{HGET\_BOX}}\.{HGET\_BOX}(\\{b001},\39\|b){}$;\5
${}\index{hwrite box+\\{hwrite\_box}}\\{hwrite\_box}({\AND}\|b){}$;\5
${}\}{}$\5
\2\&{break};\6
\4\&{case} \.{TAG}${}(\index{vbox kind+\\{vbox\_kind}}\\{vbox\_kind},\39\\{b010}){}$:\5
\1${}\{{}$\5
\index{box t+\&{box\_t}}\&{box\_t} \|b;\5
${}\index{HGET BOX+\.{HGET\_BOX}}\.{HGET\_BOX}(\\{b010},\39\|b){}$;\5
${}\index{hwrite box+\\{hwrite\_box}}\\{hwrite\_box}({\AND}\|b){}$;\5
${}\}{}$\5
\2\&{break};\6
\4\&{case} \.{TAG}${}(\index{vbox kind+\\{vbox\_kind}}\\{vbox\_kind},\39\\{b011}){}$:\5
\1${}\{{}$\5
\index{box t+\&{box\_t}}\&{box\_t} \|b;\5
${}\index{HGET BOX+\.{HGET\_BOX}}\.{HGET\_BOX}(\\{b011},\39\|b){}$;\5
${}\index{hwrite box+\\{hwrite\_box}}\\{hwrite\_box}({\AND}\|b){}$;\5
${}\}{}$\5
\2\&{break};\6
\4\&{case} \.{TAG}${}(\index{vbox kind+\\{vbox\_kind}}\\{vbox\_kind},\39\\{b100}){}$:\5
\1${}\{{}$\5
\index{box t+\&{box\_t}}\&{box\_t} \|b;\5
${}\index{HGET BOX+\.{HGET\_BOX}}\.{HGET\_BOX}(\\{b100},\39\|b){}$;\5
${}\index{hwrite box+\\{hwrite\_box}}\\{hwrite\_box}({\AND}\|b){}$;\5
${}\}{}$\5
\2\&{break};\6
\4\&{case} \.{TAG}${}(\index{vbox kind+\\{vbox\_kind}}\\{vbox\_kind},\39\\{b101}){}$:\5
\1${}\{{}$\5
\index{box t+\&{box\_t}}\&{box\_t} \|b;\5
${}\index{HGET BOX+\.{HGET\_BOX}}\.{HGET\_BOX}(\\{b101},\39\|b){}$;\5
${}\index{hwrite box+\\{hwrite\_box}}\\{hwrite\_box}({\AND}\|b){}$;\5
${}\}{}$\5
\2\&{break};\6
\4\&{case} \.{TAG}${}(\index{vbox kind+\\{vbox\_kind}}\\{vbox\_kind},\39\\{b110}){}$:\5
\1${}\{{}$\5
\index{box t+\&{box\_t}}\&{box\_t} \|b;\5
${}\index{HGET BOX+\.{HGET\_BOX}}\.{HGET\_BOX}(\\{b110},\39\|b){}$;\5
${}\index{hwrite box+\\{hwrite\_box}}\\{hwrite\_box}({\AND}\|b){}$;\5
${}\}{}$\5
\2\&{break};\6
\4\&{case} \.{TAG}${}(\index{vbox kind+\\{vbox\_kind}}\\{vbox\_kind},\39\\{b111}){}$:\5
\1${}\{{}$\5
\index{box t+\&{box\_t}}\&{box\_t} \|b;\5
${}\index{HGET BOX+\.{HGET\_BOX}}\.{HGET\_BOX}(\\{b111},\39\|b){}$;\5
${}\index{hwrite box+\\{hwrite\_box}}\\{hwrite\_box}({\AND}\|b){}$;\5
${}\}{}$\5
\2\&{break};
\Y
\fi

\M{156}

\Y\B\4\X17:get macros\X${}\mathrel+\E{}$\6
\8\#\&{define} $\index{HGET BOX+\.{HGET\_BOX}}\.{HGET\_BOX}(\|I,\39\|B){}$ \index{HGET32+\.{HGET32}}\.{HGET32} ${}(\|B.\|h);{}$\6
\&{if} ${}((\|I)\AND\\{b001}){}$\1\5
${}\index{HGET32+\.{HGET32}}\.{HGET32}(\|B.\|d){}$;\5
\2\&{else}\1\5
${}\|B.\|d\K\T{0};{}$\2\6
${}\index{HGET32+\.{HGET32}}\.{HGET32}(\|B.\|w);{}$\6
\&{if} ${}((\|I)\AND\\{b010}){}$\1\5
${}\index{HGET32+\.{HGET32}}\.{HGET32}(\|B.\|a){}$;\5
\2\&{else}\1\5
${}\|B.\|a\K\T{0};{}$\2\6
\&{if} ${}((\|I)\AND\\{b100}{}$)\6
\1${}\{{}$\5
${}\|B.\|r\K\index{hget float32+\\{hget\_float32}}\\{hget\_float32}(\,){}$;\5
${}\|B.\|s\K\index{HGET8+\.{HGET8}}\.{HGET8}{}$;\5
${}\|B.\|o\K\|B.\|s\AND\T{\^F}{}$;\5
${}\|B.\|s\K\|B.\|s\GG\T{4}{}$;\5
${}\}{}$\2\6
\&{else}\5
\1${}\{{}$\5
${}\|B.\|r\K\T{0.0}{}$;\5
${}\|B.\|o\K\|B.\|s\K\T{0}{}$;\5
${}\}{}$\2\6
${}\index{hget list+\\{hget\_list}}\\{hget\_list}({\AND}(\|B.\|l));$
\Y
\fi

\M{157}

\Y\B\4\X16:get functions\X${}\mathrel+\E{}$\6
\&{void} \index{hget hbox node+\\{hget\_hbox\_node}}\\{hget\_hbox\_node}(\&{void})\1\1\2\2\1\6
\4${}\{{}$\5
\index{box t+\&{box\_t}}\&{box\_t} \|b;\7
\X14:read the start byte \|a\X\6
\&{if} ${}(\index{KIND+\.{KIND}}\.{KIND}(\|a)\I\index{hbox kind+\\{hbox\_kind}}\\{hbox\_kind}){}$\1\5
${}\.{QUIT}(\.{"Hbox\ expected\ at\ 0x}\)\.{\%x\ got\ \%s"},\39\\{node\_pos},\39\index{NAME+\.{NAME}}\.{NAME}(\|a));{}$\2\6
${}\index{HGET BOX+\.{HGET\_BOX}}\.{HGET\_BOX}(\index{INFO+\.{INFO}}\.{INFO}(\|a),\39\|b){}$;\6
\X15:read and check the end byte \|z\X\6
\index{hwrite start+\\{hwrite\_start}}\\{hwrite\_start}(\,);\5
\index{hwritef+\\{hwritef}}\\{hwritef}(\.{"hbox"});\5
${}\index{hwrite box+\\{hwrite\_box}}\\{hwrite\_box}({\AND}\|b){}$;\5
\index{hwrite end+\\{hwrite\_end}}\\{hwrite\_end}(\,);\6
\4${}\}{}$\2\7
\&{void} \index{hget vbox node+\\{hget\_vbox\_node}}\\{hget\_vbox\_node}(\&{void})\1\1\2\2\1\6
\4${}\{{}$\5
\index{box t+\&{box\_t}}\&{box\_t} \|b;\7
\X14:read the start byte \|a\X\6
\&{if} ${}(\index{KIND+\.{KIND}}\.{KIND}(\|a)\I\index{vbox kind+\\{vbox\_kind}}\\{vbox\_kind}){}$\1\5
${}\.{QUIT}(\.{"Vbox\ expected\ at\ 0x}\)\.{\%x\ got\ \%s"},\39\\{node\_pos},\39\index{NAME+\.{NAME}}\.{NAME}(\|a));{}$\2\6
${}\index{HGET BOX+\.{HGET\_BOX}}\.{HGET\_BOX}(\index{INFO+\.{INFO}}\.{INFO}(\|a),\39\|b){}$;\6
\X15:read and check the end byte \|z\X\6
\index{hwrite start+\\{hwrite\_start}}\\{hwrite\_start}(\,);\5
\index{hwritef+\\{hwritef}}\\{hwritef}(\.{"vbox"});\5
${}\index{hwrite box+\\{hwrite\_box}}\\{hwrite\_box}({\AND}\|b){}$;\5
\index{hwrite end+\\{hwrite\_end}}\\{hwrite\_end}(\,);\6
\4${}\}{}$\2
\Y
\fi

\M{158}

\putcode
\Y\B\4\X12:put functions\X${}\mathrel+\E{}$\6
\index{info t+\&{info\_t}}\&{info\_t} \index{hput box dimen+\\{hput\_box\_dimen}}\\{hput\_box\_dimen}(\index{dimen t+\&{dimen\_t}}\&{dimen\_t} \|h${},\39{}$\index{dimen t+\&{dimen\_t}}\&{dimen\_t} \|d${},\39{}$\index{dimen t+\&{dimen\_t}}\&{dimen\_t} \|w)\1\1\2\2\1\6
\4${}\{{}$\5
\index{info t+\&{info\_t}}\&{info\_t} \|i;\5
\index{HPUT32+\.{HPUT32}}\.{HPUT32}(\|h);\6
\&{if} ${}(\|d\I\T{0}){}$\5
\1${}\{{}$\5
\index{HPUT32+\.{HPUT32}}\.{HPUT32}(\|d);\5
${}\|i\K\\{b001}{}$;\5
${}\}{}$\5
\2\&{else}\5
\1${}\|i\K\\{b000};{}$\2\6
\index{HPUT32+\.{HPUT32}}\.{HPUT32}(\|w);\6
\&{return} \|i;\6
\4${}\}{}$\2\7
\index{info t+\&{info\_t}}\&{info\_t} \index{hput box shift+\\{hput\_box\_shift}}\\{hput\_box\_shift}(\index{dimen t+\&{dimen\_t}}\&{dimen\_t} \|a)\1\1\2\2\1\6
\4${}\{{}$\5
\&{if} ${}(\|a\I\T{0}){}$\5
\1${}\{{}$\5
\index{HPUT32+\.{HPUT32}}\.{HPUT32}(\|a);\5
\&{return}\5
\\{b010};\5
${}\}{}$\5
\2\&{else}\5
\1\&{return} \\{b000};\2\6
\4${}\}{}$\2\7
\index{info t+\&{info\_t}}\&{info\_t} \index{hput box glue set+\\{hput\_box\_glue\_set}}\\{hput\_box\_glue\_set}(\&{int8\_t} \|s${},\39{}$\&{float32\_t} \|r${},\39{}$\index{order t+\&{order\_t}}\&{order\_t} \|o)\1\1\2\2\1\6
\4${}\{{}$\5
\&{if} ${}(\|r\I\T{0.0}\W\|s\I\T{0}){}$\5
\1${}\{{}$\5
\index{hput float32+\\{hput\_float32}}\\{hput\_float32}(\|r);\5
${}\index{HPUT8+\.{HPUT8}}\.{HPUT8}((\|s\LL\T{4})\OR\|o){}$;\5
\&{return} \\{b100};\5
${}\}{}$\2\6
\&{else}\1\5
\&{return} \\{b000};\2\6
\4${}\}{}$\2
\Y
\fi

\M{159}

\subsection{Extended Boxes}
Hi\TeX\ produces two kinds of extended\index{extended box} horizontal
boxes, \index{hpack kind+\\{hpack\_kind}}\\{hpack\_kind} and \index{hset kind+\\{hset\_kind}}\\{hset\_kind}, and the same for vertical boxes
using \index{vpack kind+\\{vpack\_kind}}\\{vpack\_kind} and \index{vset kind+\\{vset\_kind}}\\{vset\_kind}.  Let us focus on horizontal boxes;
the handling of vertical boxes is completely parallel.

The \index{hpack+\\{hpack}}\\{hpack} procedure of Hi\TeX\ produces an extended box of \index{hset kind+\\{hset\_kind}}\\{hset\_kind}
either if it is given an extended\index{extended dimension} dimension as its width
or if it discovers that the width of its content is an extended
dimension.  After the final width of the box has been computed in the
viewer, it just remains to set the glue; a very simple operation
indeed.

If the \index{hpack+\\{hpack}}\\{hpack} procedure of Hi\TeX\ can not determine the natural
dimensions of the box content because it contains
paragraphs\index{paragraph} or other extended boxes, it produces a box
of \index{hpack kind+\\{hpack\_kind}}\\{hpack\_kind}.  Now the viewer needs to traverse the list of content
nodes to determine the natural\index{natural dimension}
dimensions. Even the amount of stretchability\index{stretchability}
and shrinkability\index{shrinkability} has to be determined in the
viewer. For example the final stretchability of a paragraph with some
stretchability in the baseline\index{baseline skip} skip will depend
on the number of lines which, in turn, depends on \.{hsize}.  It is
not possible to merge this traversals of the box content with the
traversal necessary when displaying the box. The latter needs to
convert glue nodes into positioning instructions which requires a
fixed glue\index{glue ratio} ratio. The computation of the glue ratio,
however, requires a complete traversal of the content.

In the short format of a box of type \index{hset kind+\\{hset\_kind}}\\{hset\_kind}, \index{vset kind+\\{vset\_kind}}\\{vset\_kind}, \index{hpack kind+\\{hpack\_kind}}\\{hpack\_kind} or \index{vpack kind+\\{vpack\_kind}}\\{vpack\_kind},
info bit \\{b100} indicates if set, a complete extended dimension, and if unset,
a reference to a predefined extended dimension for the target size;
info bit \\{b010} indicates a nonzero shift amount.
For a box of type \index{hset kind+\\{hset\_kind}}\\{hset\_kind} or \index{vset kind+\\{vset\_kind}}\\{vset\_kind},   the info bit \\{b001} indicates if set a nonzero depth.
For a box of type \index{hpack kind+\\{hpack\_kind}}\\{hpack\_kind} or \index{vpack kind+\\{vpack\_kind}}\\{vpack\_kind}, the info bit \\{b001} indicates if set an additional
target size and if unset an exact target size.
For a box of type \index{vpack kind+\\{vpack\_kind}}\\{vpack\_kind} also the maximum depth is given.

\readcode
\Y\par
\par
\par
\par
\par
\par
\par
\par
\par
\par
\par
\par
\par
\Y\B\4\X2:symbols\X${}\mathrel+\E{}$\6
\8\%\&{token} \index{HPACK+\ts{HPACK}}\ts{HPACK}\5\.{"hpack"}\6
\8\%\&{token} \index{HSET+\ts{HSET}}\ts{HSET}\5\.{"hset"}\6
\8\%\&{token} \index{VPACK+\ts{VPACK}}\ts{VPACK}\5\.{"vpack"}\6
\8\%\&{token} \index{VSET+\ts{VSET}}\ts{VSET}\5\.{"vset"}\6
\8\%\&{token} \index{ADD+\ts{ADD}}\ts{ADD}\5\.{"add"}\6
\8\%\&{token} \index{TO+\ts{TO}}\ts{TO}\5\.{"to"}\6
\8\%\index{type+\&{type}}\&{type} $<$ \index{info+\\{info}}\\{info} $>$ \index{xbox+\nts{xbox}}\nts{xbox}\5
\index{box goal+\nts{box\_goal}}\nts{box\_goal}\5
\index{hpack+\nts{hpack}}\nts{hpack}\5
\index{vpack+\nts{vpack}}\nts{vpack}
\Y
\fi

\M{160}

\Y\B\4\X3:scanning rules\X${}\mathrel+\E{}$\6
${}\8\re{\vb{hpack}}{}$\ac\&{return} \index{HPACK+\ts{HPACK}}\ts{HPACK};\eac\7
${}\8\re{\vb{hset}}{}$\ac\&{return} \index{HSET+\ts{HSET}}\ts{HSET};\eac\7
${}\8\re{\vb{vpack}}{}$\ac\&{return} \index{VPACK+\ts{VPACK}}\ts{VPACK};\eac\7
${}\8\re{\vb{vset}}{}$\ac\&{return} \index{VSET+\ts{VSET}}\ts{VSET};\eac\7
${}\8\re{\vb{add}}{}$\ac\&{return} \index{ADD+\ts{ADD}}\ts{ADD};\eac\7
${}\8\re{\vb{to}}{}$\ac\&{return} \index{TO+\ts{TO}}\ts{TO};\eac
\Y
\fi

\M{161}

\Y\B\4\X5:parsing rules\X${}\mathrel+\E{}$\6
\index{box flex+\nts{box\_flex}}\nts{box\_flex}: \1\1\5
\index{plus+\nts{plus}}\nts{plus}\5
\index{minus+\nts{minus}}\nts{minus}\5
${}\{{}$\1\5
${}\index{hput stretch+\\{hput\_stretch}}\\{hput\_stretch}({\AND}(\.{\$1}));{}$\5
${}\index{hput stretch+\\{hput\_stretch}}\\{hput\_stretch}({\AND}(\.{\$2}));{}$\5
${}\}{}$\2;\2\2\7
\index{xbox+\nts{xbox}}\nts{xbox}: \1\1\5
\index{xdimen ref+\nts{xdimen\_ref}}\nts{xdimen\_ref}\5
\index{box dimen+\nts{box\_dimen}}\nts{box\_dimen}\5
\index{box shift+\nts{box\_shift}}\nts{box\_shift}\5
\index{box flex+\nts{box\_flex}}\nts{box\_flex}\5
\index{list+\nts{list}}\nts{list}\5
${}\{{}$\1\5
${}\.{\$\$}\K\.{\$2}\OR\.{\$3};{}$\5
${}\}{}$\2\6
\4\hbox to 0.5em{\hss${}\OR{}$}\5
\index{xdimen node+\nts{xdimen\_node}}\nts{xdimen\_node}\5
\index{box dimen+\nts{box\_dimen}}\nts{box\_dimen}\5
\index{box shift+\nts{box\_shift}}\nts{box\_shift}\5
\index{box flex+\nts{box\_flex}}\nts{box\_flex}\5
\index{list+\nts{list}}\nts{list}\5
${}\{{}$\1\5
${}\.{\$\$}\K\.{\$2}\OR\.{\$3}\OR\\{b100};{}$\5
${}\}{}$\2;\2\2\7
\index{box goal+\nts{box\_goal}}\nts{box\_goal}: \1\1\5
\index{TO+\ts{TO}}\ts{TO}\5
\index{xdimen ref+\nts{xdimen\_ref}}\nts{xdimen\_ref}\5
${}\{{}$\1\5
${}\.{\$\$}\K\\{b000};{}$\5
${}\}{}$\2\6
\4\hbox to 0.5em{\hss${}\OR{}$}\5
\index{ADD+\ts{ADD}}\ts{ADD}\5
\index{xdimen ref+\nts{xdimen\_ref}}\nts{xdimen\_ref}\5
${}\{{}$\1\5
${}\.{\$\$}\K\\{b001};{}$\5
${}\}{}$\2\6
\4\hbox to 0.5em{\hss${}\OR{}$}\5
\index{TO+\ts{TO}}\ts{TO}\5
\index{xdimen node+\nts{xdimen\_node}}\nts{xdimen\_node}\5
${}\{{}$\1\5
${}\.{\$\$}\K\\{b100};{}$\5
${}\}{}$\2\6
\4\hbox to 0.5em{\hss${}\OR{}$}\5
\index{ADD+\ts{ADD}}\ts{ADD}\5
\index{xdimen node+\nts{xdimen\_node}}\nts{xdimen\_node}\5
${}\{{}$\1\5
${}\.{\$\$}\K\\{b101};{}$\5
${}\}{}$\2;\2\2\7
\index{hpack+\nts{hpack}}\nts{hpack}: \1\1\5
\index{box goal+\nts{box\_goal}}\nts{box\_goal}\5
\index{box shift+\nts{box\_shift}}\nts{box\_shift}\5
\index{list+\nts{list}}\nts{list};\2\2\7
\index{vpack+\nts{vpack}}\nts{vpack}: \1\1\5
\index{box goal+\nts{box\_goal}}\nts{box\_goal}\5
\index{box shift+\nts{box\_shift}}\nts{box\_shift}\5
\index{dimension+\nts{dimension}}\nts{dimension}\5
${}\{{}$\1\5
\index{HPUT32+\.{HPUT32}}\.{HPUT32}(\.{\$3});\5
${}\}{}$\2\5
\index{list+\nts{list}}\nts{list}\5
${}\{{}$\1\5
${}\.{\$\$}\K\.{\$1}\OR\.{\$2};{}$\5
${}\}{}$\2\5
\index{vxbox node+\nts{vxbox\_node}}\nts{vxbox\_node}: \1\1\5
\index{start+\nts{start}}\nts{start}\5
\index{VSET+\ts{VSET}}\ts{VSET}\5
\index{xbox+\nts{xbox}}\nts{xbox}\5
\index{END+\ts{END}}\ts{END}\5
${}\{{}$\1\5
${}\index{hput tags+\\{hput\_tags}}\\{hput\_tags}(\.{\$1},\39\.{TAG}(\index{vset kind+\\{vset\_kind}}\\{vset\_kind},\39\.{\$3}));{}$\5
${}\}{}$\2\6
\4\hbox to 0.5em{\hss${}\OR{}$}\5
\index{start+\nts{start}}\nts{start}\5
\index{VPACK+\ts{VPACK}}\ts{VPACK}\5
\index{vpack+\nts{vpack}}\nts{vpack}\5
\index{END+\ts{END}}\ts{END}\5
${}\{{}$\1\5
${}\index{hput tags+\\{hput\_tags}}\\{hput\_tags}(\.{\$1},\39\.{TAG}(\index{vpack kind+\\{vpack\_kind}}\\{vpack\_kind},\39\.{\$3}));{}$\5
${}\}{}$\2;\2\2\7
\index{hxbox node+\nts{hxbox\_node}}\nts{hxbox\_node}: \1\1\5
\index{start+\nts{start}}\nts{start}\5
\index{HSET+\ts{HSET}}\ts{HSET}\5
\index{xbox+\nts{xbox}}\nts{xbox}\5
\index{END+\ts{END}}\ts{END}\5
${}\{{}$\1\5
${}\index{hput tags+\\{hput\_tags}}\\{hput\_tags}(\.{\$1},\39\.{TAG}(\index{hset kind+\\{hset\_kind}}\\{hset\_kind},\39\.{\$3}));{}$\5
${}\}{}$\2\6
\4\hbox to 0.5em{\hss${}\OR{}$}\5
\index{start+\nts{start}}\nts{start}\5
\index{HPACK+\ts{HPACK}}\ts{HPACK}\5
\index{hpack+\nts{hpack}}\nts{hpack}\5
\index{END+\ts{END}}\ts{END}\5
${}\{{}$\1\5
${}\index{hput tags+\\{hput\_tags}}\\{hput\_tags}(\.{\$1},\39\.{TAG}(\index{hpack kind+\\{hpack\_kind}}\\{hpack\_kind},\39\.{\$3}));{}$\5
${}\}{}$\2\5
\index{content node+\nts{content\_node}}\nts{content\_node}: \1\1\5
\index{vxbox node+\nts{vxbox\_node}}\nts{vxbox\_node}\6
\4\hbox to 0.5em{\hss${}\OR{}$}\5
\index{hxbox node+\nts{hxbox\_node}}\nts{hxbox\_node};\2\2
\Y
\fi

\M{162}

\getcode
\Y\B\4\X18:cases to get content\X${}\mathrel+\E{}$\6
\4\&{case} \.{TAG}${}(\index{hset kind+\\{hset\_kind}}\\{hset\_kind},\39\\{b000}){}$:\5
\index{HGET SET+\.{HGET\_SET}}\.{HGET\_SET}(\\{b000});\5
\&{break};\6
\4\&{case} \.{TAG}${}(\index{hset kind+\\{hset\_kind}}\\{hset\_kind},\39\\{b001}){}$:\5
\index{HGET SET+\.{HGET\_SET}}\.{HGET\_SET}(\\{b001});\5
\&{break};\6
\4\&{case} \.{TAG}${}(\index{hset kind+\\{hset\_kind}}\\{hset\_kind},\39\\{b010}){}$:\5
\index{HGET SET+\.{HGET\_SET}}\.{HGET\_SET}(\\{b010});\5
\&{break};\6
\4\&{case} \.{TAG}${}(\index{hset kind+\\{hset\_kind}}\\{hset\_kind},\39\\{b011}){}$:\5
\index{HGET SET+\.{HGET\_SET}}\.{HGET\_SET}(\\{b011});\5
\&{break};\6
\4\&{case} \.{TAG}${}(\index{hset kind+\\{hset\_kind}}\\{hset\_kind},\39\\{b100}){}$:\5
\index{HGET SET+\.{HGET\_SET}}\.{HGET\_SET}(\\{b100});\5
\&{break};\6
\4\&{case} \.{TAG}${}(\index{hset kind+\\{hset\_kind}}\\{hset\_kind},\39\\{b101}){}$:\5
\index{HGET SET+\.{HGET\_SET}}\.{HGET\_SET}(\\{b101});\5
\&{break};\6
\4\&{case} \.{TAG}${}(\index{hset kind+\\{hset\_kind}}\\{hset\_kind},\39\\{b110}){}$:\5
\index{HGET SET+\.{HGET\_SET}}\.{HGET\_SET}(\\{b110});\5
\&{break};\6
\4\&{case} \.{TAG}${}(\index{hset kind+\\{hset\_kind}}\\{hset\_kind},\39\\{b111}){}$:\5
\index{HGET SET+\.{HGET\_SET}}\.{HGET\_SET}(\\{b111});\5
\&{break};\7
\4\&{case} \.{TAG}${}(\index{vset kind+\\{vset\_kind}}\\{vset\_kind},\39\\{b000}){}$:\5
\index{HGET SET+\.{HGET\_SET}}\.{HGET\_SET}(\\{b000});\5
\&{break};\6
\4\&{case} \.{TAG}${}(\index{vset kind+\\{vset\_kind}}\\{vset\_kind},\39\\{b001}){}$:\5
\index{HGET SET+\.{HGET\_SET}}\.{HGET\_SET}(\\{b001});\5
\&{break};\6
\4\&{case} \.{TAG}${}(\index{vset kind+\\{vset\_kind}}\\{vset\_kind},\39\\{b010}){}$:\5
\index{HGET SET+\.{HGET\_SET}}\.{HGET\_SET}(\\{b010});\5
\&{break};\6
\4\&{case} \.{TAG}${}(\index{vset kind+\\{vset\_kind}}\\{vset\_kind},\39\\{b011}){}$:\5
\index{HGET SET+\.{HGET\_SET}}\.{HGET\_SET}(\\{b011});\5
\&{break};\6
\4\&{case} \.{TAG}${}(\index{vset kind+\\{vset\_kind}}\\{vset\_kind},\39\\{b100}){}$:\5
\index{HGET SET+\.{HGET\_SET}}\.{HGET\_SET}(\\{b100});\5
\&{break};\6
\4\&{case} \.{TAG}${}(\index{vset kind+\\{vset\_kind}}\\{vset\_kind},\39\\{b101}){}$:\5
\index{HGET SET+\.{HGET\_SET}}\.{HGET\_SET}(\\{b101});\5
\&{break};\6
\4\&{case} \.{TAG}${}(\index{vset kind+\\{vset\_kind}}\\{vset\_kind},\39\\{b110}){}$:\5
\index{HGET SET+\.{HGET\_SET}}\.{HGET\_SET}(\\{b110});\5
\&{break};\6
\4\&{case} \.{TAG}${}(\index{vset kind+\\{vset\_kind}}\\{vset\_kind},\39\\{b111}){}$:\5
\index{HGET SET+\.{HGET\_SET}}\.{HGET\_SET}(\\{b111});\5
\&{break};\7
\4\&{case} \.{TAG}${}(\index{hpack kind+\\{hpack\_kind}}\\{hpack\_kind},\39\\{b000}){}$:\5
${}\index{HGET PACK+\.{HGET\_PACK}}\.{HGET\_PACK}(\index{hpack kind+\\{hpack\_kind}}\\{hpack\_kind},\39\\{b000}){}$;\5
\&{break};\6
\4\&{case} \.{TAG}${}(\index{hpack kind+\\{hpack\_kind}}\\{hpack\_kind},\39\\{b001}){}$:\5
${}\index{HGET PACK+\.{HGET\_PACK}}\.{HGET\_PACK}(\index{hpack kind+\\{hpack\_kind}}\\{hpack\_kind},\39\\{b001}){}$;\5
\&{break};\6
\4\&{case} \.{TAG}${}(\index{hpack kind+\\{hpack\_kind}}\\{hpack\_kind},\39\\{b010}){}$:\5
${}\index{HGET PACK+\.{HGET\_PACK}}\.{HGET\_PACK}(\index{hpack kind+\\{hpack\_kind}}\\{hpack\_kind},\39\\{b010}){}$;\5
\&{break};\6
\4\&{case} \.{TAG}${}(\index{hpack kind+\\{hpack\_kind}}\\{hpack\_kind},\39\\{b011}){}$:\5
${}\index{HGET PACK+\.{HGET\_PACK}}\.{HGET\_PACK}(\index{hpack kind+\\{hpack\_kind}}\\{hpack\_kind},\39\\{b011}){}$;\5
\&{break};\6
\4\&{case} \.{TAG}${}(\index{hpack kind+\\{hpack\_kind}}\\{hpack\_kind},\39\\{b100}){}$:\5
${}\index{HGET PACK+\.{HGET\_PACK}}\.{HGET\_PACK}(\index{hpack kind+\\{hpack\_kind}}\\{hpack\_kind},\39\\{b100}){}$;\5
\&{break};\6
\4\&{case} \.{TAG}${}(\index{hpack kind+\\{hpack\_kind}}\\{hpack\_kind},\39\\{b101}){}$:\5
${}\index{HGET PACK+\.{HGET\_PACK}}\.{HGET\_PACK}(\index{hpack kind+\\{hpack\_kind}}\\{hpack\_kind},\39\\{b101}){}$;\5
\&{break};\6
\4\&{case} \.{TAG}${}(\index{hpack kind+\\{hpack\_kind}}\\{hpack\_kind},\39\\{b110}){}$:\5
${}\index{HGET PACK+\.{HGET\_PACK}}\.{HGET\_PACK}(\index{hpack kind+\\{hpack\_kind}}\\{hpack\_kind},\39\\{b110}){}$;\5
\&{break};\6
\4\&{case} \.{TAG}${}(\index{hpack kind+\\{hpack\_kind}}\\{hpack\_kind},\39\\{b111}){}$:\5
${}\index{HGET PACK+\.{HGET\_PACK}}\.{HGET\_PACK}(\index{hpack kind+\\{hpack\_kind}}\\{hpack\_kind},\39\\{b111}){}$;\5
\&{break};\7
\4\&{case} \.{TAG}${}(\index{vpack kind+\\{vpack\_kind}}\\{vpack\_kind},\39\\{b000}){}$:\5
${}\index{HGET PACK+\.{HGET\_PACK}}\.{HGET\_PACK}(\index{vpack kind+\\{vpack\_kind}}\\{vpack\_kind},\39\\{b000}){}$;\5
\&{break};\6
\4\&{case} \.{TAG}${}(\index{vpack kind+\\{vpack\_kind}}\\{vpack\_kind},\39\\{b001}){}$:\5
${}\index{HGET PACK+\.{HGET\_PACK}}\.{HGET\_PACK}(\index{vpack kind+\\{vpack\_kind}}\\{vpack\_kind},\39\\{b001}){}$;\5
\&{break};\6
\4\&{case} \.{TAG}${}(\index{vpack kind+\\{vpack\_kind}}\\{vpack\_kind},\39\\{b010}){}$:\5
${}\index{HGET PACK+\.{HGET\_PACK}}\.{HGET\_PACK}(\index{vpack kind+\\{vpack\_kind}}\\{vpack\_kind},\39\\{b010}){}$;\5
\&{break};\6
\4\&{case} \.{TAG}${}(\index{vpack kind+\\{vpack\_kind}}\\{vpack\_kind},\39\\{b011}){}$:\5
${}\index{HGET PACK+\.{HGET\_PACK}}\.{HGET\_PACK}(\index{vpack kind+\\{vpack\_kind}}\\{vpack\_kind},\39\\{b011}){}$;\5
\&{break};\6
\4\&{case} \.{TAG}${}(\index{vpack kind+\\{vpack\_kind}}\\{vpack\_kind},\39\\{b100}){}$:\5
${}\index{HGET PACK+\.{HGET\_PACK}}\.{HGET\_PACK}(\index{vpack kind+\\{vpack\_kind}}\\{vpack\_kind},\39\\{b100}){}$;\5
\&{break};\6
\4\&{case} \.{TAG}${}(\index{vpack kind+\\{vpack\_kind}}\\{vpack\_kind},\39\\{b101}){}$:\5
${}\index{HGET PACK+\.{HGET\_PACK}}\.{HGET\_PACK}(\index{vpack kind+\\{vpack\_kind}}\\{vpack\_kind},\39\\{b101}){}$;\5
\&{break};\6
\4\&{case} \.{TAG}${}(\index{vpack kind+\\{vpack\_kind}}\\{vpack\_kind},\39\\{b110}){}$:\5
${}\index{HGET PACK+\.{HGET\_PACK}}\.{HGET\_PACK}(\index{vpack kind+\\{vpack\_kind}}\\{vpack\_kind},\39\\{b110}){}$;\5
\&{break};\6
\4\&{case} \.{TAG}${}(\index{vpack kind+\\{vpack\_kind}}\\{vpack\_kind},\39\\{b111}){}$:\5
${}\index{HGET PACK+\.{HGET\_PACK}}\.{HGET\_PACK}(\index{vpack kind+\\{vpack\_kind}}\\{vpack\_kind},\39\\{b111}){}$;\5
\&{break};
\Y
\fi

\M{163}


\Y\B\4\X17:get macros\X${}\mathrel+\E{}$\6
\8\#\&{define} \index{HGET SET+\.{HGET\_SET}}\.{HGET\_SET}(\|I)\6
\&{if} ${}((\|I)\AND\\{b100}){}$\5
\1${}\{{}$\5
\index{xdimen t+\&{xdimen\_t}}\&{xdimen\_t} \|x;\7
${}\index{hget xdimen node+\\{hget\_xdimen\_node}}\\{hget\_xdimen\_node}({\AND}\|x){}$;\5
${}\index{hwrite xdimen node+\\{hwrite\_xdimen\_node}}\\{hwrite\_xdimen\_node}({\AND}\|x){}$;\5
${}\}{}$\2\6
\&{else}\1\5
\index{HGET REF+\.{HGET\_REF}}\.{HGET\_REF}(\index{xdimen kind+\\{xdimen\_kind}}\\{xdimen\_kind})\2\1\6
\4${}\{{}$\5
\index{dimen t+\&{dimen\_t}}\&{dimen\_t} \|h;\5
\index{HGET32+\.{HGET32}}\.{HGET32}(\|h);\5
\index{hwrite dimension+\\{hwrite\_dimension}}\\{hwrite\_dimension}(\|h);\5
${}\}{}$\2\1\6
\4${}\{{}$\5
\index{dimen t+\&{dimen\_t}}\&{dimen\_t} \|d;\5
\&{if} ${}((\|I)\AND\\{b001}){}$\1\5
\index{HGET32+\.{HGET32}}\.{HGET32}(\|d);\5
\2\&{else}\1\5
${}\|d\K\T{0}{}$;\5
\2\index{hwrite dimension+\\{hwrite\_dimension}}\\{hwrite\_dimension}(\|d);\5
${}\}{}$\2\1\6
\4${}\{{}$\5
\index{dimen t+\&{dimen\_t}}\&{dimen\_t} \|w;\5
\index{HGET32+\.{HGET32}}\.{HGET32}(\|w);\5
\index{hwrite dimension+\\{hwrite\_dimension}}\\{hwrite\_dimension}(\|w);\5
${}\}{}$\2\6
\&{if} ${}((\|I)\AND\\{b010}){}$\5
\1${}\{{}$\5
\index{dimen t+\&{dimen\_t}}\&{dimen\_t} \|a;\5
\index{HGET32+\.{HGET32}}\.{HGET32}(\|a);\6
\index{hwritef+\\{hwritef}}\\{hwritef}(\.{"\ shifted"});\5
\index{hwrite dimension+\\{hwrite\_dimension}}\\{hwrite\_dimension}(\|a);\5
${}\}{}$\2\1\6
\4${}\{{}$\5
\index{stretch t+\&{stretch\_t}}\&{stretch\_t} \|p;\5
\index{HGET STRETCH+\.{HGET\_STRETCH}}\.{HGET\_STRETCH}(\|p);\5
${}\index{hwrite plus+\\{hwrite\_plus}}\\{hwrite\_plus}({\AND}\|p){}$;\5
${}\}{}$\2\1\6
\4${}\{{}$\5
\index{stretch t+\&{stretch\_t}}\&{stretch\_t} \|m;\5
\index{HGET STRETCH+\.{HGET\_STRETCH}}\.{HGET\_STRETCH}(\|m);\5
${}\index{hwrite minus+\\{hwrite\_minus}}\\{hwrite\_minus}({\AND}\|m){}$;\5
${}\}{}$\2\1\6
\4${}\{{}$\5
\index{list t+\&{list\_t}}\&{list\_t} \|l;\5
${}\index{hget list+\\{hget\_list}}\\{hget\_list}({\AND}\|l){}$;\5
${}\index{hwrite list+\\{hwrite\_list}}\\{hwrite\_list}({\AND}\|l){}$;\5
${}\}{}$\2\7
\8\#\&{define} $\index{HGET PACK+\.{HGET\_PACK}}\.{HGET\_PACK}(\|K,\39\|I){}$\6
\&{if} ${}((\|I)\AND\\{b001}){}$\1\5
\index{hwritef+\\{hwritef}}\\{hwritef}(\.{"\ add"});\5
\2\&{else}\1\5
\index{hwritef+\\{hwritef}}\\{hwritef}(\.{"\ to"});\2\6
\&{if} ${}((\|I)\AND\\{b100}){}$\5
\1${}\{{}$\5
\index{xdimen t+\&{xdimen\_t}}\&{xdimen\_t} \|x;\7
${}\index{hget xdimen node+\\{hget\_xdimen\_node}}\\{hget\_xdimen\_node}({\AND}\|x){}$;\5
${}\index{hwrite xdimen node+\\{hwrite\_xdimen\_node}}\\{hwrite\_xdimen\_node}({\AND}\|x){}$;\5
${}\}{}$\5
\2\&{else}\1\5
\index{HGET REF+\.{HGET\_REF}}\.{HGET\_REF}(\index{xdimen kind+\\{xdimen\_kind}}\\{xdimen\_kind});\2\6
\&{if} ${}((\|I)\AND\\{b010}){}$\5
\1${}\{{}$\5
\index{dimen t+\&{dimen\_t}}\&{dimen\_t} \|d;\5
\index{HGET32+\.{HGET32}}\.{HGET32}(\|d);\6
\index{hwritef+\\{hwritef}}\\{hwritef}(\.{"\ shifted"});\5
\index{hwrite dimension+\\{hwrite\_dimension}}\\{hwrite\_dimension}(\|d);\5
${}\}{}$\2\6
\&{if} ${}(\|K\E\index{vpack kind+\\{vpack\_kind}}\\{vpack\_kind}){}$\5
\1${}\{{}$\5
\index{dimen t+\&{dimen\_t}}\&{dimen\_t} \|d;\5
\index{HGET32+\.{HGET32}}\.{HGET32}(\|d);\5
\index{hwrite dimension+\\{hwrite\_dimension}}\\{hwrite\_dimension}(\|d);\5
${}\}{}$\2\1\6
\4${}\{{}$\5
\index{list t+\&{list\_t}}\&{list\_t} \|l;\5
${}\index{hget list+\\{hget\_list}}\\{hget\_list}({\AND}\|l){}$;\5
${}\index{hwrite list+\\{hwrite\_list}}\\{hwrite\_list}({\AND}\|l){}$;\5
${}\}{}$\2
\Y
\fi

\M{164}



\subsection{Kerns}
A kern\index{kern} is a bit of white space with a certain length. If the kern is part of a
horizontal list, the length is measured in the horizontal direction,
if it is part of a vertical list, it is measured in the vertical
direction. The length of a kern is mostly given as a dimension
but provisions are made to use extended dimensions as well.

The typical
use of a kern is its insertion between two characters to make the natural
distance between them a bit wider or smaller. In the latter case, the kern
has a negative length. The typographic optimization just described is called
``kerning'' and has given the kern node its name.
Kerns inserted from font information or math mode calculations are normal kerns,
while kerns inserted from \TeX's {\tt \BS kern} or {\tt \BS/}
commands are explicit kerns.
Kern nodes do not disappear at a line break unless they are explicit\index{explicit kern}.

In the long format, explicit kerns are marked with an ``!'' sign
and in the short format with the \\{b100} info bit.
The two low order info bits are: 0 for a reference to a dimension, 1 for a reference to
an extended dimension, 2 for an immediate dimension, and 3 for an immediate extended dimension node.
To distinguish in the long format between a reference to a dimension and a reference to an extended dimension,
the latter is prefixed with the keyword ``{\tt xdimen}'' (see section~\secref{reference}).

\Y\B\4\X1:hint types\X${}\mathrel+\E{}$\6
\&{typedef} \&{struct} ${}\{{}$\5
\1\&{bool} \|x;\5
\index{xdimen t+\&{xdimen\_t}}\&{xdimen\_t} \|d;\5
\2${}\}{}$ \index{kern t+\&{kern\_t}}\&{kern\_t};
\Y
\fi

\M{165}

\readcode
\Y\par
\par
\par
\par
\Y\B\4\X2:symbols\X${}\mathrel+\E{}$\6
\8\%\&{token} \index{KERN+\ts{KERN}}\ts{KERN}\5\.{"kern"}\6
\8\%\&{token} \index{EXPLICIT+\ts{EXPLICIT}}\ts{EXPLICIT}\5\.{"!"}\6
\8\%\index{type+\&{type}}\&{type} $<$ \|b $>$ \nts{explicit} \6
\8\%\index{type+\&{type}}\&{type} $<$ \index{kt+\\{kt}}\\{kt} $>$ \index{kern+\nts{kern}}\nts{kern}
\Y
\fi

\M{166}

\Y\B\4\X3:scanning rules\X${}\mathrel+\E{}$\6
${}\8\re{\vb{kern}}{}$\ac\&{return} \index{KERN+\ts{KERN}}\ts{KERN};\eac\7
${}\8\re{\vb{!}}{}$\ac\&{return} \index{EXPLICIT+\ts{EXPLICIT}}\ts{EXPLICIT};\eac
\Y
\fi

\M{167}

\Y\B\4\X5:parsing rules\X${}\mathrel+\E{}$\6
\nts{explicit}: \1\1\5
${}\{{}$\1\5
${}\.{\$\$}\K\\{false};{}$\5
${}\}{}$\5
\2\hbox to 0.5em{\hss${}\OR{}$}\5
\index{EXPLICIT+\ts{EXPLICIT}}\ts{EXPLICIT}\5
${}\{{}$\1\5
${}\.{\$\$}\K\\{true};{}$\5
${}\}{}$\2;\2\2\7
\index{kern+\nts{kern}}\nts{kern}: \1\1\5
\nts{explicit}\5
\index{xdimen+\nts{xdimen}}\nts{xdimen}\5
${}\{{}$\1\5
${}\.{\$\$}.\|x\K\.{\$1};{}$\5
${}\.{\$\$}.\|d\K\.{\$2};{}$\5
${}\}{}$\2;\2\2\7
\index{content node+\nts{content\_node}}\nts{content\_node}: \1\1\5
\index{start+\nts{start}}\nts{start}\5
\index{KERN+\ts{KERN}}\ts{KERN}\5
\index{kern+\nts{kern}}\nts{kern}\5
\index{END+\ts{END}}\ts{END}\5
${}\{{}$\1\5
${}\index{hput tags+\\{hput\_tags}}\\{hput\_tags}(\.{\$1},\39\index{hput kern+\\{hput\_kern}}\\{hput\_kern}({\AND}(\.{\$3})));{}$\5
${}\}{}$\2
\Y
\fi

\M{168}

\writecode
\Y\B\4\X19:write functions\X${}\mathrel+\E{}$\6
\&{void} \index{hwrite explicit+\\{hwrite\_explicit}}\\{hwrite\_explicit}(\&{bool} \|x)\1\1\2\2\1\6
\4${}\{{}$\5
\&{if} (\|x)\1\5
\index{hwritef+\\{hwritef}}\\{hwritef}(\.{"\ !"});\5
\2${}\}{}$\2\7
\&{void} \index{hwrite kern+\\{hwrite\_kern}}\\{hwrite\_kern}(\index{kern t+\&{kern\_t}}\&{kern\_t} ${}{*}\|k){}$\1\1\2\2\1\6
\4${}\{{}$\5
${}\index{hwrite explicit+\\{hwrite\_explicit}}\\{hwrite\_explicit}(\|k\MG\|x);{}$\6
\&{if} ${}(\|k\MG\|d.\|h\E\T{0.0}\W\|k\MG\|d.\|v\E\T{0.0}\W\|k\MG\|d.\|w\E\T{0}){}$\1\5
\index{hwrite ref+\\{hwrite\_ref}}\\{hwrite\_ref}(\index{zero dimen no+\\{zero\_dimen\_no}}\\{zero\_dimen\_no});\2\6
\&{else}\1\5
${}\index{hwrite xdimen+\\{hwrite\_xdimen}}\\{hwrite\_xdimen}({\AND}(\|k\MG\|d));{}$\2\6
\4${}\}{}$\2
\Y
\fi

\M{169}

\getcode
\Y\B\4\X18:cases to get content\X${}\mathrel+\E{}$\6
\4\&{case} \.{TAG}${}(\index{kern kind+\\{kern\_kind}}\\{kern\_kind},\39\\{b010}){}$:\5
\1${}\{{}$\5
\index{kern t+\&{kern\_t}}\&{kern\_t} \|k;\5
${}\index{HGET KERN+\.{HGET\_KERN}}\.{HGET\_KERN}(\\{b010},\39\|k){}$;\5
${}\}{}$\5
\2\&{break};\6
\4\&{case} \.{TAG}${}(\index{kern kind+\\{kern\_kind}}\\{kern\_kind},\39\\{b011}){}$:\5
\1${}\{{}$\5
\index{kern t+\&{kern\_t}}\&{kern\_t} \|k;\5
${}\index{HGET KERN+\.{HGET\_KERN}}\.{HGET\_KERN}(\\{b011},\39\|k){}$;\5
${}\}{}$\5
\2\&{break};\6
\4\&{case} \.{TAG}${}(\index{kern kind+\\{kern\_kind}}\\{kern\_kind},\39\\{b110}){}$:\5
\1${}\{{}$\5
\index{kern t+\&{kern\_t}}\&{kern\_t} \|k;\5
${}\index{HGET KERN+\.{HGET\_KERN}}\.{HGET\_KERN}(\\{b110},\39\|k){}$;\5
${}\}{}$\5
\2\&{break};\6
\4\&{case} \.{TAG}${}(\index{kern kind+\\{kern\_kind}}\\{kern\_kind},\39\\{b111}){}$:\5
\1${}\{{}$\5
\index{kern t+\&{kern\_t}}\&{kern\_t} \|k;\5
${}\index{HGET KERN+\.{HGET\_KERN}}\.{HGET\_KERN}(\\{b111},\39\|k){}$;\5
${}\}{}$\5
\2\&{break};
\Y
\fi

\M{170}

\Y\B\4\X17:get macros\X${}\mathrel+\E{}$\6
\8\#\&{define} $\index{HGET KERN+\.{HGET\_KERN}}\.{HGET\_KERN}(\|I,\39\|K){}$ \|K${}.\|x\K(\|I)\AND\\{b100};{}$\6
\&{if} ${}(((\|I)\AND\\{b011})\E\T{2}){}$\5
\1${}\{{}$\5
${}\index{HGET32+\.{HGET32}}\.{HGET32}(\|K.\|d.\|w){}$;\5
${}\|K.\|d.\|h\K\|K.\|d.\|v\K\T{0.0}{}$;\5
${}\}{}$\2\6
\&{else} \&{if} ${}(((\|I)\AND\\{b011})\E\T{3}){}$\1\5
${}\index{hget xdimen node+\\{hget\_xdimen\_node}}\\{hget\_xdimen\_node}({\AND}(\|K.\|d));{}$\2\6
${}\index{hwrite kern+\\{hwrite\_kern}}\\{hwrite\_kern}({\AND}\|k);$
\Y
\fi

\M{171}

\putcode
\Y\B\4\X12:put functions\X${}\mathrel+\E{}$\6
\&{uint8\_t} \index{hput kern+\\{hput\_kern}}\\{hput\_kern}(\index{kern t+\&{kern\_t}}\&{kern\_t} ${}{*}\|k){}$\1\1\2\2\1\6
\4${}\{{}$\5
\index{info t+\&{info\_t}}\&{info\_t} \index{info+\\{info}}\\{info};\7
\&{if} ${}(\|k\MG\|x){}$\1\5
${}\index{info+\\{info}}\\{info}\K\\{b100}{}$;\5
\2\&{else}\1\5
${}\index{info+\\{info}}\\{info}\K\\{b000};{}$\2\6
\&{if} ${}(\|k\MG\|d.\|h\E\T{0.0}\W\|k\MG\|d.\|v\E\T{0.0}){}$\5
\1${}\{{}$\6
\&{if} ${}(\|k\MG\|d.\|w\E\T{0}){}$\1\5
\index{HPUT8+\.{HPUT8}}\.{HPUT8}(\index{zero dimen no+\\{zero\_dimen\_no}}\\{zero\_dimen\_no});\2\6
\&{else}\5
\1${}\{{}$\5
${}\index{HPUT32+\.{HPUT32}}\.{HPUT32}(\|k\MG\|d.\|w);{}$\5
${}\index{info+\\{info}}\\{info}\K\index{info+\\{info}}\\{info}\OR\T{2}{}$;\5
${}\}{}$\2\6
\4${}\}{}$\2\6
\&{else}\5
\1${}\{{}$\5
${}\index{hput xdimen node+\\{hput\_xdimen\_node}}\\{hput\_xdimen\_node}({\AND}(\|k\MG\|d));{}$\5
${}\index{info+\\{info}}\\{info}\K\index{info+\\{info}}\\{info}\OR\T{3}{}$;\5
${}\}{}$\2\6
\&{return} \.{TAG}${}(\index{kern kind+\\{kern\_kind}}\\{kern\_kind},\39\index{info+\\{info}}\\{info});{}$\6
\4${}\}{}$\2
\Y
\fi

\M{172}


\subsection{Leaders}\label{leaders}
Leaders\index{leaders} are a special type of glue that is best explained by a few
examples.
Where as ordinary glue fills its designated space with \hfil\ whiteness,\break
leaders fill their designated space with either a rule \xleaders\hrule\hfil\ or\break
some sort of repeated\leaders\hbox to 15pt{$\hss.\hss$}\hfil content.\break
In multiple leaders, the dots\leaders\hbox to 15pt{$\hss.\hss$}\hfil are usually aligned\index{alignment} across lines,\break
as in the last\leaders\hbox to 15pt{$\hss.\hss$}\hfil three lines.\break
Unless you specify centered\index{centered}\cleaders\hbox to 15pt{$\hss.\hss$}\hfil leaders\break
or you specify expanded\index{expanded}\xleaders\hbox to 15pt{$\hss.\hss$}\hfil leaders.\break
The former pack the repeated content tight and center
the repeated content in the available space, the latter distributes
the extra space between all the repeated instances.

In the short format, the two lowest info bits store the type
of leaders: 1 for aligned, 2 for centered, and 3 for expanded.


\readcode
\Y\par
\par
\par
\par
\par
\par
\Y\B\4\X2:symbols\X${}\mathrel+\E{}$\6
\8\%\&{token} \index{LEADERS+\ts{LEADERS}}\ts{LEADERS}\5\.{"leaders"}\6
\8\%\&{token} \index{ALIGN+\ts{ALIGN}}\ts{ALIGN}\5\.{"align"}\6
\8\%\&{token} \index{CENTER+\ts{CENTER}}\ts{CENTER}\5\.{"center"}\6
\8\%\&{token} \index{EXPAND+\ts{EXPAND}}\ts{EXPAND}\5\.{"expand"}\6
\8\%\index{type+\&{type}}\&{type} $<$ \index{info+\\{info}}\\{info} $>$ \index{leaders+\nts{leaders}}\nts{leaders} \6
\8\%\index{type+\&{type}}\&{type} $<$ \index{info+\\{info}}\\{info} $>$ \index{ltype+\nts{ltype}}\nts{ltype}
\Y
\fi

\M{173}

\Y\B\4\X3:scanning rules\X${}\mathrel+\E{}$\6
${}\8\re{\vb{leaders}}{}$\ac\&{return} \index{LEADERS+\ts{LEADERS}}\ts{LEADERS};\eac\7
${}\8\re{\vb{align}}{}$\ac\&{return} \index{ALIGN+\ts{ALIGN}}\ts{ALIGN};\eac\7
${}\8\re{\vb{center}}{}$\ac\&{return} \index{CENTER+\ts{CENTER}}\ts{CENTER};\eac\7
${}\8\re{\vb{expand}}{}$\ac\&{return} \index{EXPAND+\ts{EXPAND}}\ts{EXPAND};\eac
\Y
\fi

\M{174}
\Y\B\4\X5:parsing rules\X${}\mathrel+\E{}$\6
\index{ltype+\nts{ltype}}\nts{ltype}: \1\1\5
${}\{{}$\1\5
${}\.{\$\$}\K\T{1};{}$\5
${}\}{}$\2\6
\4\hbox to 0.5em{\hss${}\OR{}$}\5
\index{ALIGN+\ts{ALIGN}}\ts{ALIGN}\5
${}\{{}$\1\5
${}\.{\$\$}\K\T{1};{}$\5
${}\}{}$\5
\2\hbox to 0.5em{\hss${}\OR{}$}\5
\index{CENTER+\ts{CENTER}}\ts{CENTER}\5
${}\{{}$\1\5
${}\.{\$\$}\K\T{2};{}$\5
${}\}{}$\5
\2\hbox to 0.5em{\hss${}\OR{}$}\5
\index{EXPAND+\ts{EXPAND}}\ts{EXPAND}\5
${}\{{}$\1\5
${}\.{\$\$}\K\T{3};{}$\5
${}\}{}$\2;\2\2\7
\index{leaders+\nts{leaders}}\nts{leaders}: \1\1\5
\index{glue node+\nts{glue\_node}}\nts{glue\_node}\5
\index{ltype+\nts{ltype}}\nts{ltype}\5
\index{rule node+\nts{rule\_node}}\nts{rule\_node}\5
${}\{{}$\1\5
${}\.{\$\$}\K\.{\$2};{}$\5
${}\}{}$\2\6
\4\hbox to 0.5em{\hss${}\OR{}$}\5
\index{glue node+\nts{glue\_node}}\nts{glue\_node}\5
\index{ltype+\nts{ltype}}\nts{ltype}\5
\index{hbox node+\nts{hbox\_node}}\nts{hbox\_node}\5
${}\{{}$\1\5
${}\.{\$\$}\K\.{\$2};{}$\5
${}\}{}$\2\6
\4\hbox to 0.5em{\hss${}\OR{}$}\5
\index{glue node+\nts{glue\_node}}\nts{glue\_node}\5
\index{ltype+\nts{ltype}}\nts{ltype}\5
\index{vbox node+\nts{vbox\_node}}\nts{vbox\_node}\5
${}\{{}$\1\5
${}\.{\$\$}\K\.{\$2};{}$\5
${}\}{}$\2;\2\2\7
\index{content node+\nts{content\_node}}\nts{content\_node}: \1\1\5
\index{start+\nts{start}}\nts{start}\5
\index{LEADERS+\ts{LEADERS}}\ts{LEADERS}\5
\index{leaders+\nts{leaders}}\nts{leaders}\5
\index{END+\ts{END}}\ts{END}\5
${}\{{}$\1\5
${}\index{hput tags+\\{hput\_tags}}\\{hput\_tags}(\.{\$1},\39\.{TAG}(\index{leaders kind+\\{leaders\_kind}}\\{leaders\_kind},\39\.{\$3}));{}$\5
${}\}{}$\2
\Y
\fi

\M{175}

\writecode
\Y\B\4\X19:write functions\X${}\mathrel+\E{}$\6
\&{void} \index{hwrite leaders type+\\{hwrite\_leaders\_type}}\\{hwrite\_leaders\_type}(\&{int} \|t)\1\1\2\2\1\6
\4${}\{{}$\5
\&{if} ${}(\|t\E\T{2}){}$\1\5
\index{hwritef+\\{hwritef}}\\{hwritef}(\.{"\ center"});\2\6
\&{else} \&{if} ${}(\|t\E\T{3}){}$\1\5
\index{hwritef+\\{hwritef}}\\{hwritef}(\.{"\ expand"});\2\6
\4${}\}{}$\2
\Y
\fi

\M{176}

\getcode
\Y\B\4\X18:cases to get content\X${}\mathrel+\E{}$\6
\4\&{case} \.{TAG}${}(\index{leaders kind+\\{leaders\_kind}}\\{leaders\_kind},\39\T{1}){}$:\5
\index{HGET LEADERS+\.{HGET\_LEADERS}}\.{HGET\_LEADERS}(\T{1});\5
\&{break};\6
\4\&{case} \.{TAG}${}(\index{leaders kind+\\{leaders\_kind}}\\{leaders\_kind},\39\T{2}){}$:\5
\index{HGET LEADERS+\.{HGET\_LEADERS}}\.{HGET\_LEADERS}(\T{2});\5
\&{break};\6
\4\&{case} \.{TAG}${}(\index{leaders kind+\\{leaders\_kind}}\\{leaders\_kind},\39\T{3}){}$:\5
\index{HGET LEADERS+\.{HGET\_LEADERS}}\.{HGET\_LEADERS}(\T{3});\5
\&{break};
\Y
\fi

\M{177}
\Y\B\4\X17:get macros\X${}\mathrel+\E{}$\6
\8\#\&{define} \index{HGET LEADERS+\.{HGET\_LEADERS}}\.{HGET\_LEADERS}(\|I)\6
\index{hget glue node+\\{hget\_glue\_node}}\\{hget\_glue\_node} (\,);\5
${}\index{hwrite leaders type+\\{hwrite\_leaders\_type}}\\{hwrite\_leaders\_type}((\|I)\AND\\{b011});{}$\6
\&{if} ${}(\index{KIND+\.{KIND}}\.{KIND}({*}\index{hpos+\\{hpos}}\\{hpos})\E\index{rule kind+\\{rule\_kind}}\\{rule\_kind}){}$\1\5
\index{hget rule node+\\{hget\_rule\_node}}\\{hget\_rule\_node}(\,);\2\6
\&{else} \&{if} ${}(\index{KIND+\.{KIND}}\.{KIND}({*}\index{hpos+\\{hpos}}\\{hpos})\E\index{hbox kind+\\{hbox\_kind}}\\{hbox\_kind}){}$\1\5
\index{hget hbox node+\\{hget\_hbox\_node}}\\{hget\_hbox\_node}(\,);\2\6
\&{else}\1\5
\index{hget vbox node+\\{hget\_vbox\_node}}\\{hget\_vbox\_node}(\,);\2
\Y
\fi

\M{178}

\subsection{Baseline Skips}
Baseline\index{baseline skip} skips are small amounts of glue inserted between two consecutive lines
of text. To get nice looking pages, the amount of glue\index{glue} inserted must take into
account the depth of the line above the glue and the height of the line below the
glue to achieve a constant distance of the baselines. For example, if we have the lines
\medskip

\qquad\vbox{\hsize=0.5\hsize\noindent
``There is no\hfil\break
more gas\hfil\break
in the tank.''
}\hss

\medskip\noindent
\TeX\ will insert 7.69446pt of baseline skip between the first and the second line and
3.11111pt of baseline skip between the second and the third line. This is due to the
fact that the first line has no descenders, its depth is zero, the second line
has no ascenders but the ``g'' descends below the baseline, and the third line
has ascenders (``t'', ``h'',\dots) so it is higher than the second line.
\TeX's choice of baseline skips ensures that the baselines are exactly 12pt apart
in both cases.

Things get more complicated if the text contains mathematical formulas because then
a line can get so high or deep that it is impossible to keep the distance between
baselines constant without two adjacent lines touching each other. In such cases,
\TeX\ will insert a small minimum line skip glue\index{line skip glue}.

For the whole computation, \TeX\ uses three parameters: {\tt baselineskip},
{\tt line\-skip\-limit},\index{line skip limit} and {\tt lineskip}.
\.{baselineskip} is a glue value; its size is the normal distance of two baselines.
\TeX\ adjusts the size of the \.{baselineskip} glue for the height and the depth of
the two lines and then checks the result against \.{lineskiplimit}.
If the result is smaller than \.{lineskiplimit} it will use the \.{lineskip} glue
instead.

Because the depth and the height of lines depend on the outcome of the line breaking\index{line breaking}
routine, baseline computations must be done in the viewer.
The situation gets even more complicated because \TeX\ can manipulate the insertion
of baseline skips in various ways. Therefore \HINT/ requires the insertion of
baseline nodes wherever the viewer is supposed to perform a baseline skip
computation.

In the short format of a baseline definition, we store only the nonzero components and use the
info bits to mark them: \\{b100} implies $\\{bs}\ne0$,
\\{b010} implies $\\{ls}\ne 0$, and \\{b001} implies  $\index{lslimit+\\{lslimit}}\\{lslimit}\ne 0$.
If the baseline has only zero components, we put a reference to baseline number 0
in the output.

\Y\B\4\X6:hint basic types\X${}\mathrel+\E{}$\6
\&{typedef} \&{struct} ${}\{{}$\5
\1\index{glue t+\&{glue\_t}}\&{glue\_t} \\{bs}${},{}$ \\{ls};\5
\index{dimen t+\&{dimen\_t}}\&{dimen\_t} \\{lsl};\5
\2${}\}{}$ \index{baseline t+\&{baseline\_t}}\&{baseline\_t};
\Y
\fi

\M{179}



\readcode
\Y\par
\par
\Y\B\4\X2:symbols\X${}\mathrel+\E{}$\6
\8\%\&{token} \index{BASELINE+\ts{BASELINE}}\ts{BASELINE}\5\.{"baseline"}\6
\8\%\index{type+\&{type}}\&{type} $<$ \index{info+\\{info}}\\{info} $>$ \index{baseline+\nts{baseline}}\nts{baseline}
\Y
\fi

\M{180}
\Y\B\4\X3:scanning rules\X${}\mathrel+\E{}$\6
${}\8\re{\vb{baseline}}{}$\ac\&{return} \index{BASELINE+\ts{BASELINE}}\ts{BASELINE};\eac
\Y
\fi

\M{181}

\Y\B\4\X5:parsing rules\X${}\mathrel+\E{}$\6
\index{baseline+\nts{baseline}}\nts{baseline}: \1\1\5
\index{glue node+\nts{glue\_node}}\nts{glue\_node}\5
\index{glue node+\nts{glue\_node}}\nts{glue\_node}\5
\index{dimension+\nts{dimension}}\nts{dimension}\6
${}\{{}$\1\5
${}\.{\$\$}\K\\{b000};{}$\6
\&{if} (\.{\$1})\1\5
${}\.{\$\$}\MRL{{\OR}{\K}}\\{b100};{}$\2\6
\&{if} (\.{\$2})\1\5
${}\.{\$\$}\MRL{{\OR}{\K}}\\{b010};{}$\2\6
\&{if} ${}(\.{\$3}\I\T{0}){}$\5
\1${}\{{}$\5
\index{HPUT32+\.{HPUT32}}\.{HPUT32}(\.{\$3});\5
${}\.{\$\$}\MRL{{\OR}{\K}}\\{b001}{}$;\5
${}\}{}$\5
\2${}\}{}$\2;\2\2\7
\index{content node+\nts{content\_node}}\nts{content\_node}: \1\1\5
\index{start+\nts{start}}\nts{start}\5
\index{BASELINE+\ts{BASELINE}}\ts{BASELINE}\5
\index{baseline+\nts{baseline}}\nts{baseline}\5
\index{END+\ts{END}}\ts{END}\6
${}\{{}$\5
\1\&{if} ${}(\.{\$3}\E\\{b000}){}$\1\5
\index{HPUT8+\.{HPUT8}}\.{HPUT8}(\T{0});\5
\2${}\index{hput tags+\\{hput\_tags}}\\{hput\_tags}(\.{\$1},\39\.{TAG}(\index{baseline kind+\\{baseline\_kind}}\\{baseline\_kind},\39\.{\$3}));{}$\5
${}\}{}$\2;\2\2
\Y
\fi

\M{182}

\getcode
\Y\B\4\X18:cases to get content\X${}\mathrel+\E{}$\6
\4\&{case} \.{TAG}${}(\index{baseline kind+\\{baseline\_kind}}\\{baseline\_kind},\39\\{b001}){}$:\5
\1${}\{{}$\5
\index{baseline t+\&{baseline\_t}}\&{baseline\_t} \|b;\5
${}\index{HGET BASELINE+\.{HGET\_BASELINE}}\.{HGET\_BASELINE}(\\{b001},\39\|b){}$;\5
${}\}{}$\5
\2\&{break};\6
\4\&{case} \.{TAG}${}(\index{baseline kind+\\{baseline\_kind}}\\{baseline\_kind},\39\\{b010}){}$:\5
\1${}\{{}$\5
\index{baseline t+\&{baseline\_t}}\&{baseline\_t} \|b;\5
${}\index{HGET BASELINE+\.{HGET\_BASELINE}}\.{HGET\_BASELINE}(\\{b010},\39\|b){}$;\5
${}\}{}$\5
\2\&{break};\6
\4\&{case} \.{TAG}${}(\index{baseline kind+\\{baseline\_kind}}\\{baseline\_kind},\39\\{b011}){}$:\5
\1${}\{{}$\5
\index{baseline t+\&{baseline\_t}}\&{baseline\_t} \|b;\5
${}\index{HGET BASELINE+\.{HGET\_BASELINE}}\.{HGET\_BASELINE}(\\{b011},\39\|b){}$;\5
${}\}{}$\5
\2\&{break};\6
\4\&{case} \.{TAG}${}(\index{baseline kind+\\{baseline\_kind}}\\{baseline\_kind},\39\\{b100}){}$:\5
\1${}\{{}$\5
\index{baseline t+\&{baseline\_t}}\&{baseline\_t} \|b;\5
${}\index{HGET BASELINE+\.{HGET\_BASELINE}}\.{HGET\_BASELINE}(\\{b100},\39\|b){}$;\5
${}\}{}$\5
\2\&{break};\6
\4\&{case} \.{TAG}${}(\index{baseline kind+\\{baseline\_kind}}\\{baseline\_kind},\39\\{b101}){}$:\5
\1${}\{{}$\5
\index{baseline t+\&{baseline\_t}}\&{baseline\_t} \|b;\5
${}\index{HGET BASELINE+\.{HGET\_BASELINE}}\.{HGET\_BASELINE}(\\{b101},\39\|b){}$;\5
${}\}{}$\5
\2\&{break};\6
\4\&{case} \.{TAG}${}(\index{baseline kind+\\{baseline\_kind}}\\{baseline\_kind},\39\\{b110}){}$:\5
\1${}\{{}$\5
\index{baseline t+\&{baseline\_t}}\&{baseline\_t} \|b;\5
${}\index{HGET BASELINE+\.{HGET\_BASELINE}}\.{HGET\_BASELINE}(\\{b110},\39\|b){}$;\5
${}\}{}$\5
\2\&{break};\6
\4\&{case} \.{TAG}${}(\index{baseline kind+\\{baseline\_kind}}\\{baseline\_kind},\39\\{b111}){}$:\5
\1${}\{{}$\5
\index{baseline t+\&{baseline\_t}}\&{baseline\_t} \|b;\5
${}\index{HGET BASELINE+\.{HGET\_BASELINE}}\.{HGET\_BASELINE}(\\{b111},\39\|b){}$;\5
${}\}{}$\5
\2\&{break};
\Y
\fi

\M{183}

\Y\B\4\X17:get macros\X${}\mathrel+\E{}$\6
\8\#\&{define} $\index{HGET BASELINE+\.{HGET\_BASELINE}}\.{HGET\_BASELINE}(\|I,\39\|B)$ \6
\&{if} ${}((\|I)\AND\\{b100}){}$\1\5
\index{hget glue node+\\{hget\_glue\_node}}\\{hget\_glue\_node}(\,);\2\6
\&{else}\5
\1${}\{{}$\5
${}\|B.\\{bs}.\|p.\|o\K\|B.\\{bs}.\|m.\|o\K\|B.\\{bs}.\|w.\|w\K\T{0}{}$;\5
${}\|B.\\{bs}.\|w.\|h\K\|B.\\{bs}.\|w.\|v\K\|B.\\{bs}.\|p.\|f\K\|B.\\{bs}.\|m.\|f\K\T{0.0}{}$;\5
${}\index{hwrite glue node+\\{hwrite\_glue\_node}}\\{hwrite\_glue\_node}({\AND}(\|B.\\{bs})){}$;\5
${}\}{}$\2\6
\&{if} ${}((\|I)\AND\\{b010}){}$\1\5
\index{hget glue node+\\{hget\_glue\_node}}\\{hget\_glue\_node}(\,);\2\6
\&{else}\5
\1${}\{{}$\5
${}\|B.\\{ls}.\|p.\|o\K\|B.\\{ls}.\|m.\|o\K\|B.\\{ls}.\|w.\|w\K\T{0}{}$;\5
${}\|B.\\{ls}.\|w.\|h\K\|B.\\{ls}.\|w.\|v\K\|B.\\{ls}.\|p.\|f\K\|B.\\{ls}.\|m.\|f\K\T{0.0}{}$;\5
${}\index{hwrite glue node+\\{hwrite\_glue\_node}}\\{hwrite\_glue\_node}({\AND}(\|B.\\{ls})){}$;\5
${}\}{}$\2\6
\&{if} ${}((\|I)\AND\\{b001}){}$\1\5
${}\index{HGET32+\.{HGET32}}\.{HGET32}((\|B).\\{lsl}){}$;\5
\2\&{else}\1\5
${}\|B.\\{lsl}\K\T{0};{}$\2\6
${}\index{hwrite dimension+\\{hwrite\_dimension}}\\{hwrite\_dimension}(\|B.\\{lsl});$
\Y
\fi

\M{184}


\putcode
\Y\B\4\X12:put functions\X${}\mathrel+\E{}$\6
\&{uint8\_t} \index{hput baseline+\\{hput\_baseline}}\\{hput\_baseline}(\index{baseline t+\&{baseline\_t}}\&{baseline\_t} ${}{*}\|b){}$\1\1\2\2\1\6
\4${}\{{}$\5
\index{info t+\&{info\_t}}\&{info\_t} \index{info+\\{info}}\\{info}${}\K\\{b000};{}$\7
\&{if} ${}(\R\index{ZERO GLUE+\.{ZERO\_GLUE}}\.{ZERO\_GLUE}(\|b\MG\\{bs}){}$)\5
\1${}\index{info+\\{info}}\\{info}\MRL{{\OR}{\K}}\\{b100};{}$\2\6
\&{if} ${}(\R\index{ZERO GLUE+\.{ZERO\_GLUE}}\.{ZERO\_GLUE}(\|b\MG\\{ls}){}$)\5
\1${}\index{info+\\{info}}\\{info}\MRL{{\OR}{\K}}\\{b010};{}$\2\6
\&{if} ${}(\|b\MG\\{lsl}\I\T{0}){}$\5
\1${}\{{}$\5
${}\index{HPUT32+\.{HPUT32}}\.{HPUT32}(\|b\MG\\{lsl}){}$;\5
${}\index{info+\\{info}}\\{info}\MRL{{\OR}{\K}}\\{b001}{}$;\5
${}\}{}$\2\6
\&{return} \.{TAG}${}(\index{baseline kind+\\{baseline\_kind}}\\{baseline\_kind},\39\index{info+\\{info}}\\{info});{}$\6
\4${}\}{}$\2
\Y
\fi

\M{185}



\subsection{Ligatures}
Ligatures\index{ligature} occur only in horizontal lists.  They specify characters
that combines the glyphs of several characters into one specialized
glyph. For example in the word ``{\it difficult\/}'' the three letters
``{\it f{}f{}i\/}'' are combined into the ligature ``{\it ffi\/}''.
Hence, a ligature is very similar to a simple glyph node; the
characters that got replaced are, however, retained in the ligature
because they might be needed for example to support searching. Since
ligatures are therefore only specialized list of characters and since
we have a very efficient way to store such lists of characters, namely
as a \index{text+\nts{text}}\nts{text}, input and output of ligatures is quite simple.

The info value zero is reserved for references to a ligature.  If the
info value is between 1 and 6, it gives the number of bytes used to encode
the characters in UTF8.  Note that a ligature will always include a
glyph byte, so the minimum size is 1. A typical ligature like ``{\it fi\/}''
will need 3 byte: the ligature character ``{\it fi\/}'', and
the replacement characters ``f'' and ''i''. More byte might be
required if the character codes exceed \T{\^7F}, since we use the UTF8
encoding scheme for larger character codes.  If the info value is 7,
an additional byte following the font byte and preceding the end byte
gives the total size needed for the character codes.  In the long
format, we give the font, the character code, and then the replacement
characters coded in utf8.

\Y\B\4\X1:hint types\X${}\mathrel+\E{}$\6
\&{typedef} \&{struct} ${}\{{}$\5
\1\&{uint8\_t} \|f;\5
\index{list t+\&{list\_t}}\&{list\_t} \|l;\5
\2${}\}{}$ \index{lig t+\&{lig\_t}}\&{lig\_t};
\Y
\fi

\M{186}

\readcode
\Y\par
\par
\par
\par
\par
\Y\B\4\X2:symbols\X${}\mathrel+\E{}$\6
\8\%\&{token} \index{LIGATURE+\ts{LIGATURE}}\ts{LIGATURE}\5\.{"ligature"}\6
\8\%\index{type+\&{type}}\&{type} $<$ \|u $>$ \index{lig cc+\nts{lig\_cc}}\nts{lig\_cc} \6
\8\%\index{type+\&{type}}\&{type} $<$ \index{lg+\\{lg}}\\{lg} $>$ \index{ligature+\nts{ligature}}\nts{ligature} \6
\8\%\index{type+\&{type}}\&{type} $<$ \|u $>$ \index{ref+\nts{ref}}\nts{ref}
\Y
\fi

\M{187}
\Y\B\4\X3:scanning rules\X${}\mathrel+\E{}$\6
${}\8\re{\vb{ligature}}{}$\ac\&{return} \index{LIGATURE+\ts{LIGATURE}}\ts{LIGATURE};\eac
\Y
\fi

\M{188}

\Y\B\4\X5:parsing rules\X${}\mathrel+\E{}$\6
\index{replace cc+\nts{replace\_cc}}\nts{replace\_cc}:\5
\1\1\hbox to 0.5em{\hss${}\OR{}$}\5
\index{replace cc+\nts{replace\_cc}}\nts{replace\_cc}\5
\index{TXT CC+\ts{TXT\_CC}}\ts{TXT\_CC}\5
${}\{{}$\1\5
\index{hput utf8+\\{hput\_utf8}}\\{hput\_utf8}(\.{\$2});\5
${}\}{}$\2;\2\2\7
\index{lig cc+\nts{lig\_cc}}\nts{lig\_cc}: \1\1\5
\index{UNSIGNED+\ts{UNSIGNED}}\ts{UNSIGNED}\5
${}\{{}$\1\5
${}\.{\$\$}\K\index{hpos+\\{hpos}}\\{hpos}-\index{hstart+\\{hstart}}\\{hstart};{}$\5
\index{hput utf8+\\{hput\_utf8}}\\{hput\_utf8}(\.{\$1});\5
${}\}{}$\2;\2\2\7
\index{lig cc+\nts{lig\_cc}}\nts{lig\_cc}: \1\1\5
\index{CHARCODE+\ts{CHARCODE}}\ts{CHARCODE}\5
${}\{{}$\1\5
${}\.{\$\$}\K\index{hpos+\\{hpos}}\\{hpos}-\index{hstart+\\{hstart}}\\{hstart};{}$\5
\index{hput utf8+\\{hput\_utf8}}\\{hput\_utf8}(\.{\$1});\5
${}\}{}$\2;\2\2\7
\index{ref+\nts{ref}}\nts{ref}: \1\1\5
\index{REFERENCE+\ts{REFERENCE}}\ts{REFERENCE}\5
${}\{{}$\1\5
\index{HPUT8+\.{HPUT8}}\.{HPUT8}(\.{\$1});\5
${}\.{\$\$}\K\.{\$1};{}$\5
${}\}{}$\2;\2\2\7
\index{ligature+\nts{ligature}}\nts{ligature}: \1\1\5
\index{ref+\nts{ref}}\nts{ref}\5
${}\{{}$\1\5
${}\index{REF+\.{REF}}\.{REF}(\index{font kind+\\{font\_kind}}\\{font\_kind},\39\.{\$1});{}$\5
${}\}{}$\2\5
\index{lig cc+\nts{lig\_cc}}\nts{lig\_cc}\5
\index{TXT START+\ts{TXT\_START}}\ts{TXT\_START}\5
\index{replace cc+\nts{replace\_cc}}\nts{replace\_cc}\5
\index{TXT END+\ts{TXT\_END}}\ts{TXT\_END}\6
${}\{{}$\1\5
${}\.{\$\$}.\|f\K\.{\$1};{}$\5
${}\.{\$\$}.\|l.\|p\K\.{\$3};{}$\5
${}\.{\$\$}.\|l.\|s\K(\index{hpos+\\{hpos}}\\{hpos}-\index{hstart+\\{hstart}}\\{hstart})-\.{\$3};{}$\5
${}\.{RNG}(\.{"Ligature\ size"},\39\.{\$\$}.\|l.\|s,\39\T{0},\39\T{255});{}$\5
${}\}{}$\2;\2\2\7
\index{content node+\nts{content\_node}}\nts{content\_node}: \1\1\5
\index{start+\nts{start}}\nts{start}\5
\index{LIGATURE+\ts{LIGATURE}}\ts{LIGATURE}\5
\index{ligature+\nts{ligature}}\nts{ligature}\5
\index{END+\ts{END}}\ts{END}\5
${}\{{}$\1\5
${}\index{hput tags+\\{hput\_tags}}\\{hput\_tags}(\.{\$1},\39\index{hput ligature+\\{hput\_ligature}}\\{hput\_ligature}({\AND}(\.{\$3})));{}$\5
${}\}{}$\2;\2\2
\Y
\fi

\M{189}

\writecode
\Y\B\4\X19:write functions\X${}\mathrel+\E{}$\6
\&{void} \index{hwrite ligature+\\{hwrite\_ligature}}\\{hwrite\_ligature}(\index{lig t+\&{lig\_t}}\&{lig\_t} ${}{*}\|l){}$\1\1\2\2\1\6
\4${}\{{}$\5
\&{uint32\_t} \index{pos+\\{pos}}\\{pos}${}\K\index{hpos+\\{hpos}}\\{hpos}-\index{hstart+\\{hstart}}\\{hstart};{}$\7
${}\index{hwrite ref+\\{hwrite\_ref}}\\{hwrite\_ref}(\|l\MG\|f);{}$\6
${}\index{hpos+\\{hpos}}\\{hpos}\K\|l\MG\|l.\|p+\index{hstart+\\{hstart}}\\{hstart};{}$\6
\index{hwrite charcode+\\{hwrite\_charcode}}\\{hwrite\_charcode}(\index{hget utf8+\\{hget\_utf8}}\\{hget\_utf8}(\,));\6
\index{hwritef+\\{hwritef}}\\{hwritef}(\.{"\ \\""});\6
\&{while} ${}(\index{hpos+\\{hpos}}\\{hpos}<\index{hstart+\\{hstart}}\\{hstart}+\|l\MG\|l.\|p+\|l\MG\|l.\|s){}$\1\5
\index{hwrite txt cc+\\{hwrite\_txt\_cc}}\\{hwrite\_txt\_cc}(\index{hget utf8+\\{hget\_utf8}}\\{hget\_utf8}(\,));\2\6
\index{hwritec+\\{hwritec}}\\{hwritec}(\.{'"'});\6
${}\index{hpos+\\{hpos}}\\{hpos}\K\index{hstart+\\{hstart}}\\{hstart}+\index{pos+\\{pos}}\\{pos};{}$\6
\4${}\}{}$\2
\Y
\fi

\M{190}

\getcode
\Y\B\4\X18:cases to get content\X${}\mathrel+\E{}$\6
\4\&{case} \.{TAG}${}(\index{ligature kind+\\{ligature\_kind}}\\{ligature\_kind},\39\T{1}){}$:\5
\1${}\{{}$\5
\index{lig t+\&{lig\_t}}\&{lig\_t} \|l;\5
${}\index{HGET LIG+\.{HGET\_LIG}}\.{HGET\_LIG}(\T{1},\39\|l){}$;\5
${}\}{}$\5
\2\&{break};\6
\4\&{case} \.{TAG}${}(\index{ligature kind+\\{ligature\_kind}}\\{ligature\_kind},\39\T{2}){}$:\5
\1${}\{{}$\5
\index{lig t+\&{lig\_t}}\&{lig\_t} \|l;\5
${}\index{HGET LIG+\.{HGET\_LIG}}\.{HGET\_LIG}(\T{2},\39\|l){}$;\5
${}\}{}$\5
\2\&{break};\6
\4\&{case} \.{TAG}${}(\index{ligature kind+\\{ligature\_kind}}\\{ligature\_kind},\39\T{3}){}$:\5
\1${}\{{}$\5
\index{lig t+\&{lig\_t}}\&{lig\_t} \|l;\5
${}\index{HGET LIG+\.{HGET\_LIG}}\.{HGET\_LIG}(\T{3},\39\|l){}$;\5
${}\}{}$\5
\2\&{break};\6
\4\&{case} \.{TAG}${}(\index{ligature kind+\\{ligature\_kind}}\\{ligature\_kind},\39\T{4}){}$:\5
\1${}\{{}$\5
\index{lig t+\&{lig\_t}}\&{lig\_t} \|l;\5
${}\index{HGET LIG+\.{HGET\_LIG}}\.{HGET\_LIG}(\T{4},\39\|l){}$;\5
${}\}{}$\5
\2\&{break};\6
\4\&{case} \.{TAG}${}(\index{ligature kind+\\{ligature\_kind}}\\{ligature\_kind},\39\T{5}){}$:\5
\1${}\{{}$\5
\index{lig t+\&{lig\_t}}\&{lig\_t} \|l;\5
${}\index{HGET LIG+\.{HGET\_LIG}}\.{HGET\_LIG}(\T{5},\39\|l){}$;\5
${}\}{}$\5
\2\&{break};\6
\4\&{case} \.{TAG}${}(\index{ligature kind+\\{ligature\_kind}}\\{ligature\_kind},\39\T{6}){}$:\5
\1${}\{{}$\5
\index{lig t+\&{lig\_t}}\&{lig\_t} \|l;\5
${}\index{HGET LIG+\.{HGET\_LIG}}\.{HGET\_LIG}(\T{6},\39\|l){}$;\5
${}\}{}$\5
\2\&{break};\6
\4\&{case} \.{TAG}${}(\index{ligature kind+\\{ligature\_kind}}\\{ligature\_kind},\39\T{7}){}$:\5
\1${}\{{}$\5
\index{lig t+\&{lig\_t}}\&{lig\_t} \|l;\5
${}\index{HGET LIG+\.{HGET\_LIG}}\.{HGET\_LIG}(\T{7},\39\|l){}$;\5
${}\}{}$\5
\2\&{break};
\Y
\fi

\M{191}
\Y\B\4\X17:get macros\X${}\mathrel+\E{}$\6
\8\#\&{define} $\index{HGET LIG+\.{HGET\_LIG}}\.{HGET\_LIG}(\|I,\39\|L){}$\6
(\|L)${}.\|f\K\index{HGET8+\.{HGET8}}\.{HGET8};{}$\6
${}\index{REF+\.{REF}}\.{REF}(\index{font kind+\\{font\_kind}}\\{font\_kind},\39(\|L).\|f);{}$\6
\&{if} ${}((\|I)\E\T{7}){}$\1\5
${}(\|L).\|l.\|s\K\index{HGET8+\.{HGET8}}\.{HGET8}{}$;\5
\2\&{else}\1\5
${}(\|L).\|l.\|s\K(\|I);{}$\2\6
${}(\|L).\|l.\|p\K\index{hpos+\\{hpos}}\\{hpos}-\index{hstart+\\{hstart}}\\{hstart}{}$;\5
${}\index{hpos+\\{hpos}}\\{hpos}\MRL{+{\K}}(\|L).\|l.\|s;{}$\6
\&{if} ${}((\|I)\E\T{7}{}$)\6
\1${}\{{}$\5
\&{uint8\_t} \|n${}\K\index{HGET8+\.{HGET8}}\.{HGET8};{}$\7
\&{if} ${}(\|n\I(\|L).\|l.\|s{}$)\1\6
${}\.{QUIT}(\.{"Sizes\ in\ ligature\ d}\)\.{on't\ match\ \%d!=\%d"},\39(\|L).\|l.\|s,\39\|n);{}$\2\6
\4${}\}{}$\2\6
${}\index{hwrite ligature+\\{hwrite\_ligature}}\\{hwrite\_ligature}({\AND}(\|L));$
\Y
\fi

\M{192}

\putcode
\Y\B\4\X12:put functions\X${}\mathrel+\E{}$\6
\&{uint8\_t} \index{hput ligature+\\{hput\_ligature}}\\{hput\_ligature}(\index{lig t+\&{lig\_t}}\&{lig\_t} ${}{*}\|l){}$\1\1\2\2\1\6
\4${}\{{}$\5
\&{if} ${}(\|l\MG\|l.\|s<\T{7}){}$\1\5
\&{return} \.{TAG}${}(\index{ligature kind+\\{ligature\_kind}}\\{ligature\_kind},\39\|l\MG\|l.\|s);{}$\2\6
\&{else}\6
\1${}\{{}$\5
${}\index{memmove+\\{memmove}}\\{memmove}(\index{hstart+\\{hstart}}\\{hstart}+\|l\MG\|l.\|p+\T{1},\39\index{hstart+\\{hstart}}\\{hstart}+\|l\MG\|l.\|p,\39\|l\MG\|l.\|s){}$;\5
${}\index{hpos+\\{hpos}}\\{hpos}\PP;{}$\6
${}{*}(\index{hstart+\\{hstart}}\\{hstart}+\|l\MG\|l.\|p)\K{*}\index{hpos+\\{hpos}}\\{hpos}\PP\K\|l\MG\|l.\|s;{}$\6
\&{return} \.{TAG}${}(\index{ligature kind+\\{ligature\_kind}}\\{ligature\_kind},\39\T{7});{}$\6
\4${}\}{}$\2\6
\4${}\}{}$\2
\Y
\fi

\M{193}


\subsection{Hyphenation}\label{hyphen}\index{hyphen}
\HINT/ is capable to break lines into paragraphs. It does this
primarily at interword spaces but it might also break a line in the
middle of a word if it finds a discretionary\index{discretionary break}
line break there. These discretionary breaks are usually
provided by an automatic hyphenation algorithm but they might be also
explicitly\index{explicit hyphen} inserted by the author of a
document.

When a line break occurs at such a discretionary break, the line
before the break ends with a \index{pre break+\\{pre\_break}}\\{pre\_break} list of nodes, the line after
the break starts with a \index{post break+\\{post\_break}}\\{post\_break} list of nodes, and the next
\index{replace count+\\{replace\_count}}\\{replace\_count} nodes after the discretionary break will be
ignored. Both lists must consist entirely of glyphs\index{glyph},
kerns\index{kern}, boxes\index{box}, rules\index{rule}, or
ligatures\index{ligature}.  For example, an ordinary discretionary
hyphen will have a \index{pre break+\\{pre\_break}}\\{pre\_break} list containing ``-'', an empty
\index{post break+\\{post\_break}}\\{post\_break} list, and a \index{replace count+\\{replace\_count}}\\{replace\_count} of zero.

The long format starts with an optional ``{\tt !}'' indicating an explicit hyphen,
followed by the  \index{pre break+\\{pre\_break}}\\{pre\_break} list, then comes the replace-count followed
by the  \index{post break+\\{post\_break}}\\{post\_break} list. An empty  \index{pre break+\\{pre\_break}}\\{pre\_break} or \index{post break+\\{post\_break}}\\{post\_break} list may be omitted.

In the short format, the three components of a hyphen node are stored in this order:
\index{pre break+\\{pre\_break}}\\{pre\_break} list, \index{post break+\\{post\_break}}\\{post\_break} list, and  \index{replace count+\\{replace\_count}}\\{replace\_count}.
The \\{b100} bit in the info value indicates the presence of a
\index{pre break+\\{pre\_break}}\\{pre\_break} list, the \\{b010} bit the presence of a \index{post break+\\{post\_break}}\\{post\_break} list, and
the \\{b001} bit the presence of a replace-count.
Since the info value \\{b000} is
reserved for references, at least one of these must be specified; so we represent
a node with empty lists and a replace\index{replace count} count of zero using the info value \\{b001}
and a zero byte for the replace count.

Replace counts must be in the range 0 to 31; so the short format can set the high bit
of the replace count to indicate an explicit hyphen.

\Y\B\4\X1:hint types\X${}\mathrel+\E{}$\6
\&{typedef} \&{struct} \index{hyphen t+\&{hyphen\_t}}\&{hyphen\_t}\5
${}\{{}$\5
\1\&{bool} \|x;\5
\index{list t+\&{list\_t}}\&{list\_t} \|p${},{}$ \|q;\5
\&{uint8\_t} \|r;\5
\2${}\}{}$ \index{hyphen t+\&{hyphen\_t}}\&{hyphen\_t};
\Y
\fi

\M{194}


\readcode
\Y\par
\par
\par
\par
\Y\B\4\X2:symbols\X${}\mathrel+\E{}$\6
\8\%\&{token} \index{HYPHEN+\ts{HYPHEN}}\ts{HYPHEN}\5\.{"hyphen"}\6
\8\%\index{type+\&{type}}\&{type} $<$ \index{hy+\\{hy}}\\{hy} $>$ \index{hyphen+\nts{hyphen}}\nts{hyphen}\5
\index{hyphen node+\nts{hyphen\_node}}\nts{hyphen\_node} \6
\8\%\index{type+\&{type}}\&{type} $<$ \|l $>$ \index{opt list+\nts{opt\_list}}\nts{opt\_list}
\Y
\fi

\M{195}
\Y\B\4\X3:scanning rules\X${}\mathrel+\E{}$\6
${}\8\re{\vb{hyphen}}{}$\ac\&{return} \index{HYPHEN+\ts{HYPHEN}}\ts{HYPHEN};\eac
\Y
\fi

\M{196}

\Y\B\4\X5:parsing rules\X${}\mathrel+\E{}$\6
\index{opt list+\nts{opt\_list}}\nts{opt\_list}: \1\1\5
${}\{{}$\1\5
${}\.{\$\$}.\|p\K\index{hpos+\\{hpos}}\\{hpos}-\index{hstart+\\{hstart}}\\{hstart};{}$\5
${}\.{\$\$}.\|s\K\T{0};{}$\5
${}\.{\$\$}.\|k\K\index{list kind+\\{list\_kind}}\\{list\_kind};{}$\5
${}\}{}$\2\6
\4\hbox to 0.5em{\hss${}\OR{}$}\5
\index{list+\nts{list}}\nts{list}\5
${}\{{}$\5
\1\&{if} ${}(\.{\$1}.\|s\E\T{0}){}$\1\5
${}\index{hpos+\\{hpos}}\\{hpos}\K\index{hpos+\\{hpos}}\\{hpos}-\T{2}{}$;\5
\2${}\.{\$\$}\K\.{\$1};{}$\5
${}\}{}$\2;\2\2\7
\index{hyphen+\nts{hyphen}}\nts{hyphen}: \1\1\5
\nts{explicit}\5
\index{opt list+\nts{opt\_list}}\nts{opt\_list}\5
\index{UNSIGNED+\ts{UNSIGNED}}\ts{UNSIGNED}\5
\index{opt list+\nts{opt\_list}}\nts{opt\_list}\6
${}\{{}$\1\5
${}\.{\$\$}.\|x\K\.{\$1};{}$\5
${}\.{\$\$}.\|p\K\.{\$2};{}$\5
${}\.{RNG}(\.{"Replace\ count"},\39\.{\$3},\39\T{0},\39\T{31});{}$\5
${}\.{\$\$}.\|r\K\.{\$3};{}$\5
${}\.{\$\$}.\|q\K\.{\$4};{}$\5
${}\}{}$\2;\2\2\7
\index{hyphen node+\nts{hyphen\_node}}\nts{hyphen\_node}: \1\1\5
\index{start+\nts{start}}\nts{start}\5
\index{HYPHEN+\ts{HYPHEN}}\ts{HYPHEN}\5
\index{hyphen+\nts{hyphen}}\nts{hyphen}\5
\index{END+\ts{END}}\ts{END}\6
${}\{{}$\1\5
${}\index{hput tags+\\{hput\_tags}}\\{hput\_tags}(\.{\$1},\39\index{hput hyphen+\\{hput\_hyphen}}\\{hput\_hyphen}({\AND}(\.{\$3})));{}$\5
${}\.{\$\$}\K\.{\$3};{}$\5
${}\}{}$\2;\2\2\7
\index{content node+\nts{content\_node}}\nts{content\_node}: \1\1\5
\index{hyphen node+\nts{hyphen\_node}}\nts{hyphen\_node};\2\2
\Y
\fi

\M{197}

\writecode
\Y\B\4\X19:write functions\X${}\mathrel+\E{}$\6
\&{void} \index{hwrite hyphen+\\{hwrite\_hyphen}}\\{hwrite\_hyphen}(\index{hyphen t+\&{hyphen\_t}}\&{hyphen\_t} ${}{*}\|h){}$\1\1\2\2\1\6
\4${}\{{}$\5
${}\index{hwrite explicit+\\{hwrite\_explicit}}\\{hwrite\_explicit}(\|h\MG\|x);{}$\6
\&{if} ${}(\|h\MG\|p.\|s\I\T{0}){}$\1\5
${}\index{hwrite list+\\{hwrite\_list}}\\{hwrite\_list}({\AND}(\|h\MG\|p));{}$\2\6
${}\index{hwritef+\\{hwritef}}\\{hwritef}(\.{"\ \%d"},\39\|h\MG\|r);{}$\6
\&{if} ${}(\|h\MG\|q.\|s\I\T{0}){}$\1\5
${}\index{hwrite list+\\{hwrite\_list}}\\{hwrite\_list}({\AND}(\|h\MG\|q));{}$\2\6
\4${}\}{}$\2\7
\&{void} \index{hwrite hyphen node+\\{hwrite\_hyphen\_node}}\\{hwrite\_hyphen\_node}(\index{hyphen t+\&{hyphen\_t}}\&{hyphen\_t} ${}{*}\|h){}$\1\1\2\2\1\6
\4${}\{{}$\5
\index{hwrite start+\\{hwrite\_start}}\\{hwrite\_start}(\,);\5
\index{hwritef+\\{hwritef}}\\{hwritef}(\.{"hyphen"});\5
\index{hwrite hyphen+\\{hwrite\_hyphen}}\\{hwrite\_hyphen}(\|h);\5
\index{hwrite end+\\{hwrite\_end}}\\{hwrite\_end}(\,);\6
\4${}\}{}$\2
\Y
\fi

\M{198}

\getcode
\Y\B\4\X18:cases to get content\X${}\mathrel+\E{}$\6
\4\&{case} \.{TAG}${}(\index{hyphen kind+\\{hyphen\_kind}}\\{hyphen\_kind},\39\\{b001}){}$:\1\6
\4${}\{{}$\5
\index{hyphen t+\&{hyphen\_t}}\&{hyphen\_t} \|h;\5
${}\index{HGET HYPHEN+\.{HGET\_HYPHEN}}\.{HGET\_HYPHEN}(\\{b001},\39\|h){}$;\5
${}\index{hwrite hyphen+\\{hwrite\_hyphen}}\\{hwrite\_hyphen}({\AND}\|h){}$;\5
${}\}{}$\5
\2\&{break};\6
\4\&{case} \.{TAG}${}(\index{hyphen kind+\\{hyphen\_kind}}\\{hyphen\_kind},\39\\{b010}){}$:\1\6
\4${}\{{}$\5
\index{hyphen t+\&{hyphen\_t}}\&{hyphen\_t} \|h;\5
${}\index{HGET HYPHEN+\.{HGET\_HYPHEN}}\.{HGET\_HYPHEN}(\\{b010},\39\|h){}$;\5
${}\index{hwrite hyphen+\\{hwrite\_hyphen}}\\{hwrite\_hyphen}({\AND}\|h){}$;\5
${}\}{}$\5
\2\&{break};\6
\4\&{case} \.{TAG}${}(\index{hyphen kind+\\{hyphen\_kind}}\\{hyphen\_kind},\39\\{b011}){}$:\1\6
\4${}\{{}$\5
\index{hyphen t+\&{hyphen\_t}}\&{hyphen\_t} \|h;\5
${}\index{HGET HYPHEN+\.{HGET\_HYPHEN}}\.{HGET\_HYPHEN}(\\{b011},\39\|h){}$;\5
${}\index{hwrite hyphen+\\{hwrite\_hyphen}}\\{hwrite\_hyphen}({\AND}\|h){}$;\5
${}\}{}$\5
\2\&{break};\6
\4\&{case} \.{TAG}${}(\index{hyphen kind+\\{hyphen\_kind}}\\{hyphen\_kind},\39\\{b100}){}$:\1\6
\4${}\{{}$\5
\index{hyphen t+\&{hyphen\_t}}\&{hyphen\_t} \|h;\5
${}\index{HGET HYPHEN+\.{HGET\_HYPHEN}}\.{HGET\_HYPHEN}(\\{b100},\39\|h){}$;\5
${}\index{hwrite hyphen+\\{hwrite\_hyphen}}\\{hwrite\_hyphen}({\AND}\|h){}$;\5
${}\}{}$\5
\2\&{break};\6
\4\&{case} \.{TAG}${}(\index{hyphen kind+\\{hyphen\_kind}}\\{hyphen\_kind},\39\\{b101}){}$:\1\6
\4${}\{{}$\5
\index{hyphen t+\&{hyphen\_t}}\&{hyphen\_t} \|h;\5
${}\index{HGET HYPHEN+\.{HGET\_HYPHEN}}\.{HGET\_HYPHEN}(\\{b101},\39\|h){}$;\5
${}\index{hwrite hyphen+\\{hwrite\_hyphen}}\\{hwrite\_hyphen}({\AND}\|h){}$;\5
${}\}{}$\5
\2\&{break};\6
\4\&{case} \.{TAG}${}(\index{hyphen kind+\\{hyphen\_kind}}\\{hyphen\_kind},\39\\{b110}){}$:\1\6
\4${}\{{}$\5
\index{hyphen t+\&{hyphen\_t}}\&{hyphen\_t} \|h;\5
${}\index{HGET HYPHEN+\.{HGET\_HYPHEN}}\.{HGET\_HYPHEN}(\\{b110},\39\|h){}$;\5
${}\index{hwrite hyphen+\\{hwrite\_hyphen}}\\{hwrite\_hyphen}({\AND}\|h){}$;\5
${}\}{}$\5
\2\&{break};\6
\4\&{case} \.{TAG}${}(\index{hyphen kind+\\{hyphen\_kind}}\\{hyphen\_kind},\39\\{b111}){}$:\1\6
\4${}\{{}$\5
\index{hyphen t+\&{hyphen\_t}}\&{hyphen\_t} \|h;\5
${}\index{HGET HYPHEN+\.{HGET\_HYPHEN}}\.{HGET\_HYPHEN}(\\{b111},\39\|h){}$;\5
${}\index{hwrite hyphen+\\{hwrite\_hyphen}}\\{hwrite\_hyphen}({\AND}\|h){}$;\5
${}\}{}$\5
\2\&{break};
\Y
\fi

\M{199}

\Y\B\4\X17:get macros\X${}\mathrel+\E{}$\6
\8\#\&{define} $\index{HGET HYPHEN+\.{HGET\_HYPHEN}}\.{HGET\_HYPHEN}(\|I,\39\|Y)$ \6
\&{if} ${}((\|I)\AND\\{b100}){}$\1\5
${}\index{hget list+\\{hget\_list}}\\{hget\_list}({\AND}((\|Y).\|p));{}$\2\6
\&{else}\5
\1${}\{{}$\5
${}(\|Y).\|p.\|p\K\index{hpos+\\{hpos}}\\{hpos}-\index{hstart+\\{hstart}}\\{hstart}{}$;\5
${}(\|Y).\|p.\|s\K\T{0}{}$;\5
${}(\|Y).\|p.\|k\K\index{list kind+\\{list\_kind}}\\{list\_kind}{}$;\5
${}\}{}$\2\6
\&{if} ${}((\|I)\AND\\{b010}){}$\1\5
${}\index{hget list+\\{hget\_list}}\\{hget\_list}({\AND}((\|Y).\|q));{}$\2\6
\&{else}\5
\1${}\{{}$\5
${}(\|Y).\|q.\|p\K\index{hpos+\\{hpos}}\\{hpos}-\index{hstart+\\{hstart}}\\{hstart}{}$;\5
${}(\|Y).\|q.\|s\K\T{0}{}$;\5
${}(\|Y).\|q.\|k\K\index{list kind+\\{list\_kind}}\\{list\_kind}{}$;\5
${}\}{}$\2\6
\&{if} ${}((\|I)\AND\\{b001}){}$\5
\1${}\{{}$\5
\&{uint8\_t} \|r${}\K\index{HGET8+\.{HGET8}}\.{HGET8};{}$\7
${}(\|Y).\|r\K\|r\AND\T{\^7F}{}$;\5
${}\.{RNG}(\.{"Replace\ count"},\39(\|Y).\|r,\39\T{0},\39\T{31}){}$;\5
${}(\|Y).\|x\K(\|r\AND\T{\^80})\I\T{0}{}$;\5
${}\}{}$\5
\2\&{else}\5
\1${}\{{}$\5
${}(\|Y).\|r\K\T{0}{}$;\5
${}(\|Y).\|x\K\\{false}{}$;\5
${}\}{}$\2
\Y
\fi

\M{200}

\Y\B\4\X16:get functions\X${}\mathrel+\E{}$\6
\&{void} \index{hget hyphen node+\\{hget\_hyphen\_node}}\\{hget\_hyphen\_node}(\index{hyphen t+\&{hyphen\_t}}\&{hyphen\_t} ${}{*}\|h){}$\1\1\2\2\1\6
\4${}\{{}$\5
\X14:read the start byte \|a\X\6
\&{if} ${}(\index{KIND+\.{KIND}}\.{KIND}(\|a)\I\index{hyphen kind+\\{hyphen\_kind}}\\{hyphen\_kind}\V\index{INFO+\.{INFO}}\.{INFO}(\|a)\E\\{b000}){}$\1\5
${}\.{QUIT}(\.{"Hyphen\ expected\ at\ }\)\.{0x\%x\ got\ \%s,\%d"},\39\\{node\_pos},\39\index{NAME+\.{NAME}}\.{NAME}(\|a),\39\index{INFO+\.{INFO}}\.{INFO}(\|a));{}$\2\6
${}\index{HGET HYPHEN+\.{HGET\_HYPHEN}}\.{HGET\_HYPHEN}(\index{INFO+\.{INFO}}\.{INFO}(\|a),\39{*}\|h);{}$\6
\X15:read and check the end byte \|z\X\6
\4${}\}{}$\2
\Y
\fi

\M{201}

\putcode
\Y\B\4\X12:put functions\X${}\mathrel+\E{}$\6
\&{uint8\_t} \index{hput hyphen+\\{hput\_hyphen}}\\{hput\_hyphen}(\index{hyphen t+\&{hyphen\_t}}\&{hyphen\_t} ${}{*}\|h){}$\1\1\2\2\1\6
\4${}\{{}$\5
\index{info t+\&{info\_t}}\&{info\_t} \index{info+\\{info}}\\{info}${}\K\\{b000};{}$\7
\&{if} ${}(\|h\MG\|p.\|s>\T{0}){}$\1\5
${}\index{info+\\{info}}\\{info}\MRL{{\OR}{\K}}\\{b100};{}$\2\6
\&{if} ${}(\|h\MG\|q.\|s>\T{0}){}$\1\5
${}\index{info+\\{info}}\\{info}\MRL{{\OR}{\K}}\\{b010};{}$\2\6
\&{if} ${}(\|h\MG\|x\V\|h\MG\|r\I\T{0}\V\index{info+\\{info}}\\{info}\E\\{b000}{}$)\6
\1${}\{{}$\5
${}\index{info+\\{info}}\\{info}\MRL{{\OR}{\K}}\\{b001}{}$;\5
${}\index{HPUT8+\.{HPUT8}}\.{HPUT8}(\|h\MG\|r\OR(\|h\MG\|x\?\T{\^80}:\T{\^00})){}$;\5
${}\}{}$\2\6
\&{return} \.{TAG}${}(\index{hyphen kind+\\{hyphen\_kind}}\\{hyphen\_kind},\39\index{info+\\{info}}\\{info});{}$\6
\4${}\}{}$\2
\Y
\fi

\M{202}
\subsection{Paragraphs}
The most important procedure that the \HINT/ viewer inherits from \TeX\ is the
line breaking routine. If the horizontal size of the paragraph is not known,
breaking the paragraph\index{paragraph} into lines must be postponed and this is done by creating
a paragraph node. The paragraph node must contain all information that \TeX's
line breaking\index{line breaking} algorithm needs to do its job.

Besides the horizontal list describing the content of the paragraph and
the xdimen describing the horizontal size,
this is the set of parameters that guide the line breaking algorithm:

\itemize
\item
Integer parameters:\hfill\break
{\tt pretolerance} (badness tolerance before hyphenation),\hfill\break
{\tt tolerance} (badness tolerance after hyphenation),\hfill\break
{\tt line\_penalty} (added to the badness of every line, increase to get fewer lines),\hfill\break
{\tt hy\-phen\_pe\-nal\-ty} (penalty for break after discretionary hyphen),\hfill\break
{\tt ex\_hy\-phen\_pe\-nal\-ty} (penalty for break after explicit hyphen),\hfill\break
{\tt doub\-le\_hy\-phen\_de\-merits} (demerits for double hyphen break),\hfill\break
{\tt final\_hyphen\_de\-me\-rits} (demerits for final hyphen break),\hfill\break
{\tt adj\_demerits} (demerits for adjacent incompatible lines),\hfill\break
{\tt looseness} (make the paragraph that many lines longer than its optimal size),\hfill\break
{\tt inter\_line\_penalty} (additional penalty between lines),\hfill\break
{\tt club\_pe\-nal\-ty} (penalty for creating a club line),\hfill\break
{\tt widow\_penalty} (penalty for creating a widow line),\hfill\break
{\tt display\_widow\_penalty} (ditto, just before a display),\hfill\break
{\tt bro\-ken\_pe\-nal\-ty} (penalty for breaking a page at a broken line),\hfill\break
{\tt hang\_af\-ter} (start/end hanging indentation at this line).
\item
Dimension parameters:\hfill\break
{\tt line\_skip\_limit} (threshold for {\tt line\_skip} instead of {\tt base\-line\_skip}),\hfill\break
{\tt hang\_in\-dent} (amount of hanging indentation),\hfill\break
{\tt emergency\_stretch} (stretchability added to every line in the final pass of line breaking).
\item
Glue parameters:\hfill\break
{\tt baseline\_skip} (desired glue between baselines),\hfill\break
{\tt line\_skip} (interline glue if {\tt baseline\_skip} is infeasible),\hfill\break
{\tt left\_skip} (glue at left of justified lines),\hfill\break
{\tt right\_skip} (glue at right of justified lines),\hfill\break
{\tt par\_fill\_skip} (glue on last line of paragraph).
\enditemize


For a detailed explanation of these parameters and how they influence line breaking, you should consult the  {\TeX}book\cite{DK:texbook}; \TeX's {\tt parshape} feature is currently not implemented.
There are default values for all of these parameters (see section~\secref{defaults}); and therefore
it might not be necessary to specify any of them. Any local adjustments
are contained in a list of parameters contained in the paragraph node.

A further complication is a displayed\index{displayed formula} formula that interrupts a paragraph.
Displays are described in the next section.

To summarize, a paragraph node in the long format specifies an extended dimension, an
optional node list, and an optional parameter list.
The extended dimension is given either as a reference or as an \index{xdimen+\nts{xdimen}}\nts{xdimen} node (info bit \\{b100});
the same holds for the parameter list (info bit \\{b010}).


\readcode
\Y\par
\par
\par
\par
\Y\B\4\X2:symbols\X${}\mathrel+\E{}$\6
\8\%\&{token} \index{PAR+\ts{PAR}}\ts{PAR}\5\.{"par"}\6
\8\%\index{type+\&{type}}\&{type} $<$ \index{info+\\{info}}\\{info} $>$ \index{par+\nts{par}}\nts{par}
\Y
\fi

\M{203}

\Y\B\4\X3:scanning rules\X${}\mathrel+\E{}$\6
${}\8\re{\vb{par}}{}$\ac\&{return} \index{PAR+\ts{PAR}}\ts{PAR};\eac
\Y
\fi

\M{204}

\Y\B\4\X5:parsing rules\X${}\mathrel+\E{}$\6
\index{par+\nts{par}}\nts{par}: \1\1\5
\index{xdimen ref+\nts{xdimen\_ref}}\nts{xdimen\_ref}\5
\index{param ref+\nts{param\_ref}}\nts{param\_ref}\5
\index{list+\nts{list}}\nts{list}\5
${}\{{}$\1\5
${}\.{\$\$}\K\\{b000};{}$\5
${}\}{}$\2\6
\4\hbox to 0.5em{\hss${}\OR{}$}\5
\index{xdimen ref+\nts{xdimen\_ref}}\nts{xdimen\_ref}\5
\index{param list node+\nts{param\_list\_node}}\nts{param\_list\_node}\5
\index{list+\nts{list}}\nts{list}\5
${}\{{}$\1\5
${}\.{\$\$}\K\\{b010};{}$\5
${}\}{}$\2\6
\4\hbox to 0.5em{\hss${}\OR{}$}\5
\index{xdimen ref+\nts{xdimen\_ref}}\nts{xdimen\_ref}\5
\index{list+\nts{list}}\nts{list}\5
${}\{{}$\1\5
${}\.{\$\$}\K\\{b010};{}$\5
${}\}{}$\2\6
\4\hbox to 0.5em{\hss${}\OR{}$}\5
\index{xdimen node+\nts{xdimen\_node}}\nts{xdimen\_node}\5
\index{param ref+\nts{param\_ref}}\nts{param\_ref}\5
\index{list+\nts{list}}\nts{list}\5
${}\{{}$\1\5
${}\.{\$\$}\K\\{b100};{}$\5
${}\}{}$\2\6
\4\hbox to 0.5em{\hss${}\OR{}$}\5
\index{xdimen node+\nts{xdimen\_node}}\nts{xdimen\_node}\5
\index{param list node+\nts{param\_list\_node}}\nts{param\_list\_node}\5
\index{list+\nts{list}}\nts{list}\5
${}\{{}$\1\5
${}\.{\$\$}\K\\{b110};{}$\5
${}\}{}$\2\6
\4\hbox to 0.5em{\hss${}\OR{}$}\5
\index{xdimen node+\nts{xdimen\_node}}\nts{xdimen\_node}\5
\index{list+\nts{list}}\nts{list}\5
${}\{{}$\1\5
${}\.{\$\$}\K\\{b110};{}$\5
${}\}{}$\2;\2\2\7
\index{content node+\nts{content\_node}}\nts{content\_node}: \1\1\5
\index{start+\nts{start}}\nts{start}\5
\index{PAR+\ts{PAR}}\ts{PAR}\5
\index{par+\nts{par}}\nts{par}\5
\index{END+\ts{END}}\ts{END}\5
${}\{{}$\1\5
${}\index{hput tags+\\{hput\_tags}}\\{hput\_tags}(\.{\$1},\39\.{TAG}(\index{par kind+\\{par\_kind}}\\{par\_kind},\39\.{\$3}));{}$\5
${}\}{}$\2;\2\2
\Y
\fi

\M{205}

\getcode
\Y\B\4\X18:cases to get content\X${}\mathrel+\E{}$\6
\4\&{case} \.{TAG}${}(\index{par kind+\\{par\_kind}}\\{par\_kind},\39\\{b000}){}$:\5
\index{HGET PAR+\.{HGET\_PAR}}\.{HGET\_PAR}(\\{b000});\5
\&{break};\6
\4\&{case} \.{TAG}${}(\index{par kind+\\{par\_kind}}\\{par\_kind},\39\\{b010}){}$:\5
\index{HGET PAR+\.{HGET\_PAR}}\.{HGET\_PAR}(\\{b010});\5
\&{break};\6
\4\&{case} \.{TAG}${}(\index{par kind+\\{par\_kind}}\\{par\_kind},\39\\{b100}){}$:\5
\index{HGET PAR+\.{HGET\_PAR}}\.{HGET\_PAR}(\\{b100});\5
\&{break};\6
\4\&{case} \.{TAG}${}(\index{par kind+\\{par\_kind}}\\{par\_kind},\39\\{b110}){}$:\5
\index{HGET PAR+\.{HGET\_PAR}}\.{HGET\_PAR}(\\{b110});\5
\&{break};
\Y
\fi

\M{206}

\Y\B\4\X17:get macros\X${}\mathrel+\E{}$\6
\8\#\&{define} \index{HGET PAR+\.{HGET\_PAR}}\.{HGET\_PAR}(\|I)\6
\&{if} ${}((\|I)\AND\\{b100}){}$\5
\1${}\{{}$\5
\index{xdimen t+\&{xdimen\_t}}\&{xdimen\_t} \|x;\7
${}\index{hget xdimen node+\\{hget\_xdimen\_node}}\\{hget\_xdimen\_node}({\AND}\|x){}$;\5
${}\index{hwrite xdimen node+\\{hwrite\_xdimen\_node}}\\{hwrite\_xdimen\_node}({\AND}\|x){}$;\5
${}\}{}$\2\6
\&{else}\1\5
\index{HGET REF+\.{HGET\_REF}}\.{HGET\_REF}(\index{xdimen kind+\\{xdimen\_kind}}\\{xdimen\_kind});\2\6
\&{if} ${}((\|I)\AND\\{b010}){}$\5
\1${}\{{}$\5
\index{list t+\&{list\_t}}\&{list\_t} \|l;\5
${}\index{hget param list node+\\{hget\_param\_list\_node}}\\{hget\_param\_list\_node}({\AND}\|l){}$;\5
${}\index{hwrite param list node+\\{hwrite\_param\_list\_node}}\\{hwrite\_param\_list\_node}({\AND}\|l){}$;\5
${}\}{}$\2\6
\&{else}\1\5
\index{HGET REF+\.{HGET\_REF}}\.{HGET\_REF}(\index{param kind+\\{param\_kind}}\\{param\_kind});\2\1\6
\4${}\{{}$\5
\index{list t+\&{list\_t}}\&{list\_t} \|l;\5
${}\index{hget list+\\{hget\_list}}\\{hget\_list}({\AND}\|l){}$;\5
${}\index{hwrite list+\\{hwrite\_list}}\\{hwrite\_list}({\AND}\|l){}$;\5
${}\}{}$\2
\Y
\fi

\M{207}


\subsection{Mathematics}\index{Mathematics}\index{displayed formula}
Being able to handle mathematics\index{mathematics} nicely is one
of the primary features of \TeX\ and
so you should expect the same from \HINT/.
We start here with the more complex case---displayed formulas---and finish with the
simpler case of mathematical formulas that are part of the normal flow of text.

Displayed equations occur inside a paragraph\index{paragraph}
node. They interrupt normal processing of the paragraph and the
paragraph processing is resumed after the display. Positioning of the
display depends on several parameters, the shape of the paragraph, and
the length of the last line preceding the display.  Displayed formulas
often feature an equation number which can be placed either left or
right of the formula.  Also the size of the equation number will
influence the placement of the formula.

In a \HINT/ file, the parameter list is followed by a list of content
nodes, representing the formula, and an optional horizontal box
containing the equation number.

In the sort format, we use the info bit \\{b100} to indicate the
presence of a parameter list (which might be empty---so it's actually the absence of a
reference to a parameter list); the info bit \\{b010} to indicate the presence of
a left equation number; and the info bit \\{b001} for a right
equation\index{equation number} number.

In the long format, we use ``{\tt eqno}'' or ``{\tt left eqno}'' to indicate presence and
placement of the equation number.

\readcode
\Y\par
\par
\Y\B\4\X2:symbols\X${}\mathrel+\E{}$\6
\8\%\&{token} \index{MATH+\ts{MATH}}\ts{MATH}\5\.{"math"}\6
\8\%\index{type+\&{type}}\&{type} $<$ \index{info+\\{info}}\\{info} $>$ \index{math+\nts{math}}\nts{math}
\Y
\fi

\M{208}

\Y\B\4\X3:scanning rules\X${}\mathrel+\E{}$\6
${}\8\re{\vb{math}}{}$\ac\&{return} \index{MATH+\ts{MATH}}\ts{MATH};\eac
\Y
\fi

\M{209}

\Y\B\4\X5:parsing rules\X${}\mathrel+\E{}$\6
\index{math+\nts{math}}\nts{math}: \1\1\5
\index{list+\nts{list}}\nts{list}\5
${}\{{}$\1\5
${}\.{\$\$}\K\\{b100};{}$\5
${}\}{}$\2\6
\4\hbox to 0.5em{\hss${}\OR{}$}\5
\index{list+\nts{list}}\nts{list}\5
\index{hbox node+\nts{hbox\_node}}\nts{hbox\_node}\5
${}\{{}$\1\5
${}\.{\$\$}\K\\{b101};{}$\5
${}\}{}$\2\6
\4\hbox to 0.5em{\hss${}\OR{}$}\5
\index{hbox node+\nts{hbox\_node}}\nts{hbox\_node}\5
\index{list+\nts{list}}\nts{list}\5
${}\{{}$\1\5
${}\.{\$\$}\K\\{b110};{}$\5
${}\}{}$\2\6
\4\hbox to 0.5em{\hss${}\OR{}$}\5
\index{param ref+\nts{param\_ref}}\nts{param\_ref}\5
\index{list+\nts{list}}\nts{list}\5
${}\{{}$\1\5
${}\.{\$\$}\K\\{b000};{}$\5
${}\}{}$\2\6
\4\hbox to 0.5em{\hss${}\OR{}$}\5
\index{param ref+\nts{param\_ref}}\nts{param\_ref}\5
\index{list+\nts{list}}\nts{list}\5
\index{hbox node+\nts{hbox\_node}}\nts{hbox\_node}\5
${}\{{}$\1\5
${}\.{\$\$}\K\\{b001};{}$\5
${}\}{}$\2\6
\4\hbox to 0.5em{\hss${}\OR{}$}\5
\index{param ref+\nts{param\_ref}}\nts{param\_ref}\5
\index{hbox node+\nts{hbox\_node}}\nts{hbox\_node}\5
\index{list+\nts{list}}\nts{list}\5
${}\{{}$\1\5
${}\.{\$\$}\K\\{b010};{}$\5
${}\}{}$\2\6
\4\hbox to 0.5em{\hss${}\OR{}$}\5
\index{param list node+\nts{param\_list\_node}}\nts{param\_list\_node}\5
\index{list+\nts{list}}\nts{list}\5
${}\{{}$\1\5
${}\.{\$\$}\K\\{b100};{}$\5
${}\}{}$\2\6
\4\hbox to 0.5em{\hss${}\OR{}$}\5
\index{param list node+\nts{param\_list\_node}}\nts{param\_list\_node}\5
\index{list+\nts{list}}\nts{list}\5
\index{hbox node+\nts{hbox\_node}}\nts{hbox\_node}\5
${}\{{}$\1\5
${}\.{\$\$}\K\\{b101};{}$\5
${}\}{}$\2\6
\4\hbox to 0.5em{\hss${}\OR{}$}\5
\index{param list node+\nts{param\_list\_node}}\nts{param\_list\_node}\5
\index{hbox node+\nts{hbox\_node}}\nts{hbox\_node}\5
\index{list+\nts{list}}\nts{list}\5
${}\{{}$\1\5
${}\.{\$\$}\K\\{b110};{}$\5
${}\}{}$\2;\2\2\7
\index{content node+\nts{content\_node}}\nts{content\_node}: \1\1\5
\index{start+\nts{start}}\nts{start}\5
\index{MATH+\ts{MATH}}\ts{MATH}\5
\index{math+\nts{math}}\nts{math}\5
\index{END+\ts{END}}\ts{END}\6
${}\{{}$\1\5
${}\index{hput tags+\\{hput\_tags}}\\{hput\_tags}(\.{\$1},\39\.{TAG}(\index{math kind+\\{math\_kind}}\\{math\_kind},\39\.{\$3}));{}$\5
${}\}{}$\2;\2\2
\Y
\fi

\M{210}

\getcode
\Y\B\4\X18:cases to get content\X${}\mathrel+\E{}$\6
\4\&{case} \.{TAG}${}(\index{math kind+\\{math\_kind}}\\{math\_kind},\39\\{b000}){}$:\5
\index{HGET MATH+\.{HGET\_MATH}}\.{HGET\_MATH}(\\{b000});\5
\&{break};\6
\4\&{case} \.{TAG}${}(\index{math kind+\\{math\_kind}}\\{math\_kind},\39\\{b001}){}$:\5
\index{HGET MATH+\.{HGET\_MATH}}\.{HGET\_MATH}(\\{b001});\5
\&{break};\6
\4\&{case} \.{TAG}${}(\index{math kind+\\{math\_kind}}\\{math\_kind},\39\\{b010}){}$:\5
\index{HGET MATH+\.{HGET\_MATH}}\.{HGET\_MATH}(\\{b010});\5
\&{break};\6
\4\&{case} \.{TAG}${}(\index{math kind+\\{math\_kind}}\\{math\_kind},\39\\{b100}){}$:\5
\index{HGET MATH+\.{HGET\_MATH}}\.{HGET\_MATH}(\\{b100});\5
\&{break};\6
\4\&{case} \.{TAG}${}(\index{math kind+\\{math\_kind}}\\{math\_kind},\39\\{b101}){}$:\5
\index{HGET MATH+\.{HGET\_MATH}}\.{HGET\_MATH}(\\{b101});\5
\&{break};\6
\4\&{case} \.{TAG}${}(\index{math kind+\\{math\_kind}}\\{math\_kind},\39\\{b110}){}$:\5
\index{HGET MATH+\.{HGET\_MATH}}\.{HGET\_MATH}(\\{b110});\5
\&{break};
\Y
\fi

\M{211}

\Y\B\4\X17:get macros\X${}\mathrel+\E{}$\6
\8\#\&{define} \index{HGET MATH+\.{HGET\_MATH}}\.{HGET\_MATH}(\|I) \6
\&{if} ${}((\|I)\AND\\{b100}){}$\5
\1${}\{{}$\5
\index{list t+\&{list\_t}}\&{list\_t} \|l;\5
${}\index{hget param list node+\\{hget\_param\_list\_node}}\\{hget\_param\_list\_node}({\AND}\|l){}$;\5
${}\index{hwrite param list node+\\{hwrite\_param\_list\_node}}\\{hwrite\_param\_list\_node}({\AND}\|l){}$;\5
${}\}{}$\2\6
\&{else}\1\5
\index{HGET REF+\.{HGET\_REF}}\.{HGET\_REF}(\index{param kind+\\{param\_kind}}\\{param\_kind});\2\6
\&{if} ${}((\|I)\AND\\{b010}){}$\1\5
\index{hget hbox node+\\{hget\_hbox\_node}}\\{hget\_hbox\_node}(\,);\2\1\6
\4${}\{{}$\5
\index{list t+\&{list\_t}}\&{list\_t} \|l;\5
${}\index{hget list+\\{hget\_list}}\\{hget\_list}({\AND}\|l){}$;\5
${}\index{hwrite list+\\{hwrite\_list}}\\{hwrite\_list}({\AND}\|l){}$;\5
${}\}{}$\2\6
\&{if} ${}((\|I)\AND\\{b001}){}$\1\5
\index{hget hbox node+\\{hget\_hbox\_node}}\\{hget\_hbox\_node}(\,);\2
\Y
\fi

\M{212}

Things are much simpler if mathematical formulas are embedded in regular text.
Here it is just necessary to mark the beginning and the end of the formula
because glue inside a formula is not a possible point for a line break.
To break the line within a formula you can insert a penalty node.

In the long format, such a simple math node just consists of the keyword ``on''
or ``off''. In the short format, there are two info values still unassigned:
we use \\{b011} for ``off'' and \\{b111} for ``on''.


\readcode
\Y\par
\par
\Y\B\4\X2:symbols\X${}\mathrel+\E{}$\6
\8\%\&{token} \index{ON+\ts{ON}}\ts{ON}\5\.{"on"}\6
\8\%\&{token} \index{OFF+\ts{OFF}}\ts{OFF}\5\.{"off"}
\Y
\fi

\M{213}

\Y\B\4\X3:scanning rules\X${}\mathrel+\E{}$\6
${}\8\re{\vb{on}}{}$\ac\&{return} \index{ON+\ts{ON}}\ts{ON};\eac\7
${}\8\re{\vb{off}}{}$\ac\&{return} \index{OFF+\ts{OFF}}\ts{OFF};\eac
\Y
\fi

\M{214}

\Y\B\4\X5:parsing rules\X${}\mathrel+\E{}$\6
\index{math+\nts{math}}\nts{math}: \1\1\5
\index{ON+\ts{ON}}\ts{ON}\5
${}\{{}$\1\5
${}\.{\$\$}\K\\{b111};{}$\5
${}\}{}$\2;\2\2\7
\index{math+\nts{math}}\nts{math}: \1\1\5
\index{OFF+\ts{OFF}}\ts{OFF}\5
${}\{{}$\1\5
${}\.{\$\$}\K\\{b011};{}$\5
${}\}{}$\2;\2\2
\Y
\fi

\M{215}

\getcode
\Y\B\4\X18:cases to get content\X${}\mathrel+\E{}$\6
\4\&{case} \.{TAG}${}(\index{math kind+\\{math\_kind}}\\{math\_kind},\39\\{b111}){}$:\5
\index{hwritef+\\{hwritef}}\\{hwritef}(\.{"\ on"});\5
\&{break};\6
\4\&{case} \.{TAG}${}(\index{math kind+\\{math\_kind}}\\{math\_kind},\39\\{b011}){}$:\5
\index{hwritef+\\{hwritef}}\\{hwritef}(\.{"\ off"});\5
\&{break};
\Y
\fi

\M{216}

Note that \TeX\ allows math nodes to specify a width (the current value of
mathsouround). If this with is nonzero, it is equivalent to inserting a
kern node before the math on node or after the math off node.

\subsection{Adjustments}\label{adjust}
An adjustment\index{adjustment} occurs only in paragraphs\index{paragraph}.
When the line breaking\index{line breaking} routine finds an adjustment, it inserts
the vertical material contained in the adjustment node right after the current line.
Adjustments are implemented as just another type of list node.

\readcode
\Y\par
\par
\Y\B\4\X2:symbols\X${}\mathrel+\E{}$\6
\8\%\&{token} \ts{ADJUST}\5\.{"adjust"}\6
\8\%\index{type+\&{type}}\&{type} $<$ \|l $>$ \index{adjustment+\nts{adjustment}}\nts{adjustment}
\Y
\fi

\M{217}

\Y\B\4\X3:scanning rules\X${}\mathrel+\E{}$\6
${}\8\re{\vb{adjust}}{}$\ac\&{return} \ts{ADJUST};\eac
\Y
\fi

\M{218}

\Y\B\4\X5:parsing rules\X${}\mathrel+\E{}$\6
\index{adjustment+\nts{adjustment}}\nts{adjustment}: \1\1\5
\index{estimate+\nts{estimate}}\nts{estimate}\5
\index{content list+\nts{content\_list}}\nts{content\_list}\5
${}\{{}$\5
\1${}\.{\$\$}.\|k\K\\{adjust\_kind}{}$;\5
${}\.{\$\$}.\|p\K\.{\$2}{}$;\5
${}\.{\$\$}.\|s\K(\index{hpos+\\{hpos}}\\{hpos}-\index{hstart+\\{hstart}}\\{hstart})-\.{\$2};{}$\5
${}\}{}$\2;\2\2\7
\index{content node+\nts{content\_node}}\nts{content\_node}: \1\1\5
\index{start+\nts{start}}\nts{start}\5
\ts{ADJUST}\5
\index{adjustment+\nts{adjustment}}\nts{adjustment}\5
\index{END+\ts{END}}\ts{END}\5
${}\{{}$\1\5
${}\index{hput tags+\\{hput\_tags}}\\{hput\_tags}(\.{\$1},\39\index{hput list+\\{hput\_list}}\\{hput\_list}(\.{\$1}+\T{1},\39{\AND}(\.{\$3})));{}$\5
${}\}{}$\2;\2\2
\Y
\fi

\M{219}

\getcode
\Y\B\4\X18:cases to get content\X${}\mathrel+\E{}$\6
\4\&{case} \.{TAG}${}(\\{adjust\_kind},\39\T{1}){}$:\1\6
\4${}\{{}$\5
\index{list t+\&{list\_t}}\&{list\_t} \|l;\5
${}\index{HGET LIST+\.{HGET\_LIST}}\.{HGET\_LIST}(\T{1},\39\|l){}$;\5
${}\|l.\|k\K\\{adjust\_kind}{}$;\5
${}\index{hwrite adjustments+\\{hwrite\_adjustments}}\\{hwrite\_adjustments}({\AND}\|l){}$;\5
${}\}{}$\5
\2\&{break};\6
\4\&{case} \.{TAG}${}(\\{adjust\_kind},\39\T{2}){}$:\1\6
\4${}\{{}$\5
\index{list t+\&{list\_t}}\&{list\_t} \|l;\5
${}\index{HGET LIST+\.{HGET\_LIST}}\.{HGET\_LIST}(\T{2},\39\|l){}$;\5
${}\|l.\|k\K\\{adjust\_kind}{}$;\5
${}\index{hwrite adjustments+\\{hwrite\_adjustments}}\\{hwrite\_adjustments}({\AND}\|l){}$;\5
${}\}{}$\5
\2\&{break};\6
\4\&{case} \.{TAG}${}(\\{adjust\_kind},\39\T{3}){}$:\1\6
\4${}\{{}$\5
\index{list t+\&{list\_t}}\&{list\_t} \|l;\5
${}\index{HGET LIST+\.{HGET\_LIST}}\.{HGET\_LIST}(\T{3},\39\|l){}$;\5
${}\|l.\|k\K\\{adjust\_kind}{}$;\5
${}\index{hwrite adjustments+\\{hwrite\_adjustments}}\\{hwrite\_adjustments}({\AND}\|l){}$;\5
${}\}{}$\5
\2\&{break};\6
\4\&{case} \.{TAG}${}(\\{adjust\_kind},\39\T{4}){}$:\1\6
\4${}\{{}$\5
\index{list t+\&{list\_t}}\&{list\_t} \|l;\5
${}\index{HGET LIST+\.{HGET\_LIST}}\.{HGET\_LIST}(\T{4},\39\|l){}$;\5
${}\|l.\|k\K\\{adjust\_kind}{}$;\5
${}\index{hwrite adjustments+\\{hwrite\_adjustments}}\\{hwrite\_adjustments}({\AND}\|l){}$;\5
${}\}{}$\5
\2\&{break};\6
\4\&{case} \.{TAG}${}(\\{adjust\_kind},\39\T{5}){}$:\1\6
\4${}\{{}$\5
\index{list t+\&{list\_t}}\&{list\_t} \|l;\5
${}\index{HGET LIST+\.{HGET\_LIST}}\.{HGET\_LIST}(\T{5},\39\|l){}$;\5
${}\|l.\|k\K\\{adjust\_kind}{}$;\5
${}\index{hwrite adjustments+\\{hwrite\_adjustments}}\\{hwrite\_adjustments}({\AND}\|l){}$;\5
${}\}{}$\5
\2\&{break};
\Y
\fi

\M{220}



I guess the following should be incorporated into \index{hwrite list+\\{hwrite\_list}}\\{hwrite\_list}.
\writecode
\Y\B\4\X19:write functions\X${}\mathrel+\E{}$\6
\&{void} \index{hwrite adjustments+\\{hwrite\_adjustments}}\\{hwrite\_adjustments}(\index{list t+\&{list\_t}}\&{list\_t} ${}{*}\|l){}$\1\1\2\2\1\6
\4${}\{{}$\6
\&{if} ${}(\|l\MG\|s\E\T{0}){}$\1\5
\&{return};\2\6
\&{else}\5
\1${}\{{}$\5
\&{uint32\_t} \|h${}\K\index{hpos+\\{hpos}}\\{hpos}-\index{hstart+\\{hstart}}\\{hstart},{}$ \|e${}\K\index{hend+\\{hend}}\\{hend}-\index{hstart+\\{hstart}}\\{hstart}{}$;\C{ save \\{hpos} and \\{hend} }\7
${}\index{hpos+\\{hpos}}\\{hpos}\K\|l\MG\|p+\index{hstart+\\{hstart}}\\{hstart}{}$;\5
${}\index{hend+\\{hend}}\\{hend}\K\index{hpos+\\{hpos}}\\{hpos}+\|l\MG\|s;{}$\6
\&{if} ${}(\|l\MG\|s>\T{\^FF}){}$\1\5
${}\index{hwritef+\\{hwritef}}\\{hwritef}(\.{"\ \%d"},\39\|l\MG\|s);{}$\2\6
\&{while} ${}(\index{hpos+\\{hpos}}\\{hpos}<\index{hend+\\{hend}}\\{hend}){}$\1\5
\index{hget content node+\\{hget\_content\_node}}\\{hget\_content\_node}(\,);\2\6
${}\index{hpos+\\{hpos}}\\{hpos}\K\index{hstart+\\{hstart}}\\{hstart}+\|h{}$;\5
${}\index{hend+\\{hend}}\\{hend}\K\index{hstart+\\{hstart}}\\{hstart}+\|e{}$;\C{ restore  \\{hpos} and \\{hend} }\6
\4${}\}{}$\2\6
\4${}\}{}$\2
\Y
\fi

\M{221}

\subsection{Tables}\index{alignment}
As long as a table contains no dependencies on \.{hsize} and \.{vsize},
Hi\TeX\ can expand an alignment into a set of nested horizontal and
vertical boxes and no special processing is required in the viewer.

As long as only the size of the table itself but neither the tabskip
glues nor the table content depends on \.{hsize} or \.{vsize} the table
just needs an outer node of type \index{hset kind+\\{hset\_kind}}\\{hset\_kind} or \index{vset kind+\\{vset\_kind}}\\{vset\_kind}. If there
is non aligned material inside the table that depends on \.{hsize} or
\.{vsize} a vpack or hpack node is still sufficient.

While it is reasonable to restrict the tabskip glues to be ordinary
glue values without \.{hsize} or \.{vsize} dependencies, it might be
desirable to have content in the table that does depend on \.{hsize} or
\.{vsize}. For the latter case, we need a special kind of table
node. Here is why:

As soon as the dimension of an item in the table is an extended
dimension, it is no longer possible to compute the maximum natural with
of a column, because it is not possible to compare extended dimensions
without knowing \.{hsize} and \.{vsize}.  Hence the computation of maximum
widths needs to be done in the viewer.  After knowing the width of the columns,
the setting of tabskip glues is easy to compute.

To implement these extended tables, we will need a table node that
specifies a direction, either horizontal or vertical; a list of
tabskip glues, with the provision that the last tabskip glue in the
list is repeated as long as necessary; and a list of table content.
The table's content consists of nonaligned content, for example extra glue
or rules, and aligned content called items.
The table's content is stacked, either vertical or
horizontal, orthogonal to the alignment direction of the table.
The aligned content of a table is packed in an outer item node,
that contains a list of inner item nodes.
An inner item contains a box node (of kind \index{hbox kind+\\{hbox\_kind}}\\{hbox\_kind}, \index{vbox kind+\\{vbox\_kind}}\\{vbox\_kind},
\index{hset kind+\\{hset\_kind}}\\{hset\_kind}, \index{vset kind+\\{vset\_kind}}\\{vset\_kind}, \index{hpack kind+\\{hpack\_kind}}\\{hpack\_kind}, or \index{vpack kind+\\{vpack\_kind}}\\{vpack\_kind}) followed by
an optional span count.

The glue of the boxes in the inner items will be reset so that all boxes in the same
column reach the same maximum column with.  The span counts will be replaced by
the appropriate amount of empty boxes and tabskip glues.  Finally the
glue in the outer item will be set to obtain the desired size
of the table.

The definitions below specify just a \index{list+\nts{list}}\nts{list} for the list of tabskip glues and the
list of inner table items. This is just for convenience; the first list must contain glue
nodes and the second list must contain inner item nodes.

We reuse the \ts{H} and \ts{V} tokens, defined as part of the specification of extended dimensions,
to indicate the alignment direction of the table. To tell a reference to an extended dimension
from a reference to an ordinary dimension, we prefix the former with an \index{XDIMEN+\ts{XDIMEN}}\ts{XDIMEN} token;
for the latter, the \index{DIMEN+\ts{DIMEN}}\ts{DIMEN} token is optional. The scanner will recognize not only ``item'' as
an \index{ITEM+\ts{ITEM}}\ts{ITEM} token but also ``row'' and ''column''. This allows a more readable notation,
for example by marking the outer items as rows and the inner items as columns.

In the short format, the \\{b010} bit is used to mark a vertical table and the \\{b101} bits indicate
how the table size is specified;
an outer item node has the info value \\{b000}, an inner item node with info value \\{b111}
contains an extra byte for the span count, otherwise the info value is equal to the span count.






\readcode
\Y\par
\par
\par
\Y
\fi

\M{222}

\Y\B\4\X2:symbols\X${}\mathrel+\E{}$\6
\8\%\&{token} \index{TABLE+\ts{TABLE}}\ts{TABLE}\5\.{"table"}\6
\8\%\&{token} \index{ITEM+\ts{ITEM}}\ts{ITEM}\5\.{"item"}\6
\8\%\index{type+\&{type}}\&{type} $<$ \index{info+\\{info}}\\{info} $>$ \index{table+\nts{table}}\nts{table}
\Y
\fi

\M{223}

\Y\B\4\X3:scanning rules\X${}\mathrel+\E{}$\6
${}\8\re{\vb{table}}{}$\ac\&{return} \index{TABLE+\ts{TABLE}}\ts{TABLE};\eac\7
${}\8\re{\vb{item}}{}$\ac\&{return} \index{ITEM+\ts{ITEM}}\ts{ITEM};\eac\7
${}\8\re{\vb{row}}{}$\ac\&{return} \index{ITEM+\ts{ITEM}}\ts{ITEM};\eac\7
${}\8\re{\vb{column}}{}$\ac\&{return} \index{ITEM+\ts{ITEM}}\ts{ITEM};\eac
\Y
\fi

\M{224}

\Y\B\4\X5:parsing rules\X${}\mathrel+\E{}$\6
\index{content node+\nts{content\_node}}\nts{content\_node}: \1\1\5
\index{start+\nts{start}}\nts{start}\5
\index{ITEM+\ts{ITEM}}\ts{ITEM}\5
\index{content node+\nts{content\_node}}\nts{content\_node}\5
\index{END+\ts{END}}\ts{END}\5
${}\{{}$\1\5
${}\index{hput tags+\\{hput\_tags}}\\{hput\_tags}(\.{\$1},\39\index{hput item+\\{hput\_item}}\\{hput\_item}(\T{1}));{}$\5
${}\}{}$\2;\2\2\7
\index{content node+\nts{content\_node}}\nts{content\_node}: \1\1\5
\index{start+\nts{start}}\nts{start}\5
\index{ITEM+\ts{ITEM}}\ts{ITEM}\5
\index{content node+\nts{content\_node}}\nts{content\_node}\5
\index{UNSIGNED+\ts{UNSIGNED}}\ts{UNSIGNED}\5
\index{END+\ts{END}}\ts{END}\5
${}\{{}$\1\5
${}\index{hput tags+\\{hput\_tags}}\\{hput\_tags}(\.{\$1},\39\index{hput item+\\{hput\_item}}\\{hput\_item}(\.{\$4}));{}$\5
${}\}{}$\2;\2\2\7
\index{content node+\nts{content\_node}}\nts{content\_node}: \1\1\5
\index{start+\nts{start}}\nts{start}\5
\index{ITEM+\ts{ITEM}}\ts{ITEM}\5
\index{list+\nts{list}}\nts{list}\5
\index{END+\ts{END}}\ts{END}\5
${}\{{}$\1\5
${}\index{hput tags+\\{hput\_tags}}\\{hput\_tags}(\.{\$1},\39\.{TAG}(\index{item kind+\\{item\_kind}}\\{item\_kind},\39\\{b000}));{}$\5
${}\}{}$\2;\2\2\7
\index{table+\nts{table}}\nts{table}: \1\1\5
\ts{H}\5
\index{box goal+\nts{box\_goal}}\nts{box\_goal}\5
\index{list+\nts{list}}\nts{list}\5
\index{list+\nts{list}}\nts{list}\5
${}\{{}$\1\5
${}\.{\$\$}\K\.{\$2};{}$\5
${}\}{}$\2\5
\index{table+\nts{table}}\nts{table}: \1\1\5
\ts{V}\5
\index{box goal+\nts{box\_goal}}\nts{box\_goal}\5
\index{list+\nts{list}}\nts{list}\5
\index{list+\nts{list}}\nts{list}\5
${}\{{}$\1\5
${}\.{\$\$}\K\.{\$2}\OR\\{b010};{}$\5
${}\}{}$\2\5
\index{content node+\nts{content\_node}}\nts{content\_node}: \1\1\5
\index{start+\nts{start}}\nts{start}\5
\index{TABLE+\ts{TABLE}}\ts{TABLE}\5
\index{table+\nts{table}}\nts{table}\5
\index{END+\ts{END}}\ts{END}\5
${}\{{}$\1\5
${}\index{hput tags+\\{hput\_tags}}\\{hput\_tags}(\.{\$1},\39\.{TAG}(\index{table kind+\\{table\_kind}}\\{table\_kind},\39\.{\$3}));{}$\5
${}\}{}$\2;\2\2
\Y
\fi

\M{225}

\getcode
\Y\B\4\X18:cases to get content\X${}\mathrel+\E{}$\6
\4\&{case} \.{TAG}${}(\index{table kind+\\{table\_kind}}\\{table\_kind},\39\\{b000}){}$:\5
\index{HGET TABLE+\.{HGET\_TABLE}}\.{HGET\_TABLE}(\\{b000});\5
\&{break};\6
\4\&{case} \.{TAG}${}(\index{table kind+\\{table\_kind}}\\{table\_kind},\39\\{b001}){}$:\5
\index{HGET TABLE+\.{HGET\_TABLE}}\.{HGET\_TABLE}(\\{b001});\5
\&{break};\6
\4\&{case} \.{TAG}${}(\index{table kind+\\{table\_kind}}\\{table\_kind},\39\\{b010}){}$:\5
\index{HGET TABLE+\.{HGET\_TABLE}}\.{HGET\_TABLE}(\\{b010});\5
\&{break};\6
\4\&{case} \.{TAG}${}(\index{table kind+\\{table\_kind}}\\{table\_kind},\39\\{b011}){}$:\5
\index{HGET TABLE+\.{HGET\_TABLE}}\.{HGET\_TABLE}(\\{b011});\5
\&{break};\6
\4\&{case} \.{TAG}${}(\index{table kind+\\{table\_kind}}\\{table\_kind},\39\\{b100}){}$:\5
\index{HGET TABLE+\.{HGET\_TABLE}}\.{HGET\_TABLE}(\\{b100});\5
\&{break};\6
\4\&{case} \.{TAG}${}(\index{table kind+\\{table\_kind}}\\{table\_kind},\39\\{b101}){}$:\5
\index{HGET TABLE+\.{HGET\_TABLE}}\.{HGET\_TABLE}(\\{b101});\5
\&{break};\6
\4\&{case} \.{TAG}${}(\index{table kind+\\{table\_kind}}\\{table\_kind},\39\\{b110}){}$:\5
\index{HGET TABLE+\.{HGET\_TABLE}}\.{HGET\_TABLE}(\\{b110});\5
\&{break};\6
\4\&{case} \.{TAG}${}(\index{table kind+\\{table\_kind}}\\{table\_kind},\39\\{b111}){}$:\5
\index{HGET TABLE+\.{HGET\_TABLE}}\.{HGET\_TABLE}(\\{b111});\5
\&{break};\7
\4\&{case} \.{TAG}${}(\index{item kind+\\{item\_kind}}\\{item\_kind},\39\\{b000}){}$:\5
\1${}\{{}$\5
\index{list t+\&{list\_t}}\&{list\_t} \|l;\5
${}\index{hget list+\\{hget\_list}}\\{hget\_list}({\AND}\|l){}$;\5
${}\index{hwrite list+\\{hwrite\_list}}\\{hwrite\_list}({\AND}\|l){}$;\5
${}\}{}$\5
\2\&{break};\6
\4\&{case} \.{TAG}${}(\index{item kind+\\{item\_kind}}\\{item\_kind},\39\\{b001}){}$:\5
\index{hget content node+\\{hget\_content\_node}}\\{hget\_content\_node}(\,);\5
\&{break};\6
\4\&{case} \.{TAG}${}(\index{item kind+\\{item\_kind}}\\{item\_kind},\39\\{b010}){}$:\5
\index{hget content node+\\{hget\_content\_node}}\\{hget\_content\_node}(\,);\5
\index{hwritef+\\{hwritef}}\\{hwritef}(\.{"\ 2"});\5
\&{break};\6
\4\&{case} \.{TAG}${}(\index{item kind+\\{item\_kind}}\\{item\_kind},\39\\{b011}){}$:\5
\index{hget content node+\\{hget\_content\_node}}\\{hget\_content\_node}(\,);\5
\index{hwritef+\\{hwritef}}\\{hwritef}(\.{"\ 3"});\5
\&{break};\6
\4\&{case} \.{TAG}${}(\index{item kind+\\{item\_kind}}\\{item\_kind},\39\\{b100}){}$:\5
\index{hget content node+\\{hget\_content\_node}}\\{hget\_content\_node}(\,);\5
\index{hwritef+\\{hwritef}}\\{hwritef}(\.{"\ 4"});\5
\&{break};\6
\4\&{case} \.{TAG}${}(\index{item kind+\\{item\_kind}}\\{item\_kind},\39\\{b101}){}$:\5
\index{hget content node+\\{hget\_content\_node}}\\{hget\_content\_node}(\,);\5
\index{hwritef+\\{hwritef}}\\{hwritef}(\.{"\ 5"});\5
\&{break};\6
\4\&{case} \.{TAG}${}(\index{item kind+\\{item\_kind}}\\{item\_kind},\39\\{b110}){}$:\5
\index{hget content node+\\{hget\_content\_node}}\\{hget\_content\_node}(\,);\5
\index{hwritef+\\{hwritef}}\\{hwritef}(\.{"\ 6"});\5
\&{break};\6
\4\&{case} \.{TAG}${}(\index{item kind+\\{item\_kind}}\\{item\_kind},\39\\{b111}){}$:\5
\index{hget content node+\\{hget\_content\_node}}\\{hget\_content\_node}(\,);\5
${}\index{hwritef+\\{hwritef}}\\{hwritef}(\.{"\ \%u"},\39\index{HGET8+\.{HGET8}}\.{HGET8}){}$;\5
\&{break};
\Y
\fi

\M{226}

\Y\B\4\X17:get macros\X${}\mathrel+\E{}$\6
\8\#\&{define} \index{HGET TABLE+\.{HGET\_TABLE}}\.{HGET\_TABLE}(\|I) \6
\&{if} ${}(\|I\AND\\{b010}){}$\1\5
\index{hwritef+\\{hwritef}}\\{hwritef}(\.{"\ v"});\5
\2\&{else}\1\5
\index{hwritef+\\{hwritef}}\\{hwritef}(\.{"\ h"});\2\6
\&{if} ${}((\|I)\AND\\{b001}){}$\1\5
\index{hwritef+\\{hwritef}}\\{hwritef}(\.{"\ add"});\5
\2\&{else}\1\5
\index{hwritef+\\{hwritef}}\\{hwritef}(\.{"\ to"});\2\6
\&{if} ${}((\|I)\AND\\{b100}){}$\5
\1${}\{{}$\5
\index{xdimen t+\&{xdimen\_t}}\&{xdimen\_t} \|x;\7
${}\index{hget xdimen node+\\{hget\_xdimen\_node}}\\{hget\_xdimen\_node}({\AND}\|x){}$;\5
${}\index{hwrite xdimen node+\\{hwrite\_xdimen\_node}}\\{hwrite\_xdimen\_node}({\AND}\|x){}$;\5
${}\}{}$\2\6
\&{else}\1\5
\index{HGET REF+\.{HGET\_REF}}\.{HGET\_REF}(\index{xdimen kind+\\{xdimen\_kind}}\\{xdimen\_kind})\2\1\6
\4${}\{{}$\5
\index{list t+\&{list\_t}}\&{list\_t} \|l;\5
${}\index{hget list+\\{hget\_list}}\\{hget\_list}({\AND}\|l){}$;\5
${}\index{hwrite list+\\{hwrite\_list}}\\{hwrite\_list}({\AND}\|l){}$;\5
${}\}{}$\C{ tabskip }\6
\4${}\{{}$\5
\index{list t+\&{list\_t}}\&{list\_t} \|l;\5
${}\index{hget list+\\{hget\_list}}\\{hget\_list}({\AND}\|l){}$;\5
${}\index{hwrite list+\\{hwrite\_list}}\\{hwrite\_list}({\AND}\|l){}$;\5
${}\}{}$\C{ items }\2
\Y
\fi

\M{227}


\putcode
\Y\B\4\X12:put functions\X${}\mathrel+\E{}$\6
\&{uint8\_t} \index{hput item+\\{hput\_item}}\\{hput\_item}(\&{uint32\_t} \|n)\1\1\2\2\1\6
\4${}\{{}$\6
\&{if} ${}(\|n\E\T{0}){}$\1\5
\.{QUIT}(\.{"Span\ count\ in\ item\ }\)\.{must\ not\ be\ zero"});\2\6
\&{else} \&{if} ${}(\|n<\T{7}){}$\1\5
\&{return} \.{TAG}${}(\index{item kind+\\{item\_kind}}\\{item\_kind},\39\|n);{}$\2\6
\&{else} \&{if} ${}(\|n>\T{\^FF}){}$\1\5
${}\.{QUIT}(\.{"Span\ count\ \%d\ must\ }\)\.{be\ less\ than\ 255"},\39\|n);{}$\2\6
\&{else}\5
\1${}\{{}$\5
\index{HPUT8+\.{HPUT8}}\.{HPUT8}(\|n);\6
\&{return} \.{TAG}${}(\index{item kind+\\{item\_kind}}\\{item\_kind},\39\T{7});{}$\6
\4${}\}{}$\2\6
\4${}\}{}$\2
\Y
\fi

\M{228}
\section{Extensions to \TeX}\hascode

\subsection{Images}
Images behave pretty much like glue\index{glue}. They can stretch (or shrink)
together with the surrounding glue to fill a horizontal or vertical box.
Like glue, they stretch in the horizontal direction when filling an horizontal box
and they stretch in the vertical direction as part of a vertical box.
Stretchability and shrinkability are optional parts of an image node.

Unlike glue, images have both a width and a height.
The relation of height to width, the aspect ratio, is preserved by stretching and shrinking.

While glue often has a zero width, images usually have a nonzero natural size and making
them much smaller is undesirable.
The natural width and height of an image are optional parts of an image node;
typically however, this information is contained in the image data.

The only required part of an image node is the number of the auxiliary section
where the image data can be found.

\Y\B\4\X1:hint types\X${}\mathrel+\E{}$\6
\&{typedef} \&{struct} ${}\{{}$\5
\1\&{uint16\_t} \|n;\5
\index{dimen t+\&{dimen\_t}}\&{dimen\_t} \|w${},{}$ \|h;\5
\index{stretch t+\&{stretch\_t}}\&{stretch\_t} \|p${},{}$ \|m;\5
\2${}\}{}$ \index{image t+\&{image\_t}}\&{image\_t};
\Y
\fi

\M{229}


\readcode
\Y\par
\par
\par
\Y\B\4\X2:symbols\X${}\mathrel+\E{}$\6
\8\%\&{token} \index{IMAGE+\ts{IMAGE}}\ts{IMAGE}\5\.{"image"}\6
\8\%\index{type+\&{type}}\&{type} $<$ \|x $>$ \index{image+\nts{image}}\nts{image}\5
\index{image dimen+\nts{image\_dimen}}\nts{image\_dimen}
\Y
\fi

\M{230}

\Y\B\4\X3:scanning rules\X${}\mathrel+\E{}$\6
${}\8\re{\vb{image}}{}$\ac\&{return} \index{IMAGE+\ts{IMAGE}}\ts{IMAGE};\eac
\Y
\fi

\M{231}

\Y\B\4\X5:parsing rules\X${}\mathrel+\E{}$\6
\index{image dimen+\nts{image\_dimen}}\nts{image\_dimen}: \1\1\5
\index{dimension+\nts{dimension}}\nts{dimension}\5
\index{dimension+\nts{dimension}}\nts{dimension}\5
${}\{{}$\1\5
${}\.{\$\$}.\|w\K\.{\$1};{}$\5
${}\.{\$\$}.\|h\K\.{\$2};{}$\5
${}\}{}$\2\6
\4\hbox to 0.5em{\hss${}\OR{}$}\5
${}\{{}$\1\5
${}\.{\$\$}.\|w\K\.{\$\$}.\|h\K\T{0};{}$\5
${}\}{}$\2;\2\2\7
\index{image+\nts{image}}\nts{image}: \1\1\5
\index{UNSIGNED+\ts{UNSIGNED}}\ts{UNSIGNED}\5
\index{image dimen+\nts{image\_dimen}}\nts{image\_dimen}\5
\index{plus+\nts{plus}}\nts{plus}\5
\index{minus+\nts{minus}}\nts{minus}\5
${}\{{}$\1\5
${}\.{\$\$}.\|w\K\.{\$2}.\|w;{}$\5
${}\.{\$\$}.\|h\K\.{\$2}.\|h;{}$\5
${}\.{\$\$}.\|p\K\.{\$3};{}$\5
${}\.{\$\$}.\|m\K\.{\$4};{}$\5
${}\.{RNG}(\.{"Section\ number"},\39\.{\$1},\39\T{3},\39\index{max section no+\\{max\_section\_no}}\\{max\_section\_no});{}$\5
${}\.{\$\$}.\|n\K\.{\$1};{}$\5
${}\}{}$\2;\2\2\7
\index{content node+\nts{content\_node}}\nts{content\_node}: \1\1\5
\index{start+\nts{start}}\nts{start}\5
\index{IMAGE+\ts{IMAGE}}\ts{IMAGE}\5
\index{image+\nts{image}}\nts{image}\5
\index{END+\ts{END}}\ts{END}\5
${}\{{}$\1\5
${}\index{hput tags+\\{hput\_tags}}\\{hput\_tags}(\.{\$1},\39\index{hput image+\\{hput\_image}}\\{hput\_image}({\AND}(\.{\$3})));{}$\5
${}\}{}$\2
\Y
\fi

\M{232}

\writecode
\Y\B\4\X19:write functions\X${}\mathrel+\E{}$\6
\&{void} \index{hwrite image+\\{hwrite\_image}}\\{hwrite\_image}(\index{image t+\&{image\_t}}\&{image\_t} ${}{*}\|x){}$\1\1\2\2\1\6
\4${}\{{}$\5
${}\index{hwritef+\\{hwritef}}\\{hwritef}(\.{"\ \%u"},\39\|x\MG\|n);{}$\6
\&{if} ${}(\|x\MG\|w\I\T{0}\V\|x\MG\|h\I\T{0}){}$\5
\1${}\{{}$\5
${}\index{hwrite dimension+\\{hwrite\_dimension}}\\{hwrite\_dimension}(\|x\MG\|w);{}$\5
${}\index{hwrite dimension+\\{hwrite\_dimension}}\\{hwrite\_dimension}(\|x\MG\|h);{}$\6
\4${}\}{}$\2\6
${}\index{hwrite plus+\\{hwrite\_plus}}\\{hwrite\_plus}({\AND}\|x\MG\|p);{}$\5
${}\index{hwrite minus+\\{hwrite\_minus}}\\{hwrite\_minus}({\AND}\|x\MG\|m);{}$\6
\4${}\}{}$\2
\Y
\fi

\M{233}

\getcode
\Y\B\4\X18:cases to get content\X${}\mathrel+\E{}$\6
\4\&{case} \.{TAG}${}(\index{image kind+\\{image\_kind}}\\{image\_kind},\39\\{b100}){}$:\5
\1${}\{{}$\5
\index{image t+\&{image\_t}}\&{image\_t} \|x;\5
${}\index{HGET IMAGE+\.{HGET\_IMAGE}}\.{HGET\_IMAGE}(\\{b100},\39\|x){}$;\5
${}\}{}$\5
\2\&{break};\6
\4\&{case} \.{TAG}${}(\index{image kind+\\{image\_kind}}\\{image\_kind},\39\\{b101}){}$:\5
\1${}\{{}$\5
\index{image t+\&{image\_t}}\&{image\_t} \|x;\5
${}\index{HGET IMAGE+\.{HGET\_IMAGE}}\.{HGET\_IMAGE}(\\{b101},\39\|x){}$;\5
${}\}{}$\5
\2\&{break};\6
\4\&{case} \.{TAG}${}(\index{image kind+\\{image\_kind}}\\{image\_kind},\39\\{b110}){}$:\5
\1${}\{{}$\5
\index{image t+\&{image\_t}}\&{image\_t} \|x;\5
${}\index{HGET IMAGE+\.{HGET\_IMAGE}}\.{HGET\_IMAGE}(\\{b110},\39\|x){}$;\5
${}\}{}$\5
\2\&{break};\6
\4\&{case} \.{TAG}${}(\index{image kind+\\{image\_kind}}\\{image\_kind},\39\\{b111}){}$:\5
\1${}\{{}$\5
\index{image t+\&{image\_t}}\&{image\_t} \|x;\5
${}\index{HGET IMAGE+\.{HGET\_IMAGE}}\.{HGET\_IMAGE}(\\{b111},\39\|x){}$;\5
${}\}{}$\5
\2\&{break};
\Y
\fi

\M{234}

\Y\B\4\X17:get macros\X${}\mathrel+\E{}$\6
\8\#\&{define} $\index{HGET IMAGE+\.{HGET\_IMAGE}}\.{HGET\_IMAGE}(\|I,\39\|X){}$\6
\index{HGET16+\.{HGET16}}\.{HGET16} ${}((\|X).\|n);{}$\5
${}\.{RNG}(\.{"Section\ number"},\39(\|X).\|n,\39\T{3},\39\index{max section no+\\{max\_section\_no}}\\{max\_section\_no});{}$\6
\&{if} ${}(\|I\AND\\{b010}){}$\5
\1${}\{{}$\5
${}\index{HGET32+\.{HGET32}}\.{HGET32}((\|X).\|w);{}$\5
${}\index{HGET32+\.{HGET32}}\.{HGET32}((\|X).\|h){}$;\5
${}\}{}$\2\6
\&{else}\1\5
${}(\|X).\|w\K(\|X).\|h\K\T{0};{}$\2\6
\&{if} ${}(\|I\AND\\{b001}){}$\5
\1${}\{{}$\5
${}\index{HGET STRETCH+\.{HGET\_STRETCH}}\.{HGET\_STRETCH}((\|X).\|p);{}$\5
${}\index{HGET STRETCH+\.{HGET\_STRETCH}}\.{HGET\_STRETCH}((\|X).\|m){}$;\5
${}\}{}$\2\6
\&{else}\5
\1${}\{{}$\5
${}(\|X).\|p.\|f\K(\|X).\|m.\|f\K\T{0.0};{}$\5
${}(\|X).\|p.\|o\K(\|X).\|m.\|o\K\index{normal o+\\{normal\_o}}\\{normal\_o}{}$;\5
${}\}{}$\2\6
${}\index{hwrite image+\\{hwrite\_image}}\\{hwrite\_image}({\AND}(\|X));$
\Y
\fi

\M{235}


\putcode
\Y\B\4\X12:put functions\X${}\mathrel+\E{}$\6
\&{uint8\_t} \index{hput image+\\{hput\_image}}\\{hput\_image}(\index{image t+\&{image\_t}}\&{image\_t} ${}{*}\|x){}$\1\1\2\2\1\6
\4${}\{{}$\5
\index{info t+\&{info\_t}}\&{info\_t} \|i${}\K\\{b100};{}$\7
${}\index{HPUT16+\.{HPUT16}}\.{HPUT16}(\|x\MG\|n);{}$\6
\&{if} ${}(\|x\MG\|w\I\T{0}\V\|x\MG\|h\I\T{0}{}$)\5
\1${}\{{}$\5
${}\index{HPUT32+\.{HPUT32}}\.{HPUT32}(\|x\MG\|w);{}$\5
${}\index{HPUT32+\.{HPUT32}}\.{HPUT32}(\|x\MG\|h);{}$\5
${}\|i\MRL{{\OR}{\K}}\\{b010};{}$\6
\4${}\}{}$\2\6
\&{if} ${}(\|x\MG\|p.\|f\I\T{0.0}\V\|x\MG\|m.\|f\I\T{0.0}{}$)\5
\1${}\{{}$\5
${}\index{hput stretch+\\{hput\_stretch}}\\{hput\_stretch}({\AND}\|x\MG\|p);{}$\5
${}\index{hput stretch+\\{hput\_stretch}}\\{hput\_stretch}({\AND}\|x\MG\|m);{}$\5
${}\|i\MRL{{\OR}{\K}}\\{b001};{}$\6
\4${}\}{}$\2\6
\&{return} \.{TAG}${}(\index{image kind+\\{image\_kind}}\\{image\_kind},\39\|i);{}$\6
\4${}\}{}$\2
\Y
\fi

\M{236}

\subsection{Colors}
Colors\index{color} are certainly one of the features you will find in the final \HINT/ file format.
Here some remarks must suffice.

A \HINT/ viewer must be capable of rendering a page given just any valid
position inside the content section. Therefore \HINT/ files are stateless;
there is no need to search for preceding commands that might change a state
variable.
As a consequence, we can not just define a ``color change node''.
Colors could be specified as an optional parameter of a glyph node, but the
amount of data necessary would be considerable. In texts, on the other hand,
a color change control code would be possible because we parse texts only in forward
direction. The current font  would then become a current color and font with the appropriate
changes for positions.

A more attractive alternative would be to specify colored fonts.
This would require an optional
color argument for a font. For example one could have a cmr10 font in black as
font number 3, and a cmr10 font in blue as font number 4. Having 256 different fonts,
this is definitely a possibility because rarely you would need that many fonts
or that many colors. If necessary and desired, one could allow 16 bit font numbers
of overcome the problem.

Background colors could be associated with boxes as an optional parameter.

\subsection{Positions, Links, and Labels}\index{position}\index{link}\index{label}
A viewer can usually not display the entire content section of
a \HINT/ file. Instead it will display a page of content and will give
its user various means to change the page. This might be as simple as
a ``page down'' or ``page up'' button (or gesture) and as
sophisticated as searching using regular expressions.  More
traditional ways to navigate the content include the use of a table of
content or an index of keywords. All these methods of changing a page
have in common that a part of the content that fits nicely in the
screen area provided by the output device must be rendered given a
position inside the content section.


Let's assume that the viewer uses a \HINT/ file in short
format---after all that's the format designed for precisely this use.
A position inside the content section is then the position of the
starting byte of a node. Such a position can be stored as a 32 bit
number.  To render a page starting at that position is not difficult:
We just read content nodes, starting at the given position and feed
them to \TeX's page builder until the page is complete. To implement a
``clickable'' table of content this is good enough. We store with
every entry in the table of content the position of the section
header, and when the user clicks the entry, the viewer can display a
new page starting exactly with that section header.

Things are slightly more complex if we want to implement a ``page
down'' button. If we press this button, we want the next page to
start exactly where the current page has ended.  This is
typically in the middle of a paragraph node, and it might even be in
the middle of an hyphenated word in that paragraph. Fortunately,
paragraph and table nodes are the only nodes that can be broken across page
boundaries. But broken paragraph nodes are a common case non the less,
and unless we want to search for the enclosing node, we need to
augment in this case the primary 32 bit position inside the content
section with a secondary position. Most of the
time, 16 bit will suffice for this secondary position if we give it
relative to the primary position. Further, if the list of nodes forming the
paragraph is given as a text, we need to know the current font at the
secondary position. Of course, the viewer can find it by scanning the
initial part of the text, but when we think of a page down button, the
viewer might already know it from rendering the previous page.

Similar is the case of a ``page up'' button. Only here we need a page
that ends precisely where our current page starts. Possibly even with
the initial part of a hyphenated word. Here we need a reverse version
of \TeX's page builder that assembles a ``good'' page from the bottom
up instead of from the top down.  Sure the viewer can cache the start
position of the previous page (or the rendering of the entire page) if
the reader has reached the current page using the page down
button. But this is not possible in all cases. The reader might have
reached the current page using the table of content or even an index
or a search form.

This is the most complex case to consider: a link from an index or a
search form to the position of a keyword in the main text. Lets assume
someone looks up the word ``M\"unchen''.  Should the viewer then
generate a page that starts in the middle of a sentence with the word
``M\"unchen''? Probably not! We want a page that shows at least the whole sentence if
not the whole paragraph.  Of course the program that generates the
link could specify the position of the start of the paragraph instead
of the position of the word. But that will not solve the problem. Just
imagine reading the groundbreaking masterpiece of a German philosopher
on a small hand-held device: the paragraph will most likely be very
long and perhaps only part of the first sentence will fit on the small
screen. So the desired keyword might not be found on the page that
starts with the beginning of the paragraph; it might not even be on
the next or next to next page. Only the viewer can decide what is the
best fragment of content to display around the position of the given
keyword.

To summarize, we need three different ways to render a page for a given position:
\itemize
\item A page that starts exactly at the give position.
\item A page that ends exactly at the give position.
\item The ``best'' page that contains the given position somewhere in the middle.
\enditemize

A possible way to find the ``best'' page for the latter case could be the following:
\itemize
\item If the position is inside a paragraph, break the paragraph into lines. One line will contain
the target position. Let's call this the target line.
\item If the paragraph will not fit entirely on the page, start the page with the beginning of the
paragraph if that will place the target line on the page, otherwise
display an equal amount of lines before and after the target line.
\item Else traverse the content list backward for about $2/3$ of the page height and forward for about $2/3$
of the page height, searching for the smallest negative penalty node.
Use the  penalty node found as either the beginning
or ending of the page.
\item If there are several equally low negative penalty nodes. Prefer penalties preceding the target line
over penalty nodes following it. A good page start is more important than a good page end.
\item If there are are still several equally low negative penalty nodes, choose the one whose distance
to the target line is closest to $1/2$ of the page height.
\item If no negative penalty nodes could be found, start the page with the paragraph containing the target line.
\item Once the page start (or end) is found, use \TeX's page builder (or its reverse variant) to complete the page.
\enditemize

We call nodes that reference a position inside the content section a
link node.  As with other nodes, we can use predefined links. The
first 256 of them can be referenced by a single byte.  We should
reserve reference number 0 for a link to the beginning of the content
and reference number 1 for a link to the end of the content.  Probably
having only 256 links would be a severe restriction, hence we will
allow also 16 bit reference numbers.  If still more links are needed,
links can be embedded directly in the content stream.  We need two
types of links, a start link and an end link such that the content
between the two will constitute the visible part of the link.


In the short format, we will use the \\{b100} bit of the info value to
distinguish them: 1 indicates start link, 0 indicates end link. The
two low bits of the info value will be 0 for an 8 bit reference
number, 1 for a 16 bit reference number,  2 for an immediate
link without secondary position and current font, and  3 for an immediate
link with 32 bit secondary position and current font.
The link itself consists of a primary position, an optional
secondary position, an optional current font, and a position type. The
position type is 0 for the exact page top, 1 for the exact page
bottom, and 2 for the approximate middle as described above.

In the long format, a position can not be expressed as a byte
position; instead we use labels.  A label is identified by a unique
name expressed as a string. For example we can write
\.{<label} \.{'label10'>} and then we can use \.{'label10'} as a
symbolic reference to the position of the node
that follows the label node. When translating the long format to the
short format, these label nodes will disappear. To keep readable label
names, the links in the short format may specify an optional name that
is used for labels. If no name is given, a label name is
generated. When translating the short format to the long format, we
test just before writing a new node whether there is a link to this
node and insert a label if so. Because we write nodes in ascending
order of positions, we can sort the labels in ascending order of
position and compare \index{hpos+\\{hpos}}\\{hpos} with the position of the next label in
this order.  Immediate back links pose a problem for this translation
because the node has already been written without a label when we
encounter the link node that refers to it. If we encounter such a link
we must resort to a two pass translation: We log the information
about the back link and continue with the translation.  After the
whole file is translated, we check the log, and if unresolved back
links where found, we sort them into the previously incomplete list of
links and repeat the translation.

When translating the long format to the short format, immediate
forward links pose a similar problem: We can not encode the links
position because we have not yet encountered the label. In case we
have unused reference numbers for predefined links, we will convert the immediate link into a
predefined link. Predefined links can be completed with positions when
we find the labels, all we need to know to encode the link itself is the
reference number. If all 16 bit numbers are already in use, we reserve
the maximum amount of memory ( 8 bit for the type information,
32 bit for the primary position, 32 bit for the secondary position,
and 8 bit for the font number) in the stream and keep a linked list of
positions for the given label (the reserved space in the link nodes
can be used of that purpose) and fill in the information once we find
the corresponding label.

Links and Labels are not yet implemented.

\section{Replacing \TeX's Page Building Process}

\TeX\ uses an output\index{output routine} routine to finalize the page. It uses the accumulated material
from the page builder, found in {\tt box255}, attaches headers, footers, and floating material
like figures, tables, and footnotes. The latter material is specified by insert nodes
while headers and footers are often constructed using mark nodes.
Running an output routine requires the full power of the \TeX\ engine and will not be
part of the \HINT/ viewer. Therefore, \HINT/ replaces output routines by page templates\index{template}.
As \TeX\ can use different output routines for different parts of a book---for example
the index might use a different output routine than the main body of text---\HINT/
will allow multiple page templates. To support different output media, the page
templates will be named and a suitable user interface may offer the user a selection
of possible page layouts. In this way, the page layout remains in the hands of the
book designer, and the user has still the opportunity to pick a layout that best fits
the display device.

\TeX\ uses insertions to describe floating content that is not necessarily displayed
where it is specified. Three examples may illustrate this:
\itemize
\item Footnotes\footnote*{Like this one.}  are specified in the middle of the text but are displayed at the
bottom of the page. Long footnotes may even be split and displayed at the
bottom of the next page.  Several
footnotes\index{footnote} on the same page are collected and displayed together. The
page layout may specify a short rule to separate footnotes from the
main text, and if there are many short footnotes, it may use two columns
to display them.  In extreme cases, the page layout may demand a
footnote to be split and continued on the next page.

\item Illustrations\index{illustration} may be displayed exactly where specified if there is enough
room on the page, but may move to the top of the page, the bottom of the page,
the top of next page, or a separate page at the end of the chapter.

\item Margin notes\index{margin note} are displayed in the margin on the same page starting at the top
of the margin.
\enditemize

\HINT/ uses page templates and content streams to achieve similar effects.
But before I describe the page building\index{page building} mechanisms of \HINT/, let me summarize \TeX's
method.

\TeX's page builder ignores leading glue\index{glue}, kern\index{kern}, and penalty\index{penalty} nodes until the first
box\index{box} or rule\index{rule} is encountered;
whatsit\index{whatsit node} nodes do not really contribute anything to a page; mark\index{mark node} nodes are recorded for later use.
Once the first box, rule, or insert\index{insert node} arrives, \TeX\ makes copies of all parameters
that influence the page building process and uses these copies. These parameters
are the \index{page goal+\\{page\_goal}}\\{page\_goal} and the \index{page max depth+\\{page\_max\_depth}}\\{page\_max\_depth}; further the parameters
\index{page total+\\{page\_total}}\\{page\_total}, \index{page shrink+\\{page\_shrink}}\\{page\_shrink}, \index{page stretch+\\{page\_stretch}}\\{page\_stretch}, \index{page depth+\\{page\_depth}}\\{page\_depth},
and {\it insert\_pe\-nal\-ties\/} are initialized to zero.
The top skip\index{top skip} adjustment is made
when the first box or rule arrives---possibly after an insert.

Now the page builder accumulates material: normal material goes into box255\index{box255},
inserts specify an insert class $n$ and go into {\tt box$n$}.
Material that goes into {\tt box255} will change \index{page total+\\{page\_total}}\\{page\_total}, \index{page shrink+\\{page\_shrink}}\\{page\_shrink},
\index{page stretch+\\{page\_stretch}}\\{page\_stretch}, and \index{page depth+\\{page\_depth}}\\{page\_depth}. The latter is adjusted so that
is does not exceed \index{page max depth+\\{page\_max\_depth}}\\{page\_max\_depth}.

The handling of inserts\index{insert node} is more complex.
\TeX\ creates an insert class using \.{newinsert}. This reserves a number $n$
and four registers: {\tt box$n$} for the inserted material, {\tt count$n$} for the
magnification factor $f$, {\tt dimen$n$} for the maximum size per page $d$, and {\tt skip$n$} for the
extra space needed on a page if there are any insertions of class $n$.

For example plain \TeX\ allocates $n=254$ for footnotes\index{footnote} and sets
{\tt count254} to~$1000$, {\tt dimen254} to 8in, and {\tt skip254} to {\tt \BS bigskipamount}.

An insertion node will specify the insertion class $n$, some vertical material,
its natural height plus depth $x$, a {\it split\-\_top\-\_skip}, a {\it split\-\_max\_depth},
and a {\it floa\-ting\-\_pe\-nal\-ty}.


Now assume that an insert node with subtype 254 arrives at the page builder.
If this is the first such insert, \TeX\ will decrease the \index{page goal+\\{page\_goal}}\\{page\_goal}
by the width of skip254 and adds its stretchability and shrinkability
to the total stretchability and shrinkability of the page. Later,
the output routine will add some space and the footnote rule to fill just that
much space and add just that much shrinkability and stretchability to the page.
Then \TeX\ will normally add the vertical material in the insert node to
box254 and decrease the \index{page goal+\\{page\_goal}}\\{page\_goal} by $x\times f/1000$.

Special processing is required if \TeX\ detects that there is not enough space on
the current page to accommodate the complete insertion.
If already a previous insert did not fit on the page, simply the \index{floating penalty+\\{floating\_penalty}}\\{floating\_penalty}
as given in the insert node is added to the total \index{insert penalties+\\{insert\_penalties}}\\{insert\_penalties}.
Otherwise \TeX\ will test that the total natural height plus depth of box254
including $x$ does not exceed the maximum size $d$ and that the
$\index{page total+\\{page\_total}}\\{page\_total} + \index{page depth+\\{page\_depth}}\\{page\_depth} + x\times f/1000 - \index{page shrink+\\{page\_shrink}}\\{page\_shrink} \le \index{page goal+\\{page\_goal}}\\{page\_goal}$.
If one of these tests fails, the current insertion
is split in such a way as to make the size of the remaining insertions just pass the tests
just stated.

Whenever a glue node, or penalty node, or a kern node that is followed by glue arrives
at the page builder, it rates the current position as a possible end of the page based on
the shrinkability of the page and the difference between \index{page total+\\{page\_total}}\\{page\_total} and \index{page goal+\\{page\_goal}}\\{page\_goal}.
As the page fills, the page breaks tend to become better and better until the
page starts to get overfull and the page breaks get worse and worse until
they reach the point where they become \index{awful bad+\\{awful\_bad}}\\{awful\_bad}. At that point,
the page builder returns to the best page break found so far and fires up the
output routine.

Let's look next at the problems that show up when implementing a replacement mechanism for \HINT/.

\enumerate
\item
An insertion node can not always specify its height $x$ because insertions may contain paragraphs that need
to be broken in lines and the height of a paragraph depends in some non obvious way on
its width.

\item
Before the viewer can compute $x$ it needs to know the width of the insertion. Just imagine
displaying footnotes in two columns or setting notes in the margin. Knowing the width, it
can pack the vertical material and derive its height and depth.

\item
\TeX's plain format provides an insert macro that checks whether there is still space
on the current page, and if so, it creates a contribution to the main text body, otherwise it
creates a topinsert. Such a decision needs to be postponed to the \HINT/ viewer.

\item
\HINT/ has no output routines that would specify something like the space and the rule preceding the footnote.

\item
\TeX's output routines have the ability to inspect the content of the boxes,
split them, and distribute the content over the page.
For example, the output routine for an index set in two column format might
expect a box containing index entries up to a height of $2\times\.{vsize}$.
It will split this box in the middle and display the top part in the left
column and the bottom part in the right column. With this approach, the
last page will show two partly filled columns of about equal size.

\item
\HINT/ has no mark nodes that could be used to create page headers or footers.
Marks, like output routines, contain token lists and need the full \TeX\ interpreter
for processing them. Hence, \HINT/ does not support mark nodes.
\endenumerate

Here now is the solution I have chosen for \HINT/:

Instead of output routines, \HINT/ will use page templates.
Page templates are basically vertical boxes with placeholders marking the
positions where the content of the box registers, filled by the page builder,
should appear.
To output the page, the viewer traverses the page template,
replaces the placeholders by the appropriate box content, and
sets the glue. Inside the page template, we can use insert nodes to act
as placeholders.

It is only natural to treat the page's main body, the
inserts, and the marks using the same mechanism. We call this
mechanism a content stream\index{stream}.
Content streams are identified by a stream number in the range 0 to 254;
the number 255 is used to indicate an invalid stream number.
The stream number 0 is reserved for the main content stream; it is always defined.
Besides the main content stream, there are three types of streams:
\itemize
\item normal streams correspond to \TeX's inserts and accumulate content on the page,
\item first\index{first stream} streams correspond to \TeX's first marks and will contain only the first insertion of the page,
\item last\index{last stream} streams correspond to \TeX's bottom marks and will contain only the last insertion of the page, and
\item top\index{top stream} streams correspond to \TeX's top marks. Top streams are not yet implemented.
\enditemize

Nodes from the content section are considered contributions to stream 0 except
for insert nodes which will specify the stream number explicitly.
If the stream is not defined or is not used in the current page template, its content is simply ignored.

The page builder needs a mechanism to redirect contributions from one content
stream to another content stream based on the availability of space.
Hence a \HINT/ content stream can optionally specify a preferred stream number,
where content should go if there is still space available, a next stream number,
where content should go if the present stream has no more space available, and
a split ratio if the content is to be split between these two streams before
filling in the template.

Various stream parameters govern the treatment of contributions to the stream
and the page building process.

\itemize
\item The magnification factor $f$: Inserting a box of height $h$ to this stream will contribute $h\times f/1000$
to the height of the page under construction. For example, a stream
that uses a two column format will have an $f$ value of 500; a stream
that specifies notes that will be displayed in the page margin will
have an $f$ value of zero.

\item The height $h$: The extended dimension $h$ gives the maximum height this
stream is allowed to occupy on the current page.
To continue the previous example, a stream that will be split into two columns
will have $h=2\cdot\.{vsize}$ , and a stream that specifies
notes that will be displayed in the page margin will have
$h=1\cdot\.{vsize}$.  You can restrict the amount of space occupied by
footnotes to the bottom quarter by setting the corresponding $h$ value
to $h=0.25\cdot\.{vsize}$.

\item The depth $d$: The dimension $d$ gives the maximum depth this
stream is allowed to have after formatting.

\item The width $w$: The extended dimension $w$ gives the width of this stream
when formatting its content. For example margin notes
should have the width of the margin less some surrounding space.

\item The ``before'' list $b$: If there are any contributions to this
stream on the current page, the material in list $b$
is inserted {\it before\/} the material from the stream itself. For
example, the short line that separates the footnotes from the main
page will go, together with some surrounding space, into the list $b$.

\item The top skip glue $g$: This glue is inserted between the material
from list $b$ and the first box of the stream, reduced
by the height of the first box. Hence it specifies the distance between
the material in $b$ and the first baseline of the stream content.

\item The ``after'' list $a$: The list $a$ is treated like list $b$ but
its material is placed {\it after\/} the  material from the stream itself.

\item The ``preferred'' stream number $p$:  If $p\ne 255$, it is the number of
the {\it preferred\/} stream. If stream $p$ has still
enough room to accommodate the current contribution, move the
contribution to stream $p$, otherwise keep it.  For example, you can
move an illustration to the main content stream, provided there is
still enough space for it on the current page, by setting $p=0$.

\item The ``next'' stream number $n$: If $n\ne 255$, it is the number of the
{\it next\/} stream. If a contribution can not be
accommodated in stream $p$ nor in the current stream, treat it as an
insertion to stream $n$.  For example, you can move contributions to
the next column after the first column is full, or move illustrations
to a separate page at the end of the chapter.

\item The split ratio\index{split ratio} $r$: If $r$ is positive, both $p$ and $n$ must
be valid stream numbers and contents is not immediately moved to stream $p$ or $n$ as described before.
Instead the content is kept in the stream itself until the current page is complete.
Then, before inserting the streams into the page template, the content of
this stream is formatted as a vertical box, the vertical box is
split into a top fraction and a bottom fraction in the ratio $r/1000$
for the top and $(1000-r)/1000$ for the bottom, and finally the top
fraction is moved to stream $p$ and the bottom fraction to stream
$n$. You can use this feature for example to implement footnotes
arranged in two columns of about equal size. By collecting all the
footnotes in one stream and then splitting the footnotes with $r=500$
before placing them on the page into a right and left column.  Even
three or more columns can be implemented by cascades of streams using
this mechanism.
\enditemize

\subsection{Stream Definitions}
\index{stream}
There are four types of streams:  normal streams that work like \TeX's inserts;
and first, last, and top streams that work like \TeX's marks.
For the latter  types, the long format uses a matching keyword and the
short format the two least significant info bits. All stream definitions
start with the stream number.
In definitions of  normal streams after the number follows in this order
\itemize
\item the maximum insertion height,
\item the magnification factor, and
\item information about splitting the stream.
It consists of: a preferred stream, a next stream, and a split ratio.
An asterisk indicates a missing stream reference, in the
short format the stream number 255 serves the same purpose.
\item All stream definitions finish with the ``before'' list,
\item an extended dimension node specifying the width of the inserted material,
\item the top skip glue,
\item  the ``after'' list,
\item and the total height, stretchability, and shrinkability of the material in
the ``before'' and ``after'' list.
\enditemize

A special case is the stream definition for stream 0, the main content stream.
None of the above information is necessary for it so it is ommited.
Stream definitions, including the definition of stream 0,
occur only inside page template definititions\index{template}
where they occur twice in two different roles:
In the stream definition list, they define properties of the stream
and in the template they mark the insertion point (see section~\secref{page}).
In the latter case, stream nodes just contain the stream number.
Because a template looks like ordinary vertical material,
we like to use the same functions for parsing it.
But stream definitions are very different from stream content
nodes. To solve the problem for the long format,
the scanner will return two different tokens
when it sees the keyword ``{\tt stream}''.
In the definition section, it will return
\index{STREAMDEF+\ts{STREAMDEF}}\ts{STREAMDEF} and in the content section \index{STREAM+\ts{STREAM}}\ts{STREAM}.
The same problem is solved in the short format
by using the \\{b100} bit to mark a definition.

\goodbreak
\vbox{\readcode\vskip -\baselineskip\putcode}

\Y\par
\par
\par
\par
\par
\par
\par
\par
\par
\par
\par
\par
\par
\Y\B\4\X2:symbols\X${}\mathrel+\E{}$\6
\8\%\&{token} \index{STREAM+\ts{STREAM}}\ts{STREAM}\5\.{"stream"}\6
\8\%\&{token} \index{STREAMDEF+\ts{STREAMDEF}}\ts{STREAMDEF}\5\.{"stream\ definition"}\6
\8\%\&{token} \index{FIRST+\ts{FIRST}}\ts{FIRST}\5\.{"first"}\6
\8\%\&{token} \index{LAST+\ts{LAST}}\ts{LAST}\5\.{"last"}\6
\8\%\&{token} \index{TOP+\ts{TOP}}\ts{TOP}\5\.{"top"}\6
\8\%\&{token} \index{NOREFERENCE+\ts{NOREFERENCE}}\ts{NOREFERENCE}\5\.{"*"}\6
\8\%\index{type+\&{type}}\&{type} $<$ \index{info+\\{info}}\\{info} $>$ \index{stream type+\nts{stream\_type}}\nts{stream\_type} \6
\8\%\index{type+\&{type}}\&{type} $<$ \|u $>$ \index{stream ref+\nts{stream\_ref}}\nts{stream\_ref} \6
\8\%\index{type+\&{type}}\&{type} $<$ \index{rf+\\{rf}}\\{rf} $>$ \index{stream def node+\nts{stream\_def\_node}}\nts{stream\_def\_node}
\Y
\fi

\M{237}

\Y\B\4\X3:scanning rules\X${}\mathrel+\E{}$\6
${}\8\re{\vb{stream}}{}$\ac\&{if} ${}(\index{section no+\\{section\_no}}\\{section\_no}\E\T{1}){}$\1\5
\&{return} \index{STREAMDEF+\ts{STREAMDEF}}\ts{STREAMDEF};\2\6
\&{else}\1\5
\&{return} \index{STREAM+\ts{STREAM}}\ts{STREAM};\5
\2\eac\7
${}\8\re{\vb{first}}{}$\ac\&{return} \index{FIRST+\ts{FIRST}}\ts{FIRST};\eac\7
${}\8\re{\vb{last}}{}$\ac\&{return} \index{LAST+\ts{LAST}}\ts{LAST};\eac\7
${}\8\re{\vb{top}}{}$\ac\&{return} \index{TOP+\ts{TOP}}\ts{TOP};\eac\7
${}\8\re{\vb{\\*}}{}$\ac\&{return} \index{NOREFERENCE+\ts{NOREFERENCE}}\ts{NOREFERENCE};\eac
\Y
\fi

\M{238}

\Y\B\4\X5:parsing rules\X${}\mathrel+\E{}$\6
\index{stream link+\nts{stream\_link}}\nts{stream\_link}: \1\1\5
\index{ref+\nts{ref}}\nts{ref}\5
${}\{{}$\1\5
${}\index{REF RNG+\.{REF\_RNG}}\.{REF\_RNG}(\index{stream kind+\\{stream\_kind}}\\{stream\_kind},\39\.{\$1});{}$\5
${}\}{}$\2\6
\4\hbox to 0.5em{\hss${}\OR{}$}\5
\index{NOREFERENCE+\ts{NOREFERENCE}}\ts{NOREFERENCE}\5
${}\{{}$\1\5
\index{HPUT8+\.{HPUT8}}\.{HPUT8}(\T{255});\5
${}\}{}$\2;\2\2\7
\index{stream split+\nts{stream\_split}}\nts{stream\_split}: \1\1\5
\index{stream link+\nts{stream\_link}}\nts{stream\_link}\5
\index{stream link+\nts{stream\_link}}\nts{stream\_link}\5
\index{UNSIGNED+\ts{UNSIGNED}}\ts{UNSIGNED}\6
${}\{{}$\1\5
${}\.{RNG}(\.{"split\ ratio"},\39\.{\$3},\39\T{0},\39\T{1000});{}$\5
\index{HPUT16+\.{HPUT16}}\.{HPUT16}(\.{\$3});\5
${}\}{}$\2;\2\2\7
\index{stream info+\nts{stream\_info}}\nts{stream\_info}: \1\1\5
\index{xdimen node+\nts{xdimen\_node}}\nts{xdimen\_node}\5
\index{UNSIGNED+\ts{UNSIGNED}}\ts{UNSIGNED}\6
${}\{{}$\1\5
${}\.{RNG}(\.{"magnification\ facto}\)\.{r"},\39\.{\$2},\39\T{0},\39\T{1000});{}$\5
\index{HPUT16+\.{HPUT16}}\.{HPUT16}(\.{\$2});\5
${}\}{}$\2\5
\index{stream split+\nts{stream\_split}}\nts{stream\_split};\2\2\7
\index{stream type+\nts{stream\_type}}\nts{stream\_type}: \1\1\5
\index{stream info+\nts{stream\_info}}\nts{stream\_info}\5
${}\{{}$\1\5
${}\.{\$\$}\K\T{0};{}$\5
${}\}{}$\2\6
\4\hbox to 0.5em{\hss${}\OR{}$}\5
\index{FIRST+\ts{FIRST}}\ts{FIRST}\5
${}\{{}$\1\5
${}\.{\$\$}\K\T{1};{}$\5
${}\}{}$\5
\2\hbox to 0.5em{\hss${}\OR{}$}\5
\index{LAST+\ts{LAST}}\ts{LAST}\5
${}\{{}$\1\5
${}\.{\$\$}\K\T{2};{}$\5
${}\}{}$\5
\2\hbox to 0.5em{\hss${}\OR{}$}\5
\index{TOP+\ts{TOP}}\ts{TOP}\5
${}\{{}$\1\5
${}\.{\$\$}\K\T{3};{}$\5
${}\}{}$\2;\2\2\7
\index{stream def node+\nts{stream\_def\_node}}\nts{stream\_def\_node}: \1\1\5
\index{start+\nts{start}}\nts{start}\5
\index{STREAMDEF+\ts{STREAMDEF}}\ts{STREAMDEF}\5
\index{ref+\nts{ref}}\nts{ref}\5
\index{stream type+\nts{stream\_type}}\nts{stream\_type}\6
\index{list+\nts{list}}\nts{list}\5
\index{xdimen node+\nts{xdimen\_node}}\nts{xdimen\_node}\5
\index{glue node+\nts{glue\_node}}\nts{glue\_node}\5
\index{list+\nts{list}}\nts{list}\5
\index{glue node+\nts{glue\_node}}\nts{glue\_node}\5
\index{END+\ts{END}}\ts{END}\6
${}\{{}$\5
\1${}\index{DEF+\.{DEF}}\.{DEF}(\.{\$\$},\39\index{stream kind+\\{stream\_kind}}\\{stream\_kind},\39\.{\$3}){}$;\5
${}\index{hput tags+\\{hput\_tags}}\\{hput\_tags}(\.{\$1},\39\.{TAG}(\index{stream kind+\\{stream\_kind}}\\{stream\_kind},\39\.{\$4}\OR\\{b100}));{}$\5
${}\}{}$\2;\2\2\7
\index{stream ins node+\nts{stream\_ins\_node}}\nts{stream\_ins\_node}: \1\1\5
\index{start+\nts{start}}\nts{start}\5
\index{STREAMDEF+\ts{STREAMDEF}}\ts{STREAMDEF}\5
\index{ref+\nts{ref}}\nts{ref}\5
\index{END+\ts{END}}\ts{END}\6
${}\{{}$\1\5
${}\.{RNG}(\.{"Stream\ insertion"},\39\.{\$3},\39\T{0},\39\\{max\_ref}[\index{stream kind+\\{stream\_kind}}\\{stream\_kind}]);{}$\5
${}\index{hput tags+\\{hput\_tags}}\\{hput\_tags}(\.{\$1},\39\.{TAG}(\index{stream kind+\\{stream\_kind}}\\{stream\_kind},\39\\{b100}));{}$\5
${}\}{}$\2;\2\2\7
\index{content node+\nts{content\_node}}\nts{content\_node}: \1\1\5
\index{stream def node+\nts{stream\_def\_node}}\nts{stream\_def\_node}\5
\hbox to 0.5em{\hss${}\OR{}$}\5
\index{stream ins node+\nts{stream\_ins\_node}}\nts{stream\_ins\_node};\2\2
\Y
\fi

\M{239}


\goodbreak
\vbox{\getcode\vskip -\baselineskip\writecode}



\Y\B\4\X239:get stream information for normal streams\X${}\E{}$\1\6
\4${}\{{}$\5
\index{xdimen t+\&{xdimen\_t}}\&{xdimen\_t} \|x;\6
\&{uint16\_t} \|f${},{}$ \|r;\6
\&{uint8\_t} \|n;\7
${}\.{DBG}(\index{DBGDEF+\.{DBGDEF}}\.{DBGDEF},\39\.{"Defining\ normal\ str}\)\.{eam\ \%d\ at\ "}\.{SIZE\_F}\.{"\\n"},\39{*}(\index{hpos+\\{hpos}}\\{hpos}-\T{1}),\39\index{hpos+\\{hpos}}\\{hpos}-\index{hstart+\\{hstart}}\\{hstart}-\T{2});{}$\6
${}\index{hget xdimen node+\\{hget\_xdimen\_node}}\\{hget\_xdimen\_node}({\AND}\|x){}$;\5
${}\index{hwrite xdimen node+\\{hwrite\_xdimen\_node}}\\{hwrite\_xdimen\_node}({\AND}\|x);{}$\6
\index{HGET16+\.{HGET16}}\.{HGET16}(\|f);\5
${}\.{RNG}(\.{"magnification\ facto}\)\.{r"},\39\|f,\39\T{0},\39\T{1000}){}$;\5
${}\index{hwritef+\\{hwritef}}\\{hwritef}(\.{"\ \%d"},\39\|f);{}$\6
${}\|n\K\index{HGET8+\.{HGET8}}\.{HGET8};{}$\6
\&{if} ${}(\|n\E\T{255}){}$\1\5
\index{hwritef+\\{hwritef}}\\{hwritef}(\.{"\ *"});\2\6
\&{else}\5
\1${}\{{}$\5
${}\index{REF RNG+\.{REF\_RNG}}\.{REF\_RNG}(\index{stream kind+\\{stream\_kind}}\\{stream\_kind},\39\|n){}$;\5
\index{hwrite ref+\\{hwrite\_ref}}\\{hwrite\_ref}(\|n);\5
${}\}{}$\2\6
${}\|n\K\index{HGET8+\.{HGET8}}\.{HGET8};{}$\6
\&{if} ${}(\|n\E\T{255}){}$\1\5
\index{hwritef+\\{hwritef}}\\{hwritef}(\.{"\ *"});\2\6
\&{else}\5
\1${}\{{}$\5
${}\index{REF RNG+\.{REF\_RNG}}\.{REF\_RNG}(\index{stream kind+\\{stream\_kind}}\\{stream\_kind},\39\|n){}$;\5
\index{hwrite ref+\\{hwrite\_ref}}\\{hwrite\_ref}(\|n);\5
${}\}{}$\2\6
\index{HGET16+\.{HGET16}}\.{HGET16}(\|r);\6
${}\.{RNG}(\.{"split\ ratio"},\39\|r,\39\T{0},\39\T{1000});{}$\6
${}\index{hwritef+\\{hwritef}}\\{hwritef}(\.{"\ \%d"},\39\|r);{}$\6
\4${}\}{}$\2
\U240.\Y
\fi

\M{240}

\Y\B\4\X16:get functions\X${}\mathrel+\E{}$\6
\&{static} \&{bool} \index{hget stream def+\\{hget\_stream\_def}}\\{hget\_stream\_def}(\&{void})\1\1\2\2\1\6
\4${}\{{}$\6
\&{if} ${}(\index{KIND+\.{KIND}}\.{KIND}({*}\index{hpos+\\{hpos}}\\{hpos})\I\index{stream kind+\\{stream\_kind}}\\{stream\_kind}\V\R(\index{INFO+\.{INFO}}\.{INFO}({*}\index{hpos+\\{hpos}}\\{hpos})\AND\\{b100})){}$\1\5
\&{return} \\{false};\2\6
\&{else}\5
\1${}\{{}$\5
\index{ref t+\&{ref\_t}}\&{ref\_t} \index{df+\\{df}}\\{df};\7
\X14:read the start byte \|a\X\6
${}\.{DBG}(\index{DBGDEF+\.{DBGDEF}}\.{DBGDEF},\39\.{"Defining\ stream\ \%d\ }\)\.{at\ "}\.{SIZE\_F}\.{"\\n"},\39{*}\index{hpos+\\{hpos}}\\{hpos},\39\index{hpos+\\{hpos}}\\{hpos}-\index{hstart+\\{hstart}}\\{hstart}-\T{1});{}$\6
${}\index{DEF+\.{DEF}}\.{DEF}(\index{df+\\{df}}\\{df},\39\index{stream kind+\\{stream\_kind}}\\{stream\_kind},\39\index{HGET8+\.{HGET8}}\.{HGET8});{}$\6
\index{hwrite start+\\{hwrite\_start}}\\{hwrite\_start}(\,);\5
\index{hwritef+\\{hwritef}}\\{hwritef}(\.{"stream"});\5
\5
${}\index{hwrite ref+\\{hwrite\_ref}}\\{hwrite\_ref}(\index{df+\\{df}}\\{df}.\|n);{}$\6
\&{if} ${}(\index{df+\\{df}}\\{df}.\|n>\T{0}){}$\5
\1${}\{{}$\5
\index{xdimen t+\&{xdimen\_t}}\&{xdimen\_t} \|x;\6
\index{list t+\&{list\_t}}\&{list\_t} \|l;\7
\&{if} ${}(\index{INFO+\.{INFO}}\.{INFO}(\|a)\E\\{b100}){}$\1\5
\X239:get stream information for normal streams\X\2\6
\&{else} \&{if} ${}(\index{INFO+\.{INFO}}\.{INFO}(\|a)\E\\{b101}){}$\1\5
\index{hwritef+\\{hwritef}}\\{hwritef}(\.{"\ first"});\2\6
\&{else} \&{if} ${}(\index{INFO+\.{INFO}}\.{INFO}(\|a)\E\\{b110}){}$\1\5
\index{hwritef+\\{hwritef}}\\{hwritef}(\.{"\ last"});\2\6
\&{else} \&{if} ${}(\index{INFO+\.{INFO}}\.{INFO}(\|a)\E\\{b111}){}$\1\5
\index{hwritef+\\{hwritef}}\\{hwritef}(\.{"\ top"});\2\6
${}\index{hget list+\\{hget\_list}}\\{hget\_list}({\AND}\|l){}$;\5
${}\index{hwrite list+\\{hwrite\_list}}\\{hwrite\_list}({\AND}\|l);{}$\6
${}\index{hget xdimen node+\\{hget\_xdimen\_node}}\\{hget\_xdimen\_node}({\AND}\|x){}$;\5
${}\index{hwrite xdimen node+\\{hwrite\_xdimen\_node}}\\{hwrite\_xdimen\_node}({\AND}\|x);{}$\6
\index{hget glue node+\\{hget\_glue\_node}}\\{hget\_glue\_node}(\,);\6
${}\index{hget list+\\{hget\_list}}\\{hget\_list}({\AND}\|l){}$;\5
${}\index{hwrite list+\\{hwrite\_list}}\\{hwrite\_list}({\AND}\|l);{}$\6
\index{hget glue node+\\{hget\_glue\_node}}\\{hget\_glue\_node}(\,);\6
\4${}\}{}$\2\6
\X15:read and check the end byte \|z\X\6
\index{hwrite end+\\{hwrite\_end}}\\{hwrite\_end}(\,);\6
\&{return} \\{true};\6
\4${}\}{}$\2\6
\4${}\}{}$\2
\Y
\fi

\M{241}

When stream definitions are part of the page template, we call them
stream insertion points.
They contain only the stream reference and
are parsed by the usual content parsing functions.

\Y\B\4\X18:cases to get content\X${}\mathrel+\E{}$\6
\4\&{case} \.{TAG}${}(\index{stream kind+\\{stream\_kind}}\\{stream\_kind},\39\\{b100}){}$:\1\6
\4${}\{{}$\5
\&{uint8\_t} \|n${}\K\index{HGET8+\.{HGET8}}\.{HGET8}{}$;\5
${}\index{REF RNG+\.{REF\_RNG}}\.{REF\_RNG}(\index{stream kind+\\{stream\_kind}}\\{stream\_kind},\39\|n){}$;\5
\index{hwrite ref+\\{hwrite\_ref}}\\{hwrite\_ref}(\|n);\5
\&{break};\5
${}\}{}$\2
\Y
\fi

\M{242}


\subsection{Stream Content}
Stream\index{stream} nodes occur in the content section where they
must not be inside other nodes except toplevel
paragraph\index{paragraph} nodes.  A normal stream node contains in this
order: the stream reference number, the optional stream parameters,
and the stream content.  The content is either a vertical box or an
extended vertical box.  The stream parameters consists of the
\index{floating penalty+\\{floating\_penalty}}\\{floating\_penalty}, the \index{split max depth+\\{split\_max\_depth}}\\{split\_max\_depth}, and the
\index{split top skip+\\{split\_top\_skip}}\\{split\_top\_skip}. The parameterlist can be given
explicitely or as a reference.
In the short format, the info bits \\{b010} indicate
a normal stream content node with an explicit parameter list
and the info bits \\{b000} a normal stream with a parameter list reference.
Note that an empty parameter list is simply represented as an omited
explicit list.

If the info bit \\{b001} is set, we have a content node of type top, first,
or last. In this case, the short format has instead of the parameter list
a single byte indicating the type.
These types are currently not yet implemented.

\goodbreak
\vbox{\readcode\vskip -\baselineskip\putcode}

\Y\par
\Y\B\4\X2:symbols\X${}\mathrel+\E{}$\6
\8\%\index{type+\&{type}}\&{type} $<$ \index{info+\\{info}}\\{info} $>$ \index{stream+\nts{stream}}\nts{stream}
\Y
\fi

\M{243}


\Y
\fi

\M{244}
\Y\B\4\X5:parsing rules\X${}\mathrel+\E{}$\6
\index{stream+\nts{stream}}\nts{stream}: \1\1\5
\index{list+\nts{list}}\nts{list}\5
${}\{{}$\1\5
${}\.{\$\$}\K\\{b010};{}$\5
${}\}{}$\2\6
\4\hbox to 0.5em{\hss${}\OR{}$}\5
\index{param list node+\nts{param\_list\_node}}\nts{param\_list\_node}\5
\index{list+\nts{list}}\nts{list}\5
${}\{{}$\1\5
${}\.{\$\$}\K\\{b010};{}$\5
${}\}{}$\2\6
\4\hbox to 0.5em{\hss${}\OR{}$}\5
\index{param ref+\nts{param\_ref}}\nts{param\_ref}\5
\index{list+\nts{list}}\nts{list}\5
${}\{{}$\1\5
${}\.{\$\$}\K\\{b000};{}$\5
${}\}{}$\2;\2\2\7
\index{content node+\nts{content\_node}}\nts{content\_node}: \1\1\5
\index{start+\nts{start}}\nts{start}\5
\index{STREAM+\ts{STREAM}}\ts{STREAM}\5
\index{stream ref+\nts{stream\_ref}}\nts{stream\_ref}\5
\index{stream+\nts{stream}}\nts{stream}\5
\index{END+\ts{END}}\ts{END}\6
${}\{{}$\5
\1${}\index{hput tags+\\{hput\_tags}}\\{hput\_tags}(\.{\$1},\39\.{TAG}(\index{stream kind+\\{stream\_kind}}\\{stream\_kind},\39\.{\$4})){}$;\5
${}\}{}$\2;\2\2
\Y
\fi

\M{245}

\goodbreak
\vbox{\getcode\vskip -\baselineskip\writecode}

\Y\B\4\X18:cases to get content\X${}\mathrel+\E{}$\6
\4\&{case} \.{TAG}${}(\index{stream kind+\\{stream\_kind}}\\{stream\_kind},\39\\{b000}){}$:\5
\index{HGET STREAM+\.{HGET\_STREAM}}\.{HGET\_STREAM}(\\{b000});\5
\&{break};\6
\4\&{case} \.{TAG}${}(\index{stream kind+\\{stream\_kind}}\\{stream\_kind},\39\\{b010}){}$:\5
\index{HGET STREAM+\.{HGET\_STREAM}}\.{HGET\_STREAM}(\\{b010});\5
\&{break};
\Y
\fi

\M{246}

When we read stream numbers, we relax the define before use policy.
We just check, that the stream number is in the correct range.

\Y\B\4\X17:get macros\X${}\mathrel+\E{}$\6
\8\#\&{define} \index{HGET STREAM+\.{HGET\_STREAM}}\.{HGET\_STREAM}(\|I)\1\6
\4${}\{{}$\5
\&{uint8\_t} \|n${}\K\index{HGET8+\.{HGET8}}\.{HGET8}{}$;\5
${}\index{REF RNG+\.{REF\_RNG}}\.{REF\_RNG}(\index{stream kind+\\{stream\_kind}}\\{stream\_kind},\39\|n){}$;\5
\index{hwrite ref+\\{hwrite\_ref}}\\{hwrite\_ref}(\|n);\5
${}\}{}$\2\7
\&{if} ${}((\|I)\AND\\{b010}){}$\5
\1${}\{{}$\5
\index{list t+\&{list\_t}}\&{list\_t} \|l;\5
${}\index{hget param list node+\\{hget\_param\_list\_node}}\\{hget\_param\_list\_node}({\AND}\|l){}$;\5
${}\index{hwrite param list node+\\{hwrite\_param\_list\_node}}\\{hwrite\_param\_list\_node}({\AND}\|l){}$;\5
${}\}{}$\2\6
\&{else}\1\5
\index{HGET REF+\.{HGET\_REF}}\.{HGET\_REF}(\index{param kind+\\{param\_kind}}\\{param\_kind});\2\1\6
\4${}\{{}$\5
\index{list t+\&{list\_t}}\&{list\_t} \|l;\5
${}\index{hget list+\\{hget\_list}}\\{hget\_list}({\AND}\|l){}$;\5
${}\index{hwrite list+\\{hwrite\_list}}\\{hwrite\_list}({\AND}\|l){}$;\5
${}\}{}$\2
\Y
\fi

\M{247}




\subsection{Page Template Definitions}\label{page}
A \HINT/ file can define multiple page templates\index{template}. Not only
might an index demand a different page layout than the main body of text,
also the front page or the chapter headings might use their own page templates.
Further, the author of a \HINT/ file might define a two column format as
an alternative to a single column format to be used if the display area
is wide enough.

To help in selecting the right page template, page template definitions start with
a name and an optional priority\index{priority}; the default priority is 1.
The names might appear in a menu from which the user
can select a page layout that best fits her taste.
Without user interaction, the
system can pick the template with the highest priority. Of course,
a user interface might provide means to alter priorities. Future
versions might include sophisticated feature-vectors that
identify templates that are good for large or small displays,
landscape or portrait mode, etc \dots

After the priority follows a glue node to specify the topskip glue
and the dimension of the maximum page depth,
an extended dimension to specify the page height and
an extended dimension to specify the page width.

Then follows the main part of a page template definition: the template.
The template consists of a list of vertical material.
To construct the page, this list will be placed
into a vertical box and the glue will be set.
But of course before doing so, the viewer will
scan the list and replace all stream nsertion points
by the appropriate content streams.

Let's call the vertical box obtained this way ``the page''.
The page will fill the entire display area top to bottom and left to right.
It defines not only the appearance of the main body of text
but also the margins, the header, and the footer.
Because the \.{vsize} and  \.{hsize} variables of \TeX\ are used for
the vertical and horizontal dimension of the main body of text---they
do not include the margins---the page will usually be wider than \.{hsize}
and taller than \.{vsize}. The dimensions of the page are part
of the page template. The viewer, knowing the actual dimensions
of the display area, can derive from them the actual values of \.{hsize}
and \.{vsize}.

Stream definitions are listed after the template.

The page template with number 0 is always defined and has priority 0.
It will display just the main content stream. It puts a small margin
of $\.{hsize}/8 -4.5\hbox{pt}$ all around it.
Given a letter size page, 8.5 inch wide, this formula yields a margin of 1 inch,
matching \TeX's plain format. The margin will be positive as long as
the page is wider than $1/2$ inch. For narrower pages, there will be no
margin at all. In general, the \HINT/ viewer will never set {\tt hsize} larger
than the width of the page and {\tt vsize} larger than its height.

%8.5 in should give 1 inch margin 2/17
%612pt should give 72pt margin
%72pt = 612/8-4.5pt
%This would give a positive margin starting at 36pt or 1/2 inch

\goodbreak
\vbox{\readcode\vskip -\baselineskip\putcode}

\Y\par
\par
\par
\par
\Y\B\4\X2:symbols\X${}\mathrel+\E{}$\6
\8\%\&{token} \index{PAGE+\ts{PAGE}}\ts{PAGE}\5\.{"page"}
\Y
\fi

\M{248}

\Y\B\4\X3:scanning rules\X${}\mathrel+\E{}$\6
${}\8\re{\vb{page}}{}$\ac\&{return} \index{PAGE+\ts{PAGE}}\ts{PAGE};\eac
\Y
\fi

\M{249}

\Y\B\4\X5:parsing rules\X${}\mathrel+\E{}$\6
\index{page priority+\nts{page\_priority}}\nts{page\_priority}: \1\1\5
${}\{{}$\1\5
\index{HPUT8+\.{HPUT8}}\.{HPUT8}(\T{1});\5
${}\}{}$\2\6
\4\hbox to 0.5em{\hss${}\OR{}$}\5
\index{UNSIGNED+\ts{UNSIGNED}}\ts{UNSIGNED}\5
${}\{{}$\1\5
${}\.{RNG}(\.{"page\ priority"},\39\.{\$1},\39\T{0},\39\T{255});{}$\5
\index{HPUT8+\.{HPUT8}}\.{HPUT8}(\.{\$1});\5
${}\}{}$\2;\2\2\7
\index{stream def list+\nts{stream\_def\_list}}\nts{stream\_def\_list}: \1\1\6
\4\hbox to 0.5em{\hss${}\OR{}$}\5
\index{stream def list+\nts{stream\_def\_list}}\nts{stream\_def\_list}\5
\index{stream def node+\nts{stream\_def\_node}}\nts{stream\_def\_node};\2\2\7
\index{page+\nts{page}}\nts{page}: \1\1\5
\index{string+\nts{string}}\nts{string}\5
${}\{{}$\1\5
\index{hput string+\\{hput\_string}}\\{hput\_string}(\.{\$1});\5
${}\}{}$\2\5
\index{page priority+\nts{page\_priority}}\nts{page\_priority}\5
\index{glue node+\nts{glue\_node}}\nts{glue\_node}\5
\index{dimension+\nts{dimension}}\nts{dimension}\5
${}\{{}$\1\5
\index{HPUT32+\.{HPUT32}}\.{HPUT32}(\.{\$5});\5
${}\}{}$\2\5
\index{xdimen node+\nts{xdimen\_node}}\nts{xdimen\_node}\5
\index{xdimen node+\nts{xdimen\_node}}\nts{xdimen\_node}\5
\index{list+\nts{list}}\nts{list}\5
\index{stream def list+\nts{stream\_def\_list}}\nts{stream\_def\_list};\2\2
\Y
\fi

\M{250}

\goodbreak
\vbox{\getcode\vskip -\baselineskip\writecode}
\Y\B\4\X16:get functions\X${}\mathrel+\E{}$\6
\&{void} \index{hget page+\\{hget\_page}}\\{hget\_page}(\&{void})\1\1\2\2\1\6
\4${}\{{}$\5
\&{char} ${}{*}\|n;{}$\6
\&{uint8\_t} \|p;\6
\index{xdimen t+\&{xdimen\_t}}\&{xdimen\_t} \|x;\6
\index{list t+\&{list\_t}}\&{list\_t} \|l;\7
\index{HGET STRING+\.{HGET\_STRING}}\.{HGET\_STRING}(\|n);\5
\index{hwrite string+\\{hwrite\_string}}\\{hwrite\_string}(\|n);\6
${}\|p\K\index{HGET8+\.{HGET8}}\.{HGET8}{}$;\5
\&{if} ${}(\|p\I\T{1}){}$\1\5
${}\index{hwritef+\\{hwritef}}\\{hwritef}(\.{"\ \%d"},\39\|p);{}$\2\6
\index{hget glue node+\\{hget\_glue\_node}}\\{hget\_glue\_node}(\,);\6
\index{hget dimen+\\{hget\_dimen}}\\{hget\_dimen}(\,);\6
${}\index{hget xdimen node+\\{hget\_xdimen\_node}}\\{hget\_xdimen\_node}({\AND}\|x){}$;\5
${}\index{hwrite xdimen node+\\{hwrite\_xdimen\_node}}\\{hwrite\_xdimen\_node}({\AND}\|x){}$;\C{ page height }\6
${}\index{hget xdimen node+\\{hget\_xdimen\_node}}\\{hget\_xdimen\_node}({\AND}\|x){}$;\5
${}\index{hwrite xdimen node+\\{hwrite\_xdimen\_node}}\\{hwrite\_xdimen\_node}({\AND}\|x){}$;\C{ page width }\6
${}\index{hget list+\\{hget\_list}}\\{hget\_list}({\AND}\|l){}$;\5
${}\index{hwrite list+\\{hwrite\_list}}\\{hwrite\_list}({\AND}\|l);{}$\6
\&{while} (\index{hget stream def+\\{hget\_stream\_def}}\\{hget\_stream\_def}(\,))\1\5
\&{continue};\2\6
\4${}\}{}$\2
\Y
\fi

\M{251}

\subsection{Page Ranges}\label{range}\index{page range}
Not every template\index{template} is necessarily valid for the entire content
section.  A page range specifies a start position $a$ and an end
position $b$ in the content section and the page template is valid if
the start position $p$ of the page is within that range: $a\le p < b$.
If paging backward this definition might cause problems because the
start position of the page is known only after the page has been
build.  In this case, the viewer might choose a page template based on
the position at the bottom of the page. If it turns out that this ``bottom template''
is no longer valid when the page builder has found the start of the
page, the viewer might display the page anyway with the bottom
template, it might just display the page with the new ``top
template'', or rerun the whole page building process using this time
the ``top template''.  Neither of these alternatives is guaranteed to
produce a perfect result because changing the page template might
change the amount of material that fits on the page. A good page
template design should take this into account.

The representation of page ranges differs significantly for the short
format and the long format.  The short format will include a list of page
ranges in the definition section which consist of a page template number,
a start position, and an end position. In the long format, the start
and end position of a page
range is marked with a page range node switching the availability of a
page template on and off.  It is an error, to switch a page template
off that was not switched on, or to switch a page template on that was
already switched on.  It is permissible to omit switching off a page
template at the very end of the content section.

While we parse a long format \HINT/ file, we store page ranges and generate
the short format after reaching the end of the content section.
While we parse a short format \HINT/ file,
we check at the end of each top level node whether we should insert a
page range node into the output.
For the \.{shrink} program, it is best
to store the start and end positions of all page ranges
in an array sorted by the position\footnote*{For a \HINT/ viewer,
a data structure which allows fast retrieval of all
valid page templates for a given position is needed.}.
To check the restrictions on the switching of page templates, we
maintain for every page template an index into the range array
which identifies the position where the template was switched on.
A zero value instead of an index will identify templates that
are currently invalid. When switching a range off again, we
link the two array entries using this index. These links
are useful when producing the range nodes in short format.

A range node in short format contains the template number, the
start position and the end position.

A zero start position
is not stored, the info bit \\{b100} indicates a nonzero start position.
An end position equal to \T{\^FFFFFFFFF} is not stored,
the info bit \\{b010} indicates a smaller end position.
The info bit \\{b001} indicates that positions are stored using 2 byte
otherwise 4 byte are used for the positions.

\Y\B\4\X1:hint types\X${}\mathrel+\E{}$\6
\&{typedef} \&{struct} ${}\{{}$\5
\1\&{uint8\_t} \index{pg+\\{pg}}\\{pg};\5
\&{uint32\_t} \index{pos+\\{pos}}\\{pos};\5
\&{bool} \index{on+\\{on}}\\{on};\5
\&{int} \index{link+\\{link}}\\{link};\5
\2${}\}{}$ \index{range pos t+\&{range\_pos\_t}}\&{range\_pos\_t};
\Y
\fi

\M{252}

\Y\B\4\X252:common variables\X${}\E{}$\6
\index{range pos t+\&{range\_pos\_t}}\&{range\_pos\_t} ${}{*}\index{range pos+\\{range\_pos}}\\{range\_pos};{}$\6
\&{int} \index{next range+\\{next\_range}}\\{next\_range}${}\K\T{1},{}$ \index{max range+\\{max\_range}}\\{max\_range};\6
\&{int} ${}{*}\index{page on+\\{page\_on}}\\{page\_on}{}$;
\As268, 324, 364\ETs367.
\Us438, 439\ETs441.\Y
\fi

\M{253}

\Y\B\4\X253:initialize definitions\X${}\E{}$\6
$\index{ALLOCATE+\.{ALLOCATE}}\.{ALLOCATE}(\index{page on+\\{page\_on}}\\{page\_on},\39\\{max\_ref}[\index{page kind+\\{page\_kind}}\\{page\_kind}]+\T{1},\39\&{int});{}$\6
${}\index{ALLOCATE+\.{ALLOCATE}}\.{ALLOCATE}(\index{range pos+\\{range\_pos}}\\{range\_pos},\39\T{2}*(\\{max\_ref}[\index{range kind+\\{range\_kind}}\\{range\_kind}]+\T{1}),\39\index{range pos t+\&{range\_pos\_t}}\&{range\_pos\_t}){}$;
\As314\ET326.
\Us303\ET308.\Y
\fi

\M{254}

\Y\B\4\X11:hint macros\X${}\mathrel+\E{}$\6
\8\#\&{define} $\index{ALLOCATE+\.{ALLOCATE}}\.{ALLOCATE}(\|R,\39\|S,\39\|T){}$\6
( (\|R) $\K$  ( \|T $*$ ) $\index{calloc+\\{calloc}}\\{calloc}((\|S),\39{}$\&{sizeof} (\|T))${},\39(((\|R)\E\NULL)\?\.{QUIT}(\.{"Out\ of\ memory\ for\ "}\#\|R):\T{0})$ ) \6
\8\#\&{define} $\index{REALLOCATE+\.{REALLOCATE}}\.{REALLOCATE}(\|R,\39\|S,\39\|T){}$\6
( (\|R) $\K$  ( \|T $*$ ) $\index{realloc+\\{realloc}}\\{realloc}((\|R),\39(\|S)*{}$\&{sizeof} (\|T))${},\39(((\|R)\E\NULL)\?\.{QUIT}(\.{"Out\ of\ memory\ for\ "}\#\|R):\T{0})$ )
\Y
\fi

\M{255}

\readcode
\Y\par
\Y\B\4\X2:symbols\X${}\mathrel+\E{}$\6
\8\%\&{token} \index{RANGE+\ts{RANGE}}\ts{RANGE}\5\.{"range"}
\Y
\fi

\M{256}

\Y\B\4\X3:scanning rules\X${}\mathrel+\E{}$\6
${}\8\re{\vb{range}}{}$\ac\&{return} \index{RANGE+\ts{RANGE}}\ts{RANGE};\eac
\Y
\fi

\M{257}
\Y\B\4\X5:parsing rules\X${}\mathrel+\E{}$\6
\index{content node+\nts{content\_node}}\nts{content\_node}: \1\1\5
\index{START+\ts{START}}\ts{START}\5
\index{RANGE+\ts{RANGE}}\ts{RANGE}\5
\index{REFERENCE+\ts{REFERENCE}}\ts{REFERENCE}\5
\index{ON+\ts{ON}}\ts{ON}\5
\index{END+\ts{END}}\ts{END}\6
${}\{{}$\1\5
${}\index{REF+\.{REF}}\.{REF}(\index{page kind+\\{page\_kind}}\\{page\_kind},\39\.{\$3});{}$\5
${}\index{hput range+\\{hput\_range}}\\{hput\_range}(\.{\$3},\39\\{true});{}$\5
${}\}{}$\2\6
\4\hbox to 0.5em{\hss${}\OR{}$}\5
\index{START+\ts{START}}\ts{START}\5
\index{RANGE+\ts{RANGE}}\ts{RANGE}\5
\index{REFERENCE+\ts{REFERENCE}}\ts{REFERENCE}\5
\index{OFF+\ts{OFF}}\ts{OFF}\5
\index{END+\ts{END}}\ts{END}\6
${}\{{}$\1\5
${}\index{REF+\.{REF}}\.{REF}(\index{page kind+\\{page\_kind}}\\{page\_kind},\39\.{\$3});{}$\5
${}\index{hput range+\\{hput\_range}}\\{hput\_range}(\.{\$3},\39\\{false});{}$\5
${}\}{}$\2;\2\2
\Y
\fi

\M{258}


\writecode
\Y\B\4\X19:write functions\X${}\mathrel+\E{}$\6
\&{void} \index{hwrite range+\\{hwrite\_range}}\\{hwrite\_range}(\&{void})\C{ called in \\{hwrite\_end} }\1\6
\4${}\{{}$\5
\&{uint32\_t} \|p${}\K\index{hpos+\\{hpos}}\\{hpos}-\index{hstart+\\{hstart}}\\{hstart};{}$\7
${}\.{DBG}(\index{DBGRANGE+\.{DBGRANGE}}\.{DBGRANGE},\39\.{"Range\ check\ at\ pos\ }\)\.{0x\%x\ next\ at\ 0x\%x\\n"},\39\|p,\39\index{range pos+\\{range\_pos}}\\{range\_pos}[\index{next range+\\{next\_range}}\\{next\_range}].\index{pos+\\{pos}}\\{pos});{}$\6
\&{while} ${}(\index{next range+\\{next\_range}}\\{next\_range}<\index{max range+\\{max\_range}}\\{max\_range}\W\index{range pos+\\{range\_pos}}\\{range\_pos}[\index{next range+\\{next\_range}}\\{next\_range}].\index{pos+\\{pos}}\\{pos}\Z\|p){}$\5
\1${}\{{}$\5
\index{hwrite start+\\{hwrite\_start}}\\{hwrite\_start}(\,);\6
${}\index{hwritef+\\{hwritef}}\\{hwritef}(\.{"range\ *\%d\ "},\39\index{range pos+\\{range\_pos}}\\{range\_pos}[\index{next range+\\{next\_range}}\\{next\_range}].\index{pg+\\{pg}}\\{pg});{}$\6
\&{if} ${}(\index{range pos+\\{range\_pos}}\\{range\_pos}[\index{next range+\\{next\_range}}\\{next\_range}].\index{on+\\{on}}\\{on}){}$\1\5
\index{hwritef+\\{hwritef}}\\{hwritef}(\.{"on"});\2\6
\&{else}\1\5
\index{hwritef+\\{hwritef}}\\{hwritef}(\.{"off"});\2\6
${}\index{nesting+\\{nesting}}\\{nesting}\MM{}$;\5
\index{hwritec+\\{hwritec}}\\{hwritec}(\.{'>'});\C{ avoid a recursive call to \\{hwrite\_end} }\6
${}\index{next range+\\{next\_range}}\\{next\_range}\PP;{}$\6
\4${}\}{}$\2\6
\4${}\}{}$\2
\Y
\fi

\M{259}

\getcode
\Y\B\4\X16:get functions\X${}\mathrel+\E{}$\6
\&{void} \index{hget range+\\{hget\_range}}\\{hget\_range}(\index{info t+\&{info\_t}}\&{info\_t} \index{info+\\{info}}\\{info}${},\39{}$\&{uint8\_t} \index{pg+\\{pg}}\\{pg})\1\1\2\2\1\6
\4${}\{{}$\5
\&{uint32\_t} \index{from+\\{from}}\\{from}${},{}$ \index{to+\\{to}}\\{to};\7
${}\index{REF+\.{REF}}\.{REF}(\index{page kind+\\{page\_kind}}\\{page\_kind},\39\index{pg+\\{pg}}\\{pg});{}$\6
${}\index{REF+\.{REF}}\.{REF}(\index{range kind+\\{range\_kind}}\\{range\_kind},\39(\index{next range+\\{next\_range}}\\{next\_range}-\T{1})/\T{2});{}$\6
\&{if} ${}(\index{info+\\{info}}\\{info}\AND\\{b100}{}$)\5
\1${}\{{}$\5
\&{if} ${}(\index{info+\\{info}}\\{info}\AND\\{b001}){}$\1\5
\index{HGET32+\.{HGET32}}\.{HGET32}(\index{from+\\{from}}\\{from});\5
\2\&{else}\1\5
\index{HGET16+\.{HGET16}}\.{HGET16}(\index{from+\\{from}}\\{from});\5
\2${}\}{}$\2\6
\&{else}\1\5
${}\index{from+\\{from}}\\{from}\K\T{0};{}$\2\6
\&{if} ${}(\index{info+\\{info}}\\{info}\AND\\{b010}{}$)\5
\1${}\{{}$\5
\&{if} ${}(\index{info+\\{info}}\\{info}\AND\\{b001}){}$\1\5
\index{HGET32+\.{HGET32}}\.{HGET32}(\index{to+\\{to}}\\{to});\5
\2\&{else}\1\5
\index{HGET16+\.{HGET16}}\.{HGET16}(\index{to+\\{to}}\\{to});\5
\2${}\}{}$\2\6
\&{else}\1\5
${}\index{to+\\{to}}\\{to}\K\T{\^FFFFFFFF};{}$\2\6
${}\index{range pos+\\{range\_pos}}\\{range\_pos}[\index{next range+\\{next\_range}}\\{next\_range}].\index{pg+\\{pg}}\\{pg}\K\index{pg+\\{pg}}\\{pg};{}$\6
${}\index{range pos+\\{range\_pos}}\\{range\_pos}[\index{next range+\\{next\_range}}\\{next\_range}].\index{on+\\{on}}\\{on}\K\\{true};{}$\6
${}\index{range pos+\\{range\_pos}}\\{range\_pos}[\index{next range+\\{next\_range}}\\{next\_range}].\index{pos+\\{pos}}\\{pos}\K\index{from+\\{from}}\\{from};{}$\6
${}\.{DBG}(\index{DBGRANGE+\.{DBGRANGE}}\.{DBGRANGE},\39\.{"Range\ *\%d\ from\ 0x\%x}\)\.{\\n"},\39\index{pg+\\{pg}}\\{pg},\39\index{from+\\{from}}\\{from});{}$\6
${}\.{DBG}(\index{DBGRANGE+\.{DBGRANGE}}\.{DBGRANGE},\39\.{"Range\ *\%d\ to\ 0x\%x\\n}\)\.{"},\39\index{pg+\\{pg}}\\{pg},\39\index{to+\\{to}}\\{to});{}$\6
${}\index{next range+\\{next\_range}}\\{next\_range}\PP;{}$\6
\&{if} ${}(\index{to+\\{to}}\\{to}\I\T{\^FFFFFFFF}{}$)\6
\1${}\{{}$\5
${}\index{range pos+\\{range\_pos}}\\{range\_pos}[\index{next range+\\{next\_range}}\\{next\_range}].\index{pg+\\{pg}}\\{pg}\K\index{pg+\\{pg}}\\{pg};{}$\6
${}\index{range pos+\\{range\_pos}}\\{range\_pos}[\index{next range+\\{next\_range}}\\{next\_range}].\index{on+\\{on}}\\{on}\K\\{false};{}$\6
${}\index{range pos+\\{range\_pos}}\\{range\_pos}[\index{next range+\\{next\_range}}\\{next\_range}].\index{pos+\\{pos}}\\{pos}\K\index{to+\\{to}}\\{to};{}$\6
${}\index{next range+\\{next\_range}}\\{next\_range}\PP;{}$\6
\4${}\}{}$\2\6
\4${}\}{}$\2\7
\&{void} \index{hsort ranges+\\{hsort\_ranges}}\\{hsort\_ranges}(\&{void})\C{ simple insert sort by position }\1\6
\4${}\{{}$\5
\&{int} \|i;\7
${}\.{DBG}(\index{DBGRANGE+\.{DBGRANGE}}\.{DBGRANGE},\39\.{"Range\ sorting\ \%d\ po}\)\.{sitions\\n"},\39\index{next range+\\{next\_range}}\\{next\_range}-\T{1});{}$\6
\&{for} ${}(\|i\K\T{3};{}$ ${}\|i<\index{next range+\\{next\_range}}\\{next\_range};{}$ ${}\|i\PP{}$)\6
\1${}\{{}$\5
\&{int} \|j${}\K\|i-\T{1};{}$\7
\&{if} ${}(\index{range pos+\\{range\_pos}}\\{range\_pos}[\|i].\index{pos+\\{pos}}\\{pos}<\index{range pos+\\{range\_pos}}\\{range\_pos}[\|j].\index{pos+\\{pos}}\\{pos}{}$)\6
\1${}\{{}$\5
\index{range pos t+\&{range\_pos\_t}}\&{range\_pos\_t} \|t;\7
${}\|t\K\index{range pos+\\{range\_pos}}\\{range\_pos}[\|i];{}$\6
\&{do}\5
\1${}\{{}$\5
${}\index{range pos+\\{range\_pos}}\\{range\_pos}[\|j+\T{1}]\K\index{range pos+\\{range\_pos}}\\{range\_pos}[\|j];{}$\6
${}\|j\MM;{}$\6
\4${}\}{}$\2\5
\&{while} ${}(\index{range pos+\\{range\_pos}}\\{range\_pos}[\|i].\index{pos+\\{pos}}\\{pos}<\index{range pos+\\{range\_pos}}\\{range\_pos}[\|j].\index{pos+\\{pos}}\\{pos});{}$\6
${}\index{range pos+\\{range\_pos}}\\{range\_pos}[\|j+\T{1}]\K\|t;{}$\6
\4${}\}{}$\2\6
\4${}\}{}$\2\6
${}\index{max range+\\{max\_range}}\\{max\_range}\K\index{next range+\\{next\_range}}\\{next\_range}{}$;\5
${}\index{next range+\\{next\_range}}\\{next\_range}\K\T{1}{}$;\C{ prepare for \\{hwrite\_range} }\6
\4${}\}{}$\2
\Y
\fi

\M{260}

\putcode
\Y\B\4\X12:put functions\X${}\mathrel+\E{}$\6
\&{void} \index{hput range+\\{hput\_range}}\\{hput\_range}(\&{uint8\_t} \index{pg+\\{pg}}\\{pg}${},\39{}$\&{bool} \index{on+\\{on}}\\{on})\1\1\2\2\1\6
\4${}\{{}$\6
\&{if} ${}(((\index{next range+\\{next\_range}}\\{next\_range}-\T{1})/\T{2})>\\{max\_ref}[\index{range kind+\\{range\_kind}}\\{range\_kind}]){}$\1\5
${}\.{QUIT}(\.{"Page\ range\ \%d\ >\ \%d"},\39(\index{next range+\\{next\_range}}\\{next\_range}-\T{1})/\T{2},\39\\{max\_ref}[\index{range kind+\\{range\_kind}}\\{range\_kind}]);{}$\2\6
\&{if} ${}(\index{on+\\{on}}\\{on}\W\index{page on+\\{page\_on}}\\{page\_on}[\index{pg+\\{pg}}\\{pg}]\I\T{0}){}$\1\5
${}\.{QUIT}(\.{"Template\ \%d\ is\ swit}\)\.{ched\ on\ at\ 0x\%x\ and\ }\)\.{"}\.{SIZE\_F},\3{-1}\39\index{pg+\\{pg}}\\{pg},\39\index{range pos+\\{range\_pos}}\\{range\_pos}[\index{page on+\\{page\_on}}\\{page\_on}[\index{pg+\\{pg}}\\{pg}]].\index{pos+\\{pos}}\\{pos},\39\index{hpos+\\{hpos}}\\{hpos}-\index{hstart+\\{hstart}}\\{hstart});{}$\2\6
\&{else} \&{if} ${}(\R\index{on+\\{on}}\\{on}\W\index{page on+\\{page\_on}}\\{page\_on}[\index{pg+\\{pg}}\\{pg}]\E\T{0}){}$\1\5
${}\.{QUIT}(\.{"Template\ \%d\ is\ swit}\)\.{ched\ off\ at\ "}\.{SIZE\_F}\.{"\ but\ was\ not\ on"},\3{-1}\39\index{pg+\\{pg}}\\{pg},\39\index{hpos+\\{hpos}}\\{hpos}-\index{hstart+\\{hstart}}\\{hstart});{}$\2\6
${}\.{DBG}(\index{DBGRANGE+\.{DBGRANGE}}\.{DBGRANGE},\39\.{"Range\ *\%d\ \%s\ at\ "}\.{SIZE\_F}\.{"\\n"},\39\index{pg+\\{pg}}\\{pg},\39\index{on+\\{on}}\\{on}\?\.{"on"}:\.{"off"},\39\index{hpos+\\{hpos}}\\{hpos}-\index{hstart+\\{hstart}}\\{hstart});{}$\6
${}\index{range pos+\\{range\_pos}}\\{range\_pos}[\index{next range+\\{next\_range}}\\{next\_range}].\index{pg+\\{pg}}\\{pg}\K\index{pg+\\{pg}}\\{pg};{}$\6
${}\index{range pos+\\{range\_pos}}\\{range\_pos}[\index{next range+\\{next\_range}}\\{next\_range}].\index{pos+\\{pos}}\\{pos}\K\index{hpos+\\{hpos}}\\{hpos}-\index{hstart+\\{hstart}}\\{hstart};{}$\6
${}\index{range pos+\\{range\_pos}}\\{range\_pos}[\index{next range+\\{next\_range}}\\{next\_range}].\index{on+\\{on}}\\{on}\K\index{on+\\{on}}\\{on};{}$\6
\&{if} (\index{on+\\{on}}\\{on})\1\5
${}\index{page on+\\{page\_on}}\\{page\_on}[\index{pg+\\{pg}}\\{pg}]\K\index{next range+\\{next\_range}}\\{next\_range};{}$\2\6
\&{else}\6
\1${}\{{}$\5
${}\index{range pos+\\{range\_pos}}\\{range\_pos}[\index{next range+\\{next\_range}}\\{next\_range}].\index{link+\\{link}}\\{link}\K\index{page on+\\{page\_on}}\\{page\_on}[\index{pg+\\{pg}}\\{pg}];{}$\6
${}\index{range pos+\\{range\_pos}}\\{range\_pos}[\index{page on+\\{page\_on}}\\{page\_on}[\index{pg+\\{pg}}\\{pg}]].\index{link+\\{link}}\\{link}\K\index{next range+\\{next\_range}}\\{next\_range};{}$\6
${}\index{page on+\\{page\_on}}\\{page\_on}[\index{pg+\\{pg}}\\{pg}]\K\T{0};{}$\6
\4${}\}{}$\2\6
${}\index{next range+\\{next\_range}}\\{next\_range}\PP;{}$\6
\4${}\}{}$\2\7
\&{extern} \&{void} \index{hput definitions end+\\{hput\_definitions\_end}}\\{hput\_definitions\_end}(\&{void});\7
\&{void} \index{hput range defs+\\{hput\_range\_defs}}\\{hput\_range\_defs}(\&{void})\1\1\2\2\1\6
\4${}\{{}$\5
\&{int} \|i;\7
${}\index{section no+\\{section\_no}}\\{section\_no}\K\T{1};{}$\6
${}\index{hstart+\\{hstart}}\\{hstart}\K\index{dir+\\{dir}}\\{dir}[\T{1}].\index{buffer+\\{buffer}}\\{buffer};{}$\6
${}\index{hend+\\{hend}}\\{hend}\K\index{hstart+\\{hstart}}\\{hstart}+\index{dir+\\{dir}}\\{dir}[\T{1}].\index{bsize+\\{bsize}}\\{bsize};{}$\6
${}\index{hpos+\\{hpos}}\\{hpos}\K\index{hstart+\\{hstart}}\\{hstart}+\index{dir+\\{dir}}\\{dir}[\T{1}].\index{size+\\{size}}\\{size};{}$\6
\&{for} ${}(\|i\K\T{1};{}$ ${}\|i<\index{next range+\\{next\_range}}\\{next\_range};{}$ ${}\|i\PP){}$\1\6
\&{if} ${}(\index{range pos+\\{range\_pos}}\\{range\_pos}[\|i].\index{on+\\{on}}\\{on}{}$)\6
\1${}\{{}$\5
\index{info t+\&{info\_t}}\&{info\_t} \index{info+\\{info}}\\{info}${}\K\\{b000};{}$\6
\&{uint32\_t} \|p${}\K\index{hpos+\\{hpos}}\\{hpos}\PP-\index{hstart+\\{hstart}}\\{hstart};{}$\6
\&{uint32\_t} \index{from+\\{from}}\\{from}${},{}$ \index{to+\\{to}}\\{to};\7
${}\index{HPUT8+\.{HPUT8}}\.{HPUT8}(\index{range pos+\\{range\_pos}}\\{range\_pos}[\|i].\index{pg+\\{pg}}\\{pg});{}$\6
${}\index{from+\\{from}}\\{from}\K\index{range pos+\\{range\_pos}}\\{range\_pos}[\|i].\index{pos+\\{pos}}\\{pos};{}$\6
\&{if} ${}(\index{range pos+\\{range\_pos}}\\{range\_pos}[\|i].\index{link+\\{link}}\\{link}\I\T{0}){}$\1\5
${}\index{to+\\{to}}\\{to}\K\index{range pos+\\{range\_pos}}\\{range\_pos}[\index{range pos+\\{range\_pos}}\\{range\_pos}[\|i].\index{link+\\{link}}\\{link}].\index{pos+\\{pos}}\\{pos};{}$\2\6
\&{else}\1\5
${}\index{to+\\{to}}\\{to}\K\T{\^FFFFFFFF};{}$\2\6
\&{if} ${}(\index{from+\\{from}}\\{from}\I\T{0}{}$)\6
\1${}\{{}$\5
${}\index{info+\\{info}}\\{info}\K\index{info+\\{info}}\\{info}\OR\\{b100}{}$;\5
\&{if} ${}(\index{from+\\{from}}\\{from}>\T{\^FFFF}{}$)\5
\1${}\index{info+\\{info}}\\{info}\K\index{info+\\{info}}\\{info}\OR\\{b001}{}$;\5
\2${}\}{}$\2\6
\&{if} ${}(\index{to+\\{to}}\\{to}\I\T{\^FFFFFFFF}{}$)\6
\1${}\{{}$\5
${}\index{info+\\{info}}\\{info}\K\index{info+\\{info}}\\{info}\OR\\{b010}{}$;\5
\&{if} ${}(\index{to+\\{to}}\\{to}>\T{\^FFFF}){}$\1\5
${}\index{info+\\{info}}\\{info}\K\index{info+\\{info}}\\{info}\OR\\{b001}{}$;\5
\2${}\}{}$\2\6
\&{if} ${}(\index{info+\\{info}}\\{info}\AND\\{b100}{}$)\6
\1${}\{{}$\5
\&{if} ${}(\index{info+\\{info}}\\{info}\AND\\{b001}){}$\1\5
\index{HPUT32+\.{HPUT32}}\.{HPUT32}(\index{from+\\{from}}\\{from});\5
\2\&{else}\1\5
\index{HPUT16+\.{HPUT16}}\.{HPUT16}(\index{from+\\{from}}\\{from});\5
\2${}\}{}$\2\6
\&{if} ${}(\index{info+\\{info}}\\{info}\AND\\{b010}{}$)\6
\1${}\{{}$\5
\&{if} ${}(\index{info+\\{info}}\\{info}\AND\\{b001}){}$\1\5
\index{HPUT32+\.{HPUT32}}\.{HPUT32}(\index{to+\\{to}}\\{to});\5
\2\&{else}\1\5
\index{HPUT16+\.{HPUT16}}\.{HPUT16}(\index{to+\\{to}}\\{to});\5
\2${}\}{}$\2\6
${}\.{DBG}(\index{DBGRANGE+\.{DBGRANGE}}\.{DBGRANGE},\39\.{"Range\ *\%d\ from\ 0x\%x}\)\.{\ to\ 0x\%x\\n"},\3{-1}\39\index{range pos+\\{range\_pos}}\\{range\_pos}[\|i].\index{pg+\\{pg}}\\{pg},\39\index{from+\\{from}}\\{from},\39\index{to+\\{to}}\\{to});{}$\6
${}\index{hput tags+\\{hput\_tags}}\\{hput\_tags}(\|p,\39\.{TAG}(\index{range kind+\\{range\_kind}}\\{range\_kind},\39\index{info+\\{info}}\\{info}));{}$\6
\4${}\}{}$\2\2\6
\index{hput definitions end+\\{hput\_definitions\_end}}\\{hput\_definitions\_end}(\,);\6
\4${}\}{}$\2
\Y
\fi

\M{261}


\section{File Structure}\hascode
All \HINT/ files\index{file} start with a banner\index{banner} as
described below.  After that, they contain three mandatory
sections\index{section}: the directory\index{directory section}
section, the definition\index{definition section} section, and the
content\index{content section} section.  Usually, further
optional\index{optional section} sections follow.  In short format
files, these contain auxiliary\index{auxiliary file} files
(fonts\index{font}, images\index{image}, \dots) necessary for
rendering the content. In long format files, the directory section
will simply list the file names of the auxiliary files.



\subsection{Banner}
All \HINT/ files start with a banner\index{banner}. The banner contains only
printable ASCII characters and spaces;
its end is marked with a newline character\index{newline character}.
The first four byte are the ``magic'' number by which you recognize a \HINT/
file. It consists of the four ASCII codes `{\tt H}', `{\tt I}', `{\tt N}',
and `{\tt T}' in the long format and `{\tt h}', `{\tt i}', `{\tt n}',
and `{\tt t}' in the short format.  Then follows a space, then
the version number, a dot, the sub-version number, and another
space. Both numbers are encoded as decimal ASCII strings.  The
remainder of the banner is simply ignored but may be used to contain
other useful information about the file.  The maximum size of the
banner is 256 byte.
\Y\B\4\X11:hint macros\X${}\mathrel+\E{}$\6
\8\#\&{define} \index{MAX BANNER+\.{MAX\_BANNER}}\.{MAX\_BANNER}\5\T{256}
\Y
\fi

\M{262}

\goodbreak
To check the banner, we have the function \index{hcheck banner+\\{hcheck\_banner}}\\{hcheck\_banner};
it returns \\{true} if successful.


\Y\B\4\X262:function to check the banner\X${}\E{}$\6
\&{int} \index{version+\\{version}}\\{version}${}\K\T{1},{}$ \index{subversion+\\{subversion}}\\{subversion}${}\K\T{0};{}$\6
\&{char} ${}\index{hbanner+\\{hbanner}}\\{hbanner}[\index{MAX BANNER+\.{MAX\_BANNER}}\.{MAX\_BANNER}+\T{1}];{}$\7
\&{bool} \index{hcheck banner+\\{hcheck\_banner}}\\{hcheck\_banner}(\&{char} ${}{*}\index{magic+\\{magic}}\\{magic}){}$\1\1\2\2\1\6
\4${}\{{}$\5
\&{int} \index{hbanner size+\\{hbanner\_size}}\\{hbanner\_size}${}\K\T{0};{}$\6
\&{char} ${}{*}\|t;{}$\7
${}\|t\K\index{hbanner+\\{hbanner}}\\{hbanner};{}$\6
\&{if} ${}(\index{strncmp+\\{strncmp}}\\{strncmp}(\index{magic+\\{magic}}\\{magic},\39\index{hbanner+\\{hbanner}}\\{hbanner},\39\T{4})\I\T{0}){}$\1\5
${}\.{QUIT}(\.{"This\ is\ not\ a\ \%s\ fi}\)\.{le"},\39\index{magic+\\{magic}}\\{magic});{}$\2\6
\&{else}\1\5
${}\|t\MRL{+{\K}}\T{4};{}$\2\6
${}\index{hbanner size+\\{hbanner\_size}}\\{hbanner\_size}\K{}$(\&{int}) \index{strnlen+\\{strnlen}}\\{strnlen}${}(\index{hbanner+\\{hbanner}}\\{hbanner},\39\index{MAX BANNER+\.{MAX\_BANNER}}\.{MAX\_BANNER});{}$\6
\&{if} ${}(\index{hbanner+\\{hbanner}}\\{hbanner}[\index{hbanner size+\\{hbanner\_size}}\\{hbanner\_size}-\T{1}]\I\.{'\\n'}){}$\1\5
${}\.{QUIT}(\.{"Banner\ exceeds\ maxi}\)\.{mum\ size=0x\%x"},\39\index{MAX BANNER+\.{MAX\_BANNER}}\.{MAX\_BANNER});{}$\2\6
\&{if} ${}({*}\|t\I\.{'\ '}){}$\1\5
${}\.{QUIT}(\.{"Space\ expected\ afte}\)\.{r\ \%s"},\39\index{magic+\\{magic}}\\{magic});{}$\2\6
\&{else}\1\5
${}\|t\PP;{}$\2\6
${}\index{version+\\{version}}\\{version}\K\index{strtol+\\{strtol}}\\{strtol}(\|t,\39{\AND}\|t,\39\T{10});{}$\6
\&{if} ${}({*}\|t\I\.{'.'}){}$\1\5
${}\.{QUIT}(\.{"Dot\ expected\ after\ }\)\.{version\ number\ \%d"},\39\index{version+\\{version}}\\{version});{}$\2\6
\&{else}\1\5
${}\|t\PP;{}$\2\6
${}\index{subversion+\\{subversion}}\\{subversion}\K\index{strtol+\\{strtol}}\\{strtol}(\|t,\39{\AND}\|t,\39\T{10});{}$\6
\&{if} ${}({*}\|t\I\.{'\ '}\W{*}\|t\I\.{'\\n'}){}$\1\5
${}\.{QUIT}(\.{"Space\ expected\ afte}\)\.{r\ subversion\ number\ }\)\.{\%d"},\39\index{subversion+\\{subversion}}\\{subversion});{}$\2\6
${}\index{LOG+\.{LOG}}\.{LOG}(\.{"\%s\ file\ version\ \%d.}\)\.{\%d:\%s"},\39\index{magic+\\{magic}}\\{magic},\39\index{version+\\{version}}\\{version},\39\index{subversion+\\{subversion}}\\{subversion},\39\|t);{}$\6
${}\.{DBG}(\index{DBGDIR+\.{DBGDIR}}\.{DBGDIR},\39\.{"banner\ size=0x\%x\\n"},\39\index{hbanner size+\\{hbanner\_size}}\\{hbanner\_size});{}$\6
\&{return} \\{true};\6
\4${}\}{}$\2
\Us433, 438, 439\ETs441.\Y
\fi

\M{263}

To read a short format file, we use the macro \index{HGET8+\.{HGET8}}\.{HGET8}. It returns a single byte.
We read the banner knowing that it ends with a newline character
and is at most \index{MAX BANNER+\.{MAX\_BANNER}}\.{MAX\_BANNER} byte long.

\getcode
\Y\B\4\X263:get file functions\X${}\E{}$\6
\&{void} \index{hget banner+\\{hget\_banner}}\\{hget\_banner}(\&{void})\1\1\2\2\1\6
\4${}\{{}$\5
\&{int} \|i;\7
\&{for} ${}(\|i\K\T{0};{}$ ${}\|i<\index{MAX BANNER+\.{MAX\_BANNER}}\.{MAX\_BANNER};{}$ ${}\|i\PP){}$\5
\1${}\{{}$\5
${}\index{hbanner+\\{hbanner}}\\{hbanner}[\|i]\K\index{HGET8+\.{HGET8}}\.{HGET8};{}$\6
\&{if} ${}(\index{hbanner+\\{hbanner}}\\{hbanner}[\|i]\E\.{'\\n'}){}$\1\5
\&{break};\2\6
\4${}\}{}$\2\6
${}\index{hbanner+\\{hbanner}}\\{hbanner}[\PP\|i]\K\T{0};{}$\6
\4${}\}{}$\2
\As276, 294, 295\ETs311.
\Us433, 439\ETs441.\Y
\fi

\M{264}

To read a long format file, we use the function \index{fgetc+\\{fgetc}}\\{fgetc}.
\readcode
\Y\B\4\X264:read the banner\X${}\E{}$\1\6
\4${}\{{}$\5
\&{int} \|i${}\K\T{0},{}$ \|c;\7
\&{do}\5
\1${}\{{}$\5
${}\|c\K\index{fgetc+\\{fgetc}}\\{fgetc}(\index{hin+\\{hin}}\\{hin});{}$\6
\&{if} ${}(\|c\I\index{EOF+\.{EOF}}\.{EOF}){}$\1\5
${}\index{hbanner+\\{hbanner}}\\{hbanner}[\|i\PP]\K{}$(\&{char}) \|c;\2\6
\&{else}\1\5
\&{break};\2\6
\4${}\}{}$\2\5
\&{while} ${}(\|c\I\.{'\\n'}\W\|i<\index{MAX BANNER+\.{MAX\_BANNER}}\.{MAX\_BANNER});{}$\6
${}\index{hbanner+\\{hbanner}}\\{hbanner}[\|i]\K\T{0};{}$\6
\4${}\}{}$\2
\U438.\Y
\fi

\M{265}

Writing the banner to a short format file is accomplished by calling
\index{hput banner+\\{hput\_banner}}\\{hput\_banner} with the ``magic'' string \.{"hint"} as a first argument
and a (short) comment as the second argument.
\putcode
\Y\B\4\X265:function to write the banner\X${}\E{}$\6
\&{extern} \&{int} \index{version+\\{version}}\\{version}${},{}$ \index{subversion+\\{subversion}}\\{subversion};\7
\&{static} \&{size\_t} \index{hput banner+\\{hput\_banner}}\\{hput\_banner}(\&{char} ${}{*}\index{magic+\\{magic}}\\{magic},\39{}$\&{char} ${}{*}\|s){}$\1\1\2\2\1\6
\4${}\{{}$\5
\&{return} \index{fprintf+\\{fprintf}}\\{fprintf}${}(\index{hout+\\{hout}}\\{hout},\39\.{"\%s\ \%d.\%d\ \%s\\n"},\39\index{magic+\\{magic}}\\{magic},\39\index{version+\\{version}}\\{version},\39\index{subversion+\\{subversion}}\\{subversion},\39\|s);{}$\6
\4${}\}{}$\2
\Us435, 438\ETs439.\Y
\fi

\M{266}


\writecode
Writing the banner of a long format file is essentialy the same as for short
format file calling \index{hput banner+\\{hput\_banner}}\\{hput\_banner} with \.{"HINT"} as a first argument.

\subsection{Long Format Files}\gdef\subcodetitle{Banner}%

After reading and checking the banner, reading a long format file is
simply done by calling \index{yyparse+\\{yyparse}}\\{yyparse}. The following rule gives the big picture:
\readcode
\Y\par
\par
\Y\B\4\X5:parsing rules\X${}\mathrel+\E{}$\6
\index{hint+\nts{hint}}\nts{hint}: \1\1\5
\index{directory section+\nts{directory\_section}}\nts{directory\_section}\5
\index{definition section+\nts{definition\_section}}\nts{definition\_section}\5
\index{content section+\nts{content\_section}}\nts{content\_section};\2\2
\Y
\fi

\M{267}


\subsection{Short Format Files}\gdef\subcodetitle{Primitives}%
A short format\index{short format} file starts with the banner and continues
with a list of sections. Each section has a maximum size
of $2^{32}$ byte or 4GByte. This restriction ensures that positions\index{position}
inside a section can be stored as 32 bit integers, a feature that
we will need only for the so called ``content'' section, but it
is also nice for implementers to know in advance what sizes to expect.
The big picture is captured by the \index{put hint+\\{put\_hint}}\\{put\_hint} function:

\Y\B\4\X12:put functions\X${}\mathrel+\E{}$\6
\&{static} \&{size\_t} \index{hput root+\\{hput\_root}}\\{hput\_root}(\&{void});\6
\&{static} \&{size\_t} \index{hput section+\\{hput\_section}}\\{hput\_section}(\&{uint16\_t} \|n);\6
\&{static} \&{void} \index{hput optional sections+\\{hput\_optional\_sections}}\\{hput\_optional\_sections}(\&{void});\7
\&{void} \index{hput hint+\\{hput\_hint}}\\{hput\_hint}(\&{char} ${}{*}\index{str+\\{str}}\\{str}){}$\1\1\2\2\1\6
\4${}\{{}$\5
\&{size\_t} \|s;\7
${}\.{DBG}(\index{DBGBASIC+\.{DBGBASIC}}\.{DBGBASIC},\39\.{"Writing\ hint\ output}\)\.{\ \%s\\n"},\39\index{str+\\{str}}\\{str});{}$\6
${}\|s\K\index{hput banner+\\{hput\_banner}}\\{hput\_banner}(\.{"hint"},\39\index{str+\\{str}}\\{str});{}$\6
${}\.{DBG}(\index{DBGDIR+\.{DBGDIR}}\.{DBGDIR},\39\.{"Root\ Entry\ at\ "}\.{SIZE\_F}\.{"\\n"},\39\|s);{}$\6
${}\|s\MRL{+{\K}}\index{hput root+\\{hput\_root}}\\{hput\_root}(\,);{}$\6
${}\.{DBG}(\index{DBGDIR+\.{DBGDIR}}\.{DBGDIR},\39\.{"Directory\ section\ a}\)\.{t\ "}\.{SIZE\_F}\.{"\\n"},\39\|s);{}$\6
${}\|s\MRL{+{\K}}\index{hput section+\\{hput\_section}}\\{hput\_section}(\T{0});{}$\6
${}\.{DBG}(\index{DBGDIR+\.{DBGDIR}}\.{DBGDIR},\39\.{"Definition\ section\ }\)\.{at\ "}\.{SIZE\_F}\.{"\\n"},\39\|s);{}$\6
${}\|s\MRL{+{\K}}\index{hput section+\\{hput\_section}}\\{hput\_section}(\T{1});{}$\6
${}\.{DBG}(\index{DBGDIR+\.{DBGDIR}}\.{DBGDIR},\39\.{"Content\ section\ at\ }\)\.{"}\.{SIZE\_F}\.{"\\n"},\39\|s);{}$\6
${}\|s\MRL{+{\K}}\index{hput section+\\{hput\_section}}\\{hput\_section}(\T{2});{}$\6
${}\.{DBG}(\index{DBGDIR+\.{DBGDIR}}\.{DBGDIR},\39\.{"Auxiliary\ sections\ }\)\.{at\ "}\.{SIZE\_F}\.{"\\n"},\39\|s);{}$\6
\index{hput optional sections+\\{hput\_optional\_sections}}\\{hput\_optional\_sections}(\,);\6
\4${}\}{}$\2
\Y
\fi

\M{268}


When we work on a section, we will have the entire section in
memory and use three variables to access it:  \index{hstart+\\{hstart}}\\{hstart}
points to the first byte of the section, \index{hend+\\{hend}}\\{hend} points
to the byte after the last byte of the section, and \index{hpos+\\{hpos}}\\{hpos} points to the
current position inside the section.\label{hpos}

\Y\B\4\X252:common variables\X${}\mathrel+\E{}$\6
\&{uint8\_t} ${}{*}\index{hpos+\\{hpos}}\\{hpos}\K\NULL,{}$ ${}{*}\index{hstart+\\{hstart}}\\{hstart}\K\NULL,{}$ ${}{*}\index{hend+\\{hend}}\\{hend}\K\NULL{}$;
\Y
\fi

\M{269}

There are two sets of macros that read or write binary data at the current position
and advance the stream position accordingly.\label{HPUT}\label{HGET}

\getcode
\Y\B\4\X35:get file macros\X${}\mathrel+\E{}$\6
\8\#\&{define} \index{HGET ERROR+\.{HGET\_ERROR}}\.{HGET\_ERROR} \.{QUIT}\5${}(\.{"HGET\ overrun\ in\ sec}\)\.{tion\ \%d\ at\ "}\.{SIZE\_F}\.{"\\n"},\39\index{section no+\\{section\_no}}\\{section\_no},\39\index{hpos+\\{hpos}}\\{hpos}-\index{hstart+\\{hstart}}\\{hstart}){}$\6
\8\#\&{define} \index{HEND+\.{HEND}}\.{HEND}\5${}((\index{hpos+\\{hpos}}\\{hpos}\Z\index{hend+\\{hend}}\\{hend})\?\T{0}:(\index{HGET ERROR+\.{HGET\_ERROR}}\.{HGET\_ERROR},\39\T{0})){}$\6
\8\#\&{define} \index{HGET8+\.{HGET8}}\.{HGET8}\5${}((\index{hpos+\\{hpos}}\\{hpos}<\index{hend+\\{hend}}\\{hend})\?{*}(\index{hpos+\\{hpos}}\\{hpos}\PP):(\index{HGET ERROR+\.{HGET\_ERROR}}\.{HGET\_ERROR},\39\T{0})){}$\6
\8\#\&{define} \index{HGET16+\.{HGET16}}\.{HGET16}(\|X)\5${}((\|X)\K(\index{hpos+\\{hpos}}\\{hpos}[\T{0}]\LL\T{8})+\index{hpos+\\{hpos}}\\{hpos}[\T{1}],\39\index{hpos+\\{hpos}}\\{hpos}\MRL{+{\K}}\T{2},\39\index{HEND+\.{HEND}}\.{HEND}){}$\6
\8\#\&{define} \index{HGET24+\.{HGET24}}\.{HGET24}(\|X)\5${}((\|X)\K(\index{hpos+\\{hpos}}\\{hpos}[\T{0}]\LL\T{16})+(\index{hpos+\\{hpos}}\\{hpos}[\T{1}]\LL\T{8})+\index{hpos+\\{hpos}}\\{hpos}[\T{2}],\39\index{hpos+\\{hpos}}\\{hpos}\MRL{+{\K}}\T{3},\39\index{HEND+\.{HEND}}\.{HEND}){}$\6
\8\#\&{define} \index{HGET32+\.{HGET32}}\.{HGET32}(\|X)\5${}((\|X)\K(\index{hpos+\\{hpos}}\\{hpos}[\T{0}]\LL\T{24})+(\index{hpos+\\{hpos}}\\{hpos}[\T{1}]\LL\T{16})+(\index{hpos+\\{hpos}}\\{hpos}[\T{2}]\LL\T{8})+\index{hpos+\\{hpos}}\\{hpos}[\T{3}],\39\index{hpos+\\{hpos}}\\{hpos}\MRL{+{\K}}\T{4},\39\index{HEND+\.{HEND}}\.{HEND}){}$\6
\8\#\&{define} \index{HGETTAG+\.{HGETTAG}}\.{HGETTAG}(\|A)\5${}\|A\K\index{HGET8+\.{HGET8}}\.{HGET8},\39\.{DBGTAG}(\|A,\39\index{hpos+\\{hpos}}\\{hpos}-\T{1}){}$
\Y
\fi

\M{270}

\putcode
\Y\B\4\X12:put functions\X${}\mathrel+\E{}$\6
\&{void} \index{hput error+\\{hput\_error}}\\{hput\_error}(\&{void})\1\1${}\{{}$\6
\&{if} ${}(\index{hpos+\\{hpos}}\\{hpos}<\index{hend+\\{hend}}\\{hend}){}$\1\5
\&{return};\2\6
\.{QUIT}(\.{"HPUT\ overrun\ sectio}\)\.{n\ \%d\ pos="}\.{SIZE\_F}\.{"\\n"} $,$ $\index{section no+\\{section\_no}}\\{section\_no},\39\index{hpos+\\{hpos}}\\{hpos}-\index{hstart+\\{hstart}}\\{hstart}$ )  ;\7
${}\}{}$
\Y
\fi

\M{271}

\Y\B\4\X271:put macros\X${}\E{}$\6
\&{extern} \&{void} \index{hput error+\\{hput\_error}}\\{hput\_error}(\&{void});\6
\8\#\&{define} \index{HPUT8+\.{HPUT8}}\.{HPUT8}(\|X)\5${}(\index{hput error+\\{hput\_error}}\\{hput\_error}(\,),\39{*}(\index{hpos+\\{hpos}}\\{hpos}\PP)\K(\|X)){}$\6
\8\#\&{define} \index{HPUT16+\.{HPUT16}}\.{HPUT16}(\|X)\5${}(\index{HPUT8+\.{HPUT8}}\.{HPUT8}(((\|X)\GG\T{8})\AND\T{\^FF}),\39\index{HPUT8+\.{HPUT8}}\.{HPUT8}((\|X)\AND\T{\^FF})){}$\6
\8\#\&{define} \index{HPUT24+\.{HPUT24}}\.{HPUT24}(\|X)\5${}(\index{HPUT8+\.{HPUT8}}\.{HPUT8}(((\|X)\GG\T{16})\AND\T{\^FF}),\39\index{HPUT8+\.{HPUT8}}\.{HPUT8}(((\|X)\GG\T{8})\AND\T{\^FF}),\39\index{HPUT8+\.{HPUT8}}\.{HPUT8}((\|X)\AND\T{\^FF})){}$\6
\8\#\&{define} \index{HPUT32+\.{HPUT32}}\.{HPUT32}(\|X)\5${}(\index{HPUT8+\.{HPUT8}}\.{HPUT8}(((\|X)\GG\T{24})\AND\T{\^FF}),\39\index{HPUT8+\.{HPUT8}}\.{HPUT8}(((\|X)\GG\T{16})\AND\T{\^FF}),\39\index{HPUT8+\.{HPUT8}}\.{HPUT8}(((\|X)\GG\T{8})\AND\T{\^FF}),\39\index{HPUT8+\.{HPUT8}}\.{HPUT8}((\|X)\AND\T{\^FF})){}$
\A272.
\Us434\ET438.\Y
\fi

\M{272}

The above macros test for buffer overruns\index{buffer overrun};
allocating sufficient buffer space is done separately.

Before writing a node, we will insert a test and increase the buffer if necessary.
\Y\B\4\X271:put macros\X${}\mathrel+\E{}$\6
\&{void} \index{hput increase buffer+\\{hput\_increase\_buffer}}\\{hput\_increase\_buffer}(\&{uint32\_t} \|n);\6
\8\#\&{define} \index{HPUTX+\.{HPUTX}}\.{HPUTX}(\|N)\5${}(((\index{hend+\\{hend}}\\{hend}-\index{hpos+\\{hpos}}\\{hpos})<(\|N))\?\index{hput increase buffer+\\{hput\_increase\_buffer}}\\{hput\_increase\_buffer}(\|N):{}$(\&{void}) \T{0})\6
\8\#\&{define} \index{HPUTNODE+\.{HPUTNODE}}\.{HPUTNODE}\5\index{HPUTX+\.{HPUTX}}\.{HPUTX}(\index{MAX TAG DISTANCE+\.{MAX\_TAG\_DISTANCE}}\.{MAX\_TAG\_DISTANCE})\6
\8\#\&{define} ${}\index{HPUTTAG+\.{HPUTTAG}}\.{HPUTTAG}(\|K,\39\|I)\5(\index{HPUTNODE+\.{HPUTNODE}}\.{HPUTNODE},{}$\5
${}\.{DBGTAG}(\.{TAG}(\|K,\39\|I),\39\index{hpos+\\{hpos}}\\{hpos}),{}$\5
${}\index{HPUT8+\.{HPUT8}}\.{HPUT8}(\.{TAG}(\|K,\39\|I))){}$
\Y
\fi

\M{273}

Fortunately the only data types that have an unbounded size are strings\index{string} and texts\index{text}.
For these we insert specific tests. For all other cases a relatively
small upper bound on the maximum distance between two tags can be determined.

% My guess from grep "subq" skip.s \index{grep+\\{grep}}\\{grep}\index{hpos+\\{hpos}}\\{hpos} sort -n is 25
\Y\B\4\X11:hint macros\X${}\mathrel+\E{}$\6
\8\#\&{define} \index{MAX TAG DISTANCE+\.{MAX\_TAG\_DISTANCE}}\.{MAX\_TAG\_DISTANCE}\5\T{32}\C{ This is a guess; I need a tight upper bound. }
\Y
\fi

\M{274}

\subsection{Mapping a Short Format File}
Since modern computers with 64bit hardware have a huge address space,
mapping the entire file into virtual memory is the most efficient way
to read a large file.  ``Mapping'' is not the same as ``reading'' and it is
not the same as allocating precious memory, all that is done by the
operating system when needed. Mapping just reserves adresses.

The following functions map and unmap a short format input
file at address \index{hbase+\\{hbase}}\\{hbase}.\label{map}


\Y\B\4\X274:map functions\X${}\E{}$\6
\X275:\\{mmap} and \\{munmap} declarations\X\7
\&{static} \&{size\_t} \index{hbase size+\\{hbase\_size}}\\{hbase\_size};\6
\&{uint8\_t} ${}{*}\index{hbase+\\{hbase}}\\{hbase}\K\NULL;{}$\6
\&{extern} \&{char} ${}{*}\index{in name+\\{in\_name}}\\{in\_name};{}$\7
\&{void} \index{hget map+\\{hget\_map}}\\{hget\_map}(\&{void})\1\1\2\2\1\6
\4${}\{{}$\5
\&{struct} \index{stat+\\{stat}}\\{stat} \index{st+\\{st}}\\{st};\6
\&{int} \index{fd+\\{fd}}\\{fd};\7
${}\index{fd+\\{fd}}\\{fd}\K\index{open+\\{open}}\\{open}(\index{in name+\\{in\_name}}\\{in\_name},\39\index{O RDONLY+\.{O\_RDONLY}}\.{O\_RDONLY},\39\T{0});{}$\6
\&{if} ${}(\index{fd+\\{fd}}\\{fd}<\T{0}){}$\1\5
${}\.{QUIT}(\.{"Unable\ to\ open\ file}\)\.{\ \%s"},\39\index{in name+\\{in\_name}}\\{in\_name});{}$\2\6
\&{if} ${}(\index{fstat+\\{fstat}}\\{fstat}(\index{fd+\\{fd}}\\{fd},\39{\AND}\index{st+\\{st}}\\{st})<\T{0}){}$\1\5
\.{QUIT}(\.{"Unable\ to\ get\ file\ }\)\.{size"});\2\6
${}\index{hbase size+\\{hbase\_size}}\\{hbase\_size}\K\index{st+\\{st}}\\{st}.\index{st size+\\{st\_size}}\\{st\_size};{}$\6
${}\index{hbase+\\{hbase}}\\{hbase}\K\index{mmap+\\{mmap}}\\{mmap}(\NULL,\39\index{hbase size+\\{hbase\_size}}\\{hbase\_size},\39\index{PROT READ+\.{PROT\_READ}}\.{PROT\_READ},\39\index{MAP PRIVATE+\.{MAP\_PRIVATE}}\.{MAP\_PRIVATE},\39\index{fd+\\{fd}}\\{fd},\39\T{0});{}$\6
\&{if} ${}(\index{hbase+\\{hbase}}\\{hbase}\E\index{MAP FAILED+\.{MAP\_FAILED}}\.{MAP\_FAILED}){}$\5
\1${}\{{}$\5
${}\index{hbase+\\{hbase}}\\{hbase}\K\NULL;{}$\6
${}\index{hbase size+\\{hbase\_size}}\\{hbase\_size}\K\T{0};{}$\6
\.{QUIT}(\.{"Unable\ to\ map\ file\ }\)\.{into\ memory"});\6
\4${}\}{}$\2\6
\index{close+\\{close}}\\{close}(\index{fd+\\{fd}}\\{fd});\6
${}\index{hpos+\\{hpos}}\\{hpos}\K\index{hstart+\\{hstart}}\\{hstart}\K\index{hbase+\\{hbase}}\\{hbase};{}$\6
${}\index{hend+\\{hend}}\\{hend}\K\index{hstart+\\{hstart}}\\{hstart}+\index{hbase size+\\{hbase\_size}}\\{hbase\_size};{}$\6
\4${}\}{}$\2\7
\&{void} \index{hget unmap+\\{hget\_unmap}}\\{hget\_unmap}(\&{void})\1\1\2\2\1\6
\4${}\{{}$\5
${}\index{munmap+\\{munmap}}\\{munmap}(\index{hbase+\\{hbase}}\\{hbase},\39\index{hbase size+\\{hbase\_size}}\\{hbase\_size});{}$\6
${}\index{hbase+\\{hbase}}\\{hbase}\K\NULL;{}$\6
${}\index{hbase size+\\{hbase\_size}}\\{hbase\_size}\K\T{0};{}$\6
${}\index{hpos+\\{hpos}}\\{hpos}\K\index{hstart+\\{hstart}}\\{hstart}\K\index{hend+\\{hend}}\\{hend}\K\NULL;{}$\6
\4${}\}{}$\2
\Us433, 439\ETs441.\Y
\fi

\M{275}
A small complication arrises from the fact that the \index{mmap+\\{mmap}}\\{mmap} and
\index{munmap+\\{munmap}}\\{munmap} functions and the associated header files are not available
under the Windows operating system and not even under MinGW.

So we need to implement our own version of these functions.  We do not
implement general purpose replacements but only a replacement for the
calls with the parameters used above.  We start with the function
\\{\_get\_osfhandle} to obtain a Windows \index{HANDLE+\.{HANDLE}}\.{HANDLE} for the given file
descriptor, then use \index{GetFileSize+\\{GetFileSize}}\\{GetFileSize}, \index{CreateFileMapping+\\{CreateFileMapping}}\\{CreateFileMapping}, and finally
\index{MapViewOfFile+\\{MapViewOfFile}}\\{MapViewOfFile}.  The file is closed with \index{CloseHandle+\\{CloseHandle}}\\{CloseHandle}.


\Y\B\4\X275:\\{mmap} and \\{munmap} declarations\X${}\E{}$\6
\8\#\&{ifdef} \index{WIN32+\.{WIN32}}\.{WIN32}\6
\8\#\&{include} \.{<windows.h>}\6
\8\#\&{include} \.{<io.h>}\6
\8\#\&{define} \index{PROT READ+\.{PROT\_READ}}\.{PROT\_READ}\5\T{\^1}\6
\8\#\&{define} \index{MAP PRIVATE+\.{MAP\_PRIVATE}}\.{MAP\_PRIVATE}\5\T{\^02}\6
\8\#\&{define} \index{MAP FAILED+\.{MAP\_FAILED}}\.{MAP\_FAILED}\5((\&{void} ${}{*}){}$ ${}{-}\T{1}){}$\6
\&{static} \index{HANDLE+\.{HANDLE}}\.{HANDLE}\index{hMap+\\{hMap}}\\{hMap};\7
\&{void} ${}{*}{}$\index{mmap+\\{mmap}}\\{mmap}(\&{void} ${}{*}\\{addr},\39{}$\&{size\_t} \index{length+\\{length}}\\{length}${},\39{}$\&{int} \index{prot+\\{prot}}\\{prot}${},\39{}$\&{int} \index{flags+\\{flags}}\\{flags}${},\39{}$\&{int} \index{fd+\\{fd}}\\{fd}${},\39\index{off t+\\{off\_t}}\\{off\_t}\index{offset+\\{offset}}\\{offset}){}$\1\1\2\2\1\6
\4${}\{{}$\5
${}\index{HANDLE+\.{HANDLE}}\.{HANDLE}\index{hFile+\\{hFile}}\\{hFile}\K(\index{HANDLE+\.{HANDLE}}\.{HANDLE})\\{\_get\_osfhandle}(\index{fd+\\{fd}}\\{fd});{}$\6
\&{if} ${}(\index{hFile+\\{hFile}}\\{hFile}\E\index{INVALID HANDLE VALUE+\.{INVALID\_HANDLE\_VALUE}}\.{INVALID\_HANDLE\_VALUE}){}$\1\5
\.{QUIT}(\.{"Unable\ to\ get\ file\ }\)\.{handle"});\2\6
${}\index{hMap+\\{hMap}}\\{hMap}\K\index{CreateFileMapping+\\{CreateFileMapping}}\\{CreateFileMapping}(\index{hFile+\\{hFile}}\\{hFile},\39\NULL,\39\index{PAGE READONLY+\.{PAGE\_READONLY}}\.{PAGE\_READONLY},\39\T{0},\39\T{0},\39\NULL);{}$\6
\&{if} ${}(\index{hMap+\\{hMap}}\\{hMap}\E\NULL){}$\1\5
\.{QUIT}(\.{"Unable\ to\ map\ file\ }\)\.{into\ memory"});\2\6
${}\\{addr}\K\index{MapViewOfFile+\\{MapViewOfFile}}\\{MapViewOfFile}(\index{hMap+\\{hMap}}\\{hMap},\39\index{FILE MAP READ+\.{FILE\_MAP\_READ}}\.{FILE\_MAP\_READ},\39\T{0},\39\T{0},\39\T{0});{}$\6
\&{if} ${}(\\{addr}\E\NULL){}$\1\5
\.{QUIT}(\.{"Unable\ to\ obtain\ ad}\)\.{dress\ of\ file\ mappin}\)\.{g"});\2\6
\index{CloseHandle+\\{CloseHandle}}\\{CloseHandle}(\index{hFile+\\{hFile}}\\{hFile});\6
\&{return} \\{addr};\6
\4${}\}{}$\2\7
\&{int} \index{munmap+\\{munmap}}\\{munmap}(\&{void} ${}{*}\\{addr},\39{}$\&{size\_t} \index{length+\\{length}}\\{length})\1\1\2\2\1\6
\4${}\{{}$\5
\index{UnmapViewOfFile+\\{UnmapViewOfFile}}\\{UnmapViewOfFile}(\\{addr});\6
\index{CloseHandle+\\{CloseHandle}}\\{CloseHandle}(\index{hMap+\\{hMap}}\\{hMap});\6
${}\index{hMap+\\{hMap}}\\{hMap}\K\NULL;{}$\6
\&{return} \T{0};\6
\4${}\}{}$\2\6
\8\#\&{else}\6
\8\#\&{include} \.{<sys/mman.h>}\6
\8\#\&{endif}
\U274.\Y
\fi

\M{276}


After mapping the file at address \index{hbase+\\{hbase}}\\{hbase}, access to sections of the
file is provided by setting the three pointers \index{hpos+\\{hpos}}\\{hpos}, \index{hstart+\\{hstart}}\\{hstart}, and
\index{hend+\\{hend}}\\{hend}. The value $\index{hbase+\\{hbase}}\\{hbase}\E\NULL$ indicates, that no file is open.


\gdef\subcodetitle{Sections}%
To read sections of a short format input file, we use the function \index{hget section+\\{hget\_section}}\\{hget\_section}.

\getcode
%\codesection{\getsymbol}\getindex{1}{3}{Files}
\Y\B\4\X263:get file functions\X${}\mathrel+\E{}$\6
\X278:\\{hdecompress} function\X\7
\&{void} \index{hget section+\\{hget\_section}}\\{hget\_section}(\&{uint16\_t} \|n)\1\1\2\2\1\6
\4${}\{{}$\5
${}\.{DBG}(\index{DBGDIR+\.{DBGDIR}}\.{DBGDIR},\39\.{"Reading\ section\ \%d\\}\)\.{n"},\39\|n);{}$\6
${}\.{RNG}(\.{"Section\ number"},\39\|n,\39\T{0},\39\index{max section no+\\{max\_section\_no}}\\{max\_section\_no});{}$\6
\&{if} ${}(\index{dir+\\{dir}}\\{dir}[\|n].\index{buffer+\\{buffer}}\\{buffer}\I\NULL\W\index{dir+\\{dir}}\\{dir}[\|n].\index{xsize+\\{xsize}}\\{xsize}>\T{0}){}$\5
\1${}\{{}$\5
${}\index{hpos+\\{hpos}}\\{hpos}\K\index{hstart+\\{hstart}}\\{hstart}\K\index{dir+\\{dir}}\\{dir}[\|n].\index{buffer+\\{buffer}}\\{buffer};{}$\6
${}\index{hend+\\{hend}}\\{hend}\K\index{hstart+\\{hstart}}\\{hstart}+\index{dir+\\{dir}}\\{dir}[\|n].\index{xsize+\\{xsize}}\\{xsize};{}$\6
\4${}\}{}$\2\6
\&{else}\5
\1${}\{{}$\5
${}\index{hpos+\\{hpos}}\\{hpos}\K\index{hstart+\\{hstart}}\\{hstart}\K\index{hbase+\\{hbase}}\\{hbase}+\index{dir+\\{dir}}\\{dir}[\|n].\index{pos+\\{pos}}\\{pos};{}$\6
${}\index{hend+\\{hend}}\\{hend}\K\index{hstart+\\{hstart}}\\{hstart}+\index{dir+\\{dir}}\\{dir}[\|n].\index{size+\\{size}}\\{size};{}$\6
\&{if} ${}(\index{dir+\\{dir}}\\{dir}[\|n].\index{xsize+\\{xsize}}\\{xsize}>\T{0}){}$\1\5
\index{hdecompress+\\{hdecompress}}\\{hdecompress}(\|n);\2\6
\4${}\}{}$\2\6
\4${}\}{}$\2
\Y
\fi

\M{277}

To write a short format file, we allocate for each of the first three sections a
suitable buffer\index{buffer}, then fill these buffers, and finally write them
out in sequential order.

\Y\B\4\X12:put functions\X${}\mathrel+\E{}$\6
\8\#\&{define} \index{BUFFER SIZE+\.{BUFFER\_SIZE}}\.{BUFFER\_SIZE}\5\T{\^400}\6
\&{void} \index{new output buffers+\\{new\_output\_buffers}}\\{new\_output\_buffers}(\&{void})\1\1\2\2\1\6
\4${}\{{}$\5
${}\index{dir+\\{dir}}\\{dir}[\T{0}].\index{bsize+\\{bsize}}\\{bsize}\K\index{dir+\\{dir}}\\{dir}[\T{1}].\index{bsize+\\{bsize}}\\{bsize}\K\index{dir+\\{dir}}\\{dir}[\T{2}].\index{bsize+\\{bsize}}\\{bsize}\K\index{BUFFER SIZE+\.{BUFFER\_SIZE}}\.{BUFFER\_SIZE};{}$\6
${}\.{DBG}(\index{DBGBUFFER+\.{DBGBUFFER}}\.{DBGBUFFER},\39\.{"Allocating\ output\ b}\)\.{uffer\ size=0x\%x,\ mar}\)\.{gin=0x\%x\\n"},\39\index{BUFFER SIZE+\.{BUFFER\_SIZE}}\.{BUFFER\_SIZE},\39\index{MAX TAG DISTANCE+\.{MAX\_TAG\_DISTANCE}}\.{MAX\_TAG\_DISTANCE});{}$\6
${}\index{ALLOCATE+\.{ALLOCATE}}\.{ALLOCATE}(\index{dir+\\{dir}}\\{dir}[\T{0}].\index{buffer+\\{buffer}}\\{buffer},\39\index{dir+\\{dir}}\\{dir}[\T{0}].\index{bsize+\\{bsize}}\\{bsize}+\index{MAX TAG DISTANCE+\.{MAX\_TAG\_DISTANCE}}\.{MAX\_TAG\_DISTANCE},\39\&{uint8\_t});{}$\6
${}\index{ALLOCATE+\.{ALLOCATE}}\.{ALLOCATE}(\index{dir+\\{dir}}\\{dir}[\T{1}].\index{buffer+\\{buffer}}\\{buffer},\39\index{dir+\\{dir}}\\{dir}[\T{1}].\index{bsize+\\{bsize}}\\{bsize}+\index{MAX TAG DISTANCE+\.{MAX\_TAG\_DISTANCE}}\.{MAX\_TAG\_DISTANCE},\39\&{uint8\_t});{}$\6
${}\index{ALLOCATE+\.{ALLOCATE}}\.{ALLOCATE}(\index{dir+\\{dir}}\\{dir}[\T{2}].\index{buffer+\\{buffer}}\\{buffer},\39\index{dir+\\{dir}}\\{dir}[\T{2}].\index{bsize+\\{bsize}}\\{bsize}+\index{MAX TAG DISTANCE+\.{MAX\_TAG\_DISTANCE}}\.{MAX\_TAG\_DISTANCE},\39\&{uint8\_t});{}$\6
\4${}\}{}$\2\7
\&{void} \index{hput increase buffer+\\{hput\_increase\_buffer}}\\{hput\_increase\_buffer}(\&{uint32\_t} \|n)\1\1\2\2\1\6
\4${}\{{}$\5
\&{size\_t} \index{bsize+\\{bsize}}\\{bsize};\6
\&{uint32\_t} \index{pos+\\{pos}}\\{pos};\6
\&{const} \&{double} \index{buffer factor+\\{buffer\_factor}}\\{buffer\_factor}${}\K\T{1.4142136}{}$;\C{ $\sqrt 2$ }\7
${}\index{pos+\\{pos}}\\{pos}\K\index{hpos+\\{hpos}}\\{hpos}-\index{hstart+\\{hstart}}\\{hstart};{}$\6
${}\index{bsize+\\{bsize}}\\{bsize}\K\index{dir+\\{dir}}\\{dir}[\index{section no+\\{section\_no}}\\{section\_no}].\index{bsize+\\{bsize}}\\{bsize}*\index{buffer factor+\\{buffer\_factor}}\\{buffer\_factor}+\T{0.5};{}$\6
\&{if} ${}(\index{bsize+\\{bsize}}\\{bsize}<\index{pos+\\{pos}}\\{pos}+\|n){}$\1\5
${}\index{bsize+\\{bsize}}\\{bsize}\K\index{pos+\\{pos}}\\{pos}+\|n;{}$\2\6
\&{if} ${}(\index{bsize+\\{bsize}}\\{bsize}\G\T{\^FFFFFFFF}){}$\1\5
${}\index{bsize+\\{bsize}}\\{bsize}\K\T{\^FFFFFFFF};{}$\2\6
\&{if} ${}(\index{bsize+\\{bsize}}\\{bsize}<\index{pos+\\{pos}}\\{pos}+\|n){}$\1\5
${}\.{QUIT}(\.{"Unable\ to\ increase\ }\)\.{buffer\ size\ "}\.{SIZE\_F}\.{"\ by\ 0x\%x\ byte"},\3{-1}\39\index{hpos+\\{hpos}}\\{hpos}-\index{hstart+\\{hstart}}\\{hstart},\39\|n);{}$\2\6
${}\.{DBG}(\index{DBGBUFFER+\.{DBGBUFFER}}\.{DBGBUFFER},\39\.{"Reallocating\ output}\)\.{\ buffer\ "}\3{-1}\.{"\ for\ section\ \%d\ fro}\)\.{m\ 0x\%x\ to\ "}\.{SIZE\_F}\.{"\ byte\\n"},\39\index{section no+\\{section\_no}}\\{section\_no},\39\index{dir+\\{dir}}\\{dir}[\index{section no+\\{section\_no}}\\{section\_no}].\index{bsize+\\{bsize}}\\{bsize},\39\index{bsize+\\{bsize}}\\{bsize});{}$\6
${}\index{REALLOCATE+\.{REALLOCATE}}\.{REALLOCATE}(\index{dir+\\{dir}}\\{dir}[\index{section no+\\{section\_no}}\\{section\_no}].\index{buffer+\\{buffer}}\\{buffer},\39\index{bsize+\\{bsize}}\\{bsize},\39\&{uint8\_t});{}$\6
${}\index{dir+\\{dir}}\\{dir}[\index{section no+\\{section\_no}}\\{section\_no}].\index{bsize+\\{bsize}}\\{bsize}\K{}$(\&{uint32\_t}) \index{bsize+\\{bsize}}\\{bsize};\6
${}\index{hstart+\\{hstart}}\\{hstart}\K\index{dir+\\{dir}}\\{dir}[\index{section no+\\{section\_no}}\\{section\_no}].\index{buffer+\\{buffer}}\\{buffer};{}$\6
${}\index{hend+\\{hend}}\\{hend}\K\index{hstart+\\{hstart}}\\{hstart}+\index{bsize+\\{bsize}}\\{bsize};{}$\6
${}\index{hpos+\\{hpos}}\\{hpos}\K\index{hstart+\\{hstart}}\\{hstart}+\index{pos+\\{pos}}\\{pos};{}$\6
\4${}\}{}$\2\7
\&{static} \&{size\_t} \index{hput data+\\{hput\_data}}\\{hput\_data}(\&{uint16\_t} \|n${},\39{}$\&{uint8\_t} ${}{*}\index{buffer+\\{buffer}}\\{buffer},\39{}$\&{uint32\_t} \index{size+\\{size}}\\{size})\1\1\2\2\1\6
\4${}\{{}$\5
\&{size\_t} \|s;\7
${}\|s\K\index{fwrite+\\{fwrite}}\\{fwrite}(\index{buffer+\\{buffer}}\\{buffer},\39\T{1},\39\index{size+\\{size}}\\{size},\39\index{hout+\\{hout}}\\{hout});{}$\6
\&{if} ${}(\|s\I\index{size+\\{size}}\\{size}){}$\1\5
${}\.{QUIT}(\.{"short\ write\ "}\.{SIZE\_F}\.{"\ <\ \%d\ in\ section\ \%d}\)\.{"},\39\|s,\39\index{size+\\{size}}\\{size},\39\|n);{}$\2\6
\&{return} \|s;\6
\4${}\}{}$\2\7
\&{static} \&{size\_t} \index{hput section+\\{hput\_section}}\\{hput\_section}(\&{uint16\_t} \|n)\1\1\2\2\1\6
\4${}\{{}$\5
\&{return} \index{hput data+\\{hput\_data}}\\{hput\_data}${}(\|n,\39\index{dir+\\{dir}}\\{dir}[\|n].\index{buffer+\\{buffer}}\\{buffer},\39\index{dir+\\{dir}}\\{dir}[\|n].\index{size+\\{size}}\\{size});{}$\6
\4${}\}{}$\2
\Y
\fi

\M{278}


\subsection{Compression}
The short file format offers the possibility to store sections in
compressed\index{compression} form. We use the {\tt zlib}\index{zlib+{\tt zlib}} compression library\cite{zlib}\cite{RFC1950}
to deflate\index{deflate} and inflate\index{inflate} individual sections.  When one of the following
functions is called, we can get the section buffer, the buffer size
and the size actually used from the directory entry.  If a section
needs to be inflated, its size after decompression is found in the
\index{xsize+\\{xsize}}\\{xsize} field; if a section needs to be deflated, its size after
compression will be known after deflating it.

\Y\par
\Y\B\4\X278:\\{hdecompress} function\X${}\E{}$\6
\&{static} \&{void} \index{hdecompress+\\{hdecompress}}\\{hdecompress}(\&{uint16\_t} \|n)\1\1\2\2\1\6
\4${}\{{}$\5
\index{z stream+\&{z\_stream}}\&{z\_stream} \|z;\C{ decompression stream }\6
\&{uint8\_t} ${}{*}\index{buffer+\\{buffer}}\\{buffer};{}$\6
\&{int} \|i;\7
${}\.{DBG}(\index{DBGCOMPRESS+\.{DBGCOMPRESS}}\.{DBGCOMPRESS},\39\.{"Decompressing\ secti}\)\.{on\ \%d\ from\ 0x\%x\ to\ 0}\)\.{x\%x\ byte\\n"},\3{-1}\39\|n,\39\index{dir+\\{dir}}\\{dir}[\|n].\index{size+\\{size}}\\{size},\39\index{dir+\\{dir}}\\{dir}[\|n].\index{xsize+\\{xsize}}\\{xsize});{}$\6
${}\|z.\index{zalloc+\\{zalloc}}\\{zalloc}\K(\index{alloc func+\\{alloc\_func}}\\{alloc\_func})\T{0}{}$;\5
${}\|z.\index{zfree+\\{zfree}}\\{zfree}\K(\index{free func+\\{free\_func}}\\{free\_func})\T{0}{}$;\5
${}\|z.\index{opaque+\\{opaque}}\\{opaque}\K(\index{voidpf+\\{voidpf}}\\{voidpf})\T{0};{}$\6
${}\|z.\index{next in+\\{next\_in}}\\{next\_in}\K\index{hstart+\\{hstart}}\\{hstart};{}$\6
${}\|z.\index{avail in+\\{avail\_in}}\\{avail\_in}\K\index{hend+\\{hend}}\\{hend}-\index{hstart+\\{hstart}}\\{hstart};{}$\6
\&{if} ${}(\index{inflateInit+\\{inflateInit}}\\{inflateInit}({\AND}\|z)\I\index{Z OK+\.{Z\_OK}}\.{Z\_OK}){}$\1\5
${}\.{QUIT}(\.{"Unable\ to\ initializ}\)\.{e\ decompression:\ \%s"},\39\|z.\\{msg});{}$\2\6
${}\index{ALLOCATE+\.{ALLOCATE}}\.{ALLOCATE}(\index{buffer+\\{buffer}}\\{buffer},\39\index{dir+\\{dir}}\\{dir}[\|n].\index{xsize+\\{xsize}}\\{xsize}+\index{MAX TAG DISTANCE+\.{MAX\_TAG\_DISTANCE}}\.{MAX\_TAG\_DISTANCE},\39\&{uint8\_t});{}$\6
${}\.{DBG}(\index{DBGBUFFER+\.{DBGBUFFER}}\.{DBGBUFFER},\39\.{"Allocating\ output\ b}\)\.{uffer\ size=0x\%x,\ mar}\)\.{gin=0x\%x\\n"},\39\index{dir+\\{dir}}\\{dir}[\|n].\index{xsize+\\{xsize}}\\{xsize},\39\index{MAX TAG DISTANCE+\.{MAX\_TAG\_DISTANCE}}\.{MAX\_TAG\_DISTANCE});{}$\6
${}\|z.\index{next out+\\{next\_out}}\\{next\_out}\K\index{buffer+\\{buffer}}\\{buffer};{}$\6
${}\|z.\index{avail out+\\{avail\_out}}\\{avail\_out}\K\index{dir+\\{dir}}\\{dir}[\|n].\index{xsize+\\{xsize}}\\{xsize}+\index{MAX TAG DISTANCE+\.{MAX\_TAG\_DISTANCE}}\.{MAX\_TAG\_DISTANCE};{}$\6
${}\|i\K\index{inflate+\\{inflate}}\\{inflate}({\AND}\|z,\39\index{Z FINISH+\.{Z\_FINISH}}\.{Z\_FINISH});{}$\6
${}\.{DBG}(\index{DBGCOMPRESS+\.{DBGCOMPRESS}}\.{DBGCOMPRESS},\39\.{"in:\ avail/total=0x\%}\)\.{x/0x\%lx\ "}\3{-1}\.{"out:\ avail/total=0x}\)\.{\%x/0x\%lx,\ return\ \%d;}\)\.{\\n"},\3{-1}\39\|z.\index{avail in+\\{avail\_in}}\\{avail\_in},\39\|z.\index{total in+\\{total\_in}}\\{total\_in},\39\|z.\index{avail out+\\{avail\_out}}\\{avail\_out},\39\|z.\index{total out+\\{total\_out}}\\{total\_out},\39\|i);{}$\6
\&{if} ${}(\|i\I\index{Z STREAM END+\.{Z\_STREAM\_END}}\.{Z\_STREAM\_END}){}$\1\5
${}\.{QUIT}(\.{"Unable\ to\ complete\ }\)\.{decompression:\ \%s"},\39\|z.\\{msg});{}$\2\6
\&{if} ${}(\|z.\index{avail in+\\{avail\_in}}\\{avail\_in}\I\T{0}){}$\1\5
\.{QUIT}(\.{"Decompression\ misse}\)\.{d\ input\ data"});\2\6
\&{if} ${}(\|z.\index{total out+\\{total\_out}}\\{total\_out}\I\index{dir+\\{dir}}\\{dir}[\|n].\index{xsize+\\{xsize}}\\{xsize}){}$\1\5
${}\.{QUIT}(\.{"Decompression\ outpu}\)\.{t\ size\ missmatch\ 0x\%}\)\.{lx\ !=\ 0x\%x"},\39\|z.\index{total out+\\{total\_out}}\\{total\_out},\39\index{dir+\\{dir}}\\{dir}[\|n].\index{xsize+\\{xsize}}\\{xsize});{}$\2\6
\&{if} ${}(\index{inflateEnd+\\{inflateEnd}}\\{inflateEnd}({\AND}\|z)\I\index{Z OK+\.{Z\_OK}}\.{Z\_OK}){}$\1\5
${}\.{QUIT}(\.{"Unable\ to\ finalize\ }\)\.{decompression:\ \%s"},\39\|z.\\{msg});{}$\2\6
${}\index{dir+\\{dir}}\\{dir}[\|n].\index{buffer+\\{buffer}}\\{buffer}\K\index{buffer+\\{buffer}}\\{buffer};{}$\6
${}\index{dir+\\{dir}}\\{dir}[\|n].\index{bsize+\\{bsize}}\\{bsize}\K\index{dir+\\{dir}}\\{dir}[\|n].\index{xsize+\\{xsize}}\\{xsize};{}$\6
${}\index{hpos+\\{hpos}}\\{hpos}\K\index{hstart+\\{hstart}}\\{hstart}\K\index{buffer+\\{buffer}}\\{buffer};{}$\6
${}\index{hend+\\{hend}}\\{hend}\K\index{hstart+\\{hstart}}\\{hstart}+\index{dir+\\{dir}}\\{dir}[\|n].\index{xsize+\\{xsize}}\\{xsize};{}$\6
\4${}\}{}$\2
\U276.\Y
\fi

\M{279}

\Y\B\4\X279:\\{hcompress} function\X${}\E{}$\6
\&{static} \&{void} \index{hcompress+\\{hcompress}}\\{hcompress}(\&{uint16\_t} \|n)\1\1\2\2\1\6
\4${}\{{}$\5
\index{z stream+\&{z\_stream}}\&{z\_stream} \|z;\C{ compression stream }\6
\&{uint8\_t} ${}{*}\index{buffer+\\{buffer}}\\{buffer};{}$\6
\&{int} \|i;\7
\&{if} ${}(\index{dir+\\{dir}}\\{dir}[\|n].\index{size+\\{size}}\\{size}\E\T{0}){}$\5
\1${}\{{}$\5
${}\index{dir+\\{dir}}\\{dir}[\|n].\index{xsize+\\{xsize}}\\{xsize}\K\T{0};{}$\6
\&{return};\6
\4${}\}{}$\2\6
${}\.{DBG}(\index{DBGCOMPRESS+\.{DBGCOMPRESS}}\.{DBGCOMPRESS},\39\.{"Compressing\ section}\)\.{\ \%d\ of\ size\ 0x\%x\\n"},\39\|n,\39\index{dir+\\{dir}}\\{dir}[\|n].\index{size+\\{size}}\\{size});{}$\6
${}\|z.\index{zalloc+\\{zalloc}}\\{zalloc}\K(\index{alloc func+\\{alloc\_func}}\\{alloc\_func})\T{0}{}$;\5
${}\|z.\index{zfree+\\{zfree}}\\{zfree}\K(\index{free func+\\{free\_func}}\\{free\_func})\T{0}{}$;\5
${}\|z.\index{opaque+\\{opaque}}\\{opaque}\K(\index{voidpf+\\{voidpf}}\\{voidpf})\T{0};{}$\6
\&{if} ${}(\index{deflateInit+\\{deflateInit}}\\{deflateInit}({\AND}\|z,\39\index{Z DEFAULT COMPRESSION+\.{Z\_DEFAULT\_COMPRESSION}}\.{Z\_DEFAULT\_COMPRESSION})\I\index{Z OK+\.{Z\_OK}}\.{Z\_OK}){}$\1\5
${}\.{QUIT}(\.{"Unable\ to\ initializ}\)\.{e\ compression:\ \%s"},\39\|z.\\{msg});{}$\2\6
${}\index{ALLOCATE+\.{ALLOCATE}}\.{ALLOCATE}(\index{buffer+\\{buffer}}\\{buffer},\39\index{dir+\\{dir}}\\{dir}[\|n].\index{size+\\{size}}\\{size}+\index{MAX TAG DISTANCE+\.{MAX\_TAG\_DISTANCE}}\.{MAX\_TAG\_DISTANCE},\39\&{uint8\_t});{}$\6
${}\|z.\index{next out+\\{next\_out}}\\{next\_out}\K\index{buffer+\\{buffer}}\\{buffer};{}$\6
${}\|z.\index{avail out+\\{avail\_out}}\\{avail\_out}\K\index{dir+\\{dir}}\\{dir}[\|n].\index{size+\\{size}}\\{size}+\index{MAX TAG DISTANCE+\.{MAX\_TAG\_DISTANCE}}\.{MAX\_TAG\_DISTANCE};{}$\6
${}\|z.\index{next in+\\{next\_in}}\\{next\_in}\K\index{dir+\\{dir}}\\{dir}[\|n].\index{buffer+\\{buffer}}\\{buffer};{}$\6
${}\|z.\index{avail in+\\{avail\_in}}\\{avail\_in}\K\index{dir+\\{dir}}\\{dir}[\|n].\index{size+\\{size}}\\{size};{}$\6
${}\|i\K\index{deflate+\\{deflate}}\\{deflate}({\AND}\|z,\39\index{Z FINISH+\.{Z\_FINISH}}\.{Z\_FINISH});{}$\6
${}\.{DBG}(\index{DBGCOMPRESS+\.{DBGCOMPRESS}}\.{DBGCOMPRESS},\39\.{"deflate\ in:\ avail/t}\)\.{otal=0x\%x/0x\%lx\ out:}\)\.{\ avail/total=0x\%x/0x}\)\.{\%lx,\ return\ \%d;\\n"},\3{-1}\39\|z.\index{avail in+\\{avail\_in}}\\{avail\_in},\39\|z.\index{total in+\\{total\_in}}\\{total\_in},\39\|z.\index{avail out+\\{avail\_out}}\\{avail\_out},\39\|z.\index{total out+\\{total\_out}}\\{total\_out},\39\|i);{}$\6
\&{if} ${}(\|z.\index{avail in+\\{avail\_in}}\\{avail\_in}\I\T{0}){}$\1\5
\.{QUIT}(\.{"Compression\ missed\ }\)\.{input\ data"});\2\6
\&{if} ${}(\|i\I\index{Z STREAM END+\.{Z\_STREAM\_END}}\.{Z\_STREAM\_END}){}$\1\5
${}\.{QUIT}(\.{"Compression\ incompl}\)\.{ete:\ \%s"},\39\|z.\\{msg});{}$\2\6
\&{if} ${}(\index{deflateEnd+\\{deflateEnd}}\\{deflateEnd}({\AND}\|z)\I\index{Z OK+\.{Z\_OK}}\.{Z\_OK}){}$\1\5
${}\.{QUIT}(\.{"Unable\ to\ finalize\ }\)\.{compression:\ \%s"},\39\|z.\\{msg});{}$\2\6
${}\.{DBG}(\index{DBGCOMPRESS+\.{DBGCOMPRESS}}\.{DBGCOMPRESS},\39\.{"Compressed\ 0x\%lx\ by}\)\.{te\ to\ 0x\%lx\ byte\\n"},\3{-1}\39\|z.\index{total in+\\{total\_in}}\\{total\_in},\39\|z.\index{total out+\\{total\_out}}\\{total\_out});{}$\6
${}\index{free+\\{free}}\\{free}(\index{dir+\\{dir}}\\{dir}[\|n].\index{buffer+\\{buffer}}\\{buffer});{}$\6
${}\index{dir+\\{dir}}\\{dir}[\|n].\index{buffer+\\{buffer}}\\{buffer}\K\index{buffer+\\{buffer}}\\{buffer};{}$\6
${}\index{dir+\\{dir}}\\{dir}[\|n].\index{bsize+\\{bsize}}\\{bsize}\K\index{dir+\\{dir}}\\{dir}[\|n].\index{size+\\{size}}\\{size}+\index{MAX TAG DISTANCE+\.{MAX\_TAG\_DISTANCE}}\.{MAX\_TAG\_DISTANCE};{}$\6
${}\index{dir+\\{dir}}\\{dir}[\|n].\index{xsize+\\{xsize}}\\{xsize}\K\index{dir+\\{dir}}\\{dir}[\|n].\index{size+\\{size}}\\{size};{}$\6
${}\index{dir+\\{dir}}\\{dir}[\|n].\index{size+\\{size}}\\{size}\K\|z.\index{total out+\\{total\_out}}\\{total\_out};{}$\6
\4${}\}{}$\2
\U297.\Y
\fi

\M{280}



\section{Directory Section}
A \HINT/ file is subdivided in sections and
each section can be identified by its section number.
The first three sections, numbered 0, 1, and 2, are mandatory:
directory\index{directory section} section, definition section,  and content section.
The directory section, which we explain now, lists all sections
that make up a \HINT/ file.

A document will often contain not only plain text but also other media
for example illustrations. Illustrations are produced with specialized
tools and stored in specialized files. Because a \HINT/ file in short format
should be self contained, these special files are embedded in the \HINT/ file
as optional sections.
Because a \HINT/ file in long format should be readable, these special files
are written to disk and only the file names are retained in the directory.
Writing special files to disk has also the advantage that you can modify
them individually before embedding them in a short format file.


\subsection{Directories in Long Format}\gdef\subcodetitle{Directory Section}%
The directory\index{directory section} section of a long format \HINT/ file starts
with the  ``\.{directory}'' keyword; then follows the maximum section number used and
a list of directory entries, one for each optional section numbered 3 and above.
Each entry consists of the keyword ``\.{section}'' followed by the
section number, followed by the file name.
The section numbers must be unique and fit into 16 bit.
The directory entries must be ordered with strictly increasing section numbers.
Keeping section numbers consecutive is recommended because it reduces the
memory footprint if directories are stored as arrays indexed by the section
number as we will do below.

\readcode
\Y\par
\par
\par
\par
\par
\Y\B\4\X2:symbols\X${}\mathrel+\E{}$\6
\8\%\&{token} \index{DIRECTORY+\ts{DIRECTORY}}\ts{DIRECTORY}\5\.{"directory"}\6
\8\%\&{token} \index{SECTION+\ts{SECTION}}\ts{SECTION}\5\.{"entry"}
\Y
\fi

\M{281}

\Y\B\4\X3:scanning rules\X${}\mathrel+\E{}$\6
${}\8\re{\vb{directory}}{}$\ac\&{return} \index{DIRECTORY+\ts{DIRECTORY}}\ts{DIRECTORY};\eac\7
${}\8\re{\vb{section}}{}$\ac\&{return} \index{SECTION+\ts{SECTION}}\ts{SECTION};\eac
\Y
\fi

\M{282}

\Y\B\4\X5:parsing rules\X${}\mathrel+\E{}$\6
\index{directory section+\nts{directory\_section}}\nts{directory\_section}: \1\1\5
\index{START+\ts{START}}\ts{START}\5
\index{DIRECTORY+\ts{DIRECTORY}}\ts{DIRECTORY}\5
\index{UNSIGNED+\ts{UNSIGNED}}\ts{UNSIGNED}\3{-1}\5
${}\{{}$\1\5
${}\index{new directory+\\{new\_directory}}\\{new\_directory}(\.{\$3}+\T{1});{}$\5
\index{new output buffers+\\{new\_output\_buffers}}\\{new\_output\_buffers}(\,);\5
${}\}{}$\2\5
\index{entry list+\nts{entry\_list}}\nts{entry\_list}\5
\index{END+\ts{END}}\ts{END};\2\2\7
\index{entry list+\nts{entry\_list}}\nts{entry\_list}:\,\5
\1\1\hbox to 0.5em{\hss${}\OR{}$}\5
\index{entry list+\nts{entry\_list}}\nts{entry\_list}\5
\index{entry+\nts{entry}}\nts{entry};\2\2\7
\index{entry+\nts{entry}}\nts{entry}: \1\1\5
\index{START+\ts{START}}\ts{START}\5
\index{SECTION+\ts{SECTION}}\ts{SECTION}\5
\index{UNSIGNED+\ts{UNSIGNED}}\ts{UNSIGNED}\5
\index{string+\nts{string}}\nts{string}\5
\index{END+\ts{END}}\ts{END}\6
${}\{{}$\1\5
${}\.{RNG}(\.{"Section\ number"},\39\.{\$3},\39\T{3},\39\index{max section no+\\{max\_section\_no}}\\{max\_section\_no});{}$\5
${}\index{hset entry+\\{hset\_entry}}\\{hset\_entry}({\AND}(\index{dir+\\{dir}}\\{dir}[\.{\$3}]),\39\.{\$3},\39\T{0},\39\T{0},\39\.{\$4});{}$\5
${}\}{}$\2;\2\2
\Y
\fi

\M{283}



We use a dynamically allocated array
of directory entries to store the directory.

\Y\B\4\X283:directory entry type\X${}\E{}$\6
\&{typedef} \&{struct} ${}\{{}$\1\6
\&{uint64\_t} \index{pos+\\{pos}}\\{pos};\6
\&{uint32\_t} \index{size+\\{size}}\\{size}${},{}$ \index{xsize+\\{xsize}}\\{xsize};\6
\&{uint16\_t} \index{section no+\\{section\_no}}\\{section\_no};\6
\&{char} ${}{*}\index{file name+\\{file\_name}}\\{file\_name};{}$\6
\&{uint8\_t} ${}{*}\index{buffer+\\{buffer}}\\{buffer};{}$\6
\&{uint32\_t} \index{bsize+\\{bsize}}\\{bsize};\2\6
${}\}{}$ \index{entry t+\&{entry\_t}}\&{entry\_t};
\Us432, 434, 438, 439\ETs441.\Y
\fi

\M{284}


The function \index{new directory+\\{new\_directory}}\\{new\_directory} allocates the directory.

\Y\B\4\X284:directory functions\X${}\E{}$\6
\index{entry t+\&{entry\_t}}\&{entry\_t} ${}{*}\index{dir+\\{dir}}\\{dir}\K\NULL;{}$\6
\&{uint16\_t} \index{section no+\\{section\_no}}\\{section\_no}${},{}$ \index{max section no+\\{max\_section\_no}}\\{max\_section\_no};\7
\&{void} \index{new directory+\\{new\_directory}}\\{new\_directory}(\&{uint32\_t} \index{entries+\\{entries}}\\{entries})\1\1\2\2\1\6
\4${}\{{}$\5
${}\.{DBG}(\index{DBGDIR+\.{DBGDIR}}\.{DBGDIR},\39\.{"Creating\ directory\ }\)\.{with\ \%d\ entries\\n"},\39\index{entries+\\{entries}}\\{entries});{}$\6
${}\.{RNG}(\.{"Directory\ entries"},\39\index{entries+\\{entries}}\\{entries},\39\T{3},\39\T{\^10000});{}$\6
${}\index{max section no+\\{max\_section\_no}}\\{max\_section\_no}\K\index{entries+\\{entries}}\\{entries}-\T{1};{}$\6
${}\index{ALLOCATE+\.{ALLOCATE}}\.{ALLOCATE}(\index{dir+\\{dir}}\\{dir},\39\index{entries+\\{entries}}\\{entries},\39\index{entry t+\&{entry\_t}}\&{entry\_t});{}$\6
${}\index{dir+\\{dir}}\\{dir}[\T{0}].\index{section no+\\{section\_no}}\\{section\_no}\K\T{0}{}$;\5
${}\index{dir+\\{dir}}\\{dir}[\T{1}].\index{section no+\\{section\_no}}\\{section\_no}\K\T{1}{}$;\5
${}\index{dir+\\{dir}}\\{dir}[\T{2}].\index{section no+\\{section\_no}}\\{section\_no}\K\T{2};{}$\6
\4${}\}{}$\2
\A285.
\Us433, 435, 438, 439\ETs441.\Y
\fi

\M{285}

The function \index{hset entry+\\{hset\_entry}}\\{hset\_entry} fills in the appropriate entry.
\Y\B\4\X284:directory functions\X${}\mathrel+\E{}$\6
\&{void} \index{hset entry+\\{hset\_entry}}\\{hset\_entry}(\index{entry t+\&{entry\_t}}\&{entry\_t} ${}{*}\|e,\39{}$\&{uint16\_t} \|i${},\39{}$\&{uint32\_t} \index{size+\\{size}}\\{size}${},\39{}$\&{uint32\_t} \index{xsize+\\{xsize}}\\{xsize}${},\3{-1}\39{}$\&{char} ${}{*}\index{file name+\\{file\_name}}\\{file\_name}){}$\1\1\2\2\1\6
\4${}\{{}$\5
${}\|e\MG\index{section no+\\{section\_no}}\\{section\_no}\K\|i;{}$\6
${}\|e\MG\index{size+\\{size}}\\{size}\K\index{size+\\{size}}\\{size}{}$;\5
${}\|e\MG\index{xsize+\\{xsize}}\\{xsize}\K\index{xsize+\\{xsize}}\\{xsize};{}$\6
\&{if} ${}(\index{file name+\\{file\_name}}\\{file\_name}\E\NULL\V{*}\index{file name+\\{file\_name}}\\{file\_name}\E\T{0}){}$\1\5
${}\|e\MG\index{file name+\\{file\_name}}\\{file\_name}\K\NULL;{}$\2\6
\&{else}\1\5
${}\|e\MG\index{file name+\\{file\_name}}\\{file\_name}\K\index{strdup+\\{strdup}}\\{strdup}(\index{file name+\\{file\_name}}\\{file\_name});{}$\2\6
${}\.{DBG}(\index{DBGDIR+\.{DBGDIR}}\.{DBGDIR},\39\.{"Creating\ entry\ \%d:\ }\)\.{\\"\%s\\"\ size=0x\%x\ xsi}\)\.{ze=0x\%x\\n"},\3{-1}\39\|i,\39\index{file name+\\{file\_name}}\\{file\_name},\39\index{size+\\{size}}\\{size},\39\index{xsize+\\{xsize}}\\{xsize});{}$\6
\4${}\}{}$\2
\Y
\fi

\M{286}


Writing the auxiliary files depends on the {\tt -f} and the {\tt -g}
option.

\Y\B\4\X286:without {\tt -f} skip writing an existing file\X${}\E{}$\6
\&{if} ${}(\R\index{option force+\\{option\_force}}\\{option\_force}\W\\{access}(\index{file name+\\{file\_name}}\\{file\_name},\39\index{F OK+\.{F\_OK}}\.{F\_OK})\E\T{0}){}$\5
\1${}\{{}$\5
${}\.{MESSAGE}(\.{"File\ '\%s'\ exists.\\n}\)\.{"}\3{-1}\.{"To\ rewrite\ the\ file}\)\.{\ use\ the\ -f\ option.\\}\)\.{n"},\39\index{file name+\\{file\_name}}\\{file\_name});{}$\6
\&{continue};\6
\4${}\}{}$\2
\U291.\Y
\fi

\M{287}

The above code uses the \\{access} function, and we need to make sure it is defined:
\Y\B\4\X287:make sure \\{access} is defined\X${}\E{}$\6
\8\#\&{ifdef} \index{WIN32+\.{WIN32}}\.{WIN32}\6
\8\#\&{include} \.{<io.h>}\6
\8\#\&{define} ${}\\{access}(\|N,\39\|M)\5\\{\_access}(\|N,\39\|M){}$\6
\8\#\&{define} \index{F OK+\.{F\_OK}}\.{F\_OK}\5\T{0}\6
\8\#\&{else}\6
\8\#\&{include} \.{<unistd.h>}\6
\8\#\&{endif}
\U291.\Y
\fi

\M{288}

With the {\tt -g} option, filenames are considered global, and files
are written to the filesystem possibly overwriting the existing files.
For example a font embedded in a \HINT/ file might replace a font of
the same name in some operating systems font folder.
If the \HINT/ file is {\tt shrink}ed on one system and
{\tt stretch}ed on another system, this is usually not the desired behaviour.
Without the {\tt -g} option,\label{absrel} the files will be written in two local directories.
The names of these directories are derived from the output file name,
replacing the extension ``{\tt .HINT}'' with ``{\tt .abs}'' if the original
filename contained an absolute path, and  replacing it with ``{\tt .rel}''
if the original filename contained a relative path. Inside these directories,
the path as given in the filename is retained.
When {\tt shrink}ing a \HINT/ file without the {\tt -g} option,
the original filenames can be reconstructed.

\Y\B\4\X288:without {\tt -g} compute a local \\{file\_name}\X${}\E{}$\6
\&{if} ${}(\R\index{option global+\\{option\_global}}\\{option\_global}){}$\5
\1${}\{{}$\5
\&{int} \index{path length+\\{path\_length}}\\{path\_length}${}\K{}$(\&{int}) \index{strlen+\\{strlen}}\\{strlen}(\index{file name+\\{file\_name}}\\{file\_name});\7
\X289:determine whether \\{file\_name} is absolute or relative\X\6
\&{if} ${}(\index{file name length+\\{file\_name\_length}}\\{file\_name\_length}<\index{stem length+\\{stem\_length}}\\{stem\_length}+\index{ext length+\\{ext\_length}}\\{ext\_length}+\index{path length+\\{path\_length}}\\{path\_length}){}$\5
\1${}\{{}$\5
${}\index{file name length+\\{file\_name\_length}}\\{file\_name\_length}\K\index{stem length+\\{stem\_length}}\\{stem\_length}+\index{ext length+\\{ext\_length}}\\{ext\_length}+\index{path length+\\{path\_length}}\\{path\_length};{}$\6
${}\index{REALLOCATE+\.{REALLOCATE}}\.{REALLOCATE}(\index{stem name+\\{stem\_name}}\\{stem\_name},\39\index{file name length+\\{file\_name\_length}}\\{file\_name\_length}+\T{1},\39\&{char});{}$\6
\4${}\}{}$\2\6
${}\index{strcpy+\\{strcpy}}\\{strcpy}(\index{stem name+\\{stem\_name}}\\{stem\_name}+\index{stem length+\\{stem\_length}}\\{stem\_length},\39\index{aux ext+\\{aux\_ext}}\\{aux\_ext}[\index{name type+\\{name\_type}}\\{name\_type}]);{}$\6
${}\index{strcpy+\\{strcpy}}\\{strcpy}(\index{stem name+\\{stem\_name}}\\{stem\_name}+\index{stem length+\\{stem\_length}}\\{stem\_length}+\index{ext length+\\{ext\_length}}\\{ext\_length},\39\index{file name+\\{file\_name}}\\{file\_name});{}$\6
${}\.{DBG}(\index{DBGDIR+\.{DBGDIR}}\.{DBGDIR},\39\.{"Replacing\ auxiliar\ }\)\.{file\ name:\\n\\t\%s\\n->}\)\.{\\t\%s\\n"},\39\index{file name+\\{file\_name}}\\{file\_name},\39\index{stem name+\\{stem\_name}}\\{stem\_name});{}$\6
${}\index{file name+\\{file\_name}}\\{file\_name}\K\index{stem name+\\{stem\_name}}\\{stem\_name};{}$\6
\4${}\}{}$\2
\Us291, 296\ETs298.\Y
\fi

\M{289}

\Y\B\4\X289:determine whether \\{file\_name} is absolute or relative\X${}\E{}$\6
\&{enum} ${}\{{}$\1\6
${}\\{absolute}\K\T{0},\39\index{relative+\\{relative}}\\{relative}\K\T{1}{}$\2\6
${}\}{}$ \index{name type+\\{name\_type}}\\{name\_type};\6
\&{char} ${}{*}\index{aux ext+\\{aux\_ext}}\\{aux\_ext}[\T{2}]\K\{\.{".abs/"},\39\.{".rel/"}\};{}$\6
\&{int} \index{ext length+\\{ext\_length}}\\{ext\_length}${}\K\T{5};{}$\7
\&{if} ${}(\index{file name+\\{file\_name}}\\{file\_name}[\T{0}]\E\.{'/'}){}$\5
\1${}\{{}$\5
${}\index{name type+\\{name\_type}}\\{name\_type}\K\\{absolute};{}$\6
${}\index{file name+\\{file\_name}}\\{file\_name}\PP;{}$\6
${}\index{path length+\\{path\_length}}\\{path\_length}\MM;{}$\6
\4${}\}{}$\2\6
\&{else} \&{if} ${}(\index{path length+\\{path\_length}}\\{path\_length}>\T{3}\W\index{isalpha+\\{isalpha}}\\{isalpha}(\index{file name+\\{file\_name}}\\{file\_name}[\T{0}])\W\index{file name+\\{file\_name}}\\{file\_name}[\T{1}]\E\.{':'}\W\index{file name+\\{file\_name}}\\{file\_name}[\T{2}]\E\.{'/'}){}$\5
\1${}\{{}$\5
${}\index{name type+\\{name\_type}}\\{name\_type}\K\\{absolute};{}$\6
${}\index{file name+\\{file\_name}}\\{file\_name}[\T{1}]\K\.{'\_'};{}$\6
\4${}\}{}$\2\6
\&{else}\1\5
${}\index{name type+\\{name\_type}}\\{name\_type}\K\index{relative+\\{relative}}\\{relative}{}$;\2
\U288.\Y
\fi

\M{290}
It remains to create the direcories along the path we might have constructed.
\Y\B\4\X290:make sure the path in \\{file\_name} exists\X${}\E{}$\1\6
\4${}\{{}$\5
\&{char} ${}{*}\index{path end+\\{path\_end}}\\{path\_end};{}$\7
${}\index{path end+\\{path\_end}}\\{path\_end}\K\index{file name+\\{file\_name}}\\{file\_name}+\T{1};{}$\6
\&{while} ${}({*}\index{path end+\\{path\_end}}\\{path\_end}\I\T{0}){}$\5
\1${}\{{}$\6
\&{if} ${}({*}\index{path end+\\{path\_end}}\\{path\_end}\E\.{'/'}){}$\5
\1${}\{{}$\5
\&{struct} \index{stat+\\{stat}}\\{stat} \|s;\7
${}{*}\index{path end+\\{path\_end}}\\{path\_end}\K\T{0};{}$\6
\&{if} ${}(\index{stat+\\{stat}}\\{stat}(\index{file name+\\{file\_name}}\\{file\_name},\39{\AND}\|s)\E{-}\T{1}){}$\5
\1${}\{{}$\6
\8\#\&{ifdef} \index{WIN32+\.{WIN32}}\.{WIN32}\6
\&{if} ${}(\index{mkdir+\\{mkdir}}\\{mkdir}(\index{file name+\\{file\_name}}\\{file\_name})\I\T{0}){}$\6
\8\#\&{else}\1\6
\&{if} ${}(\index{mkdir+\\{mkdir}}\\{mkdir}(\index{file name+\\{file\_name}}\\{file\_name},\39\T{\~777})\I\T{0}){}$\6
\8\#\&{endif}\1\6
${}\.{QUIT}(\.{"Unable\ to\ create\ di}\)\.{rectory\ \%s"},\39\index{file name+\\{file\_name}}\\{file\_name});{}$\2\2\6
${}\.{DBG}(\index{DBGDIR+\.{DBGDIR}}\.{DBGDIR},\39\.{"Creating\ directory\ }\)\.{\%s\\n"},\39\index{file name+\\{file\_name}}\\{file\_name});{}$\6
\4${}\}{}$\2\6
\&{else} \&{if} ${}(\R\index{S ISDIR+\.{S\_ISDIR}}\.{S\_ISDIR}(\|s.\index{st mode+\\{st\_mode}}\\{st\_mode})){}$\1\5
${}\.{QUIT}(\.{"Unable\ to\ create\ di}\)\.{rectory\ \%s,\ file\ exi}\)\.{sts"},\39\index{file name+\\{file\_name}}\\{file\_name});{}$\2\6
${}{*}\index{path end+\\{path\_end}}\\{path\_end}\K\.{'/'};{}$\6
\4${}\}{}$\2\6
${}\index{path end+\\{path\_end}}\\{path\_end}\PP;{}$\6
\4${}\}{}$\2\6
\4${}\}{}$\2
\Us291\ET371.\Y
\fi

\M{291}

\writecode
\Y\B\4\X19:write functions\X${}\mathrel+\E{}$\6
\X287:make sure \\{access} is defined\X\7
\&{extern} \&{char} ${}{*}\index{stem name+\\{stem\_name}}\\{stem\_name};{}$\6
\&{extern} \&{int} \index{stem length+\\{stem\_length}}\\{stem\_length};\7
\&{void} \index{hwrite aux files+\\{hwrite\_aux\_files}}\\{hwrite\_aux\_files}(\&{void})\1\1\2\2\1\6
\4${}\{{}$\5
\&{int} \|i;\7
${}\.{DBG}(\index{DBGDIR+\.{DBGDIR}}\.{DBGDIR},\39\.{"Writing\ \%d\ aux\ file}\)\.{s\\n"},\39\index{max section no+\\{max\_section\_no}}\\{max\_section\_no}-\T{2});{}$\6
\&{for} ${}(\|i\K\T{3};{}$ ${}\|i\Z\index{max section no+\\{max\_section\_no}}\\{max\_section\_no};{}$ ${}\|i\PP){}$\5
\1${}\{{}$\5
\&{FILE} ${}{*}\|f;{}$\6
\&{char} ${}{*}\index{file name+\\{file\_name}}\\{file\_name}\K\index{dir+\\{dir}}\\{dir}[\|i].\index{file name+\\{file\_name}}\\{file\_name};{}$\6
\&{int} \index{file name length+\\{file\_name\_length}}\\{file\_name\_length}${}\K\T{0};{}$\7
\X288:without {\tt -g} compute a local \\{file\_name}\X\6
\X286:without {\tt -f} skip writing an existing file\X\6
\X290:make sure the path in \\{file\_name} exists\X\6
${}\|f\K\index{fopen+\\{fopen}}\\{fopen}(\index{file name+\\{file\_name}}\\{file\_name},\39\.{"wb"});{}$\6
\&{if} ${}(\|f\E\NULL){}$\1\5
${}\.{QUIT}(\.{"Unable\ to\ open\ file}\)\.{\ '\%s'\ for\ writing"},\39\index{file name+\\{file\_name}}\\{file\_name});{}$\2\6
\&{else}\5
\1${}\{{}$\5
\&{size\_t} \|s;\7
\index{hget section+\\{hget\_section}}\\{hget\_section}(\|i);\6
${}\.{DBG}(\index{DBGDIR+\.{DBGDIR}}\.{DBGDIR},\39\.{"Writing\ file\ \%s\\n"},\39\index{file name+\\{file\_name}}\\{file\_name});{}$\6
${}\|s\K\index{fwrite+\\{fwrite}}\\{fwrite}(\index{hstart+\\{hstart}}\\{hstart},\39\T{1},\39\index{dir+\\{dir}}\\{dir}[\|i].\index{size+\\{size}}\\{size},\39\|f);{}$\6
\&{if} ${}(\|s\I\index{dir+\\{dir}}\\{dir}[\|i].\index{size+\\{size}}\\{size}){}$\1\5
${}\.{QUIT}(\.{"writing\ file\ \%s"},\39\index{file name+\\{file\_name}}\\{file\_name});{}$\2\6
\index{fclose+\\{fclose}}\\{fclose}(\|f);\6
\4${}\}{}$\2\6
\4${}\}{}$\2\6
\4${}\}{}$\2
\Y
\fi

\M{292}

We write the directory, and the directory entries
in long format using the following functions.
\Y\B\4\X19:write functions\X${}\mathrel+\E{}$\6
\&{static} \&{void} \index{hwrite entry+\\{hwrite\_entry}}\\{hwrite\_entry}(\&{int} \|i)\1\1\2\2\1\6
\4${}\{{}$\5
\index{hwrite start+\\{hwrite\_start}}\\{hwrite\_start}(\,);\6
${}\index{hwritef+\\{hwritef}}\\{hwritef}(\.{"section\ \%u"},\39\index{dir+\\{dir}}\\{dir}[\|i].\index{section no+\\{section\_no}}\\{section\_no}){}$;\5
${}\index{hwrite string+\\{hwrite\_string}}\\{hwrite\_string}(\index{dir+\\{dir}}\\{dir}[\|i].\index{file name+\\{file\_name}}\\{file\_name});{}$\6
\index{hwrite end+\\{hwrite\_end}}\\{hwrite\_end}(\,);\6
\4${}\}{}$\2\7
\&{void} \index{hwrite directory+\\{hwrite\_directory}}\\{hwrite\_directory}(\&{void})\1\1\2\2\1\6
\4${}\{{}$\5
\&{int} \|i;\7
\&{if} ${}(\index{dir+\\{dir}}\\{dir}\E\NULL){}$\1\5
\.{QUIT}(\.{"Directory\ not\ alloc}\)\.{ated"});\2\6
${}\index{section no+\\{section\_no}}\\{section\_no}\K\T{0};{}$\6
${}\index{hwritef+\\{hwritef}}\\{hwritef}(\.{"<directory\ \%u"},\39\index{max section no+\\{max\_section\_no}}\\{max\_section\_no}){}$;\6
\&{for} ${}(\|i\K\T{3};{}$ ${}\|i\Z\index{max section no+\\{max\_section\_no}}\\{max\_section\_no};{}$ ${}\|i\PP){}$\1\5
\index{hwrite entry+\\{hwrite\_entry}}\\{hwrite\_entry}(\|i);\2\6
\index{hwritef+\\{hwritef}}\\{hwritef}(\.{"\\n>\\n"});\6
\4${}\}{}$\2
\Y
\fi

\M{293}

\subsection{Directories in Short Format}
The directory\index{directory section} section of a short format file contains entries
for all sections including the directory section itself. After reading the
directory section, enough information---position and size---is available to
access any section directly. As usual, a directory entry starts and ends with
a tag byte. The kind part of an entry's tag is not used; it is always zero.
The value $s$ of the two least significant bits of the info part indicate
that sizes are stored using $s+1$ byte.  The most significant bit of the info
part is 1 if the section is stored in compressed\index{compression} form. In this case the size
of the section is followed by the size of the section after decompressing it.
After the tag byte follows the section number. In the short format file,
section numbers must be strictly increasing and consecutive. This is redundant but helps
with checking. Then follows the size---or the sizes---of the section. After the size
follows the file name terminated by a zero byte. The file name might be an empty
string in which case there is just the zero byte. After the zero byte follows
a copy of the tag byte.

Here is the macro and function to read a directory\index{directory entry} entry:
\gdef\subcodetitle{Directory Entries}%
\getcode

\Y\B\4\X35:get file macros\X${}\mathrel+\E{}$\6
\8\#\&{define} \index{HGET SIZE+\.{HGET\_SIZE}}\.{HGET\_SIZE}(\|I) \6
\&{if} ${}((\|I)\AND\\{b100}){}$\5
\1${}\{{}$\6
\&{if} ${}(((\|I)\AND\\{b011})\E\T{0}){}$\1\5
${}\|s\K\index{HGET8+\.{HGET8}}\.{HGET8},\39\index{xs+\\{xs}}\\{xs}\K\index{HGET8+\.{HGET8}}\.{HGET8};{}$\2\6
\&{else} \&{if} ${}(((\|I)\AND\\{b011})\E\T{1}){}$\1\5
${}\index{HGET16+\.{HGET16}}\.{HGET16}(\|s),\39\index{HGET16+\.{HGET16}}\.{HGET16}(\index{xs+\\{xs}}\\{xs});{}$\2\6
\&{else} \&{if} ${}(((\|I)\AND\\{b011})\E\T{2}){}$\1\5
${}\index{HGET24+\.{HGET24}}\.{HGET24}(\|s),\39\index{HGET24+\.{HGET24}}\.{HGET24}(\index{xs+\\{xs}}\\{xs});{}$\2\6
\&{else} \&{if} ${}(((\|I)\AND\\{b011})\E\T{3}){}$\1\5
${}\index{HGET32+\.{HGET32}}\.{HGET32}(\|s),\39\index{HGET32+\.{HGET32}}\.{HGET32}(\index{xs+\\{xs}}\\{xs});{}$\2\6
\4${}\}{}$\2\6
\&{else}\5
\1${}\{{}$\6
\&{if} ${}(((\|I)\AND\\{b011})\E\T{0}){}$\1\5
${}\|s\K\index{HGET8+\.{HGET8}}\.{HGET8};{}$\2\6
\&{else} \&{if} ${}(((\|I)\AND\\{b011})\E\T{1}){}$\1\5
\index{HGET16+\.{HGET16}}\.{HGET16}(\|s);\2\6
\&{else} \&{if} ${}(((\|I)\AND\\{b011})\E\T{2}){}$\1\5
\index{HGET24+\.{HGET24}}\.{HGET24}(\|s);\2\6
\&{else} \&{if} ${}(((\|I)\AND\\{b011})\E\T{3}){}$\1\5
\index{HGET32+\.{HGET32}}\.{HGET32}(\|s);\2\6
\4${}\}{}$\2\6
\8\#\&{define} ${}\index{HGET ENTRY+\.{HGET\_ENTRY}}\.{HGET\_ENTRY}(\|I,\39\|E){}$\1\1\2\2\1\6
\4${}\{{}$\5
\&{uint16\_t} \|i;\6
\&{uint32\_t} \|s${}\K\T{0},{}$ \index{xs+\\{xs}}\\{xs}${}\K\T{0};{}$\6
\&{char} ${}{*}\index{file name+\\{file\_name}}\\{file\_name};{}$\7
\index{HGET16+\.{HGET16}}\.{HGET16}(\|i);\6
\index{HGET SIZE+\.{HGET\_SIZE}}\.{HGET\_SIZE}(\|I);\6
\index{HGET STRING+\.{HGET\_STRING}}\.{HGET\_STRING}(\index{file name+\\{file\_name}}\\{file\_name});\6
${}\index{hset entry+\\{hset\_entry}}\\{hset\_entry}({\AND}(\|E),\39\|i,\39\|s,\39\index{xs+\\{xs}}\\{xs},\39\index{file name+\\{file\_name}}\\{file\_name});{}$\6
\4${}\}{}$\2
\Y
\fi

\M{294}

\Y\B\4\X263:get file functions\X${}\mathrel+\E{}$\6
\&{void} \index{hget entry+\\{hget\_entry}}\\{hget\_entry}(\index{entry t+\&{entry\_t}}\&{entry\_t} ${}{*}\|e){}$\1\1\2\2\1\6
\4${}\{{}$\5
\X14:read the start byte \|a\X\6
${}\.{DBG}(\index{DBGDIR+\.{DBGDIR}}\.{DBGDIR},\39\.{"Reading\ directory\ e}\)\.{ntry\\n"});{}$\6
\&{switch} (\|a)\5
\1${}\{{}$\6
\4\&{case} \.{TAG}${}(\T{0},\39\\{b000}+\T{0}){}$:\5
${}\index{HGET ENTRY+\.{HGET\_ENTRY}}\.{HGET\_ENTRY}(\\{b000}+\T{0},\39{*}\|e){}$;\5
\&{break};\6
\4\&{case} \.{TAG}${}(\T{0},\39\\{b000}+\T{1}){}$:\5
${}\index{HGET ENTRY+\.{HGET\_ENTRY}}\.{HGET\_ENTRY}(\\{b000}+\T{1},\39{*}\|e){}$;\5
\&{break};\6
\4\&{case} \.{TAG}${}(\T{0},\39\\{b000}+\T{2}){}$:\5
${}\index{HGET ENTRY+\.{HGET\_ENTRY}}\.{HGET\_ENTRY}(\\{b000}+\T{2},\39{*}\|e){}$;\5
\&{break};\6
\4\&{case} \.{TAG}${}(\T{0},\39\\{b000}+\T{3}){}$:\5
${}\index{HGET ENTRY+\.{HGET\_ENTRY}}\.{HGET\_ENTRY}(\\{b000}+\T{3},\39{*}\|e){}$;\5
\&{break};\6
\4\&{case} \.{TAG}${}(\T{0},\39\\{b100}+\T{0}){}$:\5
${}\index{HGET ENTRY+\.{HGET\_ENTRY}}\.{HGET\_ENTRY}(\\{b100}+\T{0},\39{*}\|e){}$;\5
\&{break};\6
\4\&{case} \.{TAG}${}(\T{0},\39\\{b100}+\T{1}){}$:\5
${}\index{HGET ENTRY+\.{HGET\_ENTRY}}\.{HGET\_ENTRY}(\\{b100}+\T{1},\39{*}\|e){}$;\5
\&{break};\6
\4\&{case} \.{TAG}${}(\T{0},\39\\{b100}+\T{2}){}$:\5
${}\index{HGET ENTRY+\.{HGET\_ENTRY}}\.{HGET\_ENTRY}(\\{b100}+\T{2},\39{*}\|e){}$;\5
\&{break};\6
\4\&{case} \.{TAG}${}(\T{0},\39\\{b100}+\T{3}){}$:\5
${}\index{HGET ENTRY+\.{HGET\_ENTRY}}\.{HGET\_ENTRY}(\\{b100}+\T{3},\39{*}\|e){}$;\5
\&{break};\6
\4\&{default}:\5
\.{TAGERR}(\|a);\5
\&{break};\6
\4${}\}{}$\2\6
\X15:read and check the end byte \|z\X\6
${}\.{DBG}(\index{DBGDIR+\.{DBGDIR}}\.{DBGDIR},\39\.{"entry\ \%d:\ size=0x\%x}\)\.{\ xsize=0x\%x\\n"},\3{-1}\39\|e\MG\index{section no+\\{section\_no}}\\{section\_no},\39\|e\MG\index{size+\\{size}}\\{size},\39\|e\MG\index{xsize+\\{xsize}}\\{xsize});{}$\6
\4${}\}{}$\2
\Y
\fi

\M{295}

Because the first entry in the directory section describes the
directory section itself, we can not check its info bits in advance to determine
whether it is compressed or not. Therefore the directory section
starts with a root entry, which is always uncompressed. It describes
the position and size of the remainder of the directory which
follows.
There are two differences between the root entry and a normal entry:
it starts with the maximum section number instead of the section number zero,
and its position describes the position of the
entry for section 1 (which might already be compressed).
The name of the directory section must be the empty string.
\gdef\subcodetitle{Directory Section}%
\getcode
\Y\B\4\X263:get file functions\X${}\mathrel+\E{}$\6
\&{static} \&{void} \index{hget root+\\{hget\_root}}\\{hget\_root}(\index{entry t+\&{entry\_t}}\&{entry\_t} ${}{*}\index{root+\\{root}}\\{root}){}$\1\1\2\2\1\6
\4${}\{{}$\5
${}\.{DBG}(\index{DBGDIR+\.{DBGDIR}}\.{DBGDIR},\39\.{"Get\ Root\\n"});{}$\6
\index{hget entry+\\{hget\_entry}}\\{hget\_entry}(\index{root+\\{root}}\\{root});\6
${}\index{root+\\{root}}\\{root}\MG\index{pos+\\{pos}}\\{pos}\K\index{hpos+\\{hpos}}\\{hpos}-\index{hstart+\\{hstart}}\\{hstart};{}$\6
${}\index{max section no+\\{max\_section\_no}}\\{max\_section\_no}\K\index{root+\\{root}}\\{root}\MG\index{section no+\\{section\_no}}\\{section\_no};{}$\6
${}\index{root+\\{root}}\\{root}\MG\index{section no+\\{section\_no}}\\{section\_no}\K\T{0};{}$\6
\&{if} ${}(\index{max section no+\\{max\_section\_no}}\\{max\_section\_no}<\T{2}){}$\1\5
\.{QUIT}(\.{"Sections\ 0,\ 1,\ and\ }\)\.{2\ are\ mandatory"});\2\6
\4${}\}{}$\2\7
\&{void} \index{hget directory+\\{hget\_directory}}\\{hget\_directory}(\&{void})\1\1\2\2\1\6
\4${}\{{}$\5
\&{int} \|i;\6
\index{entry t+\&{entry\_t}}\&{entry\_t} \index{root+\\{root}}\\{root}${}\K\{\T{0}\};{}$\7
${}\index{hget root+\\{hget\_root}}\\{hget\_root}({\AND}\index{root+\\{root}}\\{root});{}$\6
${}\.{DBG}(\index{DBGDIR+\.{DBGDIR}}\.{DBGDIR},\39\.{"Get\ Directory\\n"});{}$\6
${}\index{new directory+\\{new\_directory}}\\{new\_directory}(\index{max section no+\\{max\_section\_no}}\\{max\_section\_no}+\T{1});{}$\6
${}\index{dir+\\{dir}}\\{dir}[\T{0}]\K\index{root+\\{root}}\\{root};{}$\6
\index{hget section+\\{hget\_section}}\\{hget\_section}(\T{0});\6
\&{for} ${}(\|i\K\T{1};{}$ ${}\|i\Z\index{max section no+\\{max\_section\_no}}\\{max\_section\_no};{}$ ${}\|i\PP{}$)\6
\1${}\{{}$\5
${}\index{hget entry+\\{hget\_entry}}\\{hget\_entry}({\AND}(\index{dir+\\{dir}}\\{dir}[\|i])){}$;\5
${}\index{dir+\\{dir}}\\{dir}[\|i].\index{pos+\\{pos}}\\{pos}\K\index{dir+\\{dir}}\\{dir}[\|i-\T{1}].\index{pos+\\{pos}}\\{pos}+\index{dir+\\{dir}}\\{dir}[\|i-\T{1}].\index{size+\\{size}}\\{size}{}$;\5
${}\}{}$\2\6
${}\.{DBG}(\index{DBGDIR+\.{DBGDIR}}\.{DBGDIR},\39\.{"Directory\ at\ 0x\%"}\index{PRIx64+\\{PRIx64}}\\{PRIx64}\.{"\\n"},\39\index{dir+\\{dir}}\\{dir}[\T{0}].\index{pos+\\{pos}}\\{pos});{}$\6
${}\.{DBG}(\index{DBGDIR+\.{DBGDIR}}\.{DBGDIR},\39\.{"Definitions\ at\ 0x\%"}\index{PRIx64+\\{PRIx64}}\\{PRIx64}\.{"\\n"},\39\index{dir+\\{dir}}\\{dir}[\T{1}].\index{pos+\\{pos}}\\{pos});{}$\6
${}\.{DBG}(\index{DBGDIR+\.{DBGDIR}}\.{DBGDIR},\39\.{"Content\ at\ 0x\%"}\index{PRIx64+\\{PRIx64}}\\{PRIx64}\.{"\\n"},\39\index{dir+\\{dir}}\\{dir}[\T{2}].\index{pos+\\{pos}}\\{pos});{}$\6
\4${}\}{}$\2\7
\&{void} \index{hclear dir+\\{hclear\_dir}}\\{hclear\_dir}(\&{void})\1\1\2\2\1\6
\4${}\{{}$\5
\&{int} \|i;\7
\&{if} ${}(\index{dir+\\{dir}}\\{dir}\E\NULL){}$\1\5
\&{return};\2\6
\&{for} ${}(\|i\K\T{0};{}$ ${}\|i<\T{3};{}$ ${}\|i\PP{}$)\C{ currently the only compressed sections }\1\6
\&{if} ${}(\index{dir+\\{dir}}\\{dir}[\|i].\index{xsize+\\{xsize}}\\{xsize}>\T{0}\W\index{dir+\\{dir}}\\{dir}[\|i].\index{buffer+\\{buffer}}\\{buffer}\I\NULL){}$\1\5
${}\index{free+\\{free}}\\{free}(\index{dir+\\{dir}}\\{dir}[\|i].\index{buffer+\\{buffer}}\\{buffer});{}$\2\2\6
\index{free+\\{free}}\\{free}(\index{dir+\\{dir}}\\{dir});\6
${}\index{dir+\\{dir}}\\{dir}\K\NULL;{}$\6
\4${}\}{}$\2
\Y
\fi

\M{296}

When the \.{shrink} program writes the directory section in the short format,
it needs to know the sizes of all the  sections---including the optional sections.
These sizes are not provided in the long format because it is safer and more
convenient to let the machine figure out the file sizes\index{file size}.

\Y\B\4\X296:set the file sizes for optional sections\X${}\E{}$\1\6
\4${}\{{}$\5
\&{int} \|i;\7
\&{for} ${}(\|i\K\T{3};{}$ ${}\|i\Z\index{max section no+\\{max\_section\_no}}\\{max\_section\_no};{}$ ${}\|i\PP){}$\5
\1${}\{{}$\5
\&{struct} \index{stat+\\{stat}}\\{stat} \|s;\6
\&{char} ${}{*}\index{file name+\\{file\_name}}\\{file\_name}\K\index{dir+\\{dir}}\\{dir}[\|i].\index{file name+\\{file\_name}}\\{file\_name};{}$\6
\&{int} \index{file name length+\\{file\_name\_length}}\\{file\_name\_length}${}\K\T{0};{}$\7
\X288:without {\tt -g} compute a local \\{file\_name}\X\6
\&{if} ${}(\index{stat+\\{stat}}\\{stat}(\index{file name+\\{file\_name}}\\{file\_name},\39{\AND}\|s)\I\T{0}){}$\1\5
${}\.{QUIT}(\.{"Unable\ to\ obtain\ fi}\)\.{le\ size\ for\ '\%s'"},\39\index{dir+\\{dir}}\\{dir}[\|i].\index{file name+\\{file\_name}}\\{file\_name});{}$\2\6
${}\index{dir+\\{dir}}\\{dir}[\|i].\index{size+\\{size}}\\{size}\K\|s.\index{st size+\\{st\_size}}\\{st\_size};{}$\6
${}\index{dir+\\{dir}}\\{dir}[\|i].\index{xsize+\\{xsize}}\\{xsize}\K\T{0};{}$\6
\4${}\}{}$\2\6
\4${}\}{}$\2
\U297.\Y
\fi

\M{297}

The computation of the sizes of the mandatory sections will be
explained later.
Armed with these preparations, we can put the directory into the \HINT/ file.

\gdef\subcodetitle{Directory Section}%
\putcode
\Y\B\4\X12:put functions\X${}\mathrel+\E{}$\6
\&{static} \&{void} \index{hput entry+\\{hput\_entry}}\\{hput\_entry}(\index{entry t+\&{entry\_t}}\&{entry\_t} ${}{*}\|e){}$\1\1\2\2\1\6
\4${}\{{}$\5
\&{uint8\_t} \|b;\7
\&{if} ${}(\|e\MG\index{size+\\{size}}\\{size}<\T{\^100}\W\|e\MG\index{xsize+\\{xsize}}\\{xsize}<\T{\^100}){}$\1\5
${}\|b\K\T{0};{}$\2\6
\&{else} \&{if} ${}(\|e\MG\index{size+\\{size}}\\{size}<\T{\^10000}\W\|e\MG\index{xsize+\\{xsize}}\\{xsize}<\T{\^10000}){}$\1\5
${}\|b\K\T{1};{}$\2\6
\&{else} \&{if} ${}(\|e\MG\index{size+\\{size}}\\{size}<\T{\^1000000}\W\|e\MG\index{xsize+\\{xsize}}\\{xsize}<\T{\^1000000}){}$\1\5
${}\|b\K\T{2};{}$\2\6
\&{else}\1\5
${}\|b\K\T{3};{}$\2\6
\&{if} ${}(\|e\MG\index{xsize+\\{xsize}}\\{xsize}\I\T{0}){}$\1\5
${}\|b\K\|b\OR\\{b100};{}$\2\6
${}\.{DBG}(\index{DBGTAGS+\.{DBGTAGS}}\.{DBGTAGS},\39\.{"Directory\ entry\ no=}\)\.{\%d\ size=0x\%x\ xsize=0}\)\.{x\%x\\n"},\39\|e\MG\index{section no+\\{section\_no}}\\{section\_no},\39\|e\MG\index{size+\\{size}}\\{size},\39\|e\MG\index{xsize+\\{xsize}}\\{xsize});{}$\6
${}\index{HPUTTAG+\.{HPUTTAG}}\.{HPUTTAG}(\T{0},\39\|b){}$;\6
${}\index{HPUT16+\.{HPUT16}}\.{HPUT16}(\|e\MG\index{section no+\\{section\_no}}\\{section\_no});{}$\6
\&{switch} (\|b)\5
\1${}\{{}$\6
\4\&{case} \T{0}:\5
${}\index{HPUT8+\.{HPUT8}}\.{HPUT8}(\|e\MG\index{size+\\{size}}\\{size}){}$;\5
\&{break};\6
\4\&{case} \T{1}:\5
${}\index{HPUT16+\.{HPUT16}}\.{HPUT16}(\|e\MG\index{size+\\{size}}\\{size}){}$;\5
\&{break};\6
\4\&{case} \T{2}:\5
${}\index{HPUT24+\.{HPUT24}}\.{HPUT24}(\|e\MG\index{size+\\{size}}\\{size}){}$;\5
\&{break};\6
\4\&{case} \T{3}:\5
${}\index{HPUT32+\.{HPUT32}}\.{HPUT32}(\|e\MG\index{size+\\{size}}\\{size}){}$;\5
\&{break};\6
\4\&{case} \\{b100}${}\OR\T{0}{}$:\5
${}\index{HPUT8+\.{HPUT8}}\.{HPUT8}(\|e\MG\index{size+\\{size}}\\{size}){}$;\5
${}\index{HPUT8+\.{HPUT8}}\.{HPUT8}(\|e\MG\index{xsize+\\{xsize}}\\{xsize}){}$;\5
\&{break};\6
\4\&{case} \\{b100}${}\OR\T{1}{}$:\5
${}\index{HPUT16+\.{HPUT16}}\.{HPUT16}(\|e\MG\index{size+\\{size}}\\{size}){}$;\5
${}\index{HPUT16+\.{HPUT16}}\.{HPUT16}(\|e\MG\index{xsize+\\{xsize}}\\{xsize}){}$;\5
\&{break};\6
\4\&{case} \\{b100}${}\OR\T{2}{}$:\5
${}\index{HPUT24+\.{HPUT24}}\.{HPUT24}(\|e\MG\index{size+\\{size}}\\{size}){}$;\5
${}\index{HPUT24+\.{HPUT24}}\.{HPUT24}(\|e\MG\index{xsize+\\{xsize}}\\{xsize}){}$;\5
\&{break};\6
\4\&{case} \\{b100}${}\OR\T{3}{}$:\5
${}\index{HPUT32+\.{HPUT32}}\.{HPUT32}(\|e\MG\index{size+\\{size}}\\{size}){}$;\5
${}\index{HPUT32+\.{HPUT32}}\.{HPUT32}(\|e\MG\index{xsize+\\{xsize}}\\{xsize}){}$;\5
\&{break};\6
\4\&{default}:\5
\.{QUIT}(\.{"Can't\ happen"});\5
\&{break};\6
\4${}\}{}$\2\6
${}\index{hput string+\\{hput\_string}}\\{hput\_string}(\|e\MG\index{file name+\\{file\_name}}\\{file\_name}){}$;\6
${}\.{DBGTAG}(\.{TAG}(\T{0},\39\|b),\39\index{hpos+\\{hpos}}\\{hpos}){}$;\5
${}\index{HPUT8+\.{HPUT8}}\.{HPUT8}(\.{TAG}(\T{0},\39\|b));{}$\6
\4${}\}{}$\2\7
\&{static} \&{void} \index{hput directory start+\\{hput\_directory\_start}}\\{hput\_directory\_start}(\&{void})\1\1\2\2\1\6
\4${}\{{}$\5
${}\.{DBG}(\index{DBGDIR+\.{DBGDIR}}\.{DBGDIR},\39\.{"Directory\ Section\\n}\)\.{"});{}$\6
${}\index{section no+\\{section\_no}}\\{section\_no}\K\T{0};{}$\6
${}\index{hpos+\\{hpos}}\\{hpos}\K\index{hstart+\\{hstart}}\\{hstart}\K\index{dir+\\{dir}}\\{dir}[\T{0}].\index{buffer+\\{buffer}}\\{buffer};{}$\6
${}\index{hend+\\{hend}}\\{hend}\K\index{hstart+\\{hstart}}\\{hstart}+\index{dir+\\{dir}}\\{dir}[\T{0}].\index{bsize+\\{bsize}}\\{bsize};{}$\6
\4${}\}{}$\2\7
\&{static} \&{void} \index{hput directory end+\\{hput\_directory\_end}}\\{hput\_directory\_end}(\&{void})\1\1\2\2\1\6
\4${}\{{}$\5
${}\index{dir+\\{dir}}\\{dir}[\T{0}].\index{size+\\{size}}\\{size}\K\index{hpos+\\{hpos}}\\{hpos}-\index{hstart+\\{hstart}}\\{hstart};{}$\6
${}\.{DBG}(\index{DBGDIR+\.{DBGDIR}}\.{DBGDIR},\39\.{"End\ Directory\ Secti}\)\.{on\ size=0x\%x\\n"},\39\index{dir+\\{dir}}\\{dir}[\T{0}].\index{size+\\{size}}\\{size});{}$\6
\4${}\}{}$\2\7
\&{static} \&{size\_t} \index{hput root+\\{hput\_root}}\\{hput\_root}(\&{void})\1\1\2\2\1\6
\4${}\{{}$\5
\&{uint8\_t} \index{buffer+\\{buffer}}\\{buffer}[\index{MAX TAG DISTANCE+\.{MAX\_TAG\_DISTANCE}}\.{MAX\_TAG\_DISTANCE}];\6
\&{size\_t} \|s;\7
${}\index{hpos+\\{hpos}}\\{hpos}\K\index{hstart+\\{hstart}}\\{hstart}\K\index{buffer+\\{buffer}}\\{buffer};{}$\6
${}\index{hend+\\{hend}}\\{hend}\K\index{hstart+\\{hstart}}\\{hstart}+\index{MAX TAG DISTANCE+\.{MAX\_TAG\_DISTANCE}}\.{MAX\_TAG\_DISTANCE};{}$\6
${}\index{dir+\\{dir}}\\{dir}[\T{0}].\index{section no+\\{section\_no}}\\{section\_no}\K\index{max section no+\\{max\_section\_no}}\\{max\_section\_no};{}$\6
${}\index{hput entry+\\{hput\_entry}}\\{hput\_entry}({\AND}\index{dir+\\{dir}}\\{dir}[\T{0}]);{}$\6
${}\|s\K\index{hput data+\\{hput\_data}}\\{hput\_data}(\T{0},\39\index{hstart+\\{hstart}}\\{hstart},\39\index{hpos+\\{hpos}}\\{hpos}-\index{hstart+\\{hstart}}\\{hstart});{}$\6
${}\.{DBG}(\index{DBGDIR+\.{DBGDIR}}\.{DBGDIR},\39\.{"Writing\ Root\ size="}\.{SIZE\_F}\.{"\\n"},\39\|s);{}$\6
\&{return} \|s;\6
\4${}\}{}$\2\7
\X279:\\{hcompress} function\X\7
\&{extern} \&{bool} \index{option compress+\\{option\_compress}}\\{option\_compress};\7
\&{void} \index{hput directory+\\{hput\_directory}}\\{hput\_directory}(\&{void})\1\1\2\2\1\6
\4${}\{{}$\5
\&{int} \|i;\7
\X296:set the file sizes for optional sections\X\6
\&{if} (\index{option compress+\\{option\_compress}}\\{option\_compress})\5
\1${}\{{}$\5
\index{hcompress+\\{hcompress}}\\{hcompress}(\T{1});\5
\index{hcompress+\\{hcompress}}\\{hcompress}(\T{2});\5
${}\}{}$\2\6
\index{hput directory start+\\{hput\_directory\_start}}\\{hput\_directory\_start}(\,);\6
\&{for} ${}(\|i\K\T{1};{}$ ${}\|i\Z\index{max section no+\\{max\_section\_no}}\\{max\_section\_no};{}$ ${}\|i\PP){}$\5
\1${}\{{}$\5
${}\index{dir+\\{dir}}\\{dir}[\|i].\index{pos+\\{pos}}\\{pos}\K\index{dir+\\{dir}}\\{dir}[\|i-\T{1}].\index{pos+\\{pos}}\\{pos}+\index{dir+\\{dir}}\\{dir}[\|i-\T{1}].\index{size+\\{size}}\\{size};{}$\6
${}\.{DBG}(\index{DBGDIR+\.{DBGDIR}}\.{DBGDIR},\39\.{"writing\ entry\ \%u\ at}\)\.{\ 0x\%"}\index{PRIx64+\\{PRIx64}}\\{PRIx64}\.{"\\n"},\39\|i,\39\index{dir+\\{dir}}\\{dir}[\|i].\index{pos+\\{pos}}\\{pos});{}$\6
${}\index{hput entry+\\{hput\_entry}}\\{hput\_entry}({\AND}\index{dir+\\{dir}}\\{dir}[\|i]);{}$\6
\4${}\}{}$\2\6
\index{hput directory end+\\{hput\_directory\_end}}\\{hput\_directory\_end}(\,);\6
\&{if} (\index{option compress+\\{option\_compress}}\\{option\_compress})\1\5
\index{hcompress+\\{hcompress}}\\{hcompress}(\T{0});\2\6
\4${}\}{}$\2
\Y
\fi

\M{298}


To conclude this section, here is the function that  adds the files that
are described in the directory entries 3 and above to a \HINT/ file in short format.
Where these files are found depends on the {\tt -g} option.
With that option given, the file names of the directory entrys are used unchanged.
Without that option, the files are found in the {\index{in name+\\{in\_name}}\\{in\_name}\tt .abs} and  {\index{in name+\\{in\_name}}\\{in\_name}\tt .rel}
directories, as described in section~\secref{absrel}.

\gdef\subcodetitle{Optional Sections}%
\putcode
\Y\B\4\X12:put functions\X${}\mathrel+\E{}$\6
\&{static} \&{void} \index{hput optional sections+\\{hput\_optional\_sections}}\\{hput\_optional\_sections}(\&{void})\1\1\2\2\1\6
\4${}\{{}$\5
\&{int} \|i;\7
${}\.{DBG}(\index{DBGDIR+\.{DBGDIR}}\.{DBGDIR},\39\.{"Optional\ Sections\\n}\)\.{"});{}$\6
\&{for} ${}(\|i\K\T{3};{}$ ${}\|i\Z\index{max section no+\\{max\_section\_no}}\\{max\_section\_no};{}$ ${}\|i\PP{}$)\6
\1${}\{{}$\5
\&{FILE} ${}{*}\|f;{}$\6
\&{size\_t} \index{fsize+\\{fsize}}\\{fsize};\6
\&{char} ${}{*}\index{file name+\\{file\_name}}\\{file\_name}\K\index{dir+\\{dir}}\\{dir}[\|i].\index{file name+\\{file\_name}}\\{file\_name};{}$\6
\&{int} \index{file name length+\\{file\_name\_length}}\\{file\_name\_length}${}\K\T{0};{}$\7
${}\.{DBG}(\index{DBGDIR+\.{DBGDIR}}\.{DBGDIR},\39\.{"file\ \%d:\ \%s\\n"},\39\index{dir+\\{dir}}\\{dir}[\|i].\index{section no+\\{section\_no}}\\{section\_no},\39\index{file name+\\{file\_name}}\\{file\_name});{}$\6
\&{if} ${}(\index{dir+\\{dir}}\\{dir}[\|i].\index{xsize+\\{xsize}}\\{xsize}\I\T{0}{}$)\1\6
${}\.{DBG}(\index{DBGDIR+\.{DBGDIR}}\.{DBGDIR},\39\.{"Compressing\ of\ auxi}\)\.{liary\ files\ currentl}\)\.{y\ not\ supported"});{}$\2\6
\X288:without {\tt -g} compute a local \\{file\_name}\X\6
${}\|f\K\index{fopen+\\{fopen}}\\{fopen}(\index{file name+\\{file\_name}}\\{file\_name},\39\.{"rb"});{}$\6
\&{if} ${}(\|f\E\NULL){}$\1\5
${}\.{QUIT}(\.{"Unable\ to\ read\ sect}\)\.{ion\ \%d,\ file\ \%s"},\39\index{dir+\\{dir}}\\{dir}[\|i].\index{section no+\\{section\_no}}\\{section\_no},\39\index{file name+\\{file\_name}}\\{file\_name});{}$\2\6
${}\index{fsize+\\{fsize}}\\{fsize}\K\T{0};{}$\6
\&{while} ${}(\R\index{feof+\\{feof}}\\{feof}(\|f){}$)\6
\1${}\{{}$\5
\&{size\_t} \|s${},{}$ \|t;\6
\&{char} ${}\index{buffer+\\{buffer}}\\{buffer}[\T{1}\LL\T{13}]{}$;\C{ 8kByte }\7
${}\|s\K\index{fread+\\{fread}}\\{fread}(\index{buffer+\\{buffer}}\\{buffer},\39\T{1},\39\T{1}\LL\T{13},\39\|f){}$;\6
${}\|t\K\index{fwrite+\\{fwrite}}\\{fwrite}(\index{buffer+\\{buffer}}\\{buffer},\39\T{1},\39\|s,\39\index{hout+\\{hout}}\\{hout});{}$\6
\&{if} ${}(\|s\I\|t){}$\1\5
${}\.{QUIT}(\.{"writing\ file\ \%s"},\39\index{file name+\\{file\_name}}\\{file\_name});{}$\2\6
${}\index{fsize+\\{fsize}}\\{fsize}\K\index{fsize+\\{fsize}}\\{fsize}+\|t;{}$\6
\4${}\}{}$\2\6
\index{fclose+\\{fclose}}\\{fclose}(\|f);\6
\&{if} ${}(\index{fsize+\\{fsize}}\\{fsize}\I\index{dir+\\{dir}}\\{dir}[\|i].\index{size+\\{size}}\\{size}){}$\1\5
${}\.{QUIT}(\.{"File\ size\ "}\.{SIZE\_F}\.{"\ does\ not\ match\ dir}\)\.{ectory\ size\ \%u"},\3{-1}\39\index{fsize+\\{fsize}}\\{fsize},\39\index{dir+\\{dir}}\\{dir}[\|i].\index{size+\\{size}}\\{size});{}$\2\6
\4${}\}{}$\2\6
\4${}\}{}$\2
\Y
\fi

\M{299}



\section{Definition Section}\index{definition section}
\label{defsection}
In a typical \HINT/ file, there are many things that are used over and over again.
For example the interword glue of a specific font or the indentation of
the first line of a paragraph. The definition section contains this information so that
it can be referenced in the content section by a simple reference number.
In addition there are a few parameters that guide the routines of \TeX.
An example is the ``above display skip'', which controls the amount of white space
inserted above a displayed equation, or the ``hyphen penalty'' that tells \TeX\
the ``\ae sthetic cost'' of ending a line with a hyphenated word. These parameters
also get their values in the definition section as explained in section~\secref{defaults}.


The most simple way to store these definitions is to store them in an array indexed by the
reference numbers.
To simplify the dynamic allocation of these arrays, the list of definitions
will always start with the list of maximum\index{maximum values} values: a list that contains
for each node type the maximum reference number used.

In the long format, the definition section starts with the keyword \.{definitions},
followed by the list of maximum values,
followed by the definitions proper.

When writing the short format, we start by positioning the output stream at the beginning of
the definition buffer and we end with recording the size of the definition section
in the directory.

\readcode
\Y\par
\par
\par
\par
\Y\B\4\X2:symbols\X${}\mathrel+\E{}$\6
\8\%\&{token} \index{DEFINITIONS+\ts{DEFINITIONS}}\ts{DEFINITIONS}\5\.{"definitions"}
\Y
\fi

\M{300}

\Y\B\4\X3:scanning rules\X${}\mathrel+\E{}$\6
${}\8\re{\vb{definitions}}{}$\ac\&{return} \index{DEFINITIONS+\ts{DEFINITIONS}}\ts{DEFINITIONS};\eac
\Y
\fi

\M{301}

\Y\B\4\X5:parsing rules\X${}\mathrel+\E{}$\6
\index{definition section+\nts{definition\_section}}\nts{definition\_section}: \1\1\5
\index{START+\ts{START}}\ts{START}\5
\index{DEFINITIONS+\ts{DEFINITIONS}}\ts{DEFINITIONS}\5
${}\{{}$\1\5
\index{hput definitions start+\\{hput\_definitions\_start}}\\{hput\_definitions\_start}(\,);\5
${}\}{}$\2\6
\index{max definitions+\nts{max\_definitions}}\nts{max\_definitions}\5
\index{definition list+\nts{definition\_list}}\nts{definition\_list}\6
\index{END+\ts{END}}\ts{END}\5
${}\{{}$\1\5
\index{hput definitions end+\\{hput\_definitions\_end}}\\{hput\_definitions\_end}(\,);\5
${}\}{}$\2;\2\2\7
\index{definition list+\nts{definition\_list}}\nts{definition\_list}:\5
\1\1\hbox to 0.5em{\hss${}\OR{}$}\5
\index{definition list+\nts{definition\_list}}\nts{definition\_list}\5
\index{def node+\nts{def\_node}}\nts{def\_node}\5
${}\{{}$\1\5
${}\index{REF+\.{REF}}\.{REF}(\.{\$2}.\|k,\39\.{\$2}.\|n);{}$\5
${}\}{}$\2;\2\2
\Y
\fi

\M{302}

\writecode
\Y\B\4\X19:write functions\X${}\mathrel+\E{}$\6
\&{void} \index{hwrite definitions start+\\{hwrite\_definitions\_start}}\\{hwrite\_definitions\_start}(\&{void})\1\1\2\2\1\6
\4${}\{{}$\5
${}\index{section no+\\{section\_no}}\\{section\_no}\K\T{1}{}$;\5
\index{hwritef+\\{hwritef}}\\{hwritef}(\.{"<definitions"});\6
\4${}\}{}$\2\7
\&{void} \index{hwrite definitions end+\\{hwrite\_definitions\_end}}\\{hwrite\_definitions\_end}(\&{void})\1\1\2\2\1\6
\4${}\{{}$\5
\index{hwritef+\\{hwritef}}\\{hwritef}(\.{"\\n>\\n"});\6
\4${}\}{}$\2
\Y
\fi

\M{303}



\Y\B\4\X16:get functions\X${}\mathrel+\E{}$\6
\&{void} \index{hget definition section+\\{hget\_definition\_section}}\\{hget\_definition\_section}(\&{void})\1\1\2\2\1\6
\4${}\{{}$\5
${}\.{DBG}(\index{DBGDEF+\.{DBGDEF}}\.{DBGDEF},\39\.{"Definitions\\n"});{}$\6
\index{hget section+\\{hget\_section}}\\{hget\_section}(\T{1});\6
\index{hwrite definitions start+\\{hwrite\_definitions\_start}}\\{hwrite\_definitions\_start}(\,);\6
${}\.{DBG}(\index{DBGDEF+\.{DBGDEF}}\.{DBGDEF},\39\.{"Reading\ list\ of\ max}\)\.{imum\ values\\n"});{}$\6
\index{hget max definitions+\\{hget\_max\_definitions}}\\{hget\_max\_definitions}(\,);\6
\X253:initialize definitions\X\6
\index{hwrite max definitions+\\{hwrite\_max\_definitions}}\\{hwrite\_max\_definitions}(\,);\6
${}\.{DBG}(\index{DBGDEF+\.{DBGDEF}}\.{DBGDEF},\39\.{"Reading\ list\ of\ def}\)\.{initions\\n"});{}$\6
\&{while} ${}(\index{hpos+\\{hpos}}\\{hpos}<\index{hend+\\{hend}}\\{hend}{}$)\6
\1${}\{{}$\5
\index{ref t+\&{ref\_t}}\&{ref\_t} \index{df+\\{df}}\\{df};\5
${}\index{hget def node+\\{hget\_def\_node}}\\{hget\_def\_node}({\AND}\index{df+\\{df}}\\{df});{}$\6
\&{if} ${}(\index{max fixed+\\{max\_fixed}}\\{max\_fixed}[\index{df+\\{df}}\\{df}.\|k]>\index{max default+\\{max\_default}}\\{max\_default}[\index{df+\\{df}}\\{df}.\|k]){}$\1\5
${}\.{QUIT}(\.{"Definitions\ for\ kin}\)\.{d\ \%s\ not\ supported"},\39\index{definition name+\\{definition\_name}}\\{definition\_name}[\index{df+\\{df}}\\{df}.\|k]);{}$\2\6
\&{if} ${}(\index{df+\\{df}}\\{df}.\|n>\\{max\_ref}[\index{df+\\{df}}\\{df}.\|k]\V\index{df+\\{df}}\\{df}.\|n\Z\index{max fixed+\\{max\_fixed}}\\{max\_fixed}[\index{df+\\{df}}\\{df}.\|k]){}$\1\5
${}\.{QUIT}(\.{"Definition\ \%d\ for\ \%}\)\.{s\ out\ of\ range\ [\%d\ -}\)\.{\ \%d]"},\3{-1}\39\index{df+\\{df}}\\{df}.\|n,\39\index{definition name+\\{definition\_name}}\\{definition\_name}[\index{df+\\{df}}\\{df}.\|k],\39\index{max fixed+\\{max\_fixed}}\\{max\_fixed}[\index{df+\\{df}}\\{df}.\|k]+\T{1},\39\\{max\_ref}[\index{df+\\{df}}\\{df}.\|k]);{}$\2\6
\4${}\}{}$\2\6
\index{hwrite definitions end+\\{hwrite\_definitions\_end}}\\{hwrite\_definitions\_end}(\,);\6
\4${}\}{}$\2
\Y
\fi

\M{304}

\putcode
\Y\B\4\X12:put functions\X${}\mathrel+\E{}$\6
\&{void} \index{hput definitions start+\\{hput\_definitions\_start}}\\{hput\_definitions\_start}(\&{void})\1\1\2\2\1\6
\4${}\{{}$\5
${}\.{DBG}(\index{DBGDEF+\.{DBGDEF}}\.{DBGDEF},\39\.{"Definition\ Section\\}\)\.{n"});{}$\6
${}\index{section no+\\{section\_no}}\\{section\_no}\K\T{1};{}$\6
${}\index{hpos+\\{hpos}}\\{hpos}\K\index{hstart+\\{hstart}}\\{hstart}\K\index{dir+\\{dir}}\\{dir}[\T{1}].\index{buffer+\\{buffer}}\\{buffer};{}$\6
${}\index{hend+\\{hend}}\\{hend}\K\index{hstart+\\{hstart}}\\{hstart}+\index{dir+\\{dir}}\\{dir}[\T{1}].\index{bsize+\\{bsize}}\\{bsize};{}$\6
\4${}\}{}$\2\7
\&{void} \index{hput definitions end+\\{hput\_definitions\_end}}\\{hput\_definitions\_end}(\&{void})\1\1\2\2\1\6
\4${}\{{}$\5
${}\index{dir+\\{dir}}\\{dir}[\T{1}].\index{size+\\{size}}\\{size}\K\index{hpos+\\{hpos}}\\{hpos}-\index{hstart+\\{hstart}}\\{hstart};{}$\6
${}\.{DBG}(\index{DBGDEF+\.{DBGDEF}}\.{DBGDEF},\39\.{"End\ Definition\ Sect}\)\.{ion\ size=0x\%x\\n"},\39\index{dir+\\{dir}}\\{dir}[\T{1}].\index{size+\\{size}}\\{size});{}$\6
\4${}\}{}$\2
\Y
\fi

\M{305}
\gdef\codetitle{Definitions}
\hascode
\subsection{Maximum Values}\index{maximum values}
To help implementations allocating the right amount of memory for the
definitions, the definition section starts with a list of maximum
values.  For each kind of node, we store the maximum valid reference
number in the array \\{max\_ref} which is indexed by the kind values.
For a reference number \|n and kind value $k$ we have
$0\le n\le \\{max\_ref}[\|k]$.
To make sure that a hint file without any definitions
will work, some definitions have default values.
The initialization of default and maximum values is described
in section~\secref{defaults}. The maximum
reference number that has a default value is stored in the array
\index{max default+\\{max\_default}}\\{max\_default}.  We have $-1 \le \index{max default+\\{max\_default}}\\{max\_default}[\|k] \le \\{max\_ref}[\|k] \le
255$.  Specifying maximum values that are lower than the
default\index{default value} values is not allowed in the short
format; in the long format, lower values are silently ignored.  Some
default values are permanently fixed; for example the zero glue with
reference number \index{zero skip no+\\{zero\_skip\_no}}\\{zero\_skip\_no} must never change. The array
\index{max fixed+\\{max\_fixed}}\\{max\_fixed} stores the maximum reference number that has a fixed value for a
given kind.  Definitions with reference numbers lower than the
corresponding \index{max fixed+\\{max\_fixed}}\\{max\_fixed}[\|k] number are disallowed. Usually we have
$-1 \le \index{max fixed+\\{max\_fixed}}\\{max\_fixed}[\|k] \le \index{max default+\\{max\_default}}\\{max\_default}[\|k] $, but if for a kind value
$k$ no definitions, and hence no maximum values are allowed, we set
$\index{max fixed+\\{max\_fixed}}\\{max\_fixed}[\|k]=\T{\^100}>\index{max default+\\{max\_default}}\\{max\_default}[\|k] $.


We use the \\{max\_ref} array whenever we find a
reference number in the input to check if it is within the proper range.

\Y\B\4\X305:debug macros\X${}\E{}$\6
\8\#\&{define} $\index{REF RNG+\.{REF\_RNG}}\.{REF\_RNG}(\|K,\39\|N)$ \&{if} ${}((\&{int})(\|N)>\\{max\_ref}[\|K])$ $\.{QUIT}(\.{"Reference\ \%d\ to\ \%s\ }\)\.{out\ of\ range\ [0\ -\ \%d}\)\.{]"},\39(\|N),\39\index{definition name+\\{definition\_name}}\\{definition\_name}[\|K],\39\\{max\_ref}[\|K])$
\A377.
\U336.\Y
\fi

\M{306}

In the long format file, the list of maximum values starts with
``\.{<max }'', then follow pairs of keywords and numbers like
``\.{<glue 57>}'', and it ends with ``\.{>}''.  In the short format,
we start the list of maximums with a \index{list kind+\\{list\_kind}}\\{list\_kind} tag and end it with
a \index{list kind+\\{list\_kind}}\\{list\_kind} tag.  Each maximum value is preceded and followed by a
tag byte with the appropriate kind value. The info value is always 1
because at present, reference numbers---and therefore maximum
values---are restricted to the range 0 to 255 in order to fit into a
single byte. Other info values are reserved for future extensions.
After reading the maximum values, we initialize the data structures for
the defininitions.


\readcode
\Y\par
\par
\par
\par
\Y\B\4\X2:symbols\X${}\mathrel+\E{}$\6
\8\%\&{token} \index{MAX+\ts{MAX}}\ts{MAX}\5\.{"max"}
\Y
\fi

\M{307}

\Y\B\4\X3:scanning rules\X${}\mathrel+\E{}$\6
${}\8\re{\vb{max}}{}$\ac\&{return} \index{MAX+\ts{MAX}}\ts{MAX};\eac
\Y
\fi

\M{308}
\Y\B\4\X5:parsing rules\X${}\mathrel+\E{}$\6
\index{max definitions+\nts{max\_definitions}}\nts{max\_definitions}: \1\1\5
\index{START+\ts{START}}\ts{START}\5
\index{MAX+\ts{MAX}}\ts{MAX}\5
\index{max list+\nts{max\_list}}\nts{max\_list}\5
\index{END+\ts{END}}\ts{END}\3{-1}\5
${}\{{}$\1\5
\X253:initialize definitions\X\6
\index{hput max definitions+\\{hput\_max\_definitions}}\\{hput\_max\_definitions}(\,);\5
${}\}{}$\2;\2\2\7
\index{max list+\nts{max\_list}}\nts{max\_list}:\5
\1\1\hbox to 0.5em{\hss${}\OR{}$}\5
\index{max list+\nts{max\_list}}\nts{max\_list}\5
\index{START+\ts{START}}\ts{START}\5
\index{max value+\nts{max\_value}}\nts{max\_value}\5
\index{END+\ts{END}}\ts{END};\2\2\7
\index{max value+\nts{max\_value}}\nts{max\_value}: \1\1\5
\index{FONT+\ts{FONT}}\ts{FONT}\5
\index{UNSIGNED+\ts{UNSIGNED}}\ts{UNSIGNED}\5
${}\{{}$\1\5
${}\index{hset max+\\{hset\_max}}\\{hset\_max}(\index{font kind+\\{font\_kind}}\\{font\_kind},\39\.{\$2});{}$\5
${}\}{}$\2\6
\4\hbox to 0.5em{\hss${}\OR{}$}\5
\index{INTEGER+\ts{INTEGER}}\ts{INTEGER}\5
\index{UNSIGNED+\ts{UNSIGNED}}\ts{UNSIGNED}\5
${}\{{}$\1\5
${}\index{hset max+\\{hset\_max}}\\{hset\_max}(\index{int kind+\\{int\_kind}}\\{int\_kind},\39\.{\$2});{}$\5
${}\}{}$\2\6
\4\hbox to 0.5em{\hss${}\OR{}$}\5
\index{DIMEN+\ts{DIMEN}}\ts{DIMEN}\5
\index{UNSIGNED+\ts{UNSIGNED}}\ts{UNSIGNED}\5
${}\{{}$\1\5
${}\index{hset max+\\{hset\_max}}\\{hset\_max}(\index{dimen kind+\\{dimen\_kind}}\\{dimen\_kind},\39\.{\$2});{}$\5
${}\}{}$\2\6
\4\hbox to 0.5em{\hss${}\OR{}$}\5
\index{LIGATURE+\ts{LIGATURE}}\ts{LIGATURE}\5
\index{UNSIGNED+\ts{UNSIGNED}}\ts{UNSIGNED}\5
${}\{{}$\1\5
${}\index{hset max+\\{hset\_max}}\\{hset\_max}(\index{ligature kind+\\{ligature\_kind}}\\{ligature\_kind},\39\.{\$2});{}$\5
${}\}{}$\2\6
\4\hbox to 0.5em{\hss${}\OR{}$}\5
\index{HYPHEN+\ts{HYPHEN}}\ts{HYPHEN}\5
\index{UNSIGNED+\ts{UNSIGNED}}\ts{UNSIGNED}\5
${}\{{}$\1\5
${}\index{hset max+\\{hset\_max}}\\{hset\_max}(\index{hyphen kind+\\{hyphen\_kind}}\\{hyphen\_kind},\39\.{\$2});{}$\5
${}\}{}$\2\6
\4\hbox to 0.5em{\hss${}\OR{}$}\5
\index{GLUE+\ts{GLUE}}\ts{GLUE}\5
\index{UNSIGNED+\ts{UNSIGNED}}\ts{UNSIGNED}\5
${}\{{}$\1\5
${}\index{hset max+\\{hset\_max}}\\{hset\_max}(\index{glue kind+\\{glue\_kind}}\\{glue\_kind},\39\.{\$2});{}$\5
${}\}{}$\2\6
\4\hbox to 0.5em{\hss${}\OR{}$}\5
\index{LANGUAGE+\ts{LANGUAGE}}\ts{LANGUAGE}\5
\index{UNSIGNED+\ts{UNSIGNED}}\ts{UNSIGNED}\5
${}\{{}$\1\5
${}\index{hset max+\\{hset\_max}}\\{hset\_max}(\index{language kind+\\{language\_kind}}\\{language\_kind},\39\.{\$2});{}$\5
${}\}{}$\2\6
\4\hbox to 0.5em{\hss${}\OR{}$}\5
\index{RULE+\ts{RULE}}\ts{RULE}\5
\index{UNSIGNED+\ts{UNSIGNED}}\ts{UNSIGNED}\5
${}\{{}$\1\5
${}\index{hset max+\\{hset\_max}}\\{hset\_max}(\index{rule kind+\\{rule\_kind}}\\{rule\_kind},\39\.{\$2});{}$\5
${}\}{}$\2\6
\4\hbox to 0.5em{\hss${}\OR{}$}\5
\index{IMAGE+\ts{IMAGE}}\ts{IMAGE}\5
\index{UNSIGNED+\ts{UNSIGNED}}\ts{UNSIGNED}\5
${}\{{}$\1\5
${}\index{hset max+\\{hset\_max}}\\{hset\_max}(\index{image kind+\\{image\_kind}}\\{image\_kind},\39\.{\$2});{}$\5
${}\}{}$\2\6
\4\hbox to 0.5em{\hss${}\OR{}$}\5
\index{LEADERS+\ts{LEADERS}}\ts{LEADERS}\5
\index{UNSIGNED+\ts{UNSIGNED}}\ts{UNSIGNED}\5
${}\{{}$\1\5
${}\index{hset max+\\{hset\_max}}\\{hset\_max}(\index{leaders kind+\\{leaders\_kind}}\\{leaders\_kind},\39\.{\$2});{}$\5
${}\}{}$\2\6
\4\hbox to 0.5em{\hss${}\OR{}$}\5
\index{BASELINE+\ts{BASELINE}}\ts{BASELINE}\5
\index{UNSIGNED+\ts{UNSIGNED}}\ts{UNSIGNED}\5
${}\{{}$\1\5
${}\index{hset max+\\{hset\_max}}\\{hset\_max}(\index{baseline kind+\\{baseline\_kind}}\\{baseline\_kind},\39\.{\$2});{}$\5
${}\}{}$\2\6
\4\hbox to 0.5em{\hss${}\OR{}$}\5
\index{XDIMEN+\ts{XDIMEN}}\ts{XDIMEN}\5
\index{UNSIGNED+\ts{UNSIGNED}}\ts{UNSIGNED}\5
${}\{{}$\1\5
${}\index{hset max+\\{hset\_max}}\\{hset\_max}(\index{xdimen kind+\\{xdimen\_kind}}\\{xdimen\_kind},\39\.{\$2});{}$\5
${}\}{}$\2\6
\4\hbox to 0.5em{\hss${}\OR{}$}\5
\index{PARAM+\ts{PARAM}}\ts{PARAM}\5
\index{UNSIGNED+\ts{UNSIGNED}}\ts{UNSIGNED}\5
${}\{{}$\1\5
${}\index{hset max+\\{hset\_max}}\\{hset\_max}(\index{param kind+\\{param\_kind}}\\{param\_kind},\39\.{\$2});{}$\5
${}\}{}$\2\6
\4\hbox to 0.5em{\hss${}\OR{}$}\5
\index{STREAMDEF+\ts{STREAMDEF}}\ts{STREAMDEF}\5
\index{UNSIGNED+\ts{UNSIGNED}}\ts{UNSIGNED}\5
${}\{{}$\1\5
${}\index{hset max+\\{hset\_max}}\\{hset\_max}(\index{stream kind+\\{stream\_kind}}\\{stream\_kind},\39\.{\$2});{}$\5
${}\}{}$\2\6
\4\hbox to 0.5em{\hss${}\OR{}$}\5
\index{PAGE+\ts{PAGE}}\ts{PAGE}\5
\index{UNSIGNED+\ts{UNSIGNED}}\ts{UNSIGNED}\5
${}\{{}$\1\5
${}\index{hset max+\\{hset\_max}}\\{hset\_max}(\index{page kind+\\{page\_kind}}\\{page\_kind},\39\.{\$2});{}$\5
${}\}{}$\2\6
\4\hbox to 0.5em{\hss${}\OR{}$}\5
\index{RANGE+\ts{RANGE}}\ts{RANGE}\5
\index{UNSIGNED+\ts{UNSIGNED}}\ts{UNSIGNED}\5
${}\{{}$\1\5
${}\index{hset max+\\{hset\_max}}\\{hset\_max}(\index{range kind+\\{range\_kind}}\\{range\_kind},\39\.{\$2});{}$\5
${}\}{}$\2;\2\2
\Y
\fi

\M{309}

\Y\B\4\X309:parsing functions\X${}\E{}$\6
\&{void} \index{hset max+\\{hset\_max}}\\{hset\_max}(\index{kind t+\&{kind\_t}}\&{kind\_t} \|k${},\39{}$\&{int} \|n)\1\1\2\2\1\6
\4${}\{{}$\5
${}\.{DBG}(\index{DBGDEF+\.{DBGDEF}}\.{DBGDEF},\39\.{"Setting\ max\ \%s\ to\ \%}\)\.{d\\n"},\39\index{definition name+\\{definition\_name}}\\{definition\_name}[\|k],\39\|n);{}$\6
${}\.{RNG}(\.{"Maximum"},\39\|n,\39\index{max fixed+\\{max\_fixed}}\\{max\_fixed}[\|k]+\T{1},\39\T{\^FF});{}$\6
\&{if} ${}(\|n>\\{max\_ref}[\|k]){}$\5
\1${}\{{}$\5
${}\\{max\_ref}[\|k]\K\|n;{}$\6
\4${}\}{}$\2\6
\4${}\}{}$\2
\As318\ET378.
\U437.\Y
\fi

\M{310}

\writecode
\Y\B\4\X19:write functions\X${}\mathrel+\E{}$\6
\&{void} \index{hwrite max definitions+\\{hwrite\_max\_definitions}}\\{hwrite\_max\_definitions}(\&{void})\1\1\2\2\1\6
\4${}\{{}$\5
\index{kind t+\&{kind\_t}}\&{kind\_t} \|k;\7
\index{hwrite start+\\{hwrite\_start}}\\{hwrite\_start}(\,);\5
\index{hwritef+\\{hwritef}}\\{hwritef}(\.{"max"});\6
\&{for} ${}(\|k\K\T{0};{}$ ${}\|k<\T{32};{}$ ${}\|k\PP){}$\1\6
\&{if} ${}(\\{max\_ref}[\|k]>\index{max default+\\{max\_default}}\\{max\_default}[\|k]{}$)\6
\1${}\{{}$\5
\index{hwrite start+\\{hwrite\_start}}\\{hwrite\_start}(\,);\6
${}\index{hwritef+\\{hwritef}}\\{hwritef}(\.{"\%s\ \%d"},\39\index{definition name+\\{definition\_name}}\\{definition\_name}[\|k],\39\\{max\_ref}[\|k]);{}$\6
\index{hwrite end+\\{hwrite\_end}}\\{hwrite\_end}(\,);\6
\4${}\}{}$\2\2\6
\index{hwrite end+\\{hwrite\_end}}\\{hwrite\_end}(\,);\6
\4${}\}{}$\2
\Y
\fi

\M{311}

\getcode
\Y\B\4\X263:get file functions\X${}\mathrel+\E{}$\6
\&{void} \index{hget max definitions+\\{hget\_max\_definitions}}\\{hget\_max\_definitions}(\&{void})\1\1\2\2\1\6
\4${}\{{}$\5
\index{kind t+\&{kind\_t}}\&{kind\_t} \|k;\7
\X14:read the start byte \|a\X\6
\&{if} ${}(\|a\I\.{TAG}(\index{list kind+\\{list\_kind}}\\{list\_kind},\39\T{0})){}$\1\5
\.{QUIT}(\.{"Start\ of\ maximum\ li}\)\.{st\ expected"});\2\6
\&{for} ${}(\|k\K\T{0};{}$ ${}\|k<\T{32};{}$ ${}\|k\PP){}$\1\5
${}\\{max\_ref}[\|k]\K\index{max default+\\{max\_default}}\\{max\_default}[\|k];{}$\2\6
\&{while} (\\{true})\6
\1${}\{{}$\5
\&{uint8\_t} \|n;\7
\&{if} ${}(\index{hpos+\\{hpos}}\\{hpos}\G\index{hend+\\{hend}}\\{hend}){}$\1\5
\.{QUIT}(\.{"Unexpected\ end\ of\ m}\)\.{aximum\ list"});\2\6
${}\\{node\_pos}\K\index{hpos+\\{hpos}}\\{hpos}-\index{hstart+\\{hstart}}\\{hstart};{}$\6
\index{HGETTAG+\.{HGETTAG}}\.{HGETTAG}(\|a);\6
\&{if} ${}(\index{KIND+\.{KIND}}\.{KIND}(\|a)\E\index{list kind+\\{list\_kind}}\\{list\_kind}){}$\1\5
\&{break};\2\6
\&{if} ${}(\index{INFO+\.{INFO}}\.{INFO}(\|a)\I\T{1}){}$\1\5
${}\.{QUIT}(\.{"Maximum\ info\ \%d\ not}\)\.{\ supported"},\39\index{INFO+\.{INFO}}\.{INFO}(\|a));{}$\2\6
${}\|k\K\index{KIND+\.{KIND}}\.{KIND}(\|a);{}$\6
\&{if} ${}(\index{max fixed+\\{max\_fixed}}\\{max\_fixed}[\|k]>\index{max default+\\{max\_default}}\\{max\_default}[\|k]){}$\1\5
${}\.{QUIT}(\.{"Maximum\ value\ for\ k}\)\.{ind\ \%s\ not\ supported}\)\.{"},\39\index{definition name+\\{definition\_name}}\\{definition\_name}[\|k]);{}$\2\6
${}\|n\K\index{HGET8+\.{HGET8}}\.{HGET8};{}$\6
${}\.{RNG}(\.{"Maximum\ number"},\39\|n,\39\\{max\_ref}[\|k],\39\T{\^FF});{}$\6
${}\\{max\_ref}[\|k]\K\|n;{}$\6
${}\.{DBG}(\index{DBGDEF+\.{DBGDEF}}\.{DBGDEF},\39\.{"max(\%s)\ =\ \%d\\n"},\39\index{definition name+\\{definition\_name}}\\{definition\_name}[\|k],\39\\{max\_ref}[\|k]);{}$\6
\X15:read and check the end byte \|z\X\6
\4${}\}{}$\2\6
\&{if} ${}(\index{INFO+\.{INFO}}\.{INFO}(\|a)\I\T{0}){}$\1\5
${}\.{QUIT}(\.{"End\ of\ maximum\ list}\)\.{\ with\ info\ \%d"},\39\index{INFO+\.{INFO}}\.{INFO}(\|a));{}$\2\6
\4${}\}{}$\2
\Y
\fi

\M{312}

\putcode

\Y\B\4\X12:put functions\X${}\mathrel+\E{}$\6
\&{void} \index{hput max definitions+\\{hput\_max\_definitions}}\\{hput\_max\_definitions}(\&{void})\1\1\2\2\1\6
\4${}\{{}$\5
\index{kind t+\&{kind\_t}}\&{kind\_t} \|k;\7
${}\.{DBG}(\index{DBGDEF+\.{DBGDEF}}\.{DBGDEF},\39\.{"Max\ Definitions\ Beg}\)\.{in\\n"});{}$\6
${}\index{HPUTTAG+\.{HPUTTAG}}\.{HPUTTAG}(\index{list kind+\\{list\_kind}}\\{list\_kind},\39\T{0});{}$\6
\&{for} ${}(\|k\K\T{0};{}$ ${}\|k<\T{32};{}$ ${}\|k\PP){}$\1\6
\&{if} ${}(\\{max\_ref}[\|k]>\index{max default+\\{max\_default}}\\{max\_default}[\|k]){}$\5
\1${}\{{}$\5
${}\.{DBG}(\index{DBGDEF+\.{DBGDEF}}\.{DBGDEF},\39\.{"max(\%s)\ =\ \%d\\n"},\39\index{definition name+\\{definition\_name}}\\{definition\_name}[\|k],\39\\{max\_ref}[\|k]);{}$\6
${}\index{HPUTTAG+\.{HPUTTAG}}\.{HPUTTAG}(\|k,\39\T{1});{}$\6
\index{HPUT8+\.{HPUT8}}\.{HPUT8}(\\{max\_ref}[\|k]);\6
${}\index{HPUTTAG+\.{HPUTTAG}}\.{HPUTTAG}(\|k,\39\T{1});{}$\6
\4${}\}{}$\2\2\6
${}\index{HPUTTAG+\.{HPUTTAG}}\.{HPUTTAG}(\index{list kind+\\{list\_kind}}\\{list\_kind},\39\T{0});{}$\6
${}\.{DBG}(\index{DBGDEF+\.{DBGDEF}}\.{DBGDEF},\39\.{"Max\ Definitions\ End}\)\.{\\n"});{}$\6
\4${}\}{}$\2
\Y
\fi

\M{313}


\subsection{Definitions}\label{definitions}
A definition\index{definition section} associates a reference number
with a content node.  Here is an example: A glue definition associates
a glue number, for example 71, with a glue specification. In the long
format this might look like ``{\tt \.{<}glue *71 4pt plus 5pt minus
0.5pt\.{>}}'' which makes glue number 71 refer to a 4pt glue with a
stretchability of 5pt and a shrinkability of 0.5pt.
Such a definition differs from a normal content node just by an extra
byte value immediately following the keyword respectively  start byte.

Whenever we need this glue in the content section, we can say
``{\tt \.{<}glue *71\.{>}}''.  Because we restrict the number of definitions
for every node type to at most 256, a single byte is sufficient to
store the reference number.  The \.{shrink} and \index{stretch+\.{stretch}}\.{stretch} programs
will, however, not bother to store the definitions. Instead they will
write them in the new format immediately to the output.

The parser will handle definitions in any order, but the order is relevant
if a definiton references another definition, and of course,
it never does any harm to present definitions in a systematic way.

As a rule, the definition of a reference must always preceed the
use of that reference. While this is allways the case for
references in the content section, it restricts the use of
references inside the definition section.

The definitions for integers, dimensions, extended dimensions,
languages, rules, ligatures, and images are ``simple''.
They never contain references and so it is always possible to list them first.
The definition of glues may contain extended definitions,
the definitions of baselines may reference glue nodes, and
the definitions of parameter lists contain definitions of integers, dimensions,
and glues. So these definitions should follow in this order.

The definitions of leaders and hyphens allow boxes.
While these boxes are usually
quite simple, they may contain arbitrary references---including again
references to leaders and hyphens.  So, at least in principle,
they might impose complex (or even unsatisfiable) restrictions
on the order of those definitions.

The definitions of fonts contain not only ``simple'' definitions
but also the definitions of interword glues and hyphens
introducing additional ordering restrictions.
The definition of hyphens regularly contain glyphs which in turn reference
a font---typically the font that gets just defined.
Therefore we relax the define before use policy for glyphs:
Glyphs may reference a font before the font is defined.

The definitions of page templates contain lists of arbitrary content
nodes, and while the boxes inside leaders or hyphens tend to be simple,
the content of page templates is often quite complex.
Page templates are propably the source of most ordering restrictions.
Placing page templates towards the end of the list of definitions
might be a good idea.

A special case are stream definitions. These occur only as part of
the corresponding page template definition and are listed at its end.
So references to them will occur in the page template always before their
definition.

Finally, the definitions of page ranges always reference a page template
and they should come last.

To avoid complex dependencies, an application can allways choose not to
use references in the definition section. There are only two types of
nodes where references can not be avoided: glyphs nodes which refere to fonts
and language nodes which might occur in boxes and page templates.
Possible ordering restrictions can be satisfied if languages are defined first
and fonts second.

To check the define before use policy, we use an array of bitvectors.
Where we have for every reference number \|N and every kind \|K a single
bit which is set if and only if the corresponding reference is defined.

\Y\B\4\X313:definition checks\X${}\E{}$\6
\&{uint32\_t} ${}\index{definition bits+\\{definition\_bits}}\\{definition\_bits}[\T{\^100}/\T{32}][\T{32}]\K\{\{\T{0}\}\};{}$\6
\8\#\&{define} ${}\index{SET DBIT+\.{SET\_DBIT}}\.{SET\_DBIT}(\|N,\39\|K)\5(\index{definition bits+\\{definition\_bits}}\\{definition\_bits}[\|N/\T{32}][\|K]\MRL{{\OR}{\K}}(\T{1}\LL((\|N)\AND(\T{32}-\T{1})))){}$\6
\8\#\&{define} ${}\index{GET DBIT+\.{GET\_DBIT}}\.{GET\_DBIT}(\|N,\39\|K)\5((\index{definition bits+\\{definition\_bits}}\\{definition\_bits}[\|N/\T{32}][\|K]\GG((\|N)\AND(\T{32}-\T{1})))\AND\T{1}){}$\6
\8\#\&{define} $\index{DEF+\.{DEF}}\.{DEF}(\|D,\39\|K,\39\|N){}$ (\|D)${}.\|k\K\|K{}$;\5
${}(\|D).\|n\K(\|N){}$;\5
${}\index{SET DBIT+\.{SET\_DBIT}}\.{SET\_DBIT}((\|D).\|n,\39(\|D).\|k);{}$\6
${}\.{DBG}(\index{DBGDEF+\.{DBGDEF}}\.{DBGDEF},\39\.{"Defining\ \%s\ \%d\\n"},\39\index{definition name+\\{definition\_name}}\\{definition\_name}[(\|D).\|k],\39(\|D).\|n);{}$\6
${}\.{RNG}(\.{"Definition"},\39(\|D).\|n,\39\index{max fixed+\\{max\_fixed}}\\{max\_fixed}[(\|D).\|k]+\T{1},\39\\{max\_ref}[(\|D).\|k]);$ \6
\8\#\&{define} $\index{REF+\.{REF}}\.{REF}(\|K,\39\|N){}$ \index{REF RNG+\.{REF\_RNG}}\.{REF\_RNG} ${}(\|K,\39\|N);$ \&{if} ${}(\R\index{GET DBIT+\.{GET\_DBIT}}\.{GET\_DBIT}(\|N,\39\|K))$ $\.{QUIT}(\.{"Reference\ \%d\ to\ \%s\ }\)\.{before\ definition"},\39(\|N),\39\index{definition name+\\{definition\_name}}\\{definition\_name}[\|K])$
\Us437\ET439.\Y
\fi

\M{314}

\Y\B\4\X253:initialize definitions\X${}\mathrel+\E{}$\6
$\index{definition bits+\\{definition\_bits}}\\{definition\_bits}[\T{0}][\index{int kind+\\{int\_kind}}\\{int\_kind}]\K(\T{1}\LL(\index{MAX INT DEFAULT+\.{MAX\_INT\_DEFAULT}}\.{MAX\_INT\_DEFAULT}+\T{1}))-\T{1};{}$\6
${}\index{definition bits+\\{definition\_bits}}\\{definition\_bits}[\T{0}][\index{dimen kind+\\{dimen\_kind}}\\{dimen\_kind}]\K(\T{1}\LL(\index{MAX DIMEN DEFAULT+\.{MAX\_DIMEN\_DEFAULT}}\.{MAX\_DIMEN\_DEFAULT}+\T{1}))-\T{1};{}$\6
${}\index{definition bits+\\{definition\_bits}}\\{definition\_bits}[\T{0}][\index{xdimen kind+\\{xdimen\_kind}}\\{xdimen\_kind}]\K(\T{1}\LL(\index{MAX XDIMEN DEFAULT+\.{MAX\_XDIMEN\_DEFAULT}}\.{MAX\_XDIMEN\_DEFAULT}+\T{1}))-\T{1};{}$\6
${}\index{definition bits+\\{definition\_bits}}\\{definition\_bits}[\T{0}][\index{glue kind+\\{glue\_kind}}\\{glue\_kind}]\K(\T{1}\LL(\index{MAX GLUE DEFAULT+\.{MAX\_GLUE\_DEFAULT}}\.{MAX\_GLUE\_DEFAULT}+\T{1}))-\T{1};{}$\6
${}\index{definition bits+\\{definition\_bits}}\\{definition\_bits}[\T{0}][\index{baseline kind+\\{baseline\_kind}}\\{baseline\_kind}]\K(\T{1}\LL(\index{MAX BASELINE DEFAULT+\.{MAX\_BASELINE\_DEFAULT}}\.{MAX\_BASELINE\_DEFAULT}+\T{1}))-\T{1};{}$\6
${}\index{definition bits+\\{definition\_bits}}\\{definition\_bits}[\T{0}][\index{page kind+\\{page\_kind}}\\{page\_kind}]\K(\T{1}\LL(\index{MAX PAGE DEFAULT+\.{MAX\_PAGE\_DEFAULT}}\.{MAX\_PAGE\_DEFAULT}+\T{1}))-\T{1};{}$\6
${}\index{definition bits+\\{definition\_bits}}\\{definition\_bits}[\T{0}][\index{stream kind+\\{stream\_kind}}\\{stream\_kind}]\K(\T{1}\LL(\index{MAX STREAM DEFAULT+\.{MAX\_STREAM\_DEFAULT}}\.{MAX\_STREAM\_DEFAULT}+\T{1}))-\T{1};{}$\6
${}\index{definition bits+\\{definition\_bits}}\\{definition\_bits}[\T{0}][\index{range kind+\\{range\_kind}}\\{range\_kind}]\K(\T{1}\LL(\index{MAX RANGE DEFAULT+\.{MAX\_RANGE\_DEFAULT}}\.{MAX\_RANGE\_DEFAULT}+\T{1}))-\T{1}{}$;
\Y
\fi

\M{315}

\goodbreak
\vbox{\readcode\vskip -\baselineskip\putcode}


\Y\par
\Y\B\4\X2:symbols\X${}\mathrel+\E{}$\6
\8\%\index{type+\&{type}}\&{type} $<$ \index{rf+\\{rf}}\\{rf} $>$ \index{def node+\nts{def\_node}}\nts{def\_node}
\Y
\fi

\M{316}

\Y\B\4\X5:parsing rules\X${}\mathrel+\E{}$\6
\index{def node+\nts{def\_node}}\nts{def\_node}: \1\1\5
\index{start+\nts{start}}\nts{start}\5
\index{FONT+\ts{FONT}}\ts{FONT}\5
\index{ref+\nts{ref}}\nts{ref}\5
\index{font+\nts{font}}\nts{font}\5
\index{END+\ts{END}}\ts{END}\3{-1}\5
${}\{{}$\1\5
${}\index{DEF+\.{DEF}}\.{DEF}(\.{\$\$},\39\index{font kind+\\{font\_kind}}\\{font\_kind},\39\.{\$3}){}$;\5
${}\index{hput tags+\\{hput\_tags}}\\{hput\_tags}(\.{\$1},\39\.{\$4}){}$;\5
${}\}{}$\2\6
\4\hbox to 0.5em{\hss${}\OR{}$}\5
\index{start+\nts{start}}\nts{start}\5
\index{INTEGER+\ts{INTEGER}}\ts{INTEGER}\5
\index{ref+\nts{ref}}\nts{ref}\5
\index{integer+\nts{integer}}\nts{integer}\5
\index{END+\ts{END}}\ts{END}\3{-1}\5
${}\{{}$\1\5
${}\index{DEF+\.{DEF}}\.{DEF}(\.{\$\$},\39\index{int kind+\\{int\_kind}}\\{int\_kind},\39\.{\$3}){}$;\5
${}\index{hput tags+\\{hput\_tags}}\\{hput\_tags}(\.{\$1},\39\index{hput int+\\{hput\_int}}\\{hput\_int}(\.{\$4})){}$;\5
${}\}{}$\2\6
\4\hbox to 0.5em{\hss${}\OR{}$}\5
\index{start+\nts{start}}\nts{start}\5
\index{DIMEN+\ts{DIMEN}}\ts{DIMEN}\5
\index{ref+\nts{ref}}\nts{ref}\5
\index{dimension+\nts{dimension}}\nts{dimension}\5
\index{END+\ts{END}}\ts{END}\3{-1}\5
${}\{{}$\1\5
${}\index{DEF+\.{DEF}}\.{DEF}(\.{\$\$},\39\index{dimen kind+\\{dimen\_kind}}\\{dimen\_kind},\39\.{\$3}){}$;\5
${}\index{hput tags+\\{hput\_tags}}\\{hput\_tags}(\.{\$1},\39\index{hput dimen+\\{hput\_dimen}}\\{hput\_dimen}(\.{\$4}));{}$\5
${}\}{}$\2\6
\4\hbox to 0.5em{\hss${}\OR{}$}\5
\index{start+\nts{start}}\nts{start}\5
\index{LANGUAGE+\ts{LANGUAGE}}\ts{LANGUAGE}\5
\index{ref+\nts{ref}}\nts{ref}\5
\index{string+\nts{string}}\nts{string}\5
\index{END+\ts{END}}\ts{END}\3{-1}\5
${}\{{}$\1\5
${}\index{DEF+\.{DEF}}\.{DEF}(\.{\$\$},\39\index{language kind+\\{language\_kind}}\\{language\_kind},\39\.{\$3}){}$;\5
\index{hput string+\\{hput\_string}}\\{hput\_string}(\.{\$4});\5
${}\index{hput tags+\\{hput\_tags}}\\{hput\_tags}(\.{\$1},\39\.{TAG}(\index{language kind+\\{language\_kind}}\\{language\_kind},\39\T{0}));{}$\5
${}\}{}$\2\6
\4\hbox to 0.5em{\hss${}\OR{}$}\5
\index{start+\nts{start}}\nts{start}\5
\index{GLUE+\ts{GLUE}}\ts{GLUE}\5
\index{ref+\nts{ref}}\nts{ref}\5
\index{glue+\nts{glue}}\nts{glue}\5
\index{END+\ts{END}}\ts{END}\3{-1}\5
${}\{{}$\1\5
${}\index{DEF+\.{DEF}}\.{DEF}(\.{\$\$},\39\index{glue kind+\\{glue\_kind}}\\{glue\_kind},\39\.{\$3}){}$;\5
${}\index{hput tags+\\{hput\_tags}}\\{hput\_tags}(\.{\$1},\39\index{hput glue+\\{hput\_glue}}\\{hput\_glue}({\AND}(\.{\$4})));{}$\5
${}\}{}$\2\6
\4\hbox to 0.5em{\hss${}\OR{}$}\5
\index{start+\nts{start}}\nts{start}\5
\index{XDIMEN+\ts{XDIMEN}}\ts{XDIMEN}\5
\index{ref+\nts{ref}}\nts{ref}\5
\index{xdimen+\nts{xdimen}}\nts{xdimen}\5
\index{END+\ts{END}}\ts{END}\3{-1}\5
${}\{{}$\1\5
${}\index{DEF+\.{DEF}}\.{DEF}(\.{\$\$},\39\index{xdimen kind+\\{xdimen\_kind}}\\{xdimen\_kind},\39\.{\$3}){}$;\5
${}\index{hput tags+\\{hput\_tags}}\\{hput\_tags}(\.{\$1},\39\index{hput xdimen+\\{hput\_xdimen}}\\{hput\_xdimen}({\AND}(\.{\$4})));{}$\5
${}\}{}$\2\6
\4\hbox to 0.5em{\hss${}\OR{}$}\5
\index{start+\nts{start}}\nts{start}\5
\index{RULE+\ts{RULE}}\ts{RULE}\5
\index{ref+\nts{ref}}\nts{ref}\5
\index{rule+\nts{rule}}\nts{rule}\5
\index{END+\ts{END}}\ts{END}\3{-1}\5
${}\{{}$\1\5
${}\index{DEF+\.{DEF}}\.{DEF}(\.{\$\$},\39\index{rule kind+\\{rule\_kind}}\\{rule\_kind},\39\.{\$3}){}$;\5
${}\index{hput tags+\\{hput\_tags}}\\{hput\_tags}(\.{\$1},\39\index{hput rule+\\{hput\_rule}}\\{hput\_rule}({\AND}(\.{\$4})));{}$\5
${}\}{}$\2\6
\4\hbox to 0.5em{\hss${}\OR{}$}\5
\index{start+\nts{start}}\nts{start}\5
\index{LEADERS+\ts{LEADERS}}\ts{LEADERS}\5
\index{ref+\nts{ref}}\nts{ref}\5
\index{leaders+\nts{leaders}}\nts{leaders}\5
\index{END+\ts{END}}\ts{END}\3{-1}\5
${}\{{}$\1\5
${}\index{DEF+\.{DEF}}\.{DEF}(\.{\$\$},\39\index{leaders kind+\\{leaders\_kind}}\\{leaders\_kind},\39\.{\$3}){}$;\5
${}\index{hput tags+\\{hput\_tags}}\\{hput\_tags}(\.{\$1},\39\.{TAG}(\index{leaders kind+\\{leaders\_kind}}\\{leaders\_kind},\39\.{\$4}));{}$\5
${}\}{}$\2\6
\4\hbox to 0.5em{\hss${}\OR{}$}\5
\index{start+\nts{start}}\nts{start}\5
\index{BASELINE+\ts{BASELINE}}\ts{BASELINE}\5
\index{ref+\nts{ref}}\nts{ref}\5
\index{baseline+\nts{baseline}}\nts{baseline}\5
\index{END+\ts{END}}\ts{END}\3{-1}\5
${}\{{}$\1\5
${}\index{DEF+\.{DEF}}\.{DEF}(\.{\$\$},\39\index{baseline kind+\\{baseline\_kind}}\\{baseline\_kind},\39\.{\$3}){}$;\5
${}\index{hput tags+\\{hput\_tags}}\\{hput\_tags}(\.{\$1},\39\.{TAG}(\index{baseline kind+\\{baseline\_kind}}\\{baseline\_kind},\39\.{\$4}));{}$\5
${}\}{}$\2\6
\4\hbox to 0.5em{\hss${}\OR{}$}\5
\index{start+\nts{start}}\nts{start}\5
\index{LIGATURE+\ts{LIGATURE}}\ts{LIGATURE}\5
\index{ref+\nts{ref}}\nts{ref}\5
\index{ligature+\nts{ligature}}\nts{ligature}\5
\index{END+\ts{END}}\ts{END}\3{-1}\5
${}\{{}$\1\5
${}\index{DEF+\.{DEF}}\.{DEF}(\.{\$\$},\39\index{ligature kind+\\{ligature\_kind}}\\{ligature\_kind},\39\.{\$3}){}$;\5
${}\index{hput tags+\\{hput\_tags}}\\{hput\_tags}(\.{\$1},\39\index{hput ligature+\\{hput\_ligature}}\\{hput\_ligature}({\AND}(\.{\$4})));{}$\5
${}\}{}$\2\6
\4\hbox to 0.5em{\hss${}\OR{}$}\5
\index{start+\nts{start}}\nts{start}\5
\index{HYPHEN+\ts{HYPHEN}}\ts{HYPHEN}\5
\index{ref+\nts{ref}}\nts{ref}\5
\index{hyphen+\nts{hyphen}}\nts{hyphen}\5
\index{END+\ts{END}}\ts{END}\3{-1}\5
${}\{{}$\1\5
${}\index{DEF+\.{DEF}}\.{DEF}(\.{\$\$},\39\index{hyphen kind+\\{hyphen\_kind}}\\{hyphen\_kind},\39\.{\$3}){}$;\5
${}\index{hput tags+\\{hput\_tags}}\\{hput\_tags}(\.{\$1},\39\index{hput hyphen+\\{hput\_hyphen}}\\{hput\_hyphen}({\AND}(\.{\$4})));{}$\5
${}\}{}$\2\6
\4\hbox to 0.5em{\hss${}\OR{}$}\5
\index{start+\nts{start}}\nts{start}\5
\index{IMAGE+\ts{IMAGE}}\ts{IMAGE}\5
\index{ref+\nts{ref}}\nts{ref}\5
\index{image+\nts{image}}\nts{image}\5
\index{END+\ts{END}}\ts{END}\3{-1}\5
${}\{{}$\1\5
${}\index{DEF+\.{DEF}}\.{DEF}(\.{\$\$},\39\index{image kind+\\{image\_kind}}\\{image\_kind},\39\.{\$3}){}$;\5
${}\index{hput tags+\\{hput\_tags}}\\{hput\_tags}(\.{\$1},\39\index{hput image+\\{hput\_image}}\\{hput\_image}({\AND}(\.{\$4})));{}$\5
${}\}{}$\2\6
\4\hbox to 0.5em{\hss${}\OR{}$}\5
\index{start+\nts{start}}\nts{start}\5
\index{PARAM+\ts{PARAM}}\ts{PARAM}\5
\index{ref+\nts{ref}}\nts{ref}\5
\index{param list+\nts{param\_list}}\nts{param\_list}\5
\index{END+\ts{END}}\ts{END}\3{-1}\5
${}\{{}$\1\5
${}\index{DEF+\.{DEF}}\.{DEF}(\.{\$\$},\39\index{param kind+\\{param\_kind}}\\{param\_kind},\39\.{\$3}){}$;\5
${}\index{hput tags+\\{hput\_tags}}\\{hput\_tags}(\.{\$1},\39\index{hput list+\\{hput\_list}}\\{hput\_list}(\.{\$1}+\T{2},\39{\AND}(\.{\$4})));{}$\5
${}\}{}$\2\6
\4\hbox to 0.5em{\hss${}\OR{}$}\5
\index{start+\nts{start}}\nts{start}\5
\index{PAGE+\ts{PAGE}}\ts{PAGE}\5
\index{ref+\nts{ref}}\nts{ref}\5
\index{page+\nts{page}}\nts{page}\5
\index{END+\ts{END}}\ts{END}\3{-1}\5
${}\{{}$\1\5
${}\index{DEF+\.{DEF}}\.{DEF}(\.{\$\$},\39\index{page kind+\\{page\_kind}}\\{page\_kind},\39\.{\$3}){}$;\5
${}\index{hput tags+\\{hput\_tags}}\\{hput\_tags}(\.{\$1},\39\.{TAG}(\index{page kind+\\{page\_kind}}\\{page\_kind},\39\T{0}));{}$\5
${}\}{}$\2;\2\2
\Y
\fi

\M{317}

\goodbreak
\vbox{\getcode\vskip -\baselineskip\writecode}

\Y\B\4\X16:get functions\X${}\mathrel+\E{}$\6
\&{void} \index{hget definition+\\{hget\_definition}}\\{hget\_definition}(\&{int} \|n${},\39{}$\&{uint8\_t} \|a${},\39{}$\&{uint32\_t} \\{node\_pos})\1\1\2\2\1\6
\4${}\{{}$\6
\&{switch} (\index{KIND+\.{KIND}}\.{KIND}(\|a))\5
\1${}\{{}$\6
\4\&{case} \index{font kind+\\{font\_kind}}\\{font\_kind}:\5
${}\index{hget font def+\\{hget\_font\_def}}\\{hget\_font\_def}(\index{INFO+\.{INFO}}\.{INFO}(\|a),\39\|n){}$;\5
\&{break};\6
\4\&{case} \index{param kind+\\{param\_kind}}\\{param\_kind}:\1\6
\4${}\{{}$\5
\index{list t+\&{list\_t}}\&{list\_t} \|l;\5
${}\index{HGET LIST+\.{HGET\_LIST}}\.{HGET\_LIST}(\index{INFO+\.{INFO}}\.{INFO}(\|a),\39\|l){}$;\5
${}\index{hwrite param list+\\{hwrite\_param\_list}}\\{hwrite\_param\_list}({\AND}\|l){}$;\5
\&{break};\5
${}\}{}$\2\6
\4\&{case} \index{page kind+\\{page\_kind}}\\{page\_kind}:\5
\index{hget page+\\{hget\_page}}\\{hget\_page}(\,);\5
\&{break};\6
\4\&{case} \index{dimen kind+\\{dimen\_kind}}\\{dimen\_kind}:\5
\index{hget dimen+\\{hget\_dimen}}\\{hget\_dimen}(\,);\5
\&{break};\6
\4\&{case} \index{xdimen kind+\\{xdimen\_kind}}\\{xdimen\_kind}:\1\6
\4${}\{{}$\5
\index{xdimen t+\&{xdimen\_t}}\&{xdimen\_t} \|x;\5
${}\index{hget xdimen+\\{hget\_xdimen}}\\{hget\_xdimen}(\|a,\39{\AND}\|x){}$;\5
${}\index{hwrite xdimen+\\{hwrite\_xdimen}}\\{hwrite\_xdimen}({\AND}\|x){}$;\5
\&{break};\5
${}\}{}$\2\6
\4\&{case} \index{language kind+\\{language\_kind}}\\{language\_kind}:\6
\&{if} ${}(\index{INFO+\.{INFO}}\.{INFO}(\|a)\I\\{b000}){}$\1\5
\.{QUIT}(\.{"Info\ value\ of\ langu}\)\.{age\ definition\ must\ }\)\.{be\ zero"});\2\6
\&{else}\5
\1${}\{{}$\5
\&{char} ${}{*}\|n;{}$\7
\index{HGET STRING+\.{HGET\_STRING}}\.{HGET\_STRING}(\|n);\5
\index{hwrite string+\\{hwrite\_string}}\\{hwrite\_string}(\|n);\6
\4${}\}{}$\2\6
\&{break};\6
\4\&{default}:\6
\8\#\&{if} \T{0}\6
\&{if} ${}(\index{INFO+\.{INFO}}\.{INFO}(\|a)\E\T{0}\W\|n>\index{max fixed+\\{max\_fixed}}\\{max\_fixed}[\index{KIND+\.{KIND}}\.{KIND}(\|a)]){}$\1\5
${}\.{QUIT}(\.{"References\ not\ allo}\)\.{wed\ in\ definition\ \%d}\)\.{"},\39\|n);{}$\2\6
\8\#\&{endif}\6
\index{hget content+\\{hget\_content}}\\{hget\_content}(\|a);\5
\&{break};\6
\4${}\}{}$\2\6
\4${}\}{}$\2\7
\&{void} \index{hget def node+\\{hget\_def\_node}}\\{hget\_def\_node}(\index{ref t+\&{ref\_t}}\&{ref\_t} ${}{*}\index{df+\\{df}}\\{df}){}$\1\1\2\2\1\6
\4${}\{{}$\5
\X14:read the start byte \|a\X\6
${}\index{DEF+\.{DEF}}\.{DEF}({*}\index{df+\\{df}}\\{df},\39\index{KIND+\.{KIND}}\.{KIND}(\|a),\39\index{HGET8+\.{HGET8}}\.{HGET8});{}$\6
\&{if} ${}(\index{df+\\{df}}\\{df}\MG\|k\E\index{range kind+\\{range\_kind}}\\{range\_kind}){}$\1\5
${}\index{hget range+\\{hget\_range}}\\{hget\_range}(\index{INFO+\.{INFO}}\.{INFO}(\|a),\39\index{df+\\{df}}\\{df}\MG\|n);{}$\2\6
\&{else}\5
\1${}\{{}$\5
\index{hwrite start+\\{hwrite\_start}}\\{hwrite\_start}(\,);\5
${}\index{hwritef+\\{hwritef}}\\{hwritef}(\.{"\%s\ *\%d"},\39\index{definition name+\\{definition\_name}}\\{definition\_name}[\index{df+\\{df}}\\{df}\MG\|k],\39\index{df+\\{df}}\\{df}\MG\|n);{}$\6
${}\index{hget definition+\\{hget\_definition}}\\{hget\_definition}(\index{df+\\{df}}\\{df}\MG\|n,\39\|a,\39\\{node\_pos});{}$\6
\index{hwrite end+\\{hwrite\_end}}\\{hwrite\_end}(\,);\6
\4${}\}{}$\2\6
\X15:read and check the end byte \|z\X\6
\4${}\}{}$\2
\Y
\fi

\M{318}



\subsection{Parameter Lists}\label{paramlist}\index{parameter list}
Because the content section is a ``stateless'' list of nodes, the
definitions we see in the definition section can never change. It is
however necessary to make occasionally local modifications of some of
these definitions, because some definitions are parameters of the
algorithms borrowed from \TeX. Nodes that need such modifications, for
example the paragraph nodes that are passed to \TeX's line breaking
algorithm, contain a list of local definitions called parameters.
Typically sets of related parameters are needed.  To facilitate a
simple reference to such a set of parameters, we allow predefined
parameter lists that can be referenced by a single number.  The
parameters of \TeX's routines are quite basic---integers\index{integer},
dimensions\index{dimension}, and glues\index{glue}---and all
of them have default values.
Therefore we restrict the definitions in parameter lists to such
basic definitions.

\Y\B\4\X309:parsing functions\X${}\mathrel+\E{}$\6
\&{void} \index{check param def+\\{check\_param\_def}}\\{check\_param\_def}(\index{ref t+\&{ref\_t}}\&{ref\_t} ${}{*}\index{df+\\{df}}\\{df}){}$\1\1\2\2\1\6
\4${}\{{}$\6
\&{if} ${}(\index{df+\\{df}}\\{df}\MG\|k\I\index{int kind+\\{int\_kind}}\\{int\_kind}\W\index{df+\\{df}}\\{df}\MG\|k\I\index{dimen kind+\\{dimen\_kind}}\\{dimen\_kind}\W\3{-1}\index{df+\\{df}}\\{df}\MG\|k\I\index{glue kind+\\{glue\_kind}}\\{glue\_kind}){}$\1\5
${}\.{QUIT}(\.{"Kind\ \%s\ not\ allowed}\)\.{\ in\ parameter\ list"},\39\index{definition name+\\{definition\_name}}\\{definition\_name}[\index{df+\\{df}}\\{df}\MG\|k]);{}$\2\6
\&{if} ${}(\index{df+\\{df}}\\{df}\MG\|n\Z\index{max fixed+\\{max\_fixed}}\\{max\_fixed}[\index{df+\\{df}}\\{df}\MG\|k]\V\index{max default+\\{max\_default}}\\{max\_default}[\index{df+\\{df}}\\{df}\MG\|k]<\index{df+\\{df}}\\{df}\MG\|n){}$\1\5
${}\.{QUIT}(\.{"Parameter\ \%d\ for\ \%s}\)\.{\ not\ allowed\ in\ para}\)\.{meter\ list"},\39\index{df+\\{df}}\\{df}\MG\|n,\39\index{definition name+\\{definition\_name}}\\{definition\_name}[\index{df+\\{df}}\\{df}\MG\|k]);{}$\2\6
\4${}\}{}$\2
\Y
\fi

\M{319}

The definitions below repeat the definitions we have seen for lists in section~\secref{plainlists}
with small modifications. For example we use the kind value \index{param kind+\\{param\_kind}}\\{param\_kind}. An empty parameter list
is omitted in the long format as well as in the short format.

\goodbreak
\vbox{\readcode\vskip -\baselineskip\putcode}

\Y\par
\par
\par
\par
\par
\Y\B\4\X2:symbols\X${}\mathrel+\E{}$\6
\8\%\&{token} \index{PARAM+\ts{PARAM}}\ts{PARAM}\5\.{"param"}\6
\8\%\index{type+\&{type}}\&{type} $<$ \|u $>$ \index{def list+\nts{def\_list}}\nts{def\_list} \6
\8\%\index{type+\&{type}}\&{type} $<$ \|l $>$ \index{param list+\nts{param\_list}}\nts{param\_list}\5
\index{param list node+\nts{param\_list\_node}}\nts{param\_list\_node}
\Y
\fi

\M{320}

\Y\B\4\X3:scanning rules\X${}\mathrel+\E{}$\6
${}\8\re{\vb{param}}{}$\ac\&{return} \index{PARAM+\ts{PARAM}}\ts{PARAM};\eac
\Y
\fi

\M{321}
\Y\B\4\X5:parsing rules\X${}\mathrel+\E{}$\6
\index{def list+\nts{def\_list}}\nts{def\_list}: \1\1\5
\index{position+\nts{position}}\nts{position}\5
\hbox to 0.5em{\hss${}\OR{}$}\5
\index{def list+\nts{def\_list}}\nts{def\_list}\5
\index{def node+\nts{def\_node}}\nts{def\_node}\5
${}\{{}$\1\5
${}\index{check param def+\\{check\_param\_def}}\\{check\_param\_def}({\AND}(\.{\$2}));{}$\5
${}\}{}$\2;\2\2\7
\index{param list+\nts{param\_list}}\nts{param\_list}: \1\1\5
\index{estimate+\nts{estimate}}\nts{estimate}\5
\index{def list+\nts{def\_list}}\nts{def\_list}\5
${}\{{}$\1\5
${}\.{\$\$}.\|p\K\.{\$2};{}$\5
${}\.{\$\$}.\|k\K\index{param kind+\\{param\_kind}}\\{param\_kind};{}$\5
${}\.{\$\$}.\|s\K(\index{hpos+\\{hpos}}\\{hpos}-\index{hstart+\\{hstart}}\\{hstart})-\.{\$2};{}$\5
${}\}{}$\2;\2\2\7
\index{param list node+\nts{param\_list\_node}}\nts{param\_list\_node}: \1\1\5
\index{start+\nts{start}}\nts{start}\5
\index{PARAM+\ts{PARAM}}\ts{PARAM}\5
\index{param list+\nts{param\_list}}\nts{param\_list}\5
\index{END+\ts{END}}\ts{END}\6
${}\{{}$\5
\1\&{if} ${}(\.{\$3}.\|s>\T{0}){}$\1\5
${}\index{hput tags+\\{hput\_tags}}\\{hput\_tags}(\.{\$1},\39\index{hput list+\\{hput\_list}}\\{hput\_list}(\.{\$1}+\T{1},\39{\AND}(\.{\$3}))){}$;\5
\2${}\}{}$\2;\2\2
\Y
\fi

\M{322}

\writecode
\Y\B\4\X19:write functions\X${}\mathrel+\E{}$\6
\&{void} \index{hwrite param list+\\{hwrite\_param\_list}}\\{hwrite\_param\_list}(\index{list t+\&{list\_t}}\&{list\_t} ${}{*}\|l){}$\1\1\2\2\1\6
\4${}\{{}$\5
\&{uint32\_t} \|h${}\K\index{hpos+\\{hpos}}\\{hpos}-\index{hstart+\\{hstart}}\\{hstart},{}$ \|e${}\K\index{hend+\\{hend}}\\{hend}-\index{hstart+\\{hstart}}\\{hstart}{}$;\C{ save \\{hpos} and \\{hend} }\7
${}\index{hpos+\\{hpos}}\\{hpos}\K\|l\MG\|p+\index{hstart+\\{hstart}}\\{hstart}{}$;\5
${}\index{hend+\\{hend}}\\{hend}\K\index{hpos+\\{hpos}}\\{hpos}+\|l\MG\|s;{}$\6
\&{if} ${}(\|l\MG\|s>\T{\^FF}){}$\1\5
${}\index{hwritef+\\{hwritef}}\\{hwritef}(\.{"\ \%d"},\39\|l\MG\|s);{}$\2\6
\&{while} ${}(\index{hpos+\\{hpos}}\\{hpos}<\index{hend+\\{hend}}\\{hend}{}$)\6
\1${}\{{}$\5
\index{ref t+\&{ref\_t}}\&{ref\_t} \index{df+\\{df}}\\{df};\5
${}\index{hget def node+\\{hget\_def\_node}}\\{hget\_def\_node}({\AND}\index{df+\\{df}}\\{df});{}$\6
\4${}\}{}$\2\6
${}\index{hpos+\\{hpos}}\\{hpos}\K\index{hstart+\\{hstart}}\\{hstart}+\|h{}$;\5
${}\index{hend+\\{hend}}\\{hend}\K\index{hstart+\\{hstart}}\\{hstart}+\|e{}$;\C{ restore  \\{hpos} and \\{hend} }\6
\4${}\}{}$\2\7
\&{void} \index{hwrite param list node+\\{hwrite\_param\_list\_node}}\\{hwrite\_param\_list\_node}(\index{list t+\&{list\_t}}\&{list\_t} ${}{*}\|l){}$\1\1\2\2\1\6
\4${}\{{}$\5
\&{if} ${}(\|l\MG\|s\I\T{0}{}$)\6
\1${}\{{}$\5
\index{hwrite start+\\{hwrite\_start}}\\{hwrite\_start}(\,);\5
\index{hwritef+\\{hwritef}}\\{hwritef}(\.{"param"});\6
\index{hwrite param list+\\{hwrite\_param\_list}}\\{hwrite\_param\_list}(\|l);\6
\index{hwrite end+\\{hwrite\_end}}\\{hwrite\_end}(\,);\6
\4${}\}{}$\2\6
\4${}\}{}$\2
\Y
\fi

\M{323}

\getcode
\Y\B\4\X16:get functions\X${}\mathrel+\E{}$\6
\&{void} \index{hget param list node+\\{hget\_param\_list\_node}}\\{hget\_param\_list\_node}(\index{list t+\&{list\_t}}\&{list\_t} ${}{*}\|l){}$\1\1\2\2\1\6
\4${}\{{}$\5
\&{if} ${}(\index{KIND+\.{KIND}}\.{KIND}({*}\index{hpos+\\{hpos}}\\{hpos})\I\index{param kind+\\{param\_kind}}\\{param\_kind}{}$)\6
\1${}\{{}$\5
${}\|l\MG\|p\K\index{hpos+\\{hpos}}\\{hpos}-\index{hstart+\\{hstart}}\\{hstart}{}$;\5
${}\|l\MG\|s\K\T{0}{}$;\5
${}\|l\MG\|k\K\index{param kind+\\{param\_kind}}\\{param\_kind}{}$;\5
${}\}{}$\2\6
\&{else}\1\5
\index{hget list+\\{hget\_list}}\\{hget\_list}(\|l);\2\6
\4${}\}{}$\2
\Y
\fi

\M{324}



\subsection{Fonts}\label{fonts}
Another definition that has no corresponding content node is the
font\index{font} definition.  Fonts by themselves do not constitute
content, instead they are used in glyph\index{glyph} nodes.  Fonts are
also a data type, that never occur directly in a content node; a
font is always specified by its font number. This limits the number of
fonts that can be used in a \HINT/ file to at most 256.

A long format font definition starts with the keyword ``\index{font+\.{font}}\.{font}'' and
is followed by the font number, as usual prefixed by an asterisk. Then
comes the font specification with the optional font size, the font
name, the section number of the \TeX\ font metric file, and the
section number of the file containing the glyphs for the font.
The \HINT/ format supports \.{.pk} files, the traditional font format
for \TeX, and the more modern PostScript Type 1 fonts,
TrueType fonts, and OpenType fonts.

The format of font definitions will probably change in future
versions of the \HINT/ file format.
For example,  \.{.pk} files might be replaced entirely by PostScript Type 1 fonts.
Also \HINT/ needs the \TeX\ font metric files only to obtain the sizes
of characters when running \TeX's line breaking algorithm.
But for many TrueType fonts there are no \TeX\ font metric files,
while the necessary information about character sizes should be easy
to obtain.
Another information, that is currently missing from font definitions,
is the fonts character encoding.

In a \HINT/ file, text is represented as a sequence of numbers called
character codes. \HINT/ files use the UTF-8 character encoding
scheme (CES) to map these numbers to their representation as byte
sequences.  For example the number ``\T{\^E4}'' is encoded as the byte
sequence ``\T{\^C3} \T{\^A4}''.  The same number \T{\^E4} now can represent
different characters depending on the coded character set (CCS). For
example in the common ISO-8859-1 (Latin 1) encoding the number \T{\^E4}
is the umlaut ``\"a'' where as in the ISO-8859-7 (Latin/Greek) it is
the greek letter ``$\delta$'' and in the EBCDIC encoding, used on IBM
mainframes, it is the upper case letter ``U''.

The character encoding is
irrelevant for rendering a \HINT/ file as long as the character codes
in the glyph nodes are consistent with the character codes used in the font
file, but the character encoding is necessary for all programs that
need to ``understand'' the content of the \HINT/ file. For example
programs that want to translate a \HINT/ document to a different language,
or for text-to-speech conversion.

The Internet Engineering Task Force IETF has established a character set
registry\cite{ietf:charset-mib} that defines an enumeration of all
registered coded character sets\cite{iana:charset-mib}.  The coded
character set numbers are in the range 1--2999.
This encoding number, as given in~\cite{iana:charset},
might be one possibility for specifying the font encoding as
part of a font definition.

Currently, it is only required that a font specifies
an interword glue and a default discretionary break. After that comes
a list of up to 12 font specific parameters.

The optional font size specifies the desired font\index{font size}
size. If omitted, we assign the value zero which implies the
design\index{design size} size of the font as stored in the \.{.tfm}
file.

In the short format, the font specification is given in the same order
as in the long format.  The info value will be \\{b001} if a font size
is present, otherwise it is~\\{b000}.

Our internal representation of a font just stores the font name
because in the long format we add the font name as a comment to glyph
nodes.


\Y\B\4\X252:common variables\X${}\mathrel+\E{}$\6
\&{char} ${}{*}{*}\index{hfont name+\\{hfont\_name}}\\{hfont\_name}{}$;\C{ dynamically allocated array of font names }
\Y
\fi

\M{325}

\Y\B\4\X6:hint basic types\X${}\mathrel+\E{}$\6
\8\#\&{define} \index{MAX FONT PARAMS+\.{MAX\_FONT\_PARAMS}}\.{MAX\_FONT\_PARAMS}\5\T{11}
\Y
\fi

\M{326}

\Y\B\4\X253:initialize definitions\X${}\mathrel+\E{}$\6
$\index{ALLOCATE+\.{ALLOCATE}}\.{ALLOCATE}(\index{hfont name+\\{hfont\_name}}\\{hfont\_name},\39\\{max\_ref}[\index{font kind+\\{font\_kind}}\\{font\_kind}]+\T{1},\39{}$\&{char} ${}{*}){}$;
\Y
\fi

\M{327}



\readcode
\Y\par
\par
\par
\par
\par
\Y\B\4\X2:symbols\X${}\mathrel+\E{}$\6
\8\%\&{token} \index{FONT+\ts{FONT}}\ts{FONT}\5\.{"font"}\6
\8\%\index{type+\&{type}}\&{type} $<$ \index{info+\\{info}}\\{info} $>$ \index{font+\nts{font}}\nts{font}\5
\index{font head+\nts{font\_head}}\nts{font\_head}
\Y
\fi

\M{328}

\Y\B\4\X3:scanning rules\X${}\mathrel+\E{}$\6
${}\8\re{\vb{font}}{}$\ac\&{return} \index{FONT+\ts{FONT}}\ts{FONT};\eac
\Y
\fi

\M{329}

Note that we set the definition bit early because the definition of font \|f
might involve glyphs that reference font \|f (or other fonts).

\Y\B\4\X5:parsing rules\X${}\mathrel+\E{}$\6
\index{font+\nts{font}}\nts{font}: \1\1\5
\index{font head+\nts{font\_head}}\nts{font\_head}\5
\index{font param list+\nts{font\_param\_list}}\nts{font\_param\_list};\2\2\7
\index{font head+\nts{font\_head}}\nts{font\_head}: \1\1\5
\index{string+\nts{string}}\nts{string}\5
\index{UNSIGNED+\ts{UNSIGNED}}\ts{UNSIGNED}\5
\index{UNSIGNED+\ts{UNSIGNED}}\ts{UNSIGNED}\6
${}\{{}$\1\5
\&{uint8\_t} \|f${}\K\|\$<\|u>\T{0};{}$\7
${}\index{SET DBIT+\.{SET\_DBIT}}\.{SET\_DBIT}(\|f,\39\index{font kind+\\{font\_kind}}\\{font\_kind}){}$;\5
${}\index{hfont name+\\{hfont\_name}}\\{hfont\_name}[\|f]\K\index{strdup+\\{strdup}}\\{strdup}(\.{\$1});{}$\5
${}\.{\$\$}\K\index{hput font head+\\{hput\_font\_head}}\\{hput\_font\_head}(\|f,\39\index{hfont name+\\{hfont\_name}}\\{hfont\_name}[\|f],\39\T{0},\39\.{\$2},\39\.{\$3});{}$\5
${}\}{}$\2\6
\4\hbox to 0.5em{\hss${}\OR{}$}\5
\index{string+\nts{string}}\nts{string}\5
\index{dimension+\nts{dimension}}\nts{dimension}\5
\index{UNSIGNED+\ts{UNSIGNED}}\ts{UNSIGNED}\5
\index{UNSIGNED+\ts{UNSIGNED}}\ts{UNSIGNED}\6
${}\{{}$\1\5
\&{uint8\_t} \|f${}\K\|\$<\|u>\T{0};{}$\7
${}\index{SET DBIT+\.{SET\_DBIT}}\.{SET\_DBIT}(\|f,\39\index{font kind+\\{font\_kind}}\\{font\_kind}){}$;\5
${}\index{hfont name+\\{hfont\_name}}\\{hfont\_name}[\|f]\K\index{strdup+\\{strdup}}\\{strdup}(\.{\$1});{}$\5
${}\.{\$\$}\K\index{hput font head+\\{hput\_font\_head}}\\{hput\_font\_head}(\|f,\39\index{hfont name+\\{hfont\_name}}\\{hfont\_name}[\|f],\39\.{\$2},\39\.{\$3},\39\.{\$4});{}$\5
${}\}{}$\2;\2\2\7
\index{font param list+\nts{font\_param\_list}}\nts{font\_param\_list}: \1\1\5
\index{glue node+\nts{glue\_node}}\nts{glue\_node}\5
\index{hyphen node+\nts{hyphen\_node}}\nts{hyphen\_node}\5
\hbox to 0.5em{\hss${}\OR{}$}\5
\index{font param list+\nts{font\_param\_list}}\nts{font\_param\_list}\5
\index{font param+\nts{font\_param}}\nts{font\_param};\2\2\7
\index{font param+\nts{font\_param}}\nts{font\_param}:\1\1\6
\index{start+\nts{start}}\nts{start}\5
\index{PENALTY+\ts{PENALTY}}\ts{PENALTY}\5
\index{fref+\nts{fref}}\nts{fref}\5
\index{penalty+\nts{penalty}}\nts{penalty}\5
\index{END+\ts{END}}\ts{END}\5
${}\{{}$\1\5
${}\index{hput tags+\\{hput\_tags}}\\{hput\_tags}(\.{\$1},\39\index{hput int+\\{hput\_int}}\\{hput\_int}(\.{\$4}));{}$\5
${}\}{}$\2\6
\4\hbox to 0.5em{\hss${}\OR{}$}\5
\index{start+\nts{start}}\nts{start}\5
\index{KERN+\ts{KERN}}\ts{KERN}\5
\index{fref+\nts{fref}}\nts{fref}\5
\index{kern+\nts{kern}}\nts{kern}\5
\index{END+\ts{END}}\ts{END}\5
${}\{{}$\1\5
${}\index{hput tags+\\{hput\_tags}}\\{hput\_tags}(\.{\$1},\39\index{hput kern+\\{hput\_kern}}\\{hput\_kern}({\AND}(\.{\$4})));{}$\5
${}\}{}$\2\6
\4\hbox to 0.5em{\hss${}\OR{}$}\5
\index{start+\nts{start}}\nts{start}\5
\index{LIGATURE+\ts{LIGATURE}}\ts{LIGATURE}\5
\index{fref+\nts{fref}}\nts{fref}\5
\index{ligature+\nts{ligature}}\nts{ligature}\5
\index{END+\ts{END}}\ts{END}\5
${}\{{}$\1\5
${}\index{hput tags+\\{hput\_tags}}\\{hput\_tags}(\.{\$1},\39\index{hput ligature+\\{hput\_ligature}}\\{hput\_ligature}({\AND}(\.{\$4})));{}$\5
${}\}{}$\2\6
\4\hbox to 0.5em{\hss${}\OR{}$}\5
\index{start+\nts{start}}\nts{start}\5
\index{HYPHEN+\ts{HYPHEN}}\ts{HYPHEN}\5
\index{fref+\nts{fref}}\nts{fref}\5
\index{hyphen+\nts{hyphen}}\nts{hyphen}\5
\index{END+\ts{END}}\ts{END}\5
${}\{{}$\1\5
${}\index{hput tags+\\{hput\_tags}}\\{hput\_tags}(\.{\$1},\39\index{hput hyphen+\\{hput\_hyphen}}\\{hput\_hyphen}({\AND}(\.{\$4})));{}$\5
${}\}{}$\2\6
\4\hbox to 0.5em{\hss${}\OR{}$}\5
\index{start+\nts{start}}\nts{start}\5
\index{GLUE+\ts{GLUE}}\ts{GLUE}\5
\index{fref+\nts{fref}}\nts{fref}\5
\index{glue+\nts{glue}}\nts{glue}\5
\index{END+\ts{END}}\ts{END}\5
${}\{{}$\1\5
${}\index{hput tags+\\{hput\_tags}}\\{hput\_tags}(\.{\$1},\39\index{hput glue+\\{hput\_glue}}\\{hput\_glue}({\AND}(\.{\$4})));{}$\5
${}\}{}$\2\6
\4\hbox to 0.5em{\hss${}\OR{}$}\5
\index{start+\nts{start}}\nts{start}\5
\index{LANGUAGE+\ts{LANGUAGE}}\ts{LANGUAGE}\5
\index{fref+\nts{fref}}\nts{fref}\5
\index{string+\nts{string}}\nts{string}\5
\index{END+\ts{END}}\ts{END}\5
${}\{{}$\1\5
\index{hput string+\\{hput\_string}}\\{hput\_string}(\.{\$4});\5
${}\index{hput tags+\\{hput\_tags}}\\{hput\_tags}(\.{\$1},\39\.{TAG}(\index{language kind+\\{language\_kind}}\\{language\_kind},\39\T{0}));{}$\5
${}\}{}$\2\6
\4\hbox to 0.5em{\hss${}\OR{}$}\5
\index{start+\nts{start}}\nts{start}\5
\index{RULE+\ts{RULE}}\ts{RULE}\5
\index{fref+\nts{fref}}\nts{fref}\5
\index{rule+\nts{rule}}\nts{rule}\5
\index{END+\ts{END}}\ts{END}\5
${}\{{}$\1\5
${}\index{hput tags+\\{hput\_tags}}\\{hput\_tags}(\.{\$1},\39\index{hput rule+\\{hput\_rule}}\\{hput\_rule}({\AND}(\.{\$4})));{}$\5
${}\}{}$\2\6
\4\hbox to 0.5em{\hss${}\OR{}$}\5
\index{start+\nts{start}}\nts{start}\5
\index{IMAGE+\ts{IMAGE}}\ts{IMAGE}\5
\index{fref+\nts{fref}}\nts{fref}\5
\index{image+\nts{image}}\nts{image}\5
\index{END+\ts{END}}\ts{END}\5
${}\{{}$\1\5
${}\index{hput tags+\\{hput\_tags}}\\{hput\_tags}(\.{\$1},\39\index{hput image+\\{hput\_image}}\\{hput\_image}({\AND}(\.{\$4})));{}$\5
${}\}{}$\2;\2\2\7
\index{fref+\nts{fref}}\nts{fref}: \1\1\5
\index{ref+\nts{ref}}\nts{ref}\6
${}\{{}$\1\5
${}\.{RNG}(\.{"Font\ parameter"},\39\.{\$1},\39\T{0},\39\index{MAX FONT PARAMS+\.{MAX\_FONT\_PARAMS}}\.{MAX\_FONT\_PARAMS});{}$\5
${}\}{}$\2;\2\2
\Y
\fi

\M{330}

\goodbreak
\vbox{\getcode\vskip -\baselineskip\writecode}

\Y\B\4\X16:get functions\X${}\mathrel+\E{}$\6
\&{static} \&{void} \index{hget font params+\\{hget\_font\_params}}\\{hget\_font\_params}(\&{void})\1\1\2\2\1\6
\4${}\{{}$\5
\index{hyphen t+\&{hyphen\_t}}\&{hyphen\_t} \|h;\7
\index{hget glue node+\\{hget\_glue\_node}}\\{hget\_glue\_node}(\,);\6
${}\index{hget hyphen node+\\{hget\_hyphen\_node}}\\{hget\_hyphen\_node}({\AND}(\|h)){}$;\5
${}\index{hwrite hyphen node+\\{hwrite\_hyphen\_node}}\\{hwrite\_hyphen\_node}({\AND}\|h);{}$\6
${}\.{DBG}(\index{DBGDEF+\.{DBGDEF}}\.{DBGDEF},\39\.{"Start\ font\ paramete}\)\.{rs\\n"});{}$\6
\&{while} ${}(\index{KIND+\.{KIND}}\.{KIND}({*}\index{hpos+\\{hpos}}\\{hpos})\I\index{font kind+\\{font\_kind}}\\{font\_kind}{}$)\6
\1${}\{{}$\5
\index{ref t+\&{ref\_t}}\&{ref\_t} \index{df+\\{df}}\\{df};\7
\X14:read the start byte \|a\X\6
${}\index{df+\\{df}}\\{df}.\|k\K\index{KIND+\.{KIND}}\.{KIND}(\|a);{}$\6
${}\index{df+\\{df}}\\{df}.\|n\K\index{HGET8+\.{HGET8}}\.{HGET8};{}$\6
${}\.{DBG}(\index{DBGDEF+\.{DBGDEF}}\.{DBGDEF},\39\.{"Reading\ font\ parame}\)\.{ter\ \%d:\ \%s\\n"},\39\index{df+\\{df}}\\{df}.\|n,\39\index{definition name+\\{definition\_name}}\\{definition\_name}[\index{df+\\{df}}\\{df}.\|k]);{}$\6
\&{if} ${}(\index{df+\\{df}}\\{df}.\|k\I\index{penalty kind+\\{penalty\_kind}}\\{penalty\_kind}\W\index{df+\\{df}}\\{df}.\|k\I\index{kern kind+\\{kern\_kind}}\\{kern\_kind}\W\index{df+\\{df}}\\{df}.\|k\I\index{ligature kind+\\{ligature\_kind}}\\{ligature\_kind}\W\3{-1}\index{df+\\{df}}\\{df}.\|k\I\index{hyphen kind+\\{hyphen\_kind}}\\{hyphen\_kind}\W\index{df+\\{df}}\\{df}.\|k\I\index{glue kind+\\{glue\_kind}}\\{glue\_kind}\W\index{df+\\{df}}\\{df}.\|k\I\index{language kind+\\{language\_kind}}\\{language\_kind}\W\3{-1}\index{df+\\{df}}\\{df}.\|k\I\index{rule kind+\\{rule\_kind}}\\{rule\_kind}\W\index{df+\\{df}}\\{df}.\|k\I\\{image%
\_kind}){}$\1\5
${}\.{QUIT}(\.{"Font\ parameter\ \%d\ h}\)\.{as\ invalid\ type\ \%s"},\39\index{df+\\{df}}\\{df}.\|n,\39\index{content name+\\{content\_name}}\\{content\_name}[\index{df+\\{df}}\\{df}.\|n]);{}$\2\6
${}\.{RNG}(\.{"Font\ parameter"},\39\index{df+\\{df}}\\{df}.\|n,\39\T{0},\39\index{MAX FONT PARAMS+\.{MAX\_FONT\_PARAMS}}\.{MAX\_FONT\_PARAMS});{}$\6
\index{hwrite start+\\{hwrite\_start}}\\{hwrite\_start}(\,);\5
${}\index{hwritef+\\{hwritef}}\\{hwritef}(\.{"\%s\ *\%d"},\39\index{content name+\\{content\_name}}\\{content\_name}[\index{KIND+\.{KIND}}\.{KIND}(\|a)],\39\index{df+\\{df}}\\{df}.\|n);{}$\6
${}\index{hget definition+\\{hget\_definition}}\\{hget\_definition}(\index{df+\\{df}}\\{df}.\|n,\39\|a,\39\\{node\_pos});{}$\6
\index{hwrite end+\\{hwrite\_end}}\\{hwrite\_end}(\,);\6
\X15:read and check the end byte \|z\X\6
\4${}\}{}$\2\6
${}\.{DBG}(\index{DBGDEF+\.{DBGDEF}}\.{DBGDEF},\39\.{"End\ font\ parameters}\)\.{\\n"});{}$\6
\4${}\}{}$\2\7
\&{void} \index{hget font def+\\{hget\_font\_def}}\\{hget\_font\_def}(\index{info t+\&{info\_t}}\&{info\_t} \|i${},\39{}$\&{uint8\_t} \|f)\1\1\2\2\1\6
\4${}\{{}$\5
\&{char} ${}{*}\|n{}$;\5
\index{dimen t+\&{dimen\_t}}\&{dimen\_t} \|s${}\K\T{0}{}$;\5
\&{uint16\_t} \|m${},{}$ \|y;\7
\index{HGET STRING+\.{HGET\_STRING}}\.{HGET\_STRING}(\|n);\5
\index{hwrite string+\\{hwrite\_string}}\\{hwrite\_string}(\|n);\5
${}\index{hfont name+\\{hfont\_name}}\\{hfont\_name}[\|f]\K\index{strdup+\\{strdup}}\\{strdup}(\|n);{}$\6
\&{if} ${}(\|i\AND\\{b001}{}$)\5
\1${}\{{}$\5
\index{HGET32+\.{HGET32}}\.{HGET32}(\|s);\5
\&{if} ${}(\|s\I\T{0}){}$\1\5
\index{hwrite dimension+\\{hwrite\_dimension}}\\{hwrite\_dimension}(\|s);\5
\2${}\}{}$\2\6
${}\.{DBG}(\index{DBGDEF+\.{DBGDEF}}\.{DBGDEF},\39\.{"Font\ \%s\ size\ 0x\%x\\n}\)\.{"},\39\|n,\39\|s);{}$\6
\index{HGET16+\.{HGET16}}\.{HGET16}(\|m);\5
${}\.{RNG}(\.{"Font\ metrics"},\39\|m,\39\T{3},\39\index{max section no+\\{max\_section\_no}}\\{max\_section\_no});{}$\6
\index{HGET16+\.{HGET16}}\.{HGET16}(\|y);\5
${}\.{RNG}(\.{"Font\ glyphs"},\39\|y,\39\T{3},\39\index{max section no+\\{max\_section\_no}}\\{max\_section\_no});{}$\6
${}\index{hwritef+\\{hwritef}}\\{hwritef}(\.{"\ \%d\ \%d"},\39\|m,\39\|y);{}$\6
\index{hget font params+\\{hget\_font\_params}}\\{hget\_font\_params}(\,);\6
${}\.{DBG}(\index{DBGDEF+\.{DBGDEF}}\.{DBGDEF},\39\.{"End\ font\ definition}\)\.{\\n"});{}$\6
\4${}\}{}$\2
\Y
\fi

\M{331}

\putcode
\Y\B\4\X12:put functions\X${}\mathrel+\E{}$\6
\&{uint8\_t} \index{hput font head+\\{hput\_font\_head}}\\{hput\_font\_head}(\&{uint8\_t} \|f${},\39{}$\&{char} ${}{*}\|n,\39{}$\index{dimen t+\&{dimen\_t}}\&{dimen\_t} \|s${},\3{-1}\39{}$\&{uint16\_t} \|m${},\39{}$\&{uint16\_t} \|y)\1\1\2\2\1\6
\4${}\{{}$\5
\index{info t+\&{info\_t}}\&{info\_t} \|i${}\K\\{b000};{}$\7
${}\.{DBG}(\index{DBGDEF+\.{DBGDEF}}\.{DBGDEF},\39\.{"Defining\ font\ \%d\ (\%}\)\.{s)\ size\ 0x\%x\\n"},\39\|f,\39\|n,\39\|s);{}$\6
\index{hput string+\\{hput\_string}}\\{hput\_string}(\|n);\6
\&{if} ${}(\|s\I\T{0}){}$\5
\1${}\{{}$\5
\index{HPUT32+\.{HPUT32}}\.{HPUT32}(\|s);\5
${}\|i\K\\{b001}{}$;\5
${}\}{}$\2\6
\index{HPUT16+\.{HPUT16}}\.{HPUT16}(\|m);\5
\index{HPUT16+\.{HPUT16}}\.{HPUT16}(\|y);\6
\&{return} \.{TAG}${}(\index{font kind+\\{font\_kind}}\\{font\_kind},\39\|i);{}$\6
\4${}\}{}$\2
\Y
\fi

\M{332}



\subsection{References}\label{reference}
We have seen how to make definitions, now let's see how to
reference\index{reference} them.  In the long form, we can simply
write the reference number, after the keyword like this:
``{\tt \.{<}glue *17\.{>}}''.
The asterisk\index{asterisk} is necessary to keep apart,
for example, a penalty with value 50,
written ``{\tt \.{<}penalty 50\.{>}}'',
from a penalty referencing the integer
definition number 50, written ``{\tt \.{<}penalty *50\.{>}}''.

\goodbreak
\vbox{\readcode\vskip -\baselineskip\putcode}

\Y\B\4\X5:parsing rules\X${}\mathrel+\E{}$\6
\index{xdimen ref+\nts{xdimen\_ref}}\nts{xdimen\_ref}: \1\1\5
\index{ref+\nts{ref}}\nts{ref}\5
${}\{{}$\1\5
${}\index{REF+\.{REF}}\.{REF}(\index{xdimen kind+\\{xdimen\_kind}}\\{xdimen\_kind},\39\.{\$1});{}$\5
${}\}{}$\2;\2\2\7
\index{param ref+\nts{param\_ref}}\nts{param\_ref}: \1\1\5
\index{ref+\nts{ref}}\nts{ref}\5
${}\{{}$\1\5
${}\index{REF+\.{REF}}\.{REF}(\index{param kind+\\{param\_kind}}\\{param\_kind},\39\.{\$1});{}$\5
${}\}{}$\2;\2\2\7
\index{stream ref+\nts{stream\_ref}}\nts{stream\_ref}: \1\1\5
\index{ref+\nts{ref}}\nts{ref}\5
${}\{{}$\1\5
${}\index{REF RNG+\.{REF\_RNG}}\.{REF\_RNG}(\index{stream kind+\\{stream\_kind}}\\{stream\_kind},\39\.{\$1});{}$\5
${}\}{}$\2;\2\2\7
\index{content node+\nts{content\_node}}\nts{content\_node}: \1\1\5
\index{start+\nts{start}}\nts{start}\5
\index{PENALTY+\ts{PENALTY}}\ts{PENALTY}\5
\index{ref+\nts{ref}}\nts{ref}\5
\index{END+\ts{END}}\ts{END}\6
${}\{{}$\1\5
${}\index{REF+\.{REF}}\.{REF}(\index{penalty kind+\\{penalty\_kind}}\\{penalty\_kind},\39\.{\$3}){}$;\5
${}\index{hput tags+\\{hput\_tags}}\\{hput\_tags}(\.{\$1},\39\.{TAG}(\index{penalty kind+\\{penalty\_kind}}\\{penalty\_kind},\39\T{0}));{}$\5
${}\}{}$\2\6
\4\hbox to 0.5em{\hss${}\OR{}$}\5
\index{start+\nts{start}}\nts{start}\5
\index{KERN+\ts{KERN}}\ts{KERN}\5
\nts{explicit}\5
\index{ref+\nts{ref}}\nts{ref}\5
\index{END+\ts{END}}\ts{END}\6
${}\{{}$\1\5
${}\index{REF+\.{REF}}\.{REF}(\index{dimen kind+\\{dimen\_kind}}\\{dimen\_kind},\39\.{\$4}){}$;\5
${}\index{hput tags+\\{hput\_tags}}\\{hput\_tags}(\.{\$1},\39\.{TAG}(\index{kern kind+\\{kern\_kind}}\\{kern\_kind},\39(\.{\$3})\?\\{b100}:\\{b000}));{}$\5
${}\}{}$\2\6
\4\hbox to 0.5em{\hss${}\OR{}$}\5
\index{start+\nts{start}}\nts{start}\5
\index{KERN+\ts{KERN}}\ts{KERN}\5
\nts{explicit}\5
\index{XDIMEN+\ts{XDIMEN}}\ts{XDIMEN}\5
\index{ref+\nts{ref}}\nts{ref}\5
\index{END+\ts{END}}\ts{END}\6
${}\{{}$\1\5
${}\index{REF+\.{REF}}\.{REF}(\index{xdimen kind+\\{xdimen\_kind}}\\{xdimen\_kind},\39\.{\$5}){}$;\5
${}\index{hput tags+\\{hput\_tags}}\\{hput\_tags}(\.{\$1},\39\.{TAG}(\index{kern kind+\\{kern\_kind}}\\{kern\_kind},\39(\.{\$3})\?\\{b101}:\\{b001}));{}$\5
${}\}{}$\2\6
\4\hbox to 0.5em{\hss${}\OR{}$}\5
\index{start+\nts{start}}\nts{start}\5
\index{GLUE+\ts{GLUE}}\ts{GLUE}\5
\index{ref+\nts{ref}}\nts{ref}\5
\index{END+\ts{END}}\ts{END}\6
${}\{{}$\1\5
${}\index{REF+\.{REF}}\.{REF}(\index{glue kind+\\{glue\_kind}}\\{glue\_kind},\39\.{\$3}){}$;\5
${}\index{hput tags+\\{hput\_tags}}\\{hput\_tags}(\.{\$1},\39\.{TAG}(\index{glue kind+\\{glue\_kind}}\\{glue\_kind},\39\T{0}));{}$\5
${}\}{}$\2\6
\4\hbox to 0.5em{\hss${}\OR{}$}\5
\index{start+\nts{start}}\nts{start}\5
\index{LIGATURE+\ts{LIGATURE}}\ts{LIGATURE}\5
\index{ref+\nts{ref}}\nts{ref}\5
\index{END+\ts{END}}\ts{END}\6
${}\{{}$\1\5
${}\index{REF+\.{REF}}\.{REF}(\index{ligature kind+\\{ligature\_kind}}\\{ligature\_kind},\39\.{\$3}){}$;\5
${}\index{hput tags+\\{hput\_tags}}\\{hput\_tags}(\.{\$1},\39\.{TAG}(\index{ligature kind+\\{ligature\_kind}}\\{ligature\_kind},\39\T{0}));{}$\5
${}\}{}$\2\6
\4\hbox to 0.5em{\hss${}\OR{}$}\5
\index{start+\nts{start}}\nts{start}\5
\index{HYPHEN+\ts{HYPHEN}}\ts{HYPHEN}\5
\index{ref+\nts{ref}}\nts{ref}\5
\index{END+\ts{END}}\ts{END}\6
${}\{{}$\1\5
${}\index{REF+\.{REF}}\.{REF}(\index{hyphen kind+\\{hyphen\_kind}}\\{hyphen\_kind},\39\.{\$3}){}$;\5
${}\index{hput tags+\\{hput\_tags}}\\{hput\_tags}(\.{\$1},\39\.{TAG}(\index{hyphen kind+\\{hyphen\_kind}}\\{hyphen\_kind},\39\T{0}));{}$\5
${}\}{}$\2\6
\4\hbox to 0.5em{\hss${}\OR{}$}\5
\index{start+\nts{start}}\nts{start}\5
\index{RULE+\ts{RULE}}\ts{RULE}\5
\index{ref+\nts{ref}}\nts{ref}\5
\index{END+\ts{END}}\ts{END}\6
${}\{{}$\1\5
${}\index{REF+\.{REF}}\.{REF}(\index{rule kind+\\{rule\_kind}}\\{rule\_kind},\39\.{\$3}){}$;\5
${}\index{hput tags+\\{hput\_tags}}\\{hput\_tags}(\.{\$1},\39\.{TAG}(\index{rule kind+\\{rule\_kind}}\\{rule\_kind},\39\T{0}));{}$\5
${}\}{}$\2\6
\4\hbox to 0.5em{\hss${}\OR{}$}\5
\index{start+\nts{start}}\nts{start}\5
\index{IMAGE+\ts{IMAGE}}\ts{IMAGE}\5
\index{ref+\nts{ref}}\nts{ref}\5
\index{END+\ts{END}}\ts{END}\6
${}\{{}$\1\5
${}\index{REF+\.{REF}}\.{REF}(\index{image kind+\\{image\_kind}}\\{image\_kind},\39\.{\$3}){}$;\5
${}\index{hput tags+\\{hput\_tags}}\\{hput\_tags}(\.{\$1},\39\.{TAG}(\index{image kind+\\{image\_kind}}\\{image\_kind},\39\T{0}));{}$\5
${}\}{}$\2\6
\4\hbox to 0.5em{\hss${}\OR{}$}\5
\index{start+\nts{start}}\nts{start}\5
\index{LEADERS+\ts{LEADERS}}\ts{LEADERS}\5
\index{ref+\nts{ref}}\nts{ref}\5
\index{END+\ts{END}}\ts{END}\6
${}\{{}$\1\5
${}\index{REF+\.{REF}}\.{REF}(\index{leaders kind+\\{leaders\_kind}}\\{leaders\_kind},\39\.{\$3}){}$;\5
${}\index{hput tags+\\{hput\_tags}}\\{hput\_tags}(\.{\$1},\39\.{TAG}(\index{leaders kind+\\{leaders\_kind}}\\{leaders\_kind},\39\T{0}));{}$\5
${}\}{}$\2\6
\4\hbox to 0.5em{\hss${}\OR{}$}\5
\index{start+\nts{start}}\nts{start}\5
\index{BASELINE+\ts{BASELINE}}\ts{BASELINE}\5
\index{ref+\nts{ref}}\nts{ref}\5
\index{END+\ts{END}}\ts{END}\6
${}\{{}$\1\5
${}\index{REF+\.{REF}}\.{REF}(\index{baseline kind+\\{baseline\_kind}}\\{baseline\_kind},\39\.{\$3}){}$;\5
${}\index{hput tags+\\{hput\_tags}}\\{hput\_tags}(\.{\$1},\39\.{TAG}(\index{baseline kind+\\{baseline\_kind}}\\{baseline\_kind},\39\T{0}));{}$\5
${}\}{}$\2;\2\2\7
\4\hbox to 0.5em{\hss${}\OR{}$}\5
\index{start+\nts{start}}\nts{start}\5
\index{LANGUAGE+\ts{LANGUAGE}}\ts{LANGUAGE}\5
\index{REFERENCE+\ts{REFERENCE}}\ts{REFERENCE}\5
\index{END+\ts{END}}\ts{END}\6
${}\{{}$\1\5
${}\index{REF+\.{REF}}\.{REF}(\index{language kind+\\{language\_kind}}\\{language\_kind},\39\.{\$3}){}$;\5
${}\index{hput tags+\\{hput\_tags}}\\{hput\_tags}(\.{\$1},\39\index{hput language+\\{hput\_language}}\\{hput\_language}(\.{\$3}));{}$\5
${}\}{}$\2;\2\2\7
\index{glue node+\nts{glue\_node}}\nts{glue\_node}: \1\1\5
\index{start+\nts{start}}\nts{start}\5
\index{GLUE+\ts{GLUE}}\ts{GLUE}\5
\index{ref+\nts{ref}}\nts{ref}\5
\index{END+\ts{END}}\ts{END}\6
${}\{{}$\1\5
${}\index{REF+\.{REF}}\.{REF}(\index{glue kind+\\{glue\_kind}}\\{glue\_kind},\39\.{\$3}){}$;\5
${}\index{hput tags+\\{hput\_tags}}\\{hput\_tags}(\.{\$1},\39\.{TAG}(\index{glue kind+\\{glue\_kind}}\\{glue\_kind},\39\T{0}));{}$\5
${}\}{}$\2;\2\2
\Y
\fi

\M{333}

\getcode
\Y\B\4\X18:cases to get content\X${}\mathrel+\E{}$\6
\4\&{case} \.{TAG}${}(\index{penalty kind+\\{penalty\_kind}}\\{penalty\_kind},\39\T{0}){}$:\5
\index{HGET REF+\.{HGET\_REF}}\.{HGET\_REF}(\index{penalty kind+\\{penalty\_kind}}\\{penalty\_kind});\5
\&{break};\6
\4\&{case} \.{TAG}${}(\index{kern kind+\\{kern\_kind}}\\{kern\_kind},\39\\{b000}){}$:\5
\index{HGET REF+\.{HGET\_REF}}\.{HGET\_REF}(\index{dimen kind+\\{dimen\_kind}}\\{dimen\_kind});\5
\&{break};\6
\4\&{case} \.{TAG}${}(\index{kern kind+\\{kern\_kind}}\\{kern\_kind},\39\\{b100}){}$:\5
\index{hwritef+\\{hwritef}}\\{hwritef}(\.{"\ !"});\5
\index{HGET REF+\.{HGET\_REF}}\.{HGET\_REF}(\index{dimen kind+\\{dimen\_kind}}\\{dimen\_kind});\5
\&{break};\6
\4\&{case} \.{TAG}${}(\index{kern kind+\\{kern\_kind}}\\{kern\_kind},\39\\{b001}){}$:\5
\index{hwritef+\\{hwritef}}\\{hwritef}(\.{"\ xdimen"});\5
\index{HGET REF+\.{HGET\_REF}}\.{HGET\_REF}(\index{xdimen kind+\\{xdimen\_kind}}\\{xdimen\_kind});\5
\&{break};\6
\4\&{case} \.{TAG}${}(\index{kern kind+\\{kern\_kind}}\\{kern\_kind},\39\\{b101}){}$:\5
\index{hwritef+\\{hwritef}}\\{hwritef}(\.{"\ !\ xdimen"});\5
\index{HGET REF+\.{HGET\_REF}}\.{HGET\_REF}(\index{xdimen kind+\\{xdimen\_kind}}\\{xdimen\_kind});\5
\&{break};\6
\4\&{case} \.{TAG}${}(\index{ligature kind+\\{ligature\_kind}}\\{ligature\_kind},\39\T{0}){}$:\5
\index{HGET REF+\.{HGET\_REF}}\.{HGET\_REF}(\index{ligature kind+\\{ligature\_kind}}\\{ligature\_kind});\5
\&{break};\6
\4\&{case} \.{TAG}${}(\index{hyphen kind+\\{hyphen\_kind}}\\{hyphen\_kind},\39\T{0}){}$:\5
\index{HGET REF+\.{HGET\_REF}}\.{HGET\_REF}(\index{hyphen kind+\\{hyphen\_kind}}\\{hyphen\_kind});\5
\&{break};\6
\4\&{case} \.{TAG}${}(\index{glue kind+\\{glue\_kind}}\\{glue\_kind},\39\T{0}){}$:\5
\index{HGET REF+\.{HGET\_REF}}\.{HGET\_REF}(\index{glue kind+\\{glue\_kind}}\\{glue\_kind});\5
\&{break};\6
\4\&{case} \.{TAG}${}(\index{language kind+\\{language\_kind}}\\{language\_kind},\39\\{b000}){}$:\5
\index{HGET REF+\.{HGET\_REF}}\.{HGET\_REF}(\index{language kind+\\{language\_kind}}\\{language\_kind});\5
\&{break};\6
\4\&{case} \.{TAG}${}(\index{rule kind+\\{rule\_kind}}\\{rule\_kind},\39\T{0}){}$:\5
\index{HGET REF+\.{HGET\_REF}}\.{HGET\_REF}(\index{rule kind+\\{rule\_kind}}\\{rule\_kind});\5
\&{break};\6
\4\&{case} \.{TAG}${}(\index{image kind+\\{image\_kind}}\\{image\_kind},\39\T{0}){}$:\5
\index{HGET REF+\.{HGET\_REF}}\.{HGET\_REF}(\index{image kind+\\{image\_kind}}\\{image\_kind});\5
\&{break};\6
\4\&{case} \.{TAG}${}(\index{leaders kind+\\{leaders\_kind}}\\{leaders\_kind},\39\T{0}){}$:\5
\index{HGET REF+\.{HGET\_REF}}\.{HGET\_REF}(\index{leaders kind+\\{leaders\_kind}}\\{leaders\_kind});\5
\&{break};\6
\4\&{case} \.{TAG}${}(\index{baseline kind+\\{baseline\_kind}}\\{baseline\_kind},\39\T{0}){}$:\5
\index{HGET REF+\.{HGET\_REF}}\.{HGET\_REF}(\index{baseline kind+\\{baseline\_kind}}\\{baseline\_kind});\5
\&{break};
\Y
\fi

\M{334}

\Y\B\4\X17:get macros\X${}\mathrel+\E{}$\6
\8\#\&{define} \index{HGET REF+\.{HGET\_REF}}\.{HGET\_REF}(\|K)\1\1\2\2\1\6
\4${}\{{}$\5
\&{uint8\_t} \|n${}\K\index{HGET8+\.{HGET8}}\.{HGET8}{}$;\5
${}\index{REF+\.{REF}}\.{REF}(\|K,\39\|n){}$;\5
\index{hwrite ref+\\{hwrite\_ref}}\\{hwrite\_ref}(\|n);\5
${}\}{}$\2
\Y
\fi

\M{335}
\writecode
\Y\B\4\X19:write functions\X${}\mathrel+\E{}$\6
\&{void} \index{hwrite ref+\\{hwrite\_ref}}\\{hwrite\_ref}(\&{uint8\_t} \|n)\1\1\2\2\1\6
\4${}\{{}$\5
${}\index{hwritef+\\{hwritef}}\\{hwritef}(\.{"\ *\%d"},\39\|n);{}$\6
\4${}\}{}$\2\7
\&{void} \index{hwrite ref node+\\{hwrite\_ref\_node}}\\{hwrite\_ref\_node}(\&{uint8\_t} \|k${},\39{}$\&{uint8\_t} \|n)\1\1\2\2\1\6
\4${}\{{}$\5
\index{hwrite start+\\{hwrite\_start}}\\{hwrite\_start}(\,);\5
${}\index{hwritef+\\{hwritef}}\\{hwritef}(\.{"\%s"},\39\index{content name+\\{content\_name}}\\{content\_name}[\|k]){}$;\5
\index{hwrite ref+\\{hwrite\_ref}}\\{hwrite\_ref}(\|n);\5
\index{hwrite end+\\{hwrite\_end}}\\{hwrite\_end}(\,);\6
\4${}\}{}$\2
\Y
\fi

\M{336}



\section{Defaults}\label{defaults}\index{default value}
Several of the predefined values found in the definition section are used
as parameters for the routines borrowed from \TeX\ to display the content
of a \HINT/ file. These values must be defined, but it is inconvenient if
the same standard definitions must be placed in each and every \HINT/ file.
Therefore we specify in this chapter reasonable default values.
As a consequence, even a \HINT/ file without any definitions should
produce sensible results when displayed.

The definitions that have default values are integers, dimensions,
extended dimensions, glues, baselines, page templates, streams, and page ranges.
Each of these defaults has its own subsection below.
Actually the defaults for extended dimensions and baselines are not needed by \TeX's
routines, but it is nice to have default values for the extended dimensions that represent
\.{hsize}, \.{vsize}, or a zero baseline skip.

The array \index{max default+\\{max\_default}}\\{max\_default} contains for each kind value the maximum number of
the default values. The function \index{hset max+\\{hset\_max}}\\{hset\_max} is used to initialize them.

The programs \.{shrink} and \index{stretch+\.{stretch}}\.{stretch} actually do not use the defaults.
It is, however, possible to suppress definitions if the defined value
is the same as the default.

For maximum flexibility and efficiency, this chapter defines a
header file {\tt hformat.h} and a \CEE/ program {\tt mkhformat}
that generates the corresponding {\tt hformat.c} file.
The latter contains constant arrays containing the respective default information.

\noindent
Here is the header file:
\Y\B\4\X336:\.{hformat.h }\X${}\E{}$\6
\8\#\&{ifndef} \.{\_HFORMAT\_H\_}\6
\8\#\&{define} \.{\_HFORMAT\_H\_}\6
\X305:debug macros\X\6
\X362:debug constants\X\6
\X11:hint macros\X\6
\X6:hint basic types\X\6
\X339:default names\X\7
\&{extern} \&{const} \&{char} ${}{*}\index{content name+\\{content\_name}}\\{content\_name}[\T{32}];{}$\6
\&{extern} \&{const} \&{char} ${}{*}\index{definition name+\\{definition\_name}}\\{definition\_name}[\T{32}];{}$\6
\&{extern} \&{unsigned} \&{int} \index{debugflags+\\{debugflags}}\\{debugflags};\6
\&{extern} \&{FILE} ${}{*}\index{hlog+\\{hlog}}\\{hlog};{}$\6
\&{extern} \&{int} \index{max fixed+\\{max\_fixed}}\\{max\_fixed}[\T{32}]${},{}$ \index{max default+\\{max\_default}}\\{max\_default}[\T{32}]${},{}$ \\{max\_ref}[\T{32}];\6
\&{extern} \&{int32\_t} ${}\index{int defaults+\\{int\_defaults}}\\{int\_defaults}[\index{MAX INT DEFAULT+\.{MAX\_INT\_DEFAULT}}\.{MAX\_INT\_DEFAULT}+\T{1}];{}$\6
\&{extern} \index{dimen t+\&{dimen\_t}}\&{dimen\_t} ${}\index{dimen defaults+\\{dimen\_defaults}}\\{dimen\_defaults}[\index{MAX DIMEN DEFAULT+\.{MAX\_DIMEN\_DEFAULT}}\.{MAX\_DIMEN\_DEFAULT}+\T{1}];{}$\6
\&{extern} \index{xdimen t+\&{xdimen\_t}}\&{xdimen\_t} ${}\index{xdimen defaults+\\{xdimen\_defaults}}\\{xdimen\_defaults}[\index{MAX XDIMEN DEFAULT+\.{MAX\_XDIMEN\_DEFAULT}}\.{MAX\_XDIMEN\_DEFAULT}+\T{1}];{}$\6
\&{extern} \index{glue t+\&{glue\_t}}\&{glue\_t} ${}\index{glue defaults+\\{glue\_defaults}}\\{glue\_defaults}[\index{MAX GLUE DEFAULT+\.{MAX\_GLUE\_DEFAULT}}\.{MAX\_GLUE\_DEFAULT}+\T{1}];{}$\6
\&{extern} \index{baseline t+\&{baseline\_t}}\&{baseline\_t} ${}\index{baseline defaults+\\{baseline\_defaults}}\\{baseline\_defaults}[\index{MAX BASELINE DEFAULT+\.{MAX\_BASELINE\_DEFAULT}}\.{MAX\_BASELINE\_DEFAULT}+\T{1}];{}$\6
\8\#\&{endif}
\Y
\fi

\M{337}

\noindent
And here is the \index{main+\\{main}}\\{main} program of {\tt mkhformat}:

\Y\B\4\X337:\.{mkhformat.c }\X${}\E{}$\6
\8\#\&{include} \.{<stdio.h>}\6
\8\#\&{include} \.{"basetypes.h"}\6
\8\#\&{include} \.{"hformat.h"}\6
\&{int} \index{max fixed+\\{max\_fixed}}\\{max\_fixed}[\T{32}]${},{}$ \index{max default+\\{max\_default}}\\{max\_default}[\T{32}];\6
\&{int32\_t} ${}\index{int defaults+\\{int\_defaults}}\\{int\_defaults}[\index{MAX INT DEFAULT+\.{MAX\_INT\_DEFAULT}}\.{MAX\_INT\_DEFAULT}+\T{1}]\K\{\T{0}\};{}$\6
\index{dimen t+\&{dimen\_t}}\&{dimen\_t} ${}\index{dimen defaults+\\{dimen\_defaults}}\\{dimen\_defaults}[\index{MAX DIMEN DEFAULT+\.{MAX\_DIMEN\_DEFAULT}}\.{MAX\_DIMEN\_DEFAULT}+\T{1}]\K\{\T{0}\};{}$\6
\index{xdimen t+\&{xdimen\_t}}\&{xdimen\_t} ${}\index{xdimen defaults+\\{xdimen\_defaults}}\\{xdimen\_defaults}[\index{MAX XDIMEN DEFAULT+\.{MAX\_XDIMEN\_DEFAULT}}\.{MAX\_XDIMEN\_DEFAULT}+\T{1}]\K\{\{\T{0}\}\};{}$\6
\index{glue t+\&{glue\_t}}\&{glue\_t} ${}\index{glue defaults+\\{glue\_defaults}}\\{glue\_defaults}[\index{MAX GLUE DEFAULT+\.{MAX\_GLUE\_DEFAULT}}\.{MAX\_GLUE\_DEFAULT}+\T{1}]\K\{\{\{\T{0}\}\}\};{}$\6
\index{baseline t+\&{baseline\_t}}\&{baseline\_t} ${}\index{baseline defaults+\\{baseline\_defaults}}\\{baseline\_defaults}[\index{MAX BASELINE DEFAULT+\.{MAX\_BASELINE\_DEFAULT}}\.{MAX\_BASELINE\_DEFAULT}+\T{1}]\K\{\{\{\{\T{0}\}\}\}\};{}$\7
\&{int} \index{main+\\{main}}\\{main}(\&{void})\1\1\2\2\1\6
\4${}\{{}$\5
\index{kind t+\&{kind\_t}}\&{kind\_t} \|k;\6
\&{int} \|i;\7
\index{printf+\\{printf}}\\{printf}(\.{"\#include\ \\"basetype}\)\.{s.h\\"\\n"}\6
\.{"\#include\ \\"hformat.}\)\.{h\\"\\n\\n"}\6
\X363:variables in {\tt hformat.c}\X);\6
\X7:define \\{content\_name} and \\{definition\_name}\X\6
\&{for} ${}(\|k\K\T{0};{}$ ${}\|k<\T{32};{}$ ${}\|k\PP){}$\1\5
${}\index{max default+\\{max\_default}}\\{max\_default}[\|k]\K{-}\T{1},\39\index{max fixed+\\{max\_fixed}}\\{max\_fixed}[\|k]\K\T{\^100};{}$\2\6
\X340:define \\{int\_defaults}\X\6
\X342:define \\{dimen\_defaults}\X\6
\X344:define \\{xdimen\_defaults}\X\6
\X348:define \\{baseline\_defaults}\X\6
\X352:define \\{page\_defaults}\X\6
\X350:define \\{stream\_defaults}\X\6
\X354:define \\{range\_defaults}\X\6
\X338:define \\{max\_ref}, \\{max\_fixed} and \\{max\_default}\X\6
\&{return} \T{0};\6
\4${}\}{}$\2
\Y
\fi

\M{338}

Above, we have set \index{max default+\\{max\_default}}\\{max\_default} to $-1$, meaning no defaults,
and \index{max fixed+\\{max\_fixed}}\\{max\_fixed} to \T{\^100}, meaning no definitions.
The following subsections will overwrite these values for
all kinds of definitions that have defaults.
It remains to reset \index{max fixed+\\{max\_fixed}}\\{max\_fixed} to $-1$ for all those kinds
that have no defaults but allow definitions.
Then we can print out both arrays.
\Y\B\4\X338:define \\{max\_ref}, \\{max\_fixed} and \\{max\_default}\X${}\E{}$\6
$\index{max fixed+\\{max\_fixed}}\\{max\_fixed}[\index{font kind+\\{font\_kind}}\\{font\_kind}]\K\index{max fixed+\\{max\_fixed}}\\{max\_fixed}[\index{ligature kind+\\{ligature\_kind}}\\{ligature\_kind}]\K\index{max fixed+\\{max\_fixed}}\\{max\_fixed}[\index{hyphen kind+\\{hyphen\_kind}}\\{hyphen\_kind}]\K\index{max fixed+\\{max\_fixed}}\\{max\_fixed}[\index{language kind+\\{language\_kind}}\\{language\_kind}]\K\index{max fixed+\\{max\_fixed}}\\{max\_fixed}[\index{rule kind+\\{rule\_kind}}\\{rule\_kind}]\K\index{max fixed+\\{max\_fixed}}\\{max\_fixed}[\index{image kind+\\{image\_kind}}\\{image\_kind}]\K\index{max fixed+\\{max\_fixed}}\\{max\_fixed}[\index{leaders kind+\\{leaders\_kind}}\\{leaders\_kind}]\K\index{max fixed+\\{max\_fixed}}\\{max\_fixed}[%
\index{param kind+\\{param\_kind}}\\{param\_kind}]\K{-}\T{1}{}$;\7
\index{printf+\\{printf}}\\{printf}(\.{"int\ max\_fixed[32]=\ }\)\.{\{"});\6
\&{for} ${}(\|k\K\T{0};{}$ ${}\|k<\T{32};{}$ ${}\|k\PP{}$)\6
\1${}\{{}$\5
${}\index{printf+\\{printf}}\\{printf}(\.{"\%d"},\39\index{max fixed+\\{max\_fixed}}\\{max\_fixed}[\|k]){}$;\5
\&{if} ${}(\|k<\T{31}){}$\1\5
\index{printf+\\{printf}}\\{printf}(\.{",\ "});\5
\2${}\}{}$\2\6
\index{printf+\\{printf}}\\{printf}(\.{"\};\\n\\n"});\7
\index{printf+\\{printf}}\\{printf}(\.{"int\ max\_default[32]}\)\.{=\ \{"});\6
\&{for} ${}(\|k\K\T{0};{}$ ${}\|k<\T{32};{}$ ${}\|k\PP{}$)\6
\1${}\{{}$\5
${}\index{printf+\\{printf}}\\{printf}(\.{"\%d"},\39\index{max default+\\{max\_default}}\\{max\_default}[\|k]){}$;\5
\&{if} ${}(\|k<\T{31}){}$\1\5
\index{printf+\\{printf}}\\{printf}(\.{",\ "});\5
\2${}\}{}$\2\6
\index{printf+\\{printf}}\\{printf}(\.{"\};\\n\\n"});\6
\index{printf+\\{printf}}\\{printf}(\.{"int\ max\_ref[32]=\ \{"});\6
\&{for} ${}(\|k\K\T{0};{}$ ${}\|k<\T{32};{}$ ${}\|k\PP{}$)\6
\1${}\{{}$\5
${}\index{printf+\\{printf}}\\{printf}(\.{"\%d"},\39\index{max default+\\{max\_default}}\\{max\_default}[\|k]){}$;\5
\&{if} ${}(\|k<\T{31}){}$\1\5
\index{printf+\\{printf}}\\{printf}(\.{",\ "});\5
\2${}\}{}$\2\6
\index{printf+\\{printf}}\\{printf}(\.{"\};\\n\\n"});
\U337.\Y
\fi

\M{339}

\subsection{Integers}
Integers\index{integer} are very simple objects, and it might be tempting not to
use predefined integers at all. But the \TeX\ typesetting engine,
which is used by \HINT/ uses many integer parameters to fine tune
its operations. As we will see, all these integer parameters have a predefined
integer number that refers to an integer definition.

Integers and penalties\index{penalty} share the same kind value. So a penalty node that references
one of the predefined penalties, simply contains the integer number as a reference
number.

The following integer numbers are predefined.
The zero integer is fixed with integer number zero. It is never redefined.
The default values are taken from {\tt plain.tex}.

\Y\B\4\X339:default names\X${}\E{}$\6
\&{typedef} \&{enum} ${}\{{}$\1\6
${}\index{zero int no+\\{zero\_int\_no}}\\{zero\_int\_no}\K\T{0},\39\index{pretolerance no+\\{pretolerance\_no}}\\{pretolerance\_no}\K\T{1},\39\index{tolerance no+\\{tolerance\_no}}\\{tolerance\_no}\K\T{2},\39\index{line penalty no+\\{line\_penalty\_no}}\\{line\_penalty\_no}\K\T{3},\39\index{hyphen penalty no+\\{hyphen\_penalty\_no}}\\{hyphen\_penalty\_no}\K\T{4},\39\index{ex hyphen penalty no+\\{ex\_hyphen\_penalty\_no}}\\{ex\_hyphen\_penalty\_no}\K\T{5},\39\index{club penalty no+\\{club\_penalty\_no}}\\{club\_penalty\_no}\K\T{6},\39\index{widow penalty no+\\{widow\_penalty\_no}}\\{widow\_penalty\_no}\K%
\T{7},\39\index{display widow penalty no+\\{display\_widow\_penalty\_no}}\\{display\_widow\_penalty\_no}\K\T{8},\39\index{broken penalty no+\\{broken\_penalty\_no}}\\{broken\_penalty\_no}\K\T{9},\39\index{pre display penalty no+\\{pre\_display\_penalty\_no}}\\{pre\_display\_penalty\_no}\K\T{10},\39\index{post display penalty no+\\{post\_display\_penalty\_no}}\\{post\_display\_penalty\_no}\K\T{11},\39\index{inter line penalty no+\\{inter\_line\_penalty\_no}}\\{inter\_line\_penalty\_no}\K\T{12},\39\index{double hyphen demerits no+\\{double\_hyphen\_demerits\_no}}\\{double\_hyphen\_demerits\_no}\K\T{13},\39%
\index{final hyphen demerits no+\\{final\_hyphen\_demerits\_no}}\\{final\_hyphen\_demerits\_no}\K\T{14},\39\\{adj\_demerits\_no}\K\T{15},\39\index{looseness no+\\{looseness\_no}}\\{looseness\_no}\K\T{16},\39\index{time no+\\{time\_no}}\\{time\_no}\K\T{17},\39\index{day no+\\{day\_no}}\\{day\_no}\K\T{18},\39\index{month no+\\{month\_no}}\\{month\_no}\K\T{19},\39\index{year no+\\{year\_no}}\\{year\_no}\K\T{20},\39\index{hang after no+\\{hang\_after\_no}}\\{hang\_after\_no}\K\T{21},\39\\{floating\_penalty%
\_no}\K\T{22}{}$\2\6
${}\}{}$ \index{int no t+\&{int\_no\_t}}\&{int\_no\_t};\6
\8\#\&{define} \index{MAX INT DEFAULT+\.{MAX\_INT\_DEFAULT}}\.{MAX\_INT\_DEFAULT}\5\index{floating penalty no+\\{floating\_penalty\_no}}\\{floating\_penalty\_no}
\As341, 343, 345, 347, 349, 351\ETs353.
\U336.\Y
\fi

\M{340}

\Y\B\4\X340:define \\{int\_defaults}\X${}\E{}$\6
$\index{max default+\\{max\_default}}\\{max\_default}[\index{int kind+\\{int\_kind}}\\{int\_kind}]\K\index{MAX INT DEFAULT+\.{MAX\_INT\_DEFAULT}}\.{MAX\_INT\_DEFAULT};{}$\6
${}\index{max fixed+\\{max\_fixed}}\\{max\_fixed}[\index{int kind+\\{int\_kind}}\\{int\_kind}]\K\index{zero int no+\\{zero\_int\_no}}\\{zero\_int\_no}{}$;\7
${}\index{int defaults+\\{int\_defaults}}\\{int\_defaults}[\index{zero int no+\\{zero\_int\_no}}\\{zero\_int\_no}]\K\T{0};{}$\6
${}\index{int defaults+\\{int\_defaults}}\\{int\_defaults}[\index{pretolerance no+\\{pretolerance\_no}}\\{pretolerance\_no}]\K\T{100};{}$\6
${}\index{int defaults+\\{int\_defaults}}\\{int\_defaults}[\index{tolerance no+\\{tolerance\_no}}\\{tolerance\_no}]\K\T{200};{}$\6
${}\index{int defaults+\\{int\_defaults}}\\{int\_defaults}[\index{line penalty no+\\{line\_penalty\_no}}\\{line\_penalty\_no}]\K\T{10};{}$\6
${}\index{int defaults+\\{int\_defaults}}\\{int\_defaults}[\index{hyphen penalty no+\\{hyphen\_penalty\_no}}\\{hyphen\_penalty\_no}]\K\T{50};{}$\6
${}\index{int defaults+\\{int\_defaults}}\\{int\_defaults}[\index{ex hyphen penalty no+\\{ex\_hyphen\_penalty\_no}}\\{ex\_hyphen\_penalty\_no}]\K\T{50};{}$\6
${}\index{int defaults+\\{int\_defaults}}\\{int\_defaults}[\index{club penalty no+\\{club\_penalty\_no}}\\{club\_penalty\_no}]\K\T{150};{}$\6
${}\index{int defaults+\\{int\_defaults}}\\{int\_defaults}[\index{widow penalty no+\\{widow\_penalty\_no}}\\{widow\_penalty\_no}]\K\T{150};{}$\6
${}\index{int defaults+\\{int\_defaults}}\\{int\_defaults}[\index{display widow penalty no+\\{display\_widow\_penalty\_no}}\\{display\_widow\_penalty\_no}]\K\T{50};{}$\6
${}\index{int defaults+\\{int\_defaults}}\\{int\_defaults}[\index{broken penalty no+\\{broken\_penalty\_no}}\\{broken\_penalty\_no}]\K\T{100};{}$\6
${}\index{int defaults+\\{int\_defaults}}\\{int\_defaults}[\index{pre display penalty no+\\{pre\_display\_penalty\_no}}\\{pre\_display\_penalty\_no}]\K\T{10000};{}$\6
${}\index{int defaults+\\{int\_defaults}}\\{int\_defaults}[\index{post display penalty no+\\{post\_display\_penalty\_no}}\\{post\_display\_penalty\_no}]\K\T{0};{}$\6
${}\index{int defaults+\\{int\_defaults}}\\{int\_defaults}[\index{inter line penalty no+\\{inter\_line\_penalty\_no}}\\{inter\_line\_penalty\_no}]\K\T{0};{}$\6
${}\index{int defaults+\\{int\_defaults}}\\{int\_defaults}[\index{double hyphen demerits no+\\{double\_hyphen\_demerits\_no}}\\{double\_hyphen\_demerits\_no}]\K\T{10000};{}$\6
${}\index{int defaults+\\{int\_defaults}}\\{int\_defaults}[\index{final hyphen demerits no+\\{final\_hyphen\_demerits\_no}}\\{final\_hyphen\_demerits\_no}]\K\T{5000};{}$\6
${}\index{int defaults+\\{int\_defaults}}\\{int\_defaults}[\\{adj\_demerits\_no}]\K\T{10000};{}$\6
${}\index{int defaults+\\{int\_defaults}}\\{int\_defaults}[\index{looseness no+\\{looseness\_no}}\\{looseness\_no}]\K\T{0};{}$\6
${}\index{int defaults+\\{int\_defaults}}\\{int\_defaults}[\index{time no+\\{time\_no}}\\{time\_no}]\K\T{720};{}$\6
${}\index{int defaults+\\{int\_defaults}}\\{int\_defaults}[\index{day no+\\{day\_no}}\\{day\_no}]\K\T{4};{}$\6
${}\index{int defaults+\\{int\_defaults}}\\{int\_defaults}[\index{month no+\\{month\_no}}\\{month\_no}]\K\T{7};{}$\6
${}\index{int defaults+\\{int\_defaults}}\\{int\_defaults}[\index{year no+\\{year\_no}}\\{year\_no}]\K\T{1776};{}$\6
${}\index{int defaults+\\{int\_defaults}}\\{int\_defaults}[\index{hang after no+\\{hang\_after\_no}}\\{hang\_after\_no}]\K\T{1};{}$\6
${}\index{int defaults+\\{int\_defaults}}\\{int\_defaults}[\index{floating penalty no+\\{floating\_penalty\_no}}\\{floating\_penalty\_no}]\K\T{20000}{}$;\7
\index{printf+\\{printf}}\\{printf}(\.{"int32\_t\ int\_default}\)\.{s[MAX\_INT\_DEFAULT+1]}\)\.{=\{"});\6
\&{for} ${}(\|i\K\T{0};{}$ ${}\|i\Z\index{max default+\\{max\_default}}\\{max\_default}[\index{int kind+\\{int\_kind}}\\{int\_kind}];{}$ ${}\|i\PP){}$\5
\1${}\{{}$\5
${}\index{printf+\\{printf}}\\{printf}(\.{"\%d"},\39\index{int defaults+\\{int\_defaults}}\\{int\_defaults}[\|i]);{}$\6
\&{if} ${}(\|i<\index{max default+\\{max\_default}}\\{max\_default}[\index{int kind+\\{int\_kind}}\\{int\_kind}]){}$\1\5
\index{printf+\\{printf}}\\{printf}(\.{",\ "});\2\6
\4${}\}{}$\2\6
\index{printf+\\{printf}}\\{printf}(\.{"\};\\n\\n"});
\U337.\Y
\fi

\M{341}

\subsection{Dimensions}

Notice that there are default values for the two dimensions \.{hsize} and \.{vsize}.
These are the ``design sizes'' for the hint file. While it might not be possible
to display the \HINT/ file using these values of \.{hsize} and \.{vsize},
these are the authors recommendation for the best ``viewing experience''.

\noindent
\Y\B\4\X339:default names\X${}\mathrel+\E{}$\6
\&{typedef} \&{enum} ${}\{{}$\1\6
${}\index{zero dimen no+\\{zero\_dimen\_no}}\\{zero\_dimen\_no}\K\T{0},\39\index{hsize dimen no+\\{hsize\_dimen\_no}}\\{hsize\_dimen\_no}\K\T{1},\39\index{vsize dimen no+\\{vsize\_dimen\_no}}\\{vsize\_dimen\_no}\K\T{2},\39\index{line skip limit no+\\{line\_skip\_limit\_no}}\\{line\_skip\_limit\_no}\K\T{3},\39\index{max depth no+\\{max\_depth\_no}}\\{max\_depth\_no}\K\T{4},\39\index{split max depth no+\\{split\_max\_depth\_no}}\\{split\_max\_depth\_no}\K\T{5},\39\index{hang indent no+\\{hang\_indent\_no}}\\{hang\_indent\_no}\K\T{6},\39\index{emergency stretch no+\\{emergency\_stretch\_no}}\\{emergency\_stretch\_no}%
\K\T{7},\39\index{quad no+\\{quad\_no}}\\{quad\_no}\K\T{8},\39\index{math quad no+\\{math\_quad\_no}}\\{math\_quad\_no}\K\T{9}{}$\2\6
${}\}{}$ \index{dimen no t+\&{dimen\_no\_t}}\&{dimen\_no\_t};\6
\8\#\&{define} \index{MAX DIMEN DEFAULT+\.{MAX\_DIMEN\_DEFAULT}}\.{MAX\_DIMEN\_DEFAULT}\5\index{math quad no+\\{math\_quad\_no}}\\{math\_quad\_no}
\Y
\fi

\M{342}

\Y\B\4\X342:define \\{dimen\_defaults}\X${}\E{}$\6
$\index{max default+\\{max\_default}}\\{max\_default}[\index{dimen kind+\\{dimen\_kind}}\\{dimen\_kind}]\K\index{MAX DIMEN DEFAULT+\.{MAX\_DIMEN\_DEFAULT}}\.{MAX\_DIMEN\_DEFAULT};{}$\6
${}\index{max fixed+\\{max\_fixed}}\\{max\_fixed}[\index{dimen kind+\\{dimen\_kind}}\\{dimen\_kind}]\K\index{zero dimen no+\\{zero\_dimen\_no}}\\{zero\_dimen\_no}{}$;\7
${}\index{dimen defaults+\\{dimen\_defaults}}\\{dimen\_defaults}[\index{zero dimen no+\\{zero\_dimen\_no}}\\{zero\_dimen\_no}]\K\T{0};{}$\6
${}\index{dimen defaults+\\{dimen\_defaults}}\\{dimen\_defaults}[\index{hsize dimen no+\\{hsize\_dimen\_no}}\\{hsize\_dimen\_no}]\K\T{6.5}*\T{72}*\index{ONE+\.{ONE}}\.{ONE};{}$\6
${}\index{dimen defaults+\\{dimen\_defaults}}\\{dimen\_defaults}[\index{vsize dimen no+\\{vsize\_dimen\_no}}\\{vsize\_dimen\_no}]\K\T{8.9}*\T{72}*\index{ONE+\.{ONE}}\.{ONE};{}$\6
${}\index{dimen defaults+\\{dimen\_defaults}}\\{dimen\_defaults}[\index{line skip limit no+\\{line\_skip\_limit\_no}}\\{line\_skip\_limit\_no}]\K\T{0};{}$\6
${}\index{dimen defaults+\\{dimen\_defaults}}\\{dimen\_defaults}[\index{split max depth no+\\{split\_max\_depth\_no}}\\{split\_max\_depth\_no}]\K\T{3.5}*\index{ONE+\.{ONE}}\.{ONE};{}$\6
${}\index{dimen defaults+\\{dimen\_defaults}}\\{dimen\_defaults}[\index{hang indent no+\\{hang\_indent\_no}}\\{hang\_indent\_no}]\K\T{0};{}$\6
${}\index{dimen defaults+\\{dimen\_defaults}}\\{dimen\_defaults}[\index{emergency stretch no+\\{emergency\_stretch\_no}}\\{emergency\_stretch\_no}]\K\T{0};{}$\6
${}\index{dimen defaults+\\{dimen\_defaults}}\\{dimen\_defaults}[\index{quad no+\\{quad\_no}}\\{quad\_no}]\K\T{10}*\index{ONE+\.{ONE}}\.{ONE};{}$\6
${}\index{dimen defaults+\\{dimen\_defaults}}\\{dimen\_defaults}[\index{math quad no+\\{math\_quad\_no}}\\{math\_quad\_no}]\K\T{10}*\index{ONE+\.{ONE}}\.{ONE}{}$;\7
\index{printf+\\{printf}}\\{printf}(\.{"dimen\_t\ dimen\_defau}\)\.{lts[MAX\_DIMEN\_DEFAUL}\)\.{T+1]=\{"});\6
\&{for} ${}(\|i\K\T{0};{}$ ${}\|i\Z\index{max default+\\{max\_default}}\\{max\_default}[\index{dimen kind+\\{dimen\_kind}}\\{dimen\_kind}];{}$ ${}\|i\PP){}$\5
\1${}\{{}$\5
${}\index{printf+\\{printf}}\\{printf}(\.{"0x\%x"},\39\index{dimen defaults+\\{dimen\_defaults}}\\{dimen\_defaults}[\|i]);{}$\6
\&{if} ${}(\|i<\index{max default+\\{max\_default}}\\{max\_default}[\index{dimen kind+\\{dimen\_kind}}\\{dimen\_kind}]){}$\1\5
\index{printf+\\{printf}}\\{printf}(\.{",\ "});\2\6
\4${}\}{}$\2\6
\index{printf+\\{printf}}\\{printf}(\.{"\};\\n\\n"});
\A346.
\U337.\Y
\fi

\M{343}

\subsection{Extended Dimensions}
Extended dimensions\index{extended dimension} can be used in a variety of nodes for example
kern\index{kern} and box\index{box} nodes.
We define three fixed extended dimensions: zero, hsize, and vsize.
In contrast to the \.{hsize} and \.{vsize} dimensions defined in the previous
section, the extended dimensions defined here are linear functions that always evaluate
to the current horizontal and vertical size in the viewer.

\Y\B\4\X339:default names\X${}\mathrel+\E{}$\6
\&{typedef} \&{enum} ${}\{{}$\1\6
${}\index{zero xdimen no+\\{zero\_xdimen\_no}}\\{zero\_xdimen\_no}\K\T{0},\39\index{hsize xdimen no+\\{hsize\_xdimen\_no}}\\{hsize\_xdimen\_no}\K\T{1},\39\index{vsize xdimen no+\\{vsize\_xdimen\_no}}\\{vsize\_xdimen\_no}\K\T{2}{}$\2\6
${}\}{}$ \index{xdimen no t+\&{xdimen\_no\_t}}\&{xdimen\_no\_t};\6
\8\#\&{define} \index{MAX XDIMEN DEFAULT+\.{MAX\_XDIMEN\_DEFAULT}}\.{MAX\_XDIMEN\_DEFAULT}\5\index{vsize xdimen no+\\{vsize\_xdimen\_no}}\\{vsize\_xdimen\_no}
\Y
\fi

\M{344}

\Y\B\4\X344:define \\{xdimen\_defaults}\X${}\E{}$\6
$\index{max default+\\{max\_default}}\\{max\_default}[\index{xdimen kind+\\{xdimen\_kind}}\\{xdimen\_kind}]\K\index{MAX XDIMEN DEFAULT+\.{MAX\_XDIMEN\_DEFAULT}}\.{MAX\_XDIMEN\_DEFAULT};{}$\6
${}\index{max fixed+\\{max\_fixed}}\\{max\_fixed}[\index{xdimen kind+\\{xdimen\_kind}}\\{xdimen\_kind}]\K\index{vsize xdimen no+\\{vsize\_xdimen\_no}}\\{vsize\_xdimen\_no}{}$;\7
\index{printf+\\{printf}}\\{printf}(\.{"xdimen\_t\ xdimen\_def}\)\.{aults[MAX\_XDIMEN\_DEF}\)\.{AULT+1]=\{"}\6
\.{"\{0x0,\ 0.0,\ 0.0\},\ \{0}\)\.{x0,\ 1.0,\ 0.0\},\ \{0x0,}\)\.{\ 0.0,\ 1.0\}"}\6
\.{"\};\\n\\n"});
\U337.\Y
\fi

\M{345}


\subsection{Glue}

There are predefined glue\index{glue} numbers that correspond to the skip parameters of \TeX.
The default values are taken from {\tt plain.tex}.

\Y\B\4\X339:default names\X${}\mathrel+\E{}$\6
\&{typedef} \&{enum} ${}\{{}$\1\6
${}\index{zero skip no+\\{zero\_skip\_no}}\\{zero\_skip\_no}\K\T{0},\39\index{fil skip no+\\{fil\_skip\_no}}\\{fil\_skip\_no}\K\T{1},\39\index{fill skip no+\\{fill\_skip\_no}}\\{fill\_skip\_no}\K\T{2},\39\index{line skip no+\\{line\_skip\_no}}\\{line\_skip\_no}\K\T{3},\39\index{baseline skip no+\\{baseline\_skip\_no}}\\{baseline\_skip\_no}\K\T{4},\39\\{above\_display\_skip\_no}\K\T{5},\39\index{below display skip no+\\{below\_display\_skip\_no}}\\{below\_display\_skip\_no}\K\T{6},\39\\{above\_display\_short%
\_skip\_no}\K\T{7},\39\index{below display short skip no+\\{below\_display\_short\_skip\_no}}\\{below\_display\_short\_skip\_no}\K\T{8},\39\index{left skip no+\\{left\_skip\_no}}\\{left\_skip\_no}\K\T{9},\39\index{right skip no+\\{right\_skip\_no}}\\{right\_skip\_no}\K\T{10},\39\index{top skip no+\\{top\_skip\_no}}\\{top\_skip\_no}\K\T{11},{}$\C{ used for page template 0 }\6
\index{split top skip no+\\{split\_top\_skip\_no}}\\{split\_top\_skip\_no}${}\K\T{12},\39\index{tab skip no+\\{tab\_skip\_no}}\\{tab\_skip\_no}\K\T{13},\39\index{par fill skip no+\\{par\_fill\_skip\_no}}\\{par\_fill\_skip\_no}\K\T{14}{}$\2\6
${}\}{}$ \index{glue no t+\&{glue\_no\_t}}\&{glue\_no\_t};\6
\8\#\&{define} \index{MAX GLUE DEFAULT+\.{MAX\_GLUE\_DEFAULT}}\.{MAX\_GLUE\_DEFAULT}\5\index{par fill skip no+\\{par\_fill\_skip\_no}}\\{par\_fill\_skip\_no}
\Y
\fi

\M{346}

\Y\B\4\X342:define \\{dimen\_defaults}\X${}\mathrel+\E{}$\6
$\index{max default+\\{max\_default}}\\{max\_default}[\index{glue kind+\\{glue\_kind}}\\{glue\_kind}]\K\index{MAX GLUE DEFAULT+\.{MAX\_GLUE\_DEFAULT}}\.{MAX\_GLUE\_DEFAULT};{}$\6
${}\index{max fixed+\\{max\_fixed}}\\{max\_fixed}[\index{glue kind+\\{glue\_kind}}\\{glue\_kind}]\K\index{fill skip no+\\{fill\_skip\_no}}\\{fill\_skip\_no};{}$\6
${}\index{glue defaults+\\{glue\_defaults}}\\{glue\_defaults}[\index{fil skip no+\\{fil\_skip\_no}}\\{fil\_skip\_no}].\|p.\|f\K\T{1.0};{}$\6
${}\index{glue defaults+\\{glue\_defaults}}\\{glue\_defaults}[\index{fil skip no+\\{fil\_skip\_no}}\\{fil\_skip\_no}].\|p.\|o\K\index{fil o+\\{fil\_o}}\\{fil\_o};{}$\6
${}\index{glue defaults+\\{glue\_defaults}}\\{glue\_defaults}[\index{fill skip no+\\{fill\_skip\_no}}\\{fill\_skip\_no}].\|p.\|f\K\T{1.0};{}$\6
${}\index{glue defaults+\\{glue\_defaults}}\\{glue\_defaults}[\index{fill skip no+\\{fill\_skip\_no}}\\{fill\_skip\_no}].\|p.\|o\K\index{fill o+\\{fill\_o}}\\{fill\_o}{}$;\7
${}\index{glue defaults+\\{glue\_defaults}}\\{glue\_defaults}[\index{line skip no+\\{line\_skip\_no}}\\{line\_skip\_no}].\|w.\|w\K\T{1}*\index{ONE+\.{ONE}}\.{ONE};{}$\6
${}\index{glue defaults+\\{glue\_defaults}}\\{glue\_defaults}[\index{baseline skip no+\\{baseline\_skip\_no}}\\{baseline\_skip\_no}].\|w.\|w\K\T{12}*\index{ONE+\.{ONE}}\.{ONE};{}$\6
${}\index{glue defaults+\\{glue\_defaults}}\\{glue\_defaults}[\\{above\_display\_skip\_no}].\|w.\|w\K\T{12}*\index{ONE+\.{ONE}}\.{ONE};{}$\6
${}\index{glue defaults+\\{glue\_defaults}}\\{glue\_defaults}[\\{above\_display\_skip\_no}].\|p.\|f\K\T{3.0};{}$\6
${}\index{glue defaults+\\{glue\_defaults}}\\{glue\_defaults}[\\{above\_display\_skip\_no}].\|p.\|o\K\index{normal o+\\{normal\_o}}\\{normal\_o};{}$\6
${}\index{glue defaults+\\{glue\_defaults}}\\{glue\_defaults}[\\{above\_display\_skip\_no}].\|m.\|f\K\T{9.0};{}$\6
${}\index{glue defaults+\\{glue\_defaults}}\\{glue\_defaults}[\\{above\_display\_skip\_no}].\|m.\|o\K\index{normal o+\\{normal\_o}}\\{normal\_o};{}$\6
${}\index{glue defaults+\\{glue\_defaults}}\\{glue\_defaults}[\index{below display skip no+\\{below\_display\_skip\_no}}\\{below\_display\_skip\_no}].\|w.\|w\K\T{12}*\index{ONE+\.{ONE}}\.{ONE};{}$\6
${}\index{glue defaults+\\{glue\_defaults}}\\{glue\_defaults}[\index{below display skip no+\\{below\_display\_skip\_no}}\\{below\_display\_skip\_no}].\|p.\|f\K\T{3.0};{}$\6
${}\index{glue defaults+\\{glue\_defaults}}\\{glue\_defaults}[\index{below display skip no+\\{below\_display\_skip\_no}}\\{below\_display\_skip\_no}].\|p.\|o\K\index{normal o+\\{normal\_o}}\\{normal\_o};{}$\6
${}\index{glue defaults+\\{glue\_defaults}}\\{glue\_defaults}[\index{below display skip no+\\{below\_display\_skip\_no}}\\{below\_display\_skip\_no}].\|m.\|f\K\T{9.0};{}$\6
${}\index{glue defaults+\\{glue\_defaults}}\\{glue\_defaults}[\index{below display skip no+\\{below\_display\_skip\_no}}\\{below\_display\_skip\_no}].\|m.\|o\K\index{normal o+\\{normal\_o}}\\{normal\_o};{}$\6
${}\index{glue defaults+\\{glue\_defaults}}\\{glue\_defaults}[\\{above\_display\_short\_skip\_no}].\|p.\|f\K\T{3.0};{}$\6
${}\index{glue defaults+\\{glue\_defaults}}\\{glue\_defaults}[\\{above\_display\_short\_skip\_no}].\|p.\|o\K\index{normal o+\\{normal\_o}}\\{normal\_o};{}$\6
${}\index{glue defaults+\\{glue\_defaults}}\\{glue\_defaults}[\index{below display short skip no+\\{below\_display\_short\_skip\_no}}\\{below\_display\_short\_skip\_no}].\|w.\|w\K\T{7}*\index{ONE+\.{ONE}}\.{ONE};{}$\6
${}\index{glue defaults+\\{glue\_defaults}}\\{glue\_defaults}[\index{below display short skip no+\\{below\_display\_short\_skip\_no}}\\{below\_display\_short\_skip\_no}].\|p.\|f\K\T{3.0};{}$\6
${}\index{glue defaults+\\{glue\_defaults}}\\{glue\_defaults}[\index{below display short skip no+\\{below\_display\_short\_skip\_no}}\\{below\_display\_short\_skip\_no}].\|p.\|o\K\index{normal o+\\{normal\_o}}\\{normal\_o};{}$\6
${}\index{glue defaults+\\{glue\_defaults}}\\{glue\_defaults}[\index{below display short skip no+\\{below\_display\_short\_skip\_no}}\\{below\_display\_short\_skip\_no}].\|m.\|f\K\T{4.0};{}$\6
${}\index{glue defaults+\\{glue\_defaults}}\\{glue\_defaults}[\index{below display short skip no+\\{below\_display\_short\_skip\_no}}\\{below\_display\_short\_skip\_no}].\|m.\|o\K\index{normal o+\\{normal\_o}}\\{normal\_o};{}$\6
${}\index{glue defaults+\\{glue\_defaults}}\\{glue\_defaults}[\index{top skip no+\\{top\_skip\_no}}\\{top\_skip\_no}].\|w.\|w\K\T{10}*\index{ONE+\.{ONE}}\.{ONE};{}$\6
${}\index{glue defaults+\\{glue\_defaults}}\\{glue\_defaults}[\index{split top skip no+\\{split\_top\_skip\_no}}\\{split\_top\_skip\_no}].\|w.\|w\K\T{8.5}*\index{ONE+\.{ONE}}\.{ONE};{}$\6
${}\index{glue defaults+\\{glue\_defaults}}\\{glue\_defaults}[\index{par fill skip no+\\{par\_fill\_skip\_no}}\\{par\_fill\_skip\_no}].\|p.\|f\K\T{1.0};{}$\6
${}\index{glue defaults+\\{glue\_defaults}}\\{glue\_defaults}[\index{par fill skip no+\\{par\_fill\_skip\_no}}\\{par\_fill\_skip\_no}].\|p.\|o\K\index{fil o+\\{fil\_o}}\\{fil\_o};{}$\6
\8\#\&{define} \index{PRINT GLUE+\.{PRINT\_GLUE}}\.{PRINT\_GLUE}(\|G)\5${}\index{printf+\\{printf}}\\{printf}(\.{"\{\{0x\%x,\ \%f,\ \%f\},\{\%f}\)\.{,\ \%d\},\{\%f,\ \%d\}\}"},\39\|G.\|w.\|w,\39\|G.\|w.\|h,\39\|G.\|w.\|v,\39\|G.\|p.\|f,\39\|G.\|p.\|o,\39\|G.\|m.\|f,\39\|G.\|m.\|o){}$\7
\index{printf+\\{printf}}\\{printf}(\.{"glue\_t\ glue\_default}\)\.{s[MAX\_GLUE\_DEFAULT+1}\)\.{]=\{\\n"});\6
\&{for} ${}(\|i\K\T{0};{}$ ${}\|i\Z\index{max default+\\{max\_default}}\\{max\_default}[\index{glue kind+\\{glue\_kind}}\\{glue\_kind}];{}$ ${}\|i\PP{}$)\6
\1${}\{{}$\5
\index{PRINT GLUE+\.{PRINT\_GLUE}}\.{PRINT\_GLUE}(\index{glue defaults+\\{glue\_defaults}}\\{glue\_defaults}[\|i]);\5
\&{if} ${}(\|i<\index{max default+\\{max\_default}}\\{max\_default}[\index{int kind+\\{int\_kind}}\\{int\_kind}]){}$\1\5
\index{printf+\\{printf}}\\{printf}(\.{",\\n"});\2\6
\4${}\}{}$\2\6
\index{printf+\\{printf}}\\{printf}(\.{"\};\\n\\n"});
\Y
\fi

\M{347}

We fix the glue definition with number zero to be the ``zero glue'': a
glue with width zero and zero stretchability and shrinkability. Here
is the reason: In the short format, the info bits of a glue node
indicate which components of a glue are nonzero.  Therefore the zero
glue should have an info value of zero---which on the other hand is
reserved for a reference to a glue definition. Hence, the best way to
represent a zero glue is as a predefined glue.


\subsection{Baseline Skips}

The zero baseline\index{baseline skip} which inserts no baseline skip is predefined.

\Y\B\4\X339:default names\X${}\mathrel+\E{}$\6
\&{typedef} \&{enum} ${}\{{}$\1\6
${}\index{zero baseline no+\\{zero\_baseline\_no}}\\{zero\_baseline\_no}\K\T{0}{}$\2\6
${}\}{}$ \index{baseline no t+\&{baseline\_no\_t}}\&{baseline\_no\_t};\6
\8\#\&{define} \index{MAX BASELINE DEFAULT+\.{MAX\_BASELINE\_DEFAULT}}\.{MAX\_BASELINE\_DEFAULT}\5\index{zero baseline no+\\{zero\_baseline\_no}}\\{zero\_baseline\_no}
\Y
\fi

\M{348}
\Y\B\4\X348:define \\{baseline\_defaults}\X${}\E{}$\6
$\index{max default+\\{max\_default}}\\{max\_default}[\index{baseline kind+\\{baseline\_kind}}\\{baseline\_kind}]\K\index{MAX BASELINE DEFAULT+\.{MAX\_BASELINE\_DEFAULT}}\.{MAX\_BASELINE\_DEFAULT};{}$\6
${}\index{max fixed+\\{max\_fixed}}\\{max\_fixed}[\index{baseline kind+\\{baseline\_kind}}\\{baseline\_kind}]\K\index{zero baseline no+\\{zero\_baseline\_no}}\\{zero\_baseline\_no}{}$;\1\7
\4${}\{{}$\5
\index{baseline t+\&{baseline\_t}}\&{baseline\_t} \|z${}\K\{\{\{\T{0}\}\}\};{}$\7
\index{printf+\\{printf}}\\{printf}(\.{"baseline\_t\ baseline}\)\.{\_defaults[MAX\_BASELI}\)\.{NE\_DEFAULT+1]=\{\{"});\6
${}\index{PRINT GLUE+\.{PRINT\_GLUE}}\.{PRINT\_GLUE}(\|z.\\{bs}){}$;\5
\index{printf+\\{printf}}\\{printf}(\.{",\ "});\5
${}\index{PRINT GLUE+\.{PRINT\_GLUE}}\.{PRINT\_GLUE}(\|z.\\{ls});{}$\6
${}\index{printf+\\{printf}}\\{printf}(\.{",\ 0x\%x\}\};\\n\\n"},\39\|z.\\{lsl});{}$\6
\4${}\}{}$\2
\U337.\Y
\fi

\M{349}

\subsection{Streams}
The zero stream\index{stream} is predefined for the main content.
\Y\B\4\X339:default names\X${}\mathrel+\E{}$\6
\&{typedef} \&{enum} ${}\{{}$\1\6
${}\index{zero stream no+\\{zero\_stream\_no}}\\{zero\_stream\_no}\K\T{0}{}$\2\6
${}\}{}$ \index{stream no t+\&{stream\_no\_t}}\&{stream\_no\_t};\6
\8\#\&{define} \index{MAX STREAM DEFAULT+\.{MAX\_STREAM\_DEFAULT}}\.{MAX\_STREAM\_DEFAULT}\5\index{zero stream no+\\{zero\_stream\_no}}\\{zero\_stream\_no}
\Y
\fi

\M{350}

\Y\B\4\X350:define \\{stream\_defaults}\X${}\E{}$\6
$\index{max default+\\{max\_default}}\\{max\_default}[\index{stream kind+\\{stream\_kind}}\\{stream\_kind}]\K\index{MAX STREAM DEFAULT+\.{MAX\_STREAM\_DEFAULT}}\.{MAX\_STREAM\_DEFAULT};{}$\6
${}\index{max fixed+\\{max\_fixed}}\\{max\_fixed}[\index{stream kind+\\{stream\_kind}}\\{stream\_kind}]\K\index{zero stream no+\\{zero\_stream\_no}}\\{zero\_stream\_no}{}$;
\U337.\Y
\fi

\M{351}


\subsection{Page Templates}

The zero page template\index{template} is a predefined, built-in page template.
\Y\B\4\X339:default names\X${}\mathrel+\E{}$\6
\&{typedef} \&{enum} ${}\{{}$\1\6
${}\index{zero page no+\\{zero\_page\_no}}\\{zero\_page\_no}\K\T{0}{}$\2\6
${}\}{}$ \index{page no t+\&{page\_no\_t}}\&{page\_no\_t};\6
\8\#\&{define} \index{MAX PAGE DEFAULT+\.{MAX\_PAGE\_DEFAULT}}\.{MAX\_PAGE\_DEFAULT}\5\index{zero page no+\\{zero\_page\_no}}\\{zero\_page\_no}
\Y
\fi

\M{352}

\Y\B\4\X352:define \\{page\_defaults}\X${}\E{}$\6
$\index{max default+\\{max\_default}}\\{max\_default}[\index{page kind+\\{page\_kind}}\\{page\_kind}]\K\index{MAX PAGE DEFAULT+\.{MAX\_PAGE\_DEFAULT}}\.{MAX\_PAGE\_DEFAULT};{}$\6
${}\index{max fixed+\\{max\_fixed}}\\{max\_fixed}[\index{page kind+\\{page\_kind}}\\{page\_kind}]\K\index{zero page no+\\{zero\_page\_no}}\\{zero\_page\_no}{}$;
\U337.\Y
\fi

\M{353}

\subsection{Page Ranges}

The page\index{page range} range for the zero page template is
the entire content section. It is predefined.

\Y\B\4\X339:default names\X${}\mathrel+\E{}$\6
\&{typedef} \&{enum} ${}\{{}$\1\6
${}\index{zero range no+\\{zero\_range\_no}}\\{zero\_range\_no}\K\T{0}{}$\2\6
${}\}{}$ \index{range no t+\&{range\_no\_t}}\&{range\_no\_t};\6
\8\#\&{define} \index{MAX RANGE DEFAULT+\.{MAX\_RANGE\_DEFAULT}}\.{MAX\_RANGE\_DEFAULT}\5\index{zero range no+\\{zero\_range\_no}}\\{zero\_range\_no}
\Y
\fi

\M{354}

\Y\B\4\X354:define \\{range\_defaults}\X${}\E{}$\6
$\index{max default+\\{max\_default}}\\{max\_default}[\index{range kind+\\{range\_kind}}\\{range\_kind}]\K\index{MAX RANGE DEFAULT+\.{MAX\_RANGE\_DEFAULT}}\.{MAX\_RANGE\_DEFAULT};{}$\6
${}\index{max fixed+\\{max\_fixed}}\\{max\_fixed}[\index{range kind+\\{range\_kind}}\\{range\_kind}]\K\index{zero range no+\\{zero\_range\_no}}\\{zero\_range\_no}{}$;
\U337.\Y
\fi

\M{355}


\section{Content Section}
The content section\index{content section} is just a list of nodes. Within the \.{shrink} program,
reading a node in long format will trigger writing the node in short format.
Similarly within the \index{stretch+\.{stretch}}\.{stretch} program, reading a node
in short form will cause writing it in long format. As a consequence,
the main task of writing the content section in long format is accomplished
by calling \index{get content+\\{get\_content}}\\{get\_content} and writing it in the short format
is accomplished by parsing the \index{content list+\nts{content\_list}}\nts{content\_list}.

%\readcode
\codesection{\redsymbol}{Reading the Long Format}\redindex{1}{6}{Content Section}
\label{content}%
\Y\par
\Y\B\4\X2:symbols\X${}\mathrel+\E{}$\6
\8\%\&{token} \index{CONTENT+\ts{CONTENT}}\ts{CONTENT}\5\.{"content"}
\Y
\fi

\M{356}

\Y\B\4\X3:scanning rules\X${}\mathrel+\E{}$\6
${}\8\re{\vb{content}}{}$\ac\&{return} \index{CONTENT+\ts{CONTENT}}\ts{CONTENT};\eac
\Y
\fi

\M{357}


\Y\B\4\X5:parsing rules\X${}\mathrel+\E{}$\6
\index{content section+\nts{content\_section}}\nts{content\_section}: \1\1\5
\index{START+\ts{START}}\ts{START}\5
\index{CONTENT+\ts{CONTENT}}\ts{CONTENT}\6
${}\{{}$\1\5
\index{hput content start+\\{hput\_content\_start}}\\{hput\_content\_start}(\,);\5
${}\}{}$\2\5
\index{content list+\nts{content\_list}}\nts{content\_list}\5
\index{END+\ts{END}}\ts{END}\5
${}\{{}$\1\5
\index{hput content end+\\{hput\_content\_end}}\\{hput\_content\_end}(\,);\5
\index{hput range defs+\\{hput\_range\_defs}}\\{hput\_range\_defs}(\,);\5
${}\}{}$\2;\2\2
\Y
\fi

\M{358}

%\writecode
\codesection{\wrtsymbol}{Writing the Long Format}\wrtindex{1}{6}{Content Section}

\Y\B\4\X19:write functions\X${}\mathrel+\E{}$\6
\&{void} \index{hwrite content section+\\{hwrite\_content\_section}}\\{hwrite\_content\_section}(\&{void})\1\1\2\2\1\6
\4${}\{{}$\5
${}\index{section no+\\{section\_no}}\\{section\_no}\K\T{2};{}$\6
\index{hwritef+\\{hwritef}}\\{hwritef}(\.{"<content"});\6
\index{hsort ranges+\\{hsort\_ranges}}\\{hsort\_ranges}(\,);\6
\index{hget content section+\\{hget\_content\_section}}\\{hget\_content\_section}(\,);\6
\index{hwritef+\\{hwritef}}\\{hwritef}(\.{"\\n>\\n"});\6
\4${}\}{}$\2
\Y
\fi

\M{359}

%\getcode
\codesection{\getsymbol}{Reading the Short Format}\getindex{1}{6}{Content Section}
\Y\B\4\X16:get functions\X${}\mathrel+\E{}$\6
\&{void} \index{hget content section+\\{hget\_content\_section}}\\{hget\_content\_section}(\,)\1\1\2\2\1\6
\4${}\{{}$\5
${}\.{DBG}(\index{DBGDIR+\.{DBGDIR}}\.{DBGDIR},\39\.{"Content\\n"});{}$\6
\index{hget section+\\{hget\_section}}\\{hget\_section}(\T{2});\6
\index{hwrite range+\\{hwrite\_range}}\\{hwrite\_range}(\,);\6
\&{while} ${}(\index{hpos+\\{hpos}}\\{hpos}<\index{hend+\\{hend}}\\{hend}){}$\1\5
\index{hget content node+\\{hget\_content\_node}}\\{hget\_content\_node}(\,);\2\6
\4${}\}{}$\2
\Y
\fi

\M{360}

%\putcode
\codesection{\putsymbol}{Writing the Short Format}\putindex{1}{6}{Content Section}
\Y\B\4\X12:put functions\X${}\mathrel+\E{}$\6
\&{void} \index{hput content start+\\{hput\_content\_start}}\\{hput\_content\_start}(\&{void})\1\1\2\2\1\6
\4${}\{{}$\5
${}\.{DBG}(\index{DBGDIR+\.{DBGDIR}}\.{DBGDIR},\39\.{"Content\ Section\\n"});{}$\6
${}\index{section no+\\{section\_no}}\\{section\_no}\K\T{2};{}$\6
${}\index{hpos+\\{hpos}}\\{hpos}\K\index{hstart+\\{hstart}}\\{hstart}\K\index{dir+\\{dir}}\\{dir}[\T{2}].\index{buffer+\\{buffer}}\\{buffer};{}$\6
${}\index{hend+\\{hend}}\\{hend}\K\index{hstart+\\{hstart}}\\{hstart}+\index{dir+\\{dir}}\\{dir}[\T{2}].\index{bsize+\\{bsize}}\\{bsize};{}$\6
\4${}\}{}$\2\7
\&{void} \index{hput content end+\\{hput\_content\_end}}\\{hput\_content\_end}(\&{void})\1\1\2\2\1\6
\4${}\{{}$\5
${}\index{dir+\\{dir}}\\{dir}[\T{2}].\index{size+\\{size}}\\{size}\K\index{hpos+\\{hpos}}\\{hpos}-\index{hstart+\\{hstart}}\\{hstart}{}$;\C{ Updating the directory entry }\6
${}\.{DBG}(\index{DBGDIR+\.{DBGDIR}}\.{DBGDIR},\39\.{"End\ Content\ Section}\)\.{,\ size=0x\%x\\n"},\39\index{dir+\\{dir}}\\{dir}[\T{2}].\index{size+\\{size}}\\{size});{}$\6
\4${}\}{}$\2
\Y
\fi

\M{361}


\section{Processing the Command Line}
The following code explains the command line\index{command line}
parameters and options\index{option}\index{debugging}.
It tells us what to expect in the rest of this section.
\Y\B\4\X361:explain usage\X${}\E{}$\6
$\index{fprintf+\\{fprintf}}\\{fprintf}(\index{stderr+\\{stderr}}\\{stderr},\39\.{"Usage:\ \%s\ [options]}\)\.{\ filename\%s\\n"},\39\index{prog name+\\{prog\_name}}\\{prog\_name},\39\index{in ext+\\{in\_ext}}\\{in\_ext}){}$;\6
${}\index{fprintf+\\{fprintf}}\\{fprintf}(\index{stderr+\\{stderr}}\\{stderr},\39{}$\.{"Options:\\n"}\6
\.{"\\t\ -o\ file\\t\ specif}\)\.{y\ an\ output\ file\ nam}\)\.{e\\n"}\6
\.{"\\t\ -g\ \ \ \ \ \\t\ assume}\)\.{\ global\ names\ for\ au}\)\.{xiliar\ files\\n"}\6
\.{"\\t\ -l\ \ \ \ \ \\t\ redire}\)\.{ct\ stderr\ to\ a\ log\ f}\)\.{ile\\n"}\6
\.{"\\t\ -u\ \ \ \ \ \\t\ enable}\)\.{\ writing\ utf8\ charac}\)\.{ter\ codes\\n"}\6
\.{"\\t\ -x\ \ \ \ \ \\t\ enable}\)\.{\ writing\ hexadecimal}\)\.{\ character\ codes\\n"}\6
\.{"\\t\ -c\ \ \ \ \ \\t\ enable}\)\.{\ compression\ of\ sect}\)\.{ion\ 1\ and\ 2\\n"});\6
\8\#\&{ifdef} \index{DEBUG+\.{DEBUG}}\.{DEBUG}\6
${}\index{fprintf+\\{fprintf}}\\{fprintf}(\index{stderr+\\{stderr}}\\{stderr},\39\.{"\\t\ -d\ XXX\ \\t\ hexade}\)\.{cimal\ value.\ OR\ toge}\)\.{ther\ these\ values:\\n}\)\.{"}){}$;\6
${}\index{fprintf+\\{fprintf}}\\{fprintf}(\index{stderr+\\{stderr}}\\{stderr},\39\.{"\\t\\t\\t\ XX=\%03X\ \ \ ba}\)\.{sic\ debugging\\n"},\39\index{DBGBASIC+\.{DBGBASIC}}\.{DBGBASIC}){}$;\6
${}\index{fprintf+\\{fprintf}}\\{fprintf}(\index{stderr+\\{stderr}}\\{stderr},\39\.{"\\t\\t\\t\ XX=\%03X\ \ \ ta}\)\.{g\ debugging\\n"},\39\index{DBGTAGS+\.{DBGTAGS}}\.{DBGTAGS}){}$;\6
${}\index{fprintf+\\{fprintf}}\\{fprintf}(\index{stderr+\\{stderr}}\\{stderr},\39\.{"\\t\\t\\t\ XX=\%03X\ \ \ no}\)\.{de\ debugging\\n"},\39\index{DBGNODE+\.{DBGNODE}}\.{DBGNODE}){}$;\6
${}\index{fprintf+\\{fprintf}}\\{fprintf}(\index{stderr+\\{stderr}}\\{stderr},\39\.{"\\t\\t\\t\ XX=\%03X\ \ \ de}\)\.{finition\ debugging\\n}\)\.{"},\39\index{DBGDEF+\.{DBGDEF}}\.{DBGDEF}){}$;\6
${}\index{fprintf+\\{fprintf}}\\{fprintf}(\index{stderr+\\{stderr}}\\{stderr},\39\.{"\\t\\t\\t\ XX=\%03X\ \ \ di}\)\.{rectory\ debugging\\n"},\39\index{DBGDIR+\.{DBGDIR}}\.{DBGDIR}){}$;\6
${}\index{fprintf+\\{fprintf}}\\{fprintf}(\index{stderr+\\{stderr}}\\{stderr},\39\.{"\\t\\t\\t\ XX=\%03X\ \ \ ra}\)\.{nge\ debugging\\n"},\39\index{DBGRANGE+\.{DBGRANGE}}\.{DBGRANGE}){}$;\6
${}\index{fprintf+\\{fprintf}}\\{fprintf}(\index{stderr+\\{stderr}}\\{stderr},\39\.{"\\t\\t\\t\ XX=\%03X\ \ \ fl}\)\.{oat\ debugging\\n"},\39\index{DBGFLOAT+\.{DBGFLOAT}}\.{DBGFLOAT}){}$;\6
${}\index{fprintf+\\{fprintf}}\\{fprintf}(\index{stderr+\\{stderr}}\\{stderr},\39\.{"\\t\\t\\t\ XX=\%03X\ \ \ co}\)\.{mpression\ debugging\\}\)\.{n"},\39\index{DBGCOMPRESS+\.{DBGCOMPRESS}}\.{DBGCOMPRESS}){}$;\6
${}\index{fprintf+\\{fprintf}}\\{fprintf}(\index{stderr+\\{stderr}}\\{stderr},\39\.{"\\t\\t\\t\ XX=\%03X\ \ \ bu}\)\.{ffer\ debugging\\n"},\39\index{DBGBUFFER+\.{DBGBUFFER}}\.{DBGBUFFER}){}$;\6
${}\index{fprintf+\\{fprintf}}\\{fprintf}(\index{stderr+\\{stderr}}\\{stderr},\39\.{"\\t\\t\\t\ XX=\%03X\ \ \ fl}\)\.{ex\ debugging\\n"},\39\index{DBGFLEX+\.{DBGFLEX}}\.{DBGFLEX}){}$;\6
${}\index{fprintf+\\{fprintf}}\\{fprintf}(\index{stderr+\\{stderr}}\\{stderr},\39\.{"\\t\\t\\t\ XX=\%03X\ \ \ bi}\)\.{son\ debugging\\n"},\39\index{DBGBISON+\.{DBGBISON}}\.{DBGBISON}){}$;\6
${}\index{fprintf+\\{fprintf}}\\{fprintf}(\index{stderr+\\{stderr}}\\{stderr},\39\.{"\\t\\t\\t\ XX=\%03X\ \ \ Te}\)\.{X\ debugging\\n"},\39\index{DBGTEX+\.{DBGTEX}}\.{DBGTEX}){}$;\6
${}\index{fprintf+\\{fprintf}}\\{fprintf}(\index{stderr+\\{stderr}}\\{stderr},\39\.{"\\t\\t\\t\ XX=\%03X\ \ \ Pa}\)\.{ge\ debugging\\n"},\39\index{DBGPAGE+\.{DBGPAGE}}\.{DBGPAGE}){}$;\6
${}\index{fprintf+\\{fprintf}}\\{fprintf}(\index{stderr+\\{stderr}}\\{stderr},\39\.{"\\t\\t\\t\ XX=\%03X\ \ \ Fo}\)\.{nt\ debugging\\n"},\39\index{DBGFONT+\.{DBGFONT}}\.{DBGFONT}){}$;\6
${}\index{fprintf+\\{fprintf}}\\{fprintf}(\index{stderr+\\{stderr}}\\{stderr},\39\.{"\\t\\t\\t\ XX=\%03X\ \ \ Re}\)\.{nder\ debugging\\n"},\39\index{DBGRENDER+\.{DBGRENDER}}\.{DBGRENDER}){}$;\6
\8\#\&{endif}
\Us438, 439\ETs441.\Y
\fi

\M{362}
We define constants for different debug flags.
\Y\B\4\X362:debug constants\X${}\E{}$\6
\8\#\&{define} \index{DBGNONE+\.{DBGNONE}}\.{DBGNONE}\5\T{\^0}\6
\8\#\&{define} \index{DBGBASIC+\.{DBGBASIC}}\.{DBGBASIC}\5\T{\^1}\6
\8\#\&{define} \index{DBGTAGS+\.{DBGTAGS}}\.{DBGTAGS}\5\T{\^2}\6
\8\#\&{define} \index{DBGNODE+\.{DBGNODE}}\.{DBGNODE}\5\T{\^4}\6
\8\#\&{define} \index{DBGDEF+\.{DBGDEF}}\.{DBGDEF}\5\T{\^8}\6
\8\#\&{define} \index{DBGDIR+\.{DBGDIR}}\.{DBGDIR}\5\T{\^10}\6
\8\#\&{define} \index{DBGRANGE+\.{DBGRANGE}}\.{DBGRANGE}\5\T{\^20}\6
\8\#\&{define} \index{DBGFLOAT+\.{DBGFLOAT}}\.{DBGFLOAT}\5\T{\^40}\6
\8\#\&{define} \index{DBGCOMPRESS+\.{DBGCOMPRESS}}\.{DBGCOMPRESS}\5\T{\^80}\6
\8\#\&{define} \index{DBGBUFFER+\.{DBGBUFFER}}\.{DBGBUFFER}\5\T{\^100}\6
\8\#\&{define} \index{DBGFLEX+\.{DBGFLEX}}\.{DBGFLEX}\5\T{\^200}\6
\8\#\&{define} \index{DBGBISON+\.{DBGBISON}}\.{DBGBISON}\5\T{\^400}\6
\8\#\&{define} \index{DBGTEX+\.{DBGTEX}}\.{DBGTEX}\5\T{\^800}\6
\8\#\&{define} \index{DBGPAGE+\.{DBGPAGE}}\.{DBGPAGE}\5\T{\^1000}\6
\8\#\&{define} \index{DBGFONT+\.{DBGFONT}}\.{DBGFONT}\5\T{\^2000}\6
\8\#\&{define} \index{DBGRENDER+\.{DBGRENDER}}\.{DBGRENDER}\5\T{\^4000}
\U336.\Y
\fi

\M{363}

Next we define variables.
Some of these variables go into {\tt hformat.c} because
it enables us to reuse them in other programs.
Some are common variables that are
needed in all three programs defined here.
And some variables are just local variables in the \index{main+\\{main}}\\{main} program.

The variable \index{in name+\\{in\_name}}\\{in\_name} is not local to \index{main+\\{main}}\\{main} because it is
used in the function \index{hget map+\\{hget\_map}}\\{hget\_map} (see page\pageref{map}).
The variable \index{stem name+\\{stem\_name}}\\{stem\_name} contains the name of the input file
not including the extension. The space allocated for it
is large enough to append an extension with up to five characters.
It can be used with the extension {\tt .log} for the log file,
with {\tt .HINT} or {\tt .hnt} for the output file,
and with {\tt .abs} or {\tt .rel} when writing or reading the auxiliar sections.

The {\tt stretch} program will overwrite the \index{stem name+\\{stem\_name}}\\{stem\_name}
using the name of the output file if it is set with the {\tt -o}
option.

\Y\B\4\X363:variables in {\tt hformat.c}\X${}\E{}$\6
\.{"unsigned\ int\ debugf}\)\.{lags=DBGNONE;\\n"}
\A368.
\U337.\Y
\fi

\M{364}

\Y\B\4\X252:common variables\X${}\mathrel+\E{}$\6
\&{bool} \index{option utf8+\\{option\_utf8}}\\{option\_utf8}${}\K\\{false};{}$\6
\&{bool} \index{option hex+\\{option\_hex}}\\{option\_hex}${}\K\\{false};{}$\6
\&{bool} \index{option force+\\{option\_force}}\\{option\_force}${}\K\\{false};{}$\6
\&{bool} \index{option global+\\{option\_global}}\\{option\_global}${}\K\\{false};{}$\6
\&{bool} \index{option compress+\\{option\_compress}}\\{option\_compress}${}\K\\{false};{}$\6
\&{char} ${}{*}\index{in name+\\{in\_name}}\\{in\_name};{}$\6
\&{char} ${}{*}\index{stem name+\\{stem\_name}}\\{stem\_name};{}$\6
\&{int} \index{stem length+\\{stem\_length}}\\{stem\_length}${}\K\T{0}{}$;
\Y
\fi

\M{365}

\Y\B\4\X365:local variables in \\{main}\X${}\E{}$\6
\&{char} ${}{*}\index{prog name+\\{prog\_name}}\\{prog\_name};{}$\6
\&{char} ${}{*}\index{in ext+\\{in\_ext}}\\{in\_ext};{}$\6
\&{char} ${}{*}\index{out ext+\\{out\_ext}}\\{out\_ext};{}$\6
\&{char} ${}{*}\index{file name+\\{file\_name}}\\{file\_name}\K\NULL;{}$\6
\&{int} \index{file name length+\\{file\_name\_length}}\\{file\_name\_length}${}\K\T{0};{}$\6
\&{bool} \index{option log+\\{option\_log}}\\{option\_log}${}\K\\{false}{}$;
\Us438, 439\ETs441.\Y
\fi

\M{366}

Processing the command line looks for options and then sets the
input file name\index{file name}.

\Y\B\4\X366:process the command line\X${}\E{}$\6
$\index{debugflags+\\{debugflags}}\\{debugflags}\K\index{DBGBASIC+\.{DBGBASIC}}\.{DBGBASIC};{}$\6
${}\index{prog name+\\{prog\_name}}\\{prog\_name}\K\index{argv+\\{argv}}\\{argv}[\T{0}];{}$\6
\&{if} ${}(\index{argc+\\{argc}}\\{argc}<\T{2}){}$\1\5
\&{goto} \index{explain usage+\\{explain\_usage}}\\{explain\_usage};\2\6
${}\index{argv+\\{argv}}\\{argv}\PP{}$;\C{ skip the program name }\6
\&{while} ${}({*}\index{argv+\\{argv}}\\{argv}\I\NULL){}$\5
\1${}\{{}$\6
\&{if} ${}(({*}\index{argv+\\{argv}}\\{argv})[\T{0}]\E\.{'-'}){}$\5
\1${}\{{}$\5
\&{char} \index{option+\\{option}}\\{option}${}\K({*}\index{argv+\\{argv}}\\{argv})[\T{1}];{}$\7
\&{switch} (\index{option+\\{option}}\\{option})\5
\1${}\{{}$\6
\4\&{default}:\5
\&{goto} \index{explain usage+\\{explain\_usage}}\\{explain\_usage};\6
\4\&{case} \.{'o'}:\5
${}\index{argv+\\{argv}}\\{argv}\PP;{}$\6
${}\index{file name length+\\{file\_name\_length}}\\{file\_name\_length}\K{}$(\&{int}) \index{strlen+\\{strlen}}\\{strlen}${}({*}\index{argv+\\{argv}}\\{argv});{}$\6
${}\index{ALLOCATE+\.{ALLOCATE}}\.{ALLOCATE}(\index{file name+\\{file\_name}}\\{file\_name},\39\index{file name length+\\{file\_name\_length}}\\{file\_name\_length}+\T{6},\39\&{char}){}$;\C{ extra space for extension }\6
${}\index{strcpy+\\{strcpy}}\\{strcpy}(\index{file name+\\{file\_name}}\\{file\_name},\39{*}\index{argv+\\{argv}}\\{argv}){}$;\5
\&{break};\6
\4\&{case} \.{'l'}:\5
${}\index{option log+\\{option\_log}}\\{option\_log}\K\\{true}{}$;\5
\&{break};\6
\4\&{case} \.{'u'}:\5
${}\index{option utf8+\\{option\_utf8}}\\{option\_utf8}\K\\{true}{}$;\5
\&{break};\6
\4\&{case} \.{'x'}:\5
${}\index{option hex+\\{option\_hex}}\\{option\_hex}\K\\{true}{}$;\5
\&{break};\6
\4\&{case} \.{'f'}:\5
${}\index{option force+\\{option\_force}}\\{option\_force}\K\\{true}{}$;\5
\&{break};\6
\4\&{case} \.{'g'}:\5
${}\index{option global+\\{option\_global}}\\{option\_global}\K\\{true}{}$;\5
\&{break};\6
\4\&{case} \.{'c'}:\5
${}\index{option compress+\\{option\_compress}}\\{option\_compress}\K\\{true}{}$;\5
\&{break};\6
\4\&{case} \.{'d'}:\6
${}\index{argv+\\{argv}}\\{argv}\PP;{}$\6
\&{if} ${}({*}\index{argv+\\{argv}}\\{argv}\E\NULL){}$\1\5
\&{goto} \index{explain usage+\\{explain\_usage}}\\{explain\_usage};\2\6
${}\index{debugflags+\\{debugflags}}\\{debugflags}\K\index{strtol+\\{strtol}}\\{strtol}({*}\index{argv+\\{argv}}\\{argv},\39\NULL,\39\T{16});{}$\6
\&{break};\6
\4${}\}{}$\2\6
\4${}\}{}$\2\6
\&{else}\C{ the input file name }\6
\1${}\{{}$\5
\&{int} \index{path length+\\{path\_length}}\\{path\_length}${}\K{}$(\&{int}) \index{strlen+\\{strlen}}\\{strlen}${}({*}\index{argv+\\{argv}}\\{argv});{}$\6
\&{int} \index{ext length+\\{ext\_length}}\\{ext\_length}${}\K{}$(\&{int}) \index{strlen+\\{strlen}}\\{strlen}(\index{in ext+\\{in\_ext}}\\{in\_ext});\7
${}\index{ALLOCATE+\.{ALLOCATE}}\.{ALLOCATE}(\index{in name+\\{in\_name}}\\{in\_name},\39\index{path length+\\{path\_length}}\\{path\_length}+\index{ext length+\\{ext\_length}}\\{ext\_length}+\T{1},\39\&{char});{}$\6
${}\index{strcpy+\\{strcpy}}\\{strcpy}(\index{in name+\\{in\_name}}\\{in\_name},\39{*}\index{argv+\\{argv}}\\{argv});{}$\6
\&{if} ${}(\index{path length+\\{path\_length}}\\{path\_length}<\index{ext length+\\{ext\_length}}\\{ext\_length}\V\index{strncmp+\\{strncmp}}\\{strncmp}(\index{in name+\\{in\_name}}\\{in\_name}+\index{path length+\\{path\_length}}\\{path\_length}-\index{ext length+\\{ext\_length}}\\{ext\_length},\39\index{in ext+\\{in\_ext}}\\{in\_ext},\39\index{ext length+\\{ext\_length}}\\{ext\_length})\I\T{0}){}$\5
\1${}\{{}$\5
${}\index{strcat+\\{strcat}}\\{strcat}(\index{in name+\\{in\_name}}\\{in\_name},\39\index{in ext+\\{in\_ext}}\\{in\_ext});{}$\6
${}\index{path length+\\{path\_length}}\\{path\_length}\MRL{+{\K}}\index{ext length+\\{ext\_length}}\\{ext\_length};{}$\6
\4${}\}{}$\2\6
${}\index{stem length+\\{stem\_length}}\\{stem\_length}\K\index{path length+\\{path\_length}}\\{path\_length}-\index{ext length+\\{ext\_length}}\\{ext\_length};{}$\6
${}\index{ALLOCATE+\.{ALLOCATE}}\.{ALLOCATE}(\index{stem name+\\{stem\_name}}\\{stem\_name},\39\index{stem length+\\{stem\_length}}\\{stem\_length}+\T{6},\39\&{char});{}$\6
${}\index{strncpy+\\{strncpy}}\\{strncpy}(\index{stem name+\\{stem\_name}}\\{stem\_name},\39\index{in name+\\{in\_name}}\\{in\_name},\39\index{stem length+\\{stem\_length}}\\{stem\_length});{}$\6
${}\index{stem name+\\{stem\_name}}\\{stem\_name}[\index{stem length+\\{stem\_length}}\\{stem\_length}]\K\T{0};{}$\6
\&{if} ${}({*}(\index{argv+\\{argv}}\\{argv}+\T{1})\I\NULL){}$\1\5
\&{goto} \index{explain usage+\\{explain\_usage}}\\{explain\_usage};\2\6
\4${}\}{}$\2\6
${}\index{argv+\\{argv}}\\{argv}\PP;{}$\6
\4${}\}{}$\2
\Us438, 439\ETs441.\Y
\fi

\M{367}

After the command line has been processed, three file streams need to be opened:
The input file \index{hin+\\{hin}}\\{hin}\index{input file} and the output file \index{hout+\\{hout}}\\{hout}\index{output file}.
Further we need a log file \index{hlog+\\{hlog}}\\{hlog}\index{log file} if debugging is enabled.
For technical reasons, the scanner\index{scanning} generated by \.{flex} needs
an input file \index{yyin+\\{yyin}}\\{yyin}\index{input file} which is set to \index{hin+\\{hin}}\\{hin}
and an output file \index{yyout+\\{yyout}}\\{yyout} (which is not used).

\Y\B\4\X252:common variables\X${}\mathrel+\E{}$\6
\&{FILE} ${}{*}\index{hin+\\{hin}}\\{hin}\K\NULL,{}$ ${}{*}\index{hout+\\{hout}}\\{hout}\K\NULL{}$;
\Y
\fi

\M{368}
\Y\B\4\X363:variables in {\tt hformat.c}\X${}\mathrel+\E{}$\6
\.{"FILE\ *hlog=NULL;\\n"}
\Y
\fi

\M{369}

The log file is opened first because
this is the place where error messages\index{error message}
should go while the other files are opened.
It inherits its name from the input file name.

\Y\B\4\X369:open the log file\X${}\E{}$\6
\8\#\&{ifdef} \index{DEBUG+\.{DEBUG}}\.{DEBUG}\6
\&{if} (\index{option log+\\{option\_log}}\\{option\_log})\5
\1${}\{{}$\5
${}\index{strcat+\\{strcat}}\\{strcat}(\index{stem name+\\{stem\_name}}\\{stem\_name},\39\.{".log"});{}$\6
${}\index{hlog+\\{hlog}}\\{hlog}\K\index{freopen+\\{freopen}}\\{freopen}(\index{stem name+\\{stem\_name}}\\{stem\_name},\39\.{"w"},\39\index{stderr+\\{stderr}}\\{stderr});{}$\6
\&{if} ${}(\index{hlog+\\{hlog}}\\{hlog}\E\NULL){}$\5
\1${}\{{}$\5
${}\index{fprintf+\\{fprintf}}\\{fprintf}(\index{stderr+\\{stderr}}\\{stderr},\39\.{"Unable\ to\ open\ logf}\)\.{ile\ \%s"},\39\index{stem name+\\{stem\_name}}\\{stem\_name});{}$\6
${}\index{hlog+\\{hlog}}\\{hlog}\K\index{stderr+\\{stderr}}\\{stderr};{}$\6
\4${}\}{}$\2\6
${}\index{stem name+\\{stem\_name}}\\{stem\_name}[\index{stem length+\\{stem\_length}}\\{stem\_length}]\K\T{0};{}$\6
\4${}\}{}$\2\6
\&{else}\1\5
${}\index{hlog+\\{hlog}}\\{hlog}\K\index{stderr+\\{stderr}}\\{stderr};{}$\2\6
\8\#\&{else}\6
${}\index{hlog+\\{hlog}}\\{hlog}\K\index{stderr+\\{stderr}}\\{stderr};{}$\6
\8\#\&{endif}
\Us438, 439\ETs441.\Y
\fi

\M{370}

Once we have established logging, we can try to open the other files.
\Y\B\4\X370:open the input file\X${}\E{}$\6
$\index{hin+\\{hin}}\\{hin}\K\index{fopen+\\{fopen}}\\{fopen}(\index{in name+\\{in\_name}}\\{in\_name},\39\.{"rb"});{}$\6
\&{if} ${}(\index{hin+\\{hin}}\\{hin}\E\NULL){}$\1\5
${}\.{QUIT}(\.{"Unable\ to\ open\ inpu}\)\.{t\ file\ \%s"},\39\index{in name+\\{in\_name}}\\{in\_name}){}$;\2
\U438.\Y
\fi

\M{371}

\Y\B\4\X371:open the output file\X${}\E{}$\6
\&{if} ${}(\index{file name+\\{file\_name}}\\{file\_name}\I\NULL){}$\5
\1${}\{{}$\5
\&{int} \index{ext length+\\{ext\_length}}\\{ext\_length}${}\K{}$(\&{int}) \index{strlen+\\{strlen}}\\{strlen}(\index{out ext+\\{out\_ext}}\\{out\_ext});\7
\&{if} ${}(\index{file name length+\\{file\_name\_length}}\\{file\_name\_length}\Z\index{ext length+\\{ext\_length}}\\{ext\_length}\V\index{strncmp+\\{strncmp}}\\{strncmp}(\index{file name+\\{file\_name}}\\{file\_name}+\index{file name length+\\{file\_name\_length}}\\{file\_name\_length}-\index{ext length+\\{ext\_length}}\\{ext\_length},\39\index{out ext+\\{out\_ext}}\\{out\_ext},\39\index{ext length+\\{ext\_length}}\\{ext\_length})\I\T{0}){}$\5
\1${}\{{}$\5
${}\index{strcat+\\{strcat}}\\{strcat}(\index{file name+\\{file\_name}}\\{file\_name},\39\index{out ext+\\{out\_ext}}\\{out\_ext});{}$\6
${}\index{file name length+\\{file\_name\_length}}\\{file\_name\_length}\MRL{+{\K}}\index{ext length+\\{ext\_length}}\\{ext\_length};{}$\6
\4${}\}{}$\2\6
\4${}\}{}$\2\6
\&{else}\5
\1${}\{{}$\5
${}\index{file name length+\\{file\_name\_length}}\\{file\_name\_length}\K\index{stem length+\\{stem\_length}}\\{stem\_length}+{}$(\&{int}) \index{strlen+\\{strlen}}\\{strlen}(\index{out ext+\\{out\_ext}}\\{out\_ext});\6
${}\index{ALLOCATE+\.{ALLOCATE}}\.{ALLOCATE}(\index{file name+\\{file\_name}}\\{file\_name},\39\index{file name length+\\{file\_name\_length}}\\{file\_name\_length}+\T{1},\39\&{char});{}$\6
${}\index{strcpy+\\{strcpy}}\\{strcpy}(\index{file name+\\{file\_name}}\\{file\_name},\39\index{stem name+\\{stem\_name}}\\{stem\_name}){}$;\5
${}\index{strcpy+\\{strcpy}}\\{strcpy}(\index{file name+\\{file\_name}}\\{file\_name}+\index{stem length+\\{stem\_length}}\\{stem\_length},\39\index{out ext+\\{out\_ext}}\\{out\_ext});{}$\6
\4${}\}{}$\2\6
\X290:make sure the path in \\{file\_name} exists\X\6
${}\index{hout+\\{hout}}\\{hout}\K\index{fopen+\\{fopen}}\\{fopen}(\index{file name+\\{file\_name}}\\{file\_name},\39\.{"wb"});{}$\6
\&{if} ${}(\index{hout+\\{hout}}\\{hout}\E\NULL){}$\1\5
${}\.{QUIT}(\.{"Unable\ to\ open\ outp}\)\.{ut\ file\ \%s"},\39\index{file name+\\{file\_name}}\\{file\_name}){}$;\2
\Us438\ET439.\Y
\fi

\M{372}

The {\tt stretch} program will replace the \index{stem name+\\{stem\_name}}\\{stem\_name} using the stem of the
output file.
\Y\B\4\X372:determine the \\{stem\_name} from the output \\{file\_name}\X${}\E{}$\6
$\index{stem length+\\{stem\_length}}\\{stem\_length}\K\index{file name length+\\{file\_name\_length}}\\{file\_name\_length}-{}$(\&{int}) \index{strlen+\\{strlen}}\\{strlen}(\index{out ext+\\{out\_ext}}\\{out\_ext});\6
${}\index{ALLOCATE+\.{ALLOCATE}}\.{ALLOCATE}(\index{stem name+\\{stem\_name}}\\{stem\_name},\39\index{stem length+\\{stem\_length}}\\{stem\_length}+\T{6},\39\&{char});{}$\6
${}\index{strncpy+\\{strncpy}}\\{strncpy}(\index{stem name+\\{stem\_name}}\\{stem\_name},\39\index{file name+\\{file\_name}}\\{file\_name},\39\index{stem length+\\{stem\_length}}\\{stem\_length});{}$\6
${}\index{stem name+\\{stem\_name}}\\{stem\_name}[\index{stem length+\\{stem\_length}}\\{stem\_length}]\K\T{0}{}$;
\U439.\Y
\fi

\M{373}

At the very end, we will close the files again.
\Y\B\4\X373:close the input file\X${}\E{}$\6
\&{if} ${}(\index{in name+\\{in\_name}}\\{in\_name}\I\NULL){}$\1\5
\index{free+\\{free}}\\{free}(\index{in name+\\{in\_name}}\\{in\_name});\2\6
\&{if} ${}(\index{hin+\\{hin}}\\{hin}\I\NULL){}$\1\5
\index{fclose+\\{fclose}}\\{fclose}(\index{hin+\\{hin}}\\{hin});\2
\U438.\Y
\fi

\M{374}
\Y\B\4\X374:close the output file\X${}\E{}$\6
\&{if} ${}(\index{file name+\\{file\_name}}\\{file\_name}\I\NULL){}$\1\5
\index{free+\\{free}}\\{free}(\index{file name+\\{file\_name}}\\{file\_name});\2\6
\&{if} ${}(\index{hout+\\{hout}}\\{hout}\I\NULL){}$\1\5
\index{fclose+\\{fclose}}\\{fclose}(\index{hout+\\{hout}}\\{hout});\2
\Us438\ET439.\Y
\fi

\M{375}
\Y\B\4\X375:close the log file\X${}\E{}$\6
\&{if} ${}(\index{hlog+\\{hlog}}\\{hlog}\I\NULL){}$\1\5
\index{fclose+\\{fclose}}\\{fclose}(\index{hlog+\\{hlog}}\\{hlog});\2\6
\&{if} ${}(\index{stem name+\\{stem\_name}}\\{stem\_name}\I\NULL){}$\1\5
\index{free+\\{free}}\\{free}(\index{stem name+\\{stem\_name}}\\{stem\_name});\2
\Us438, 439\ETs441.\Y
\fi

\M{376}



\section{Error Handling and Debugging}\label{error_section}
There is no good program without good error handling\index{error message}\index{debugging}.
To print messages\index{message} or indicate errors, I define the following macros:
\index{MESSAGE+\.{MESSAGE}}\index{QUIT+\.{QUIT}}

\Y\B\4\X376:\.{error.h }\X${}\E{}$\6
\8\#\&{ifndef} \.{\_ERROR\_H}\6
\8\#\&{define} \.{\_ERROR\_H}\6
\8\#\&{include} \.{<stdlib.h>}\6
\8\#\&{include} \.{<stdio.h>}\6
\&{extern} \&{FILE} ${}{*}\index{hlog+\\{hlog}}\\{hlog};{}$\6
\&{extern} \&{uint8\_t} ${}{*}\index{hpos+\\{hpos}}\\{hpos},{}$ ${}{*}\index{hstart+\\{hstart}}\\{hstart};{}$\6
\8\#\&{define} ${}\index{LOG+\.{LOG}}\.{LOG}(\,\ldots\,)\5(\index{fprintf+\\{fprintf}}\\{fprintf}(\index{hlog+\\{hlog}}\\{hlog},\39\.{\_\_VA\_ARGS\_\_}),\39\index{fflush+\\{fflush}}\\{fflush}(\index{hlog+\\{hlog}}\\{hlog})){}$\6
\8\#\&{define} ${}\.{MESSAGE}(\,\ldots\,)\5(\index{fprintf+\\{fprintf}}\\{fprintf}(\index{hlog+\\{hlog}}\\{hlog},\39\.{\_\_VA\_ARGS\_\_}),\39\index{fflush+\\{fflush}}\\{fflush}(\index{hlog+\\{hlog}}\\{hlog})){}$\6
\8\#\&{define} ${}\.{QUIT}(\,\ldots\,)\5(\.{MESSAGE}(\.{"ERROR:\ "}\.{\_\_VA\_ARGS\_\_}),\39\index{fprintf+\\{fprintf}}\\{fprintf}(\index{hlog+\\{hlog}}\\{hlog},\39\.{"\\n"}),\39\index{exit+\\{exit}}\\{exit}(\T{1})){}$\6
\8\#\&{endif}
\Y
\fi

\M{377}


The amount of debugging\index{debugging} depends on the debugging flags.
For portability, we first define the output specifier for expressions of type \&{size\_t}.
\index{DBG+\.{DBG}}\index{SIZE F+\.{SIZE\_F}}\index{DBGTAG+\.{DBGTAG}}
\index{RNG+\.{RNG}}\index{TAGERR+\.{TAGERR}}
\Y\B\4\X305:debug macros\X${}\mathrel+\E{}$\6
\8\#\&{ifdef} \index{WIN32+\.{WIN32}}\.{WIN32}\6
\8\#\&{define} \.{SIZE\_F}\5\.{"0x\%x"}\6
\8\#\&{else}\6
\8\#\&{define} \.{SIZE\_F}\5\.{"0x\%zx"}\6
\8\#\&{endif}\6
\8\#\&{ifdef} \index{DEBUG+\.{DEBUG}}\.{DEBUG}\6
\8\#\&{define} ${}\.{DBG}(\index{FLAGS+\.{FLAGS}}\.{FLAGS},\39\,\ldots\,)\5((\index{debugflags+\\{debugflags}}\\{debugflags}\AND(\index{FLAGS+\.{FLAGS}}\.{FLAGS}))\?\index{LOG+\.{LOG}}\.{LOG}(\.{\_\_VA\_ARGS\_\_}):\T{0}){}$\6
\8\#\&{else}\6
\8\#\&{define} ${}\.{DBG}(\index{FLAGS+\.{FLAGS}}\.{FLAGS},\39\,\ldots\,)\5\T{0}{}$\6
\8\#\&{endif}\6
\8\#\&{define} ${}\.{DBGTAG}(\|A,\39\|P)\5\.{DBG}(\index{DBGTAGS+\.{DBGTAGS}}\.{DBGTAGS},\39\.{"tag\ [\%s,\%d]\ at\ "}\.{SIZE\_F}\.{"\\n"},\3{-1}\39\index{NAME+\.{NAME}}\.{NAME}(\|A),\39\index{INFO+\.{INFO}}\.{INFO}(\|A),\39(\|P)-\index{hstart+\\{hstart}}\\{hstart}){}$\6
\8\#\&{define} $\.{RNG}(\|S,\39\|N,\39\|A,\39\|Z){}$\6
\&{if} ${}((\&{int})(\|N)<(\&{int})(\|A)\V(\&{int})(\|N)>(\&{int})(\|Z))$ $\.{QUIT}(\|S\,\.{"\ \%d\ out\ of\ range\ [\%}\)\.{d\ -\ \%d]"},\39\|N,\39\|A,\39\|Z)$ \6
\8\#\&{define} \.{TAGERR}(\|A)\5${}\.{QUIT}(\.{"Unknown\ tag\ [\%s,\%d]}\)\.{\ at\ "}\.{SIZE\_F}\.{"\\n"},\39\index{NAME+\.{NAME}}\.{NAME}(\|A),\39\index{INFO+\.{INFO}}\.{INFO}(\|A),\39\index{hpos+\\{hpos}}\\{hpos}-\index{hstart+\\{hstart}}\\{hstart}){}$
\Y
\fi

\M{378}

The \.{bison} generated parser will need a function \index{yyerror+\\{yyerror}}\\{yyerror} for
error reporting. We can define it now:

\Y\B\4\X309:parsing functions\X${}\mathrel+\E{}$\6
\&{extern} \&{int} \index{yylineno+\\{yylineno}}\\{yylineno};\7
\&{int} \index{yyerror+\\{yyerror}}\\{yyerror}(\&{const} \&{char} ${}{*}\\{msg}){}$\1\1\2\2\1\6
\4${}\{{}$\5
${}\.{QUIT}(\.{"\ in\ line\ \%d\ \%s"},\39\index{yylineno+\\{yylineno}}\\{yylineno},\39\\{msg});{}$\6
\&{return} \T{0};\6
\4${}\}{}$\2
\Y
\fi

\M{379}

To enable the generation of debugging code \.{bison} needs also the following:
\Y\B\4\X379:enable bison debugging\X${}\E{}$\6
\8\#\&{ifdef} \index{DEBUG+\.{DEBUG}}\.{DEBUG}\6
\8\#\&{define} \index{YYDEBUG+\.{YYDEBUG}}\.{YYDEBUG}\5\T{1}\6
\&{extern} \&{int} \index{yydebug+\\{yydebug}}\\{yydebug};\6
\8\#\&{else}\6
\8\#\&{define} \index{YYDEBUG+\.{YYDEBUG}}\.{YYDEBUG}\5\T{0}\6
\8\#\&{endif}
\Us436\ET437.\Y
\fi

\M{380}


\appendix

\section{Reading Short Format Files Backwards}
This section is not really part of the file format definition, but it illustrates
an important property of the content section in short format: it can be read in both
directions. This is important because we want to be able
to start at an arbitrary point in the content and from there move pagewise backward.

The program {\tt skip}\index{skip+{\tt skip}} described in this section does just that. As wee see in section~\secref{skip}
its \index{main+\\{main}}\\{main} program is almost the same as the \index{main+\\{main}}\\{main} program of the program {\tt stretch}.
The only difference is the removal of an output file and the replacement of the call to
\index{hwrite content section+\\{hwrite\_content\_section}}\\{hwrite\_content\_section} by \index{hskip content section+\\{hskip\_content\_section}}\\{hskip\_content\_section}.

\Y\B\4\X380:skip functions\X${}\E{}$\6
\&{static} \&{void} \index{hskip content section+\\{hskip\_content\_section}}\\{hskip\_content\_section}(\&{void})\1\1\2\2\1\6
\4${}\{{}$\5
${}\.{DBG}(\index{DBGBASIC+\.{DBGBASIC}}\.{DBGBASIC},\39\.{"Skipping\ Content\ Se}\)\.{ction\\n"});{}$\6
\index{hget section+\\{hget\_section}}\\{hget\_section}(\T{2});\6
${}\index{hpos+\\{hpos}}\\{hpos}\K\index{hend+\\{hend}}\\{hend};{}$\6
\&{while} ${}(\index{hpos+\\{hpos}}\\{hpos}>\index{hstart+\\{hstart}}\\{hstart}){}$\1\5
\index{hteg content node+\\{hteg\_content\_node}}\\{hteg\_content\_node}(\,);\2\6
\4${}\}{}$\2
\As381, 385, 387, 398, 401, 404\ETs421.
\U441.\Y
\fi

\M{381}

The function \index{hteg content node+\\{hteg\_content\_node}}\\{hteg\_content\_node} used above is the reverse version of the function \index{hget content node+\\{hget\_content\_node}}\\{hget\_content\_node}.
Many such ``reverse functions'' will follow now and we will consistently use the same
naming scheme: replacing ``{\it get\/}`` by ``{\it teg\/}'' or ``{\tt GET}'' by ``{\tt TEG}''.
There is of course no need for a long format file in reverse order, and hence, the
{\tt skip} program differs in another aspect from {\tt stretch}: it does not produce
any output and it does not do much input checking. It will just extract enough information
from a content node to skip a node and ``advance'' or better ``retreat'' to the previous node.

\Y\B\4\X380:skip functions\X${}\mathrel+\E{}$\6
\&{static} \&{void} \index{hteg content node+\\{hteg\_content\_node}}\\{hteg\_content\_node}(\&{void})\1\1\2\2\1\6
\4${}\{{}$\5
\X382:skip the end byte \|z\X\6
\index{hteg content+\\{hteg\_content}}\\{hteg\_content}(\|z);\6
\X383:skip and check the start byte \|a\X\6
\4${}\}{}$\2\7
\&{static} \&{void} \index{hteg content+\\{hteg\_content}}\\{hteg\_content}(\&{uint8\_t} \|z)\1\1\2\2\1\6
\4${}\{{}$\5
\&{switch} (\|z)\6
\1${}\{{}$\5
\X390:cases to skip content\X\hbox{\1}\6
\4\&{default}:\5
\.{TAGERR}(\|z);\6
\&{break};\hbox{\2}\6
\4${}\}{}$\2\6
\4${}\}{}$\2
\Y
\fi

\M{382}

The code to skip the end\index{end byte} byte \|z and to check the start\index{start byte} byte \|a is used repeatedly.

\Y\B\4\X382:skip the end byte \|z\X${}\E{}$\6
\&{uint8\_t} \|a${},{}$ \|z;\C{ the start and the end byte}\6
\&{uint32\_t} \\{node\_pos}${}\K\index{hpos+\\{hpos}}\\{hpos}-\index{hstart+\\{hstart}}\\{hstart};{}$\7
\&{if} ${}(\index{hpos+\\{hpos}}\\{hpos}\Z\index{hstart+\\{hstart}}\\{hstart}){}$\1\5
\&{return};\2\6
\index{HTEGTAG+\.{HTEGTAG}}\.{HTEGTAG}(\|z);
\Us381, 387, 398, 401, 404\ETs421.\Y
\fi

\M{383}

\Y\B\4\X383:skip and check the start byte \|a\X${}\E{}$\6
\index{HTEGTAG+\.{HTEGTAG}}\.{HTEGTAG}(\|a);\6
\&{if} ${}(\|a\I\|z){}$\1\5
${}\.{QUIT}(\.{"Tag\ mismatch\ [\%s,\%d}\)\.{]!=[\%s,\%d]\ at\ "}\.{SIZE\_F}\.{"\ to\ 0x\%x\\n"},\3{-1}\39\index{NAME+\.{NAME}}\.{NAME}(\|a),\39\index{INFO+\.{INFO}}\.{INFO}(\|a),\39\index{NAME+\.{NAME}}\.{NAME}(\|z),\39\index{INFO+\.{INFO}}\.{INFO}(\|z),\3{-1}\39\index{hpos+\\{hpos}}\\{hpos}-\index{hstart+\\{hstart}}\\{hstart},\39\\{node\_pos}-\T{1}){}$;\2
\Us381, 387, 398, 401, 404\ETs421.\Y
\fi

\M{384}

We replace the ``{\tt GET}'' macros by the following ``{\tt TEG}'' macros:

\Y\B\4\X384:skip macros\X${}\E{}$\6
\8\#\&{define} \index{HTEG8+\.{HTEG8}}\.{HTEG8}(\|X)\5${}(\index{hpos+\\{hpos}}\\{hpos}\MRL{-{\K}}\T{1},\39(\|X)\K\index{hpos+\\{hpos}}\\{hpos}[\T{0}]){}$\6
\8\#\&{define} \index{HTEG16+\.{HTEG16}}\.{HTEG16}(\|X)\5${}(\index{hpos+\\{hpos}}\\{hpos}\MRL{-{\K}}\T{2},\39(\|X)\K(\index{hpos+\\{hpos}}\\{hpos}[\T{0}]\LL\T{8})+\index{hpos+\\{hpos}}\\{hpos}[\T{1}]){}$\6
\8\#\&{define} \index{HTEG24+\.{HTEG24}}\.{HTEG24}(\|X)\5${}(\index{hpos+\\{hpos}}\\{hpos}\MRL{-{\K}}\T{3},\39(\|X)\K(\index{hpos+\\{hpos}}\\{hpos}[\T{0}]\LL\T{16})+(\index{hpos+\\{hpos}}\\{hpos}[\T{1}]\LL\T{8})+\index{hpos+\\{hpos}}\\{hpos}[\T{2}]){}$\6
\8\#\&{define} \index{HTEG32+\.{HTEG32}}\.{HTEG32}(\|X)\5${}(\index{hpos+\\{hpos}}\\{hpos}\MRL{-{\K}}\T{4},\39(\|X)\K(\index{hpos+\\{hpos}}\\{hpos}[\T{0}]\LL\T{24})+(\index{hpos+\\{hpos}}\\{hpos}[\T{1}]\LL\T{16})+(\index{hpos+\\{hpos}}\\{hpos}[\T{2}]\LL\T{8})+\index{hpos+\\{hpos}}\\{hpos}[\T{3}]){}$\6
\8\#\&{define} \index{HTEGTAG+\.{HTEGTAG}}\.{HTEGTAG}(\|X)\5${}\index{HTEG8+\.{HTEG8}}\.{HTEG8}(\|X),\39\.{DBGTAG}(\|X,\39\index{hpos+\\{hpos}}\\{hpos}){}$
\As386, 388, 389, 391, 393, 396, 399, 402, 405, 407, 409, 411, 413, 415, 417, 419, 423, 425\ETs427.
\U441.\Y
\fi

\M{385}

Now we review step by step the different kinds of nodes.
\subsection{Floating Point Numbers}\index{floating point number}
\noindent
\Y\B\4\X380:skip functions\X${}\mathrel+\E{}$\6
\&{static} \&{float32\_t} \index{hteg float32+\\{hteg\_float32}}\\{hteg\_float32}(\&{void})\1\1\2\2\1\6
\4${}\{{}$\5
\&{union} ${}\{{}$\5
\1\&{float32\_t} \|d;\5
\&{uint32\_t} \index{bits+\\{bits}}\\{bits};\5
\2${}\}{}$ \|u;\7
${}\index{HTEG32+\.{HTEG32}}\.{HTEG32}(\|u.\index{bits+\\{bits}}\\{bits});{}$\6
\&{return} \|u${}.\|d;{}$\6
\4${}\}{}$\2
\Y
\fi

\M{386}


\subsection{Extended Dimensions}\index{extended dimension}
\noindent
\Y\B\4\X384:skip macros\X${}\mathrel+\E{}$\6
\8\#\&{define} $\index{HTEG XDIMEN+\.{HTEG\_XDIMEN}}\.{HTEG\_XDIMEN}(\|I,\39\|X)$ \6
\&{if} ${}((\|I)\AND\\{b001}){}$\1\5
${}\index{HTEG32+\.{HTEG32}}\.{HTEG32}((\|X).\|v);{}$\2\6
\&{if} ${}((\|I)\AND\\{b010}){}$\1\5
${}\index{HTEG32+\.{HTEG32}}\.{HTEG32}((\|X).\|h);{}$\2\6
\&{if} ${}((\|I)\AND\\{b100}){}$\1\5
${}\index{HTEG32+\.{HTEG32}}\.{HTEG32}((\|X).\|w);{}$\2
\Y
\fi

\M{387}

\Y\B\4\X380:skip functions\X${}\mathrel+\E{}$\6
\&{static} \&{void} \index{hteg xdimen node+\\{hteg\_xdimen\_node}}\\{hteg\_xdimen\_node}(\index{xdimen t+\&{xdimen\_t}}\&{xdimen\_t} ${}{*}\|x){}$\1\1\2\2\1\6
\4${}\{{}$\5
\X382:skip the end byte \|z\X\6
\&{switch} (\|z)\5
\1${}\{{}$\6
\8\#\&{if} \T{0}\C{  currently the info value 0 is not supported }\6
\4\&{case} \.{TAG}${}(\index{xdimen kind+\\{xdimen\_kind}}\\{xdimen\_kind},\39\\{b000}){}$:\C{ see section~\secref{reference} }\1\6
\4${}\{{}$\5
\&{uint8\_t} \|n;\5
\index{HTEG8+\.{HTEG8}}\.{HTEG8}(\|n);\6
\4${}\}{}$\5
\2\&{break};\6
\8\#\&{endif}\6
\4\&{case} \.{TAG}${}(\index{xdimen kind+\\{xdimen\_kind}}\\{xdimen\_kind},\39\\{b001}){}$:\5
${}\index{HTEG XDIMEN+\.{HTEG\_XDIMEN}}\.{HTEG\_XDIMEN}(\\{b001},\39{*}\|x){}$;\5
\&{break};\6
\4\&{case} \.{TAG}${}(\index{xdimen kind+\\{xdimen\_kind}}\\{xdimen\_kind},\39\\{b010}){}$:\5
${}\index{HTEG XDIMEN+\.{HTEG\_XDIMEN}}\.{HTEG\_XDIMEN}(\\{b010},\39{*}\|x){}$;\5
\&{break};\6
\4\&{case} \.{TAG}${}(\index{xdimen kind+\\{xdimen\_kind}}\\{xdimen\_kind},\39\\{b011}){}$:\5
${}\index{HTEG XDIMEN+\.{HTEG\_XDIMEN}}\.{HTEG\_XDIMEN}(\\{b011},\39{*}\|x){}$;\5
\&{break};\6
\4\&{case} \.{TAG}${}(\index{xdimen kind+\\{xdimen\_kind}}\\{xdimen\_kind},\39\\{b100}){}$:\5
${}\index{HTEG XDIMEN+\.{HTEG\_XDIMEN}}\.{HTEG\_XDIMEN}(\\{b100},\39{*}\|x){}$;\5
\&{break};\6
\4\&{case} \.{TAG}${}(\index{xdimen kind+\\{xdimen\_kind}}\\{xdimen\_kind},\39\\{b101}){}$:\5
${}\index{HTEG XDIMEN+\.{HTEG\_XDIMEN}}\.{HTEG\_XDIMEN}(\\{b101},\39{*}\|x){}$;\5
\&{break};\6
\4\&{case} \.{TAG}${}(\index{xdimen kind+\\{xdimen\_kind}}\\{xdimen\_kind},\39\\{b110}){}$:\5
${}\index{HTEG XDIMEN+\.{HTEG\_XDIMEN}}\.{HTEG\_XDIMEN}(\\{b110},\39{*}\|x){}$;\5
\&{break};\6
\4\&{case} \.{TAG}${}(\index{xdimen kind+\\{xdimen\_kind}}\\{xdimen\_kind},\39\\{b111}){}$:\5
${}\index{HTEG XDIMEN+\.{HTEG\_XDIMEN}}\.{HTEG\_XDIMEN}(\\{b111},\39{*}\|x){}$;\5
\&{break};\6
\4\&{default}:\5
${}\.{QUIT}(\.{"Extent\ expected\ at\ }\)\.{0x\%x\ got\ \%s"},\39\\{node\_pos},\39\index{NAME+\.{NAME}}\.{NAME}(\|z)){}$;\5
\&{break};\6
\4${}\}{}$\2\6
\X383:skip and check the start byte \|a\X\6
\4${}\}{}$\2
\Y
\fi

\M{388}


\subsection{Stretch and Shrink}\index{stretchability}\index{shrinkability}
\noindent
\Y\B\4\X384:skip macros\X${}\mathrel+\E{}$\6
\8\#\&{define} \index{HTEG STRETCH+\.{HTEG\_STRETCH}}\.{HTEG\_STRETCH}(\|S)\1\1\2\2\1\6
\4${}\{{}$\5
\index{stch t+\&{stch\_t}}\&{stch\_t} \index{st+\\{st}}\\{st};\5
${}\index{HTEG32+\.{HTEG32}}\.{HTEG32}(\index{st+\\{st}}\\{st}.\|u){}$;\5
${}\|S.\|o\K\index{st+\\{st}}\\{st}.\|u\AND\T{3}{}$;\5
${}\index{st+\\{st}}\\{st}.\|u\MRL{\AND{\K}}\CM\T{3}{}$;\5
${}\|S.\|f\K\index{st+\\{st}}\\{st}.\|f{}$;\5
${}\}{}$\2
\Y
\fi

\M{389}

\subsection{Glyphs}\index{glyph}
\noindent
\Y\B\4\X384:skip macros\X${}\mathrel+\E{}$\6
\8\#\&{define} \index{HTEG GLYPH+\.{HTEG\_GLYPH}}\.{HTEG\_GLYPH} ${}(\|I,\39\|G){}$ \index{HTEG8+\.{HTEG8}}\.{HTEG8} ${}((\|G).\|f);{}$\6
\&{if} ${}(\|I\E\T{1}){}$\1\5
${}\index{HTEG8+\.{HTEG8}}\.{HTEG8}((\|G).\|c);{}$\2\6
\&{else} \&{if} ${}(\|I\E\T{2}){}$\1\5
${}\index{HTEG16+\.{HTEG16}}\.{HTEG16}((\|G).\|c);{}$\2\6
\&{else} \&{if} ${}(\|I\E\T{3}){}$\1\5
${}\index{HTEG24+\.{HTEG24}}\.{HTEG24}((\|G).\|c);{}$\2\6
\&{else} \&{if} ${}(\|I\E\T{4}){}$\1\5
${}\index{HTEG32+\.{HTEG32}}\.{HTEG32}((\|G).\|c);{}$\2
\Y
\fi

\M{390}

\Y\B\4\X390:cases to skip content\X${}\E{}$\6
\4\&{case} \.{TAG}${}(\index{glyph kind+\\{glyph\_kind}}\\{glyph\_kind},\39\T{1}){}$:\5
\1${}\{{}$\5
\index{glyph t+\&{glyph\_t}}\&{glyph\_t} \|g;\5
${}\index{HTEG GLYPH+\.{HTEG\_GLYPH}}\.{HTEG\_GLYPH}(\T{1},\39\|g){}$;\5
${}\}{}$\5
\2\&{break};\6
\4\&{case} \.{TAG}${}(\index{glyph kind+\\{glyph\_kind}}\\{glyph\_kind},\39\T{2}){}$:\5
\1${}\{{}$\5
\index{glyph t+\&{glyph\_t}}\&{glyph\_t} \|g;\5
${}\index{HTEG GLYPH+\.{HTEG\_GLYPH}}\.{HTEG\_GLYPH}(\T{2},\39\|g){}$;\5
${}\}{}$\5
\2\&{break};\6
\4\&{case} \.{TAG}${}(\index{glyph kind+\\{glyph\_kind}}\\{glyph\_kind},\39\T{3}){}$:\5
\1${}\{{}$\5
\index{glyph t+\&{glyph\_t}}\&{glyph\_t} \|g;\5
${}\index{HTEG GLYPH+\.{HTEG\_GLYPH}}\.{HTEG\_GLYPH}(\T{3},\39\|g){}$;\5
${}\}{}$\5
\2\&{break};\6
\4\&{case} \.{TAG}${}(\index{glyph kind+\\{glyph\_kind}}\\{glyph\_kind},\39\T{4}){}$:\5
\1${}\{{}$\5
\index{glyph t+\&{glyph\_t}}\&{glyph\_t} \|g;\5
${}\index{HTEG GLYPH+\.{HTEG\_GLYPH}}\.{HTEG\_GLYPH}(\T{4},\39\|g){}$;\5
${}\}{}$\5
\2\&{break};
\As392, 394, 395, 397, 400, 403, 406, 408, 410, 412, 414, 416, 418, 420, 422, 424, 426\ETs428.
\U381.\Y
\fi

\M{391}


\subsection{Penalties}\index{penalty}
\noindent
\Y\B\4\X384:skip macros\X${}\mathrel+\E{}$\6
\8\#\&{define} $\index{HTEG PENALTY+\.{HTEG\_PENALTY}}\.{HTEG\_PENALTY}(\|I,\39\|P)$ \6
\&{if} ${}(\|I\E\T{1}){}$\5
\1${}\{{}$\5
\&{int8\_t} \|n;\5
\index{HTEG8+\.{HTEG8}}\.{HTEG8}(\|n);\5
${}\|P\K\|n{}$;\5
${}\}{}$\2\6
\&{else}\5
\1${}\{{}$\5
\&{int16\_t} \|n;\5
\index{HTEG16+\.{HTEG16}}\.{HTEG16}(\|n);\5
${}\|P\K\|n{}$;\5
${}\}{}$\2
\Y
\fi

\M{392}

\Y\B\4\X390:cases to skip content\X${}\mathrel+\E{}$\6
\4\&{case} \.{TAG}${}(\index{penalty kind+\\{penalty\_kind}}\\{penalty\_kind},\39\T{1}){}$:\5
\1${}\{{}$\5
\&{int32\_t} \|p;\5
${}\index{HTEG PENALTY+\.{HTEG\_PENALTY}}\.{HTEG\_PENALTY}(\T{1},\39\|p){}$;\5
${}\}{}$\5
\2\&{break};\6
\4\&{case} \.{TAG}${}(\index{penalty kind+\\{penalty\_kind}}\\{penalty\_kind},\39\T{2}){}$:\5
\1${}\{{}$\5
\&{int32\_t} \|p;\5
${}\index{HTEG PENALTY+\.{HTEG\_PENALTY}}\.{HTEG\_PENALTY}(\T{2},\39\|p){}$;\5
${}\}{}$\5
\2\&{break};
\Y
\fi

\M{393}


\subsection{Kerns}\index{kern}
\noindent
\Y\B\4\X384:skip macros\X${}\mathrel+\E{}$\6
\8\#\&{define} ${}\index{HTEG KERN+\.{HTEG\_KERN}}\.{HTEG\_KERN}(\|I,\39\|X)\5{}$\6
\&{if} ${}(((\|I)\AND\\{b011})\E\T{2}){}$\1\5
${}\index{HTEG32+\.{HTEG32}}\.{HTEG32}(\|X.\|w);{}$\2\6
\&{else} \&{if} ${}(((\|I)\AND\\{b011})\E\T{3})\index{hteg xdimen node+\\{hteg\_xdimen\_node}}\\{hteg\_xdimen\_node}({\AND}(\|X)){}$
\Y
\fi

\M{394}

\Y\B\4\X390:cases to skip content\X${}\mathrel+\E{}$\6
\4\&{case} \.{TAG}${}(\index{kern kind+\\{kern\_kind}}\\{kern\_kind},\39\\{b010}){}$:\5
\1${}\{{}$\5
\index{xdimen t+\&{xdimen\_t}}\&{xdimen\_t} \|x;\5
${}\index{HTEG KERN+\.{HTEG\_KERN}}\.{HTEG\_KERN}(\\{b010},\39\|x){}$;\5
${}\}{}$\5
\2\&{break};\6
\4\&{case} \.{TAG}${}(\index{kern kind+\\{kern\_kind}}\\{kern\_kind},\39\\{b011}){}$:\5
\1${}\{{}$\5
\index{xdimen t+\&{xdimen\_t}}\&{xdimen\_t} \|x;\5
${}\index{HTEG KERN+\.{HTEG\_KERN}}\.{HTEG\_KERN}(\\{b011},\39\|x){}$;\5
${}\}{}$\5
\2\&{break};\6
\4\&{case} \.{TAG}${}(\index{kern kind+\\{kern\_kind}}\\{kern\_kind},\39\\{b110}){}$:\5
\1${}\{{}$\5
\index{xdimen t+\&{xdimen\_t}}\&{xdimen\_t} \|x;\5
${}\index{HTEG KERN+\.{HTEG\_KERN}}\.{HTEG\_KERN}(\\{b110},\39\|x){}$;\5
${}\}{}$\5
\2\&{break};\6
\4\&{case} \.{TAG}${}(\index{kern kind+\\{kern\_kind}}\\{kern\_kind},\39\\{b111}){}$:\5
\1${}\{{}$\5
\index{xdimen t+\&{xdimen\_t}}\&{xdimen\_t} \|x;\5
${}\index{HTEG KERN+\.{HTEG\_KERN}}\.{HTEG\_KERN}(\\{b111},\39\|x){}$;\5
${}\}{}$\5
\2\&{break};
\Y
\fi

\M{395}

\subsection{Language}\index{language}

\Y\B\4\X390:cases to skip content\X${}\mathrel+\E{}$\6
\4\&{case} \.{TAG}${}(\index{language kind+\\{language\_kind}}\\{language\_kind},\39\T{1}){}$:\5
\&{case} \.{TAG}${}(\index{language kind+\\{language\_kind}}\\{language\_kind},\39\T{2}){}$:\5
\&{case} \.{TAG}${}(\index{language kind+\\{language\_kind}}\\{language\_kind},\39\T{3}){}$:\5
\&{case} \.{TAG}${}(\index{language kind+\\{language\_kind}}\\{language\_kind},\39\T{4}){}$:\5
\&{case} \.{TAG}${}(\index{language kind+\\{language\_kind}}\\{language\_kind},\39\T{5}){}$:\5
\&{case} \.{TAG}${}(\index{language kind+\\{language\_kind}}\\{language\_kind},\39\T{6}){}$:\5
\&{case} \.{TAG}${}(\index{language kind+\\{language\_kind}}\\{language\_kind},\39\T{7}){}$:\5
\&{break};
\Y
\fi

\M{396}

\subsection{Rules}\index{rule}
\noindent
\Y\B\4\X384:skip macros\X${}\mathrel+\E{}$\6
\8\#\&{define} $\index{HTEG RULE+\.{HTEG\_RULE}}\.{HTEG\_RULE}(\|I,\39\|R){}$\6
\&{if} ${}((\|I)\AND\\{b001}){}$\1\5
${}\index{HTEG32+\.{HTEG32}}\.{HTEG32}((\|R).\|w){}$;\5
\2\&{else}\1\5
${}(\|R).\|w\K\index{RUNNING DIMEN+\.{RUNNING\_DIMEN}}\.{RUNNING\_DIMEN};{}$\2\6
\&{if} ${}((\|I)\AND\\{b010}){}$\1\5
${}\index{HTEG32+\.{HTEG32}}\.{HTEG32}((\|R).\|d){}$;\5
\2\&{else}\1\5
${}(\|R).\|d\K\index{RUNNING DIMEN+\.{RUNNING\_DIMEN}}\.{RUNNING\_DIMEN};{}$\2\6
\&{if} ${}((\|I)\AND\\{b100}){}$\1\5
${}\index{HTEG32+\.{HTEG32}}\.{HTEG32}((\|R).\|h){}$;\5
\2\&{else}\1\5
${}(\|R).\|h\K\index{RUNNING DIMEN+\.{RUNNING\_DIMEN}}\.{RUNNING\_DIMEN};{}$\2
\Y
\fi

\M{397}

\Y\B\4\X390:cases to skip content\X${}\mathrel+\E{}$\6
\4\&{case} \.{TAG}${}(\index{rule kind+\\{rule\_kind}}\\{rule\_kind},\39\\{b011}){}$:\5
\1${}\{{}$\5
\index{rule t+\&{rule\_t}}\&{rule\_t} \|r;\5
${}\index{HTEG RULE+\.{HTEG\_RULE}}\.{HTEG\_RULE}(\\{b011},\39\|r){}$;\5
${}\}{}$\5
\2\&{break};\6
\4\&{case} \.{TAG}${}(\index{rule kind+\\{rule\_kind}}\\{rule\_kind},\39\\{b101}){}$:\5
\1${}\{{}$\5
\index{rule t+\&{rule\_t}}\&{rule\_t} \|r;\5
${}\index{HTEG RULE+\.{HTEG\_RULE}}\.{HTEG\_RULE}(\\{b101},\39\|r){}$;\5
${}\}{}$\5
\2\&{break};\6
\4\&{case} \.{TAG}${}(\index{rule kind+\\{rule\_kind}}\\{rule\_kind},\39\\{b001}){}$:\5
\1${}\{{}$\5
\index{rule t+\&{rule\_t}}\&{rule\_t} \|r;\5
${}\index{HTEG RULE+\.{HTEG\_RULE}}\.{HTEG\_RULE}(\\{b001},\39\|r){}$;\5
${}\}{}$\5
\2\&{break};\6
\4\&{case} \.{TAG}${}(\index{rule kind+\\{rule\_kind}}\\{rule\_kind},\39\\{b110}){}$:\5
\1${}\{{}$\5
\index{rule t+\&{rule\_t}}\&{rule\_t} \|r;\5
${}\index{HTEG RULE+\.{HTEG\_RULE}}\.{HTEG\_RULE}(\\{b110},\39\|r){}$;\5
${}\}{}$\5
\2\&{break};\6
\4\&{case} \.{TAG}${}(\index{rule kind+\\{rule\_kind}}\\{rule\_kind},\39\\{b111}){}$:\5
\1${}\{{}$\5
\index{rule t+\&{rule\_t}}\&{rule\_t} \|r;\5
${}\index{HTEG RULE+\.{HTEG\_RULE}}\.{HTEG\_RULE}(\\{b111},\39\|r){}$;\5
${}\}{}$\5
\2\&{break};
\Y
\fi

\M{398}

\Y\B\4\X380:skip functions\X${}\mathrel+\E{}$\6
\&{static} \&{void} \index{hteg rule node+\\{hteg\_rule\_node}}\\{hteg\_rule\_node}(\&{void})\1\1\2\2\1\6
\4${}\{{}$\5
\X382:skip the end byte \|z\X\6
\&{if} ${}(\index{KIND+\.{KIND}}\.{KIND}(\|z)\E\index{rule kind+\\{rule\_kind}}\\{rule\_kind}){}$\5
\1${}\{{}$\5
\index{rule t+\&{rule\_t}}\&{rule\_t} \|r;\5
${}\index{HTEG RULE+\.{HTEG\_RULE}}\.{HTEG\_RULE}(\index{INFO+\.{INFO}}\.{INFO}(\|z),\39\|r){}$;\5
${}\}{}$\2\6
\&{else}\1\5
${}\.{QUIT}(\.{"Rule\ expected\ at\ 0x}\)\.{\%x\ got\ \%s"},\39\\{node\_pos},\39\index{NAME+\.{NAME}}\.{NAME}(\|z));{}$\2\6
\X383:skip and check the start byte \|a\X\6
\4${}\}{}$\2
\Y
\fi

\M{399}
\subsection{Glue}\index{glue}
\noindent
\Y\B\4\X384:skip macros\X${}\mathrel+\E{}$\6
\8\#\&{define} $\index{HTEG GLUE+\.{HTEG\_GLUE}}\.{HTEG\_GLUE}(\|I,\39\|G){}$\6
\&{if} ${}((\|I)\AND\\{b001})$ $\index{HTEG STRETCH+\.{HTEG\_STRETCH}}\.{HTEG\_STRETCH}((\|G).\|m{}$)\5
\&{else}\1\5
${}(\|G).\|m.\|f\K\T{0.0},\39(\|G).\|m.\|o\K\T{0};{}$\2\6
\&{if} ${}((\|I)\AND\\{b010})$ $\index{HTEG STRETCH+\.{HTEG\_STRETCH}}\.{HTEG\_STRETCH}((\|G).\|p{}$)\5
\&{else}\1\5
${}(\|G).\|p.\|f\K\T{0.0},\39(\|G).\|p.\|o\K\T{0};{}$\2\6
\&{if} ${}(\|I\E\\{b111}){}$\1\5
${}\index{hteg xdimen node+\\{hteg\_xdimen\_node}}\\{hteg\_xdimen\_node}({\AND}((\|G).\|w));{}$\2\6
\&{else}\6
\1${}\{{}$\5
${}(\|G).\|w.\|h\K\T{0.0}{}$;\5
${}(\|G).\|w.\|v\K\T{0.0};{}$\6
\&{if} ${}((\|I)\AND\\{b100}){}$\1\5
${}\index{HTEG32+\.{HTEG32}}\.{HTEG32}((\|G).\|w.\|w){}$;\5
\2\&{else}\1\5
${}(\|G).\|w.\|w\K\T{0}{}$;\5
\2${}\}{}$\2
\Y
\fi

\M{400}

\Y\B\4\X390:cases to skip content\X${}\mathrel+\E{}$\6
\4\&{case} \.{TAG}${}(\index{glue kind+\\{glue\_kind}}\\{glue\_kind},\39\\{b001}){}$:\5
\1${}\{{}$\5
\index{glue t+\&{glue\_t}}\&{glue\_t} \|g;\5
${}\index{HTEG GLUE+\.{HTEG\_GLUE}}\.{HTEG\_GLUE}(\\{b001},\39\|g){}$;\5
${}\}{}$\5
\2\&{break};\6
\4\&{case} \.{TAG}${}(\index{glue kind+\\{glue\_kind}}\\{glue\_kind},\39\\{b010}){}$:\5
\1${}\{{}$\5
\index{glue t+\&{glue\_t}}\&{glue\_t} \|g;\5
${}\index{HTEG GLUE+\.{HTEG\_GLUE}}\.{HTEG\_GLUE}(\\{b010},\39\|g){}$;\5
${}\}{}$\5
\2\&{break};\6
\4\&{case} \.{TAG}${}(\index{glue kind+\\{glue\_kind}}\\{glue\_kind},\39\\{b011}){}$:\5
\1${}\{{}$\5
\index{glue t+\&{glue\_t}}\&{glue\_t} \|g;\5
${}\index{HTEG GLUE+\.{HTEG\_GLUE}}\.{HTEG\_GLUE}(\\{b011},\39\|g){}$;\5
${}\}{}$\5
\2\&{break};\6
\4\&{case} \.{TAG}${}(\index{glue kind+\\{glue\_kind}}\\{glue\_kind},\39\\{b100}){}$:\5
\1${}\{{}$\5
\index{glue t+\&{glue\_t}}\&{glue\_t} \|g;\5
${}\index{HTEG GLUE+\.{HTEG\_GLUE}}\.{HTEG\_GLUE}(\\{b100},\39\|g){}$;\5
${}\}{}$\5
\2\&{break};\6
\4\&{case} \.{TAG}${}(\index{glue kind+\\{glue\_kind}}\\{glue\_kind},\39\\{b101}){}$:\5
\1${}\{{}$\5
\index{glue t+\&{glue\_t}}\&{glue\_t} \|g;\5
${}\index{HTEG GLUE+\.{HTEG\_GLUE}}\.{HTEG\_GLUE}(\\{b101},\39\|g){}$;\5
${}\}{}$\5
\2\&{break};\6
\4\&{case} \.{TAG}${}(\index{glue kind+\\{glue\_kind}}\\{glue\_kind},\39\\{b110}){}$:\5
\1${}\{{}$\5
\index{glue t+\&{glue\_t}}\&{glue\_t} \|g;\5
${}\index{HTEG GLUE+\.{HTEG\_GLUE}}\.{HTEG\_GLUE}(\\{b110},\39\|g){}$;\5
${}\}{}$\5
\2\&{break};\6
\4\&{case} \.{TAG}${}(\index{glue kind+\\{glue\_kind}}\\{glue\_kind},\39\\{b111}){}$:\5
\1${}\{{}$\5
\index{glue t+\&{glue\_t}}\&{glue\_t} \|g;\5
${}\index{HTEG GLUE+\.{HTEG\_GLUE}}\.{HTEG\_GLUE}(\\{b111},\39\|g){}$;\5
${}\}{}$\5
\2\&{break};
\Y
\fi

\M{401}

\Y\B\4\X380:skip functions\X${}\mathrel+\E{}$\6
\&{static} \&{void} \index{hteg glue node+\\{hteg\_glue\_node}}\\{hteg\_glue\_node}(\&{void})\1\1\2\2\1\6
\4${}\{{}$\5
\X382:skip the end byte \|z\X\6
\&{if} ${}(\index{INFO+\.{INFO}}\.{INFO}(\|z)\E\\{b000}){}$\1\5
\index{HTEG REF+\.{HTEG\_REF}}\.{HTEG\_REF}(\index{glue kind+\\{glue\_kind}}\\{glue\_kind});\2\6
\&{else}\5
\1${}\{{}$\5
\index{glue t+\&{glue\_t}}\&{glue\_t} \|g;\5
${}\index{HTEG GLUE+\.{HTEG\_GLUE}}\.{HTEG\_GLUE}(\index{INFO+\.{INFO}}\.{INFO}(\|z),\39\|g){}$;\5
${}\}{}$\2\6
\X383:skip and check the start byte \|a\X\6
\4${}\}{}$\2
\Y
\fi

\M{402}

\subsection{Boxes}\index{box}
\noindent
\Y\B\4\X384:skip macros\X${}\mathrel+\E{}$\6
\8\#\&{define} $\index{HTEG BOX+\.{HTEG\_BOX}}\.{HTEG\_BOX}(\|I,\39\|B){}$ \index{hteg list+\\{hteg\_list}}\\{hteg\_list} ${}({\AND}(\|B.\|l));{}$\6
\&{if} ${}((\|I)\AND\\{b100}{}$)\6
\1${}\{{}$\5
${}\index{HTEG8+\.{HTEG8}}\.{HTEG8}(\|B.\|s){}$;\5
${}\|B.\|r\K\index{hteg float32+\\{hteg\_float32}}\\{hteg\_float32}(\,){}$;\5
${}\|B.\|o\K\|B.\|s\AND\T{\^F}{}$;\5
${}\|B.\|s\K\|B.\|s\GG\T{4}{}$;\5
${}\}{}$\2\6
\&{else}\5
\1${}\{{}$\5
${}\|B.\|r\K\T{0.0}{}$;\5
${}\|B.\|o\K\|B.\|s\K\T{0}{}$;\5
${}\}{}$\2\6
\&{if} ${}((\|I)\AND\\{b010}){}$\1\5
${}\index{HTEG32+\.{HTEG32}}\.{HTEG32}(\|B.\|a){}$;\5
\2\&{else}\1\5
${}\|B.\|a\K\T{0};{}$\2\6
${}\index{HTEG32+\.{HTEG32}}\.{HTEG32}(\|B.\|w);{}$\6
\&{if} ${}((\|I)\AND\\{b001}){}$\1\5
${}\index{HTEG32+\.{HTEG32}}\.{HTEG32}(\|B.\|d){}$;\5
\2\&{else}\1\5
${}\|B.\|d\K\T{0};{}$\2\6
${}\index{HTEG32+\.{HTEG32}}\.{HTEG32}(\|B.\|h){}$;
\Y
\fi

\M{403}

\Y\B\4\X390:cases to skip content\X${}\mathrel+\E{}$\6
\4\&{case} \.{TAG}${}(\index{hbox kind+\\{hbox\_kind}}\\{hbox\_kind},\39\\{b000}){}$:\5
\1${}\{{}$\5
\index{box t+\&{box\_t}}\&{box\_t} \|b;\5
${}\index{HTEG BOX+\.{HTEG\_BOX}}\.{HTEG\_BOX}(\\{b000},\39\|b){}$;\5
${}\}{}$\5
\2\&{break};\6
\4\&{case} \.{TAG}${}(\index{hbox kind+\\{hbox\_kind}}\\{hbox\_kind},\39\\{b001}){}$:\5
\1${}\{{}$\5
\index{box t+\&{box\_t}}\&{box\_t} \|b;\5
${}\index{HTEG BOX+\.{HTEG\_BOX}}\.{HTEG\_BOX}(\\{b001},\39\|b){}$;\5
${}\}{}$\5
\2\&{break};\6
\4\&{case} \.{TAG}${}(\index{hbox kind+\\{hbox\_kind}}\\{hbox\_kind},\39\\{b010}){}$:\5
\1${}\{{}$\5
\index{box t+\&{box\_t}}\&{box\_t} \|b;\5
${}\index{HTEG BOX+\.{HTEG\_BOX}}\.{HTEG\_BOX}(\\{b010},\39\|b){}$;\5
${}\}{}$\5
\2\&{break};\6
\4\&{case} \.{TAG}${}(\index{hbox kind+\\{hbox\_kind}}\\{hbox\_kind},\39\\{b011}){}$:\5
\1${}\{{}$\5
\index{box t+\&{box\_t}}\&{box\_t} \|b;\5
${}\index{HTEG BOX+\.{HTEG\_BOX}}\.{HTEG\_BOX}(\\{b011},\39\|b){}$;\5
${}\}{}$\5
\2\&{break};\6
\4\&{case} \.{TAG}${}(\index{hbox kind+\\{hbox\_kind}}\\{hbox\_kind},\39\\{b100}){}$:\5
\1${}\{{}$\5
\index{box t+\&{box\_t}}\&{box\_t} \|b;\5
${}\index{HTEG BOX+\.{HTEG\_BOX}}\.{HTEG\_BOX}(\\{b100},\39\|b){}$;\5
${}\}{}$\5
\2\&{break};\6
\4\&{case} \.{TAG}${}(\index{hbox kind+\\{hbox\_kind}}\\{hbox\_kind},\39\\{b101}){}$:\5
\1${}\{{}$\5
\index{box t+\&{box\_t}}\&{box\_t} \|b;\5
${}\index{HTEG BOX+\.{HTEG\_BOX}}\.{HTEG\_BOX}(\\{b101},\39\|b){}$;\5
${}\}{}$\5
\2\&{break};\6
\4\&{case} \.{TAG}${}(\index{hbox kind+\\{hbox\_kind}}\\{hbox\_kind},\39\\{b110}){}$:\5
\1${}\{{}$\5
\index{box t+\&{box\_t}}\&{box\_t} \|b;\5
${}\index{HTEG BOX+\.{HTEG\_BOX}}\.{HTEG\_BOX}(\\{b110},\39\|b){}$;\5
${}\}{}$\5
\2\&{break};\6
\4\&{case} \.{TAG}${}(\index{hbox kind+\\{hbox\_kind}}\\{hbox\_kind},\39\\{b111}){}$:\5
\1${}\{{}$\5
\index{box t+\&{box\_t}}\&{box\_t} \|b;\5
${}\index{HTEG BOX+\.{HTEG\_BOX}}\.{HTEG\_BOX}(\\{b111},\39\|b){}$;\5
${}\}{}$\5
\2\&{break};\6
\4\&{case} \.{TAG}${}(\index{vbox kind+\\{vbox\_kind}}\\{vbox\_kind},\39\\{b000}){}$:\5
\1${}\{{}$\5
\index{box t+\&{box\_t}}\&{box\_t} \|b;\5
${}\index{HTEG BOX+\.{HTEG\_BOX}}\.{HTEG\_BOX}(\\{b000},\39\|b){}$;\5
${}\}{}$\5
\2\&{break};\6
\4\&{case} \.{TAG}${}(\index{vbox kind+\\{vbox\_kind}}\\{vbox\_kind},\39\\{b001}){}$:\5
\1${}\{{}$\5
\index{box t+\&{box\_t}}\&{box\_t} \|b;\5
${}\index{HTEG BOX+\.{HTEG\_BOX}}\.{HTEG\_BOX}(\\{b001},\39\|b){}$;\5
${}\}{}$\5
\2\&{break};\6
\4\&{case} \.{TAG}${}(\index{vbox kind+\\{vbox\_kind}}\\{vbox\_kind},\39\\{b010}){}$:\5
\1${}\{{}$\5
\index{box t+\&{box\_t}}\&{box\_t} \|b;\5
${}\index{HTEG BOX+\.{HTEG\_BOX}}\.{HTEG\_BOX}(\\{b010},\39\|b){}$;\5
${}\}{}$\5
\2\&{break};\6
\4\&{case} \.{TAG}${}(\index{vbox kind+\\{vbox\_kind}}\\{vbox\_kind},\39\\{b011}){}$:\5
\1${}\{{}$\5
\index{box t+\&{box\_t}}\&{box\_t} \|b;\5
${}\index{HTEG BOX+\.{HTEG\_BOX}}\.{HTEG\_BOX}(\\{b011},\39\|b){}$;\5
${}\}{}$\5
\2\&{break};\6
\4\&{case} \.{TAG}${}(\index{vbox kind+\\{vbox\_kind}}\\{vbox\_kind},\39\\{b100}){}$:\5
\1${}\{{}$\5
\index{box t+\&{box\_t}}\&{box\_t} \|b;\5
${}\index{HTEG BOX+\.{HTEG\_BOX}}\.{HTEG\_BOX}(\\{b100},\39\|b){}$;\5
${}\}{}$\5
\2\&{break};\6
\4\&{case} \.{TAG}${}(\index{vbox kind+\\{vbox\_kind}}\\{vbox\_kind},\39\\{b101}){}$:\5
\1${}\{{}$\5
\index{box t+\&{box\_t}}\&{box\_t} \|b;\5
${}\index{HTEG BOX+\.{HTEG\_BOX}}\.{HTEG\_BOX}(\\{b101},\39\|b){}$;\5
${}\}{}$\5
\2\&{break};\6
\4\&{case} \.{TAG}${}(\index{vbox kind+\\{vbox\_kind}}\\{vbox\_kind},\39\\{b110}){}$:\5
\1${}\{{}$\5
\index{box t+\&{box\_t}}\&{box\_t} \|b;\5
${}\index{HTEG BOX+\.{HTEG\_BOX}}\.{HTEG\_BOX}(\\{b110},\39\|b){}$;\5
${}\}{}$\5
\2\&{break};\6
\4\&{case} \.{TAG}${}(\index{vbox kind+\\{vbox\_kind}}\\{vbox\_kind},\39\\{b111}){}$:\5
\1${}\{{}$\5
\index{box t+\&{box\_t}}\&{box\_t} \|b;\5
${}\index{HTEG BOX+\.{HTEG\_BOX}}\.{HTEG\_BOX}(\\{b111},\39\|b){}$;\5
${}\}{}$\5
\2\&{break};
\Y
\fi

\M{404}

\Y\B\4\X380:skip functions\X${}\mathrel+\E{}$\6
\&{static} \&{void} \index{hteg hbox node+\\{hteg\_hbox\_node}}\\{hteg\_hbox\_node}(\&{void})\1\1\2\2\1\6
\4${}\{{}$\5
\index{box t+\&{box\_t}}\&{box\_t} \|b;\7
\X382:skip the end byte \|z\X\6
\&{if} ${}(\index{KIND+\.{KIND}}\.{KIND}(\|z)\I\index{hbox kind+\\{hbox\_kind}}\\{hbox\_kind}){}$\1\5
${}\.{QUIT}(\.{"Hbox\ expected\ at\ 0x}\)\.{\%x\ got\ \%s"},\39\\{node\_pos},\39\index{NAME+\.{NAME}}\.{NAME}(\|z));{}$\2\6
${}\index{HTEG BOX+\.{HTEG\_BOX}}\.{HTEG\_BOX}(\index{INFO+\.{INFO}}\.{INFO}(\|z),\39\|b){}$;\6
\X383:skip and check the start byte \|a\X\6
\4${}\}{}$\2\7
\&{static} \&{void} \index{hteg vbox node+\\{hteg\_vbox\_node}}\\{hteg\_vbox\_node}(\&{void})\1\1\2\2\1\6
\4${}\{{}$\5
\index{box t+\&{box\_t}}\&{box\_t} \|b;\7
\X382:skip the end byte \|z\X\6
\&{if} ${}(\index{KIND+\.{KIND}}\.{KIND}(\|z)\I\index{vbox kind+\\{vbox\_kind}}\\{vbox\_kind}){}$\1\5
${}\.{QUIT}(\.{"Vbox\ expected\ at\ 0x}\)\.{\%x\ got\ \%s"},\39\\{node\_pos},\39\index{NAME+\.{NAME}}\.{NAME}(\|z));{}$\2\6
${}\index{HTEG BOX+\.{HTEG\_BOX}}\.{HTEG\_BOX}(\index{INFO+\.{INFO}}\.{INFO}(\|z),\39\|b){}$;\6
\X383:skip and check the start byte \|a\X\6
\4${}\}{}$\2
\Y
\fi

\M{405}


\subsection{Extended Boxes}\index{extended box}
\noindent
\Y\B\4\X384:skip macros\X${}\mathrel+\E{}$\6
\8\#\&{define} \index{HTEG SET+\.{HTEG\_SET}}\.{HTEG\_SET}(\|I)\1\6
\4${}\{{}$\5
\index{list t+\&{list\_t}}\&{list\_t} \|l;\5
${}\index{hteg list+\\{hteg\_list}}\\{hteg\_list}({\AND}\|l){}$;\5
${}\}{}$\2\1\7
\4${}\{{}$\5
\index{stretch t+\&{stretch\_t}}\&{stretch\_t} \|m;\5
\index{HTEG STRETCH+\.{HTEG\_STRETCH}}\.{HTEG\_STRETCH}(\|m);\5
${}\}{}$\2\1\6
\4${}\{{}$\5
\index{stretch t+\&{stretch\_t}}\&{stretch\_t} \|p;\5
\index{HTEG STRETCH+\.{HTEG\_STRETCH}}\.{HTEG\_STRETCH}(\|p);\5
${}\}{}$\2\6
\&{if} ${}((\|I)\AND\\{b010}){}$\5
\1${}\{{}$\5
\index{dimen t+\&{dimen\_t}}\&{dimen\_t} \|a;\5
\index{HTEG32+\.{HTEG32}}\.{HTEG32}(\|a);\5
${}\}{}$\2\1\6
\4${}\{{}$\5
\index{dimen t+\&{dimen\_t}}\&{dimen\_t} \|w;\5
\index{HTEG32+\.{HTEG32}}\.{HTEG32}(\|w);\5
${}\}{}$\2\1\6
\4${}\{{}$\5
\index{dimen t+\&{dimen\_t}}\&{dimen\_t} \|d;\5
\&{if} ${}((\|I)\AND\\{b001}){}$\1\5
\index{HTEG32+\.{HTEG32}}\.{HTEG32}(\|d);\5
\2\&{else}\1\5
${}\|d\K\T{0}{}$;\5
\2${}\}{}$\2\1\6
\4${}\{{}$\5
\index{dimen t+\&{dimen\_t}}\&{dimen\_t} \|h;\5
\index{HTEG32+\.{HTEG32}}\.{HTEG32}(\|h);\5
${}\}{}$\2\6
\&{if} ${}((\|I)\AND\\{b100}){}$\5
\1${}\{{}$\5
\index{xdimen t+\&{xdimen\_t}}\&{xdimen\_t} \|x;\7
${}\index{hteg xdimen node+\\{hteg\_xdimen\_node}}\\{hteg\_xdimen\_node}({\AND}\|x){}$;\5
${}\}{}$\2\6
\&{else}\1\5
\index{HTEG REF+\.{HTEG\_REF}}\.{HTEG\_REF}(\index{xdimen kind+\\{xdimen\_kind}}\\{xdimen\_kind});\2\7
\8\#\&{define} $\index{HTEG PACK+\.{HTEG\_PACK}}\.{HTEG\_PACK}(\|K,\39\|I){}$\1\6
\4${}\{{}$\5
\index{list t+\&{list\_t}}\&{list\_t} \|l;\5
${}\index{hteg list+\\{hteg\_list}}\\{hteg\_list}({\AND}\|l){}$;\5
${}\}{}$\2\7
\&{if} ${}(\|K\E\index{vpack kind+\\{vpack\_kind}}\\{vpack\_kind}){}$\5
\1${}\{{}$\5
\index{dimen t+\&{dimen\_t}}\&{dimen\_t} \|d;\5
\index{HTEG32+\.{HTEG32}}\.{HTEG32}(\|d);\5
${}\}{}$\2\6
\&{if} ${}((\|I)\AND\\{b010}){}$\5
\1${}\{{}$\5
\index{dimen t+\&{dimen\_t}}\&{dimen\_t} \|d;\5
\index{HTEG32+\.{HTEG32}}\.{HTEG32}(\|d);\5
${}\}{}$\2\6
\&{if} ${}((\|I)\AND\\{b100}){}$\5
\1${}\{{}$\5
\index{xdimen t+\&{xdimen\_t}}\&{xdimen\_t} \|x;\7
${}\index{hteg xdimen node+\\{hteg\_xdimen\_node}}\\{hteg\_xdimen\_node}({\AND}\|x){}$;\5
${}\}{}$\5
\2\&{else}\1\5
\index{HTEG REF+\.{HTEG\_REF}}\.{HTEG\_REF}(\index{xdimen kind+\\{xdimen\_kind}}\\{xdimen\_kind});\2
\Y
\fi

\M{406}

\Y\B\4\X390:cases to skip content\X${}\mathrel+\E{}$\6
\4\&{case} \.{TAG}${}(\index{hset kind+\\{hset\_kind}}\\{hset\_kind},\39\\{b000}){}$:\5
\index{HTEG SET+\.{HTEG\_SET}}\.{HTEG\_SET}(\\{b000});\5
\&{break};\6
\4\&{case} \.{TAG}${}(\index{hset kind+\\{hset\_kind}}\\{hset\_kind},\39\\{b001}){}$:\5
\index{HTEG SET+\.{HTEG\_SET}}\.{HTEG\_SET}(\\{b001});\5
\&{break};\6
\4\&{case} \.{TAG}${}(\index{hset kind+\\{hset\_kind}}\\{hset\_kind},\39\\{b010}){}$:\5
\index{HTEG SET+\.{HTEG\_SET}}\.{HTEG\_SET}(\\{b010});\5
\&{break};\6
\4\&{case} \.{TAG}${}(\index{hset kind+\\{hset\_kind}}\\{hset\_kind},\39\\{b011}){}$:\5
\index{HTEG SET+\.{HTEG\_SET}}\.{HTEG\_SET}(\\{b011});\5
\&{break};\6
\4\&{case} \.{TAG}${}(\index{hset kind+\\{hset\_kind}}\\{hset\_kind},\39\\{b100}){}$:\5
\index{HTEG SET+\.{HTEG\_SET}}\.{HTEG\_SET}(\\{b100});\5
\&{break};\6
\4\&{case} \.{TAG}${}(\index{hset kind+\\{hset\_kind}}\\{hset\_kind},\39\\{b101}){}$:\5
\index{HTEG SET+\.{HTEG\_SET}}\.{HTEG\_SET}(\\{b101});\5
\&{break};\6
\4\&{case} \.{TAG}${}(\index{hset kind+\\{hset\_kind}}\\{hset\_kind},\39\\{b110}){}$:\5
\index{HTEG SET+\.{HTEG\_SET}}\.{HTEG\_SET}(\\{b110});\5
\&{break};\6
\4\&{case} \.{TAG}${}(\index{hset kind+\\{hset\_kind}}\\{hset\_kind},\39\\{b111}){}$:\5
\index{HTEG SET+\.{HTEG\_SET}}\.{HTEG\_SET}(\\{b111});\5
\&{break};\7
\4\&{case} \.{TAG}${}(\index{vset kind+\\{vset\_kind}}\\{vset\_kind},\39\\{b000}){}$:\5
\index{HTEG SET+\.{HTEG\_SET}}\.{HTEG\_SET}(\\{b000});\5
\&{break};\6
\4\&{case} \.{TAG}${}(\index{vset kind+\\{vset\_kind}}\\{vset\_kind},\39\\{b001}){}$:\5
\index{HTEG SET+\.{HTEG\_SET}}\.{HTEG\_SET}(\\{b001});\5
\&{break};\6
\4\&{case} \.{TAG}${}(\index{vset kind+\\{vset\_kind}}\\{vset\_kind},\39\\{b010}){}$:\5
\index{HTEG SET+\.{HTEG\_SET}}\.{HTEG\_SET}(\\{b010});\5
\&{break};\6
\4\&{case} \.{TAG}${}(\index{vset kind+\\{vset\_kind}}\\{vset\_kind},\39\\{b011}){}$:\5
\index{HTEG SET+\.{HTEG\_SET}}\.{HTEG\_SET}(\\{b011});\5
\&{break};\6
\4\&{case} \.{TAG}${}(\index{vset kind+\\{vset\_kind}}\\{vset\_kind},\39\\{b100}){}$:\5
\index{HTEG SET+\.{HTEG\_SET}}\.{HTEG\_SET}(\\{b100});\5
\&{break};\6
\4\&{case} \.{TAG}${}(\index{vset kind+\\{vset\_kind}}\\{vset\_kind},\39\\{b101}){}$:\5
\index{HTEG SET+\.{HTEG\_SET}}\.{HTEG\_SET}(\\{b101});\5
\&{break};\6
\4\&{case} \.{TAG}${}(\index{vset kind+\\{vset\_kind}}\\{vset\_kind},\39\\{b110}){}$:\5
\index{HTEG SET+\.{HTEG\_SET}}\.{HTEG\_SET}(\\{b110});\5
\&{break};\6
\4\&{case} \.{TAG}${}(\index{vset kind+\\{vset\_kind}}\\{vset\_kind},\39\\{b111}){}$:\5
\index{HTEG SET+\.{HTEG\_SET}}\.{HTEG\_SET}(\\{b111});\5
\&{break};\7
\4\&{case} \.{TAG}${}(\index{hpack kind+\\{hpack\_kind}}\\{hpack\_kind},\39\\{b000}){}$:\5
${}\index{HTEG PACK+\.{HTEG\_PACK}}\.{HTEG\_PACK}(\index{hpack kind+\\{hpack\_kind}}\\{hpack\_kind},\39\\{b000}){}$;\5
\&{break};\6
\4\&{case} \.{TAG}${}(\index{hpack kind+\\{hpack\_kind}}\\{hpack\_kind},\39\\{b001}){}$:\5
${}\index{HTEG PACK+\.{HTEG\_PACK}}\.{HTEG\_PACK}(\index{hpack kind+\\{hpack\_kind}}\\{hpack\_kind},\39\\{b001}){}$;\5
\&{break};\6
\4\&{case} \.{TAG}${}(\index{hpack kind+\\{hpack\_kind}}\\{hpack\_kind},\39\\{b010}){}$:\5
${}\index{HTEG PACK+\.{HTEG\_PACK}}\.{HTEG\_PACK}(\index{hpack kind+\\{hpack\_kind}}\\{hpack\_kind},\39\\{b010}){}$;\5
\&{break};\6
\4\&{case} \.{TAG}${}(\index{hpack kind+\\{hpack\_kind}}\\{hpack\_kind},\39\\{b011}){}$:\5
${}\index{HTEG PACK+\.{HTEG\_PACK}}\.{HTEG\_PACK}(\index{hpack kind+\\{hpack\_kind}}\\{hpack\_kind},\39\\{b011}){}$;\5
\&{break};\6
\4\&{case} \.{TAG}${}(\index{hpack kind+\\{hpack\_kind}}\\{hpack\_kind},\39\\{b100}){}$:\5
${}\index{HTEG PACK+\.{HTEG\_PACK}}\.{HTEG\_PACK}(\index{hpack kind+\\{hpack\_kind}}\\{hpack\_kind},\39\\{b100}){}$;\5
\&{break};\6
\4\&{case} \.{TAG}${}(\index{hpack kind+\\{hpack\_kind}}\\{hpack\_kind},\39\\{b101}){}$:\5
${}\index{HTEG PACK+\.{HTEG\_PACK}}\.{HTEG\_PACK}(\index{hpack kind+\\{hpack\_kind}}\\{hpack\_kind},\39\\{b101}){}$;\5
\&{break};\6
\4\&{case} \.{TAG}${}(\index{hpack kind+\\{hpack\_kind}}\\{hpack\_kind},\39\\{b110}){}$:\5
${}\index{HTEG PACK+\.{HTEG\_PACK}}\.{HTEG\_PACK}(\index{hpack kind+\\{hpack\_kind}}\\{hpack\_kind},\39\\{b110}){}$;\5
\&{break};\6
\4\&{case} \.{TAG}${}(\index{hpack kind+\\{hpack\_kind}}\\{hpack\_kind},\39\\{b111}){}$:\5
${}\index{HTEG PACK+\.{HTEG\_PACK}}\.{HTEG\_PACK}(\index{hpack kind+\\{hpack\_kind}}\\{hpack\_kind},\39\\{b111}){}$;\5
\&{break};\7
\4\&{case} \.{TAG}${}(\index{vpack kind+\\{vpack\_kind}}\\{vpack\_kind},\39\\{b000}){}$:\5
${}\index{HTEG PACK+\.{HTEG\_PACK}}\.{HTEG\_PACK}(\index{vpack kind+\\{vpack\_kind}}\\{vpack\_kind},\39\\{b000}){}$;\5
\&{break};\6
\4\&{case} \.{TAG}${}(\index{vpack kind+\\{vpack\_kind}}\\{vpack\_kind},\39\\{b001}){}$:\5
${}\index{HTEG PACK+\.{HTEG\_PACK}}\.{HTEG\_PACK}(\index{vpack kind+\\{vpack\_kind}}\\{vpack\_kind},\39\\{b001}){}$;\5
\&{break};\6
\4\&{case} \.{TAG}${}(\index{vpack kind+\\{vpack\_kind}}\\{vpack\_kind},\39\\{b010}){}$:\5
${}\index{HTEG PACK+\.{HTEG\_PACK}}\.{HTEG\_PACK}(\index{vpack kind+\\{vpack\_kind}}\\{vpack\_kind},\39\\{b010}){}$;\5
\&{break};\6
\4\&{case} \.{TAG}${}(\index{vpack kind+\\{vpack\_kind}}\\{vpack\_kind},\39\\{b011}){}$:\5
${}\index{HTEG PACK+\.{HTEG\_PACK}}\.{HTEG\_PACK}(\index{vpack kind+\\{vpack\_kind}}\\{vpack\_kind},\39\\{b011}){}$;\5
\&{break};\6
\4\&{case} \.{TAG}${}(\index{vpack kind+\\{vpack\_kind}}\\{vpack\_kind},\39\\{b100}){}$:\5
${}\index{HTEG PACK+\.{HTEG\_PACK}}\.{HTEG\_PACK}(\index{vpack kind+\\{vpack\_kind}}\\{vpack\_kind},\39\\{b100}){}$;\5
\&{break};\6
\4\&{case} \.{TAG}${}(\index{vpack kind+\\{vpack\_kind}}\\{vpack\_kind},\39\\{b101}){}$:\5
${}\index{HTEG PACK+\.{HTEG\_PACK}}\.{HTEG\_PACK}(\index{vpack kind+\\{vpack\_kind}}\\{vpack\_kind},\39\\{b101}){}$;\5
\&{break};\6
\4\&{case} \.{TAG}${}(\index{vpack kind+\\{vpack\_kind}}\\{vpack\_kind},\39\\{b110}){}$:\5
${}\index{HTEG PACK+\.{HTEG\_PACK}}\.{HTEG\_PACK}(\index{vpack kind+\\{vpack\_kind}}\\{vpack\_kind},\39\\{b110}){}$;\5
\&{break};\6
\4\&{case} \.{TAG}${}(\index{vpack kind+\\{vpack\_kind}}\\{vpack\_kind},\39\\{b111}){}$:\5
${}\index{HTEG PACK+\.{HTEG\_PACK}}\.{HTEG\_PACK}(\index{vpack kind+\\{vpack\_kind}}\\{vpack\_kind},\39\\{b111}){}$;\5
\&{break};
\Y
\fi

\M{407}


\subsection{Leaders}\index{leaders}
\noindent
\Y\B\4\X384:skip macros\X${}\mathrel+\E{}$\6
\8\#\&{define} \index{HTEG LEADERS+\.{HTEG\_LEADERS}}\.{HTEG\_LEADERS}(\|I)\6
\&{if} ${}(\index{KIND+\.{KIND}}\.{KIND}(\index{hpos+\\{hpos}}\\{hpos}[{-}\T{1}])\E\index{rule kind+\\{rule\_kind}}\\{rule\_kind}){}$\1\5
\index{hteg rule node+\\{hteg\_rule\_node}}\\{hteg\_rule\_node}(\,);\2\6
\&{else} \&{if} ${}(\index{KIND+\.{KIND}}\.{KIND}(\index{hpos+\\{hpos}}\\{hpos}[{-}\T{1}])\E\index{hbox kind+\\{hbox\_kind}}\\{hbox\_kind}){}$\1\5
\index{hteg hbox node+\\{hteg\_hbox\_node}}\\{hteg\_hbox\_node}(\,);\2\6
\&{else}\1\5
\index{hteg vbox node+\\{hteg\_vbox\_node}}\\{hteg\_vbox\_node}(\,);\2\6
\&{if} ${}(\index{KIND+\.{KIND}}\.{KIND}(\index{hpos+\\{hpos}}\\{hpos}[{-}\T{1}])\E\index{glue kind+\\{glue\_kind}}\\{glue\_kind}){}$\1\5
\index{hteg glue node+\\{hteg\_glue\_node}}\\{hteg\_glue\_node}(\,);\2
\Y
\fi

\M{408}

\Y\B\4\X390:cases to skip content\X${}\mathrel+\E{}$\6
\4\&{case} \.{TAG}${}(\index{leaders kind+\\{leaders\_kind}}\\{leaders\_kind},\39\T{1}){}$:\5
\index{HTEG LEADERS+\.{HTEG\_LEADERS}}\.{HTEG\_LEADERS}(\T{1});\5
\&{break};\6
\4\&{case} \.{TAG}${}(\index{leaders kind+\\{leaders\_kind}}\\{leaders\_kind},\39\T{2}){}$:\5
\index{HTEG LEADERS+\.{HTEG\_LEADERS}}\.{HTEG\_LEADERS}(\T{2});\5
\&{break};\6
\4\&{case} \.{TAG}${}(\index{leaders kind+\\{leaders\_kind}}\\{leaders\_kind},\39\T{3}){}$:\5
\index{HTEG LEADERS+\.{HTEG\_LEADERS}}\.{HTEG\_LEADERS}(\T{3});\5
\&{break};
\Y
\fi

\M{409}

\subsection{Baseline Skips}\index{baseline skip}
\noindent
\Y\B\4\X384:skip macros\X${}\mathrel+\E{}$\6
\8\#\&{define} $\index{HTEG BASELINE+\.{HTEG\_BASELINE}}\.{HTEG\_BASELINE}(\|I,\39\|B)$ \6
\&{if} ${}((\|I)\AND\\{b001}){}$\1\5
${}\index{HTEG32+\.{HTEG32}}\.{HTEG32}((\|B).\\{lsl}){}$;\5
\2\&{else}\1\5
${}\|B.\\{lsl}\K\T{0};{}$\2\6
\&{if} ${}((\|I)\AND\\{b010}){}$\1\5
\index{hteg glue node+\\{hteg\_glue\_node}}\\{hteg\_glue\_node}(\,);\2\6
\&{else}\5
\1${}\{{}$\5
${}\|B.\\{ls}.\|p.\|o\K\|B.\\{ls}.\|m.\|o\K\|B.\\{ls}.\|w.\|w\K\T{0}{}$;\5
${}\|B.\\{ls}.\|w.\|h\K\|B.\\{ls}.\|w.\|v\K\|B.\\{ls}.\|p.\|f\K\|B.\\{ls}.\|m.\|f\K\T{0.0}{}$;\5
${}\}{}$\2\6
\&{if} ${}((\|I)\AND\\{b100}){}$\1\5
\index{hteg glue node+\\{hteg\_glue\_node}}\\{hteg\_glue\_node}(\,);\2\6
\&{else}\5
\1${}\{{}$\5
${}\|B.\\{bs}.\|p.\|o\K\|B.\\{bs}.\|m.\|o\K\|B.\\{bs}.\|w.\|w\K\T{0}{}$;\5
${}\|B.\\{bs}.\|w.\|h\K\|B.\\{bs}.\|w.\|v\K\|B.\\{bs}.\|p.\|f\K\|B.\\{bs}.\|m.\|f\K\T{0.0}{}$;\5
${}\}{}$\2
\Y
\fi

\M{410}

\Y\B\4\X390:cases to skip content\X${}\mathrel+\E{}$\6
\4\&{case} \.{TAG}${}(\index{baseline kind+\\{baseline\_kind}}\\{baseline\_kind},\39\\{b001}){}$:\5
\1${}\{{}$\5
\index{baseline t+\&{baseline\_t}}\&{baseline\_t} \|b;\5
${}\index{HTEG BASELINE+\.{HTEG\_BASELINE}}\.{HTEG\_BASELINE}(\\{b001},\39\|b){}$;\5
${}\}{}$\5
\2\&{break};\6
\4\&{case} \.{TAG}${}(\index{baseline kind+\\{baseline\_kind}}\\{baseline\_kind},\39\\{b010}){}$:\5
\1${}\{{}$\5
\index{baseline t+\&{baseline\_t}}\&{baseline\_t} \|b;\5
${}\index{HTEG BASELINE+\.{HTEG\_BASELINE}}\.{HTEG\_BASELINE}(\\{b010},\39\|b){}$;\5
${}\}{}$\5
\2\&{break};\6
\4\&{case} \.{TAG}${}(\index{baseline kind+\\{baseline\_kind}}\\{baseline\_kind},\39\\{b011}){}$:\5
\1${}\{{}$\5
\index{baseline t+\&{baseline\_t}}\&{baseline\_t} \|b;\5
${}\index{HTEG BASELINE+\.{HTEG\_BASELINE}}\.{HTEG\_BASELINE}(\\{b011},\39\|b){}$;\5
${}\}{}$\5
\2\&{break};\6
\4\&{case} \.{TAG}${}(\index{baseline kind+\\{baseline\_kind}}\\{baseline\_kind},\39\\{b100}){}$:\5
\1${}\{{}$\5
\index{baseline t+\&{baseline\_t}}\&{baseline\_t} \|b;\5
${}\index{HTEG BASELINE+\.{HTEG\_BASELINE}}\.{HTEG\_BASELINE}(\\{b100},\39\|b){}$;\5
${}\}{}$\5
\2\&{break};\6
\4\&{case} \.{TAG}${}(\index{baseline kind+\\{baseline\_kind}}\\{baseline\_kind},\39\\{b101}){}$:\5
\1${}\{{}$\5
\index{baseline t+\&{baseline\_t}}\&{baseline\_t} \|b;\5
${}\index{HTEG BASELINE+\.{HTEG\_BASELINE}}\.{HTEG\_BASELINE}(\\{b101},\39\|b){}$;\5
${}\}{}$\5
\2\&{break};\6
\4\&{case} \.{TAG}${}(\index{baseline kind+\\{baseline\_kind}}\\{baseline\_kind},\39\\{b110}){}$:\5
\1${}\{{}$\5
\index{baseline t+\&{baseline\_t}}\&{baseline\_t} \|b;\5
${}\index{HTEG BASELINE+\.{HTEG\_BASELINE}}\.{HTEG\_BASELINE}(\\{b110},\39\|b){}$;\5
${}\}{}$\5
\2\&{break};\6
\4\&{case} \.{TAG}${}(\index{baseline kind+\\{baseline\_kind}}\\{baseline\_kind},\39\\{b111}){}$:\5
\1${}\{{}$\5
\index{baseline t+\&{baseline\_t}}\&{baseline\_t} \|b;\5
${}\index{HTEG BASELINE+\.{HTEG\_BASELINE}}\.{HTEG\_BASELINE}(\\{b111},\39\|b){}$;\5
${}\}{}$\5
\2\&{break};
\Y
\fi

\M{411}
\subsection{Ligatures}\index{ligature}
\noindent
\Y\B\4\X384:skip macros\X${}\mathrel+\E{}$\6
\8\#\&{define} $\index{HTEG LIG+\.{HTEG\_LIG}}\.{HTEG\_LIG}(\|I,\39\|L){}$\6
\&{if} ${}((\|I)\E\T{7}{}$)\5
\1${}\index{HTEG8+\.{HTEG8}}\.{HTEG8}((\|L).\|l.\|s){}$;\5
\2\&{else}\1\5
${}(\|L).\|l.\|s\K(\|I);{}$\2\6
${}\index{hpos+\\{hpos}}\\{hpos}\MRL{-{\K}}(\|L).\|l.\|s{}$;\5
${}(\|L).\|l.\|p\K\index{hpos+\\{hpos}}\\{hpos}-\index{hstart+\\{hstart}}\\{hstart};{}$\6
\&{if} ${}((\|I)\E\T{7}){}$\5
\1${}\{{}$\5
\&{uint8\_t} \|n;\5
\index{HTEG8+\.{HTEG8}}\.{HTEG8}(\|n);\6
\&{if} ${}(\|n\I(\|L).\|l.\|s){}$\1\5
${}\.{QUIT}(\.{"Sizes\ in\ ligature\ d}\)\.{on't\ match\ \%d!=\%d"},\39(\|L).\|l.\|s,\39\|n);{}$\2\6
\4${}\}{}$\2\6
${}\index{HTEG8+\.{HTEG8}}\.{HTEG8}((\|L).\|f);$
\Y
\fi

\M{412}

\Y\B\4\X390:cases to skip content\X${}\mathrel+\E{}$\6
\4\&{case} \.{TAG}${}(\index{ligature kind+\\{ligature\_kind}}\\{ligature\_kind},\39\T{1}){}$:\5
\1${}\{{}$\5
\index{lig t+\&{lig\_t}}\&{lig\_t} \|l;\5
${}\index{HTEG LIG+\.{HTEG\_LIG}}\.{HTEG\_LIG}(\T{1},\39\|l){}$;\5
${}\}{}$\5
\2\&{break};\6
\4\&{case} \.{TAG}${}(\index{ligature kind+\\{ligature\_kind}}\\{ligature\_kind},\39\T{2}){}$:\5
\1${}\{{}$\5
\index{lig t+\&{lig\_t}}\&{lig\_t} \|l;\5
${}\index{HTEG LIG+\.{HTEG\_LIG}}\.{HTEG\_LIG}(\T{2},\39\|l){}$;\5
${}\}{}$\5
\2\&{break};\6
\4\&{case} \.{TAG}${}(\index{ligature kind+\\{ligature\_kind}}\\{ligature\_kind},\39\T{3}){}$:\5
\1${}\{{}$\5
\index{lig t+\&{lig\_t}}\&{lig\_t} \|l;\5
${}\index{HTEG LIG+\.{HTEG\_LIG}}\.{HTEG\_LIG}(\T{3},\39\|l){}$;\5
${}\}{}$\5
\2\&{break};\6
\4\&{case} \.{TAG}${}(\index{ligature kind+\\{ligature\_kind}}\\{ligature\_kind},\39\T{4}){}$:\5
\1${}\{{}$\5
\index{lig t+\&{lig\_t}}\&{lig\_t} \|l;\5
${}\index{HTEG LIG+\.{HTEG\_LIG}}\.{HTEG\_LIG}(\T{4},\39\|l){}$;\5
${}\}{}$\5
\2\&{break};\6
\4\&{case} \.{TAG}${}(\index{ligature kind+\\{ligature\_kind}}\\{ligature\_kind},\39\T{5}){}$:\5
\1${}\{{}$\5
\index{lig t+\&{lig\_t}}\&{lig\_t} \|l;\5
${}\index{HTEG LIG+\.{HTEG\_LIG}}\.{HTEG\_LIG}(\T{5},\39\|l){}$;\5
${}\}{}$\5
\2\&{break};\6
\4\&{case} \.{TAG}${}(\index{ligature kind+\\{ligature\_kind}}\\{ligature\_kind},\39\T{6}){}$:\5
\1${}\{{}$\5
\index{lig t+\&{lig\_t}}\&{lig\_t} \|l;\5
${}\index{HTEG LIG+\.{HTEG\_LIG}}\.{HTEG\_LIG}(\T{6},\39\|l){}$;\5
${}\}{}$\5
\2\&{break};\6
\4\&{case} \.{TAG}${}(\index{ligature kind+\\{ligature\_kind}}\\{ligature\_kind},\39\T{7}){}$:\5
\1${}\{{}$\5
\index{lig t+\&{lig\_t}}\&{lig\_t} \|l;\5
${}\index{HTEG LIG+\.{HTEG\_LIG}}\.{HTEG\_LIG}(\T{7},\39\|l){}$;\5
${}\}{}$\5
\2\&{break};
\Y
\fi

\M{413}


\subsection{Hyphenation}\index{hyphen}
\noindent
\Y\B\4\X384:skip macros\X${}\mathrel+\E{}$\6
\8\#\&{define} $\index{HTEG HYPHEN+\.{HTEG\_HYPHEN}}\.{HTEG\_HYPHEN}(\|I,\39\ts{H})$ \6
\&{if} ${}((\|I)\AND\\{b001}){}$\1\5
${}\index{HTEG8+\.{HTEG8}}\.{HTEG8}((\ts{H}).\|r){}$;\5
\2\&{else}\1\5
${}(\ts{H}).\|r\K\T{0};{}$\2\6
\&{if} ${}((\|I)\AND\\{b010}){}$\1\5
${}\index{hteg list+\\{hteg\_list}}\\{hteg\_list}({\AND}((\ts{H}).\|q));{}$\2\6
\&{else}\5
\1${}\{{}$\5
${}(\ts{H}).\|q.\|p\K\index{hpos+\\{hpos}}\\{hpos}-\index{hstart+\\{hstart}}\\{hstart}{}$;\5
${}(\ts{H}).\|q.\|s\K\T{0}{}$;\5
${}(\ts{H}).\|q.\|k\K\index{list kind+\\{list\_kind}}\\{list\_kind}{}$;\5
${}\}{}$\2\6
\&{if} ${}((\|I)\AND\\{b100}){}$\1\5
${}\index{hteg list+\\{hteg\_list}}\\{hteg\_list}({\AND}((\ts{H}).\|p));{}$\2\6
\&{else}\5
\1${}\{{}$\5
${}(\ts{H}).\|p.\|p\K\index{hpos+\\{hpos}}\\{hpos}-\index{hstart+\\{hstart}}\\{hstart}{}$;\5
${}(\ts{H}).\|p.\|s\K\T{0}{}$;\5
${}(\ts{H}).\|p.\|k\K\index{list kind+\\{list\_kind}}\\{list\_kind}{}$;\5
${}\}{}$\2
\Y
\fi

\M{414}
\Y\B\4\X390:cases to skip content\X${}\mathrel+\E{}$\6
\4\&{case} \.{TAG}${}(\index{hyphen kind+\\{hyphen\_kind}}\\{hyphen\_kind},\39\\{b001}){}$:\1\6
\4${}\{{}$\5
\index{hyphen t+\&{hyphen\_t}}\&{hyphen\_t} \|h;\5
${}\index{HTEG HYPHEN+\.{HTEG\_HYPHEN}}\.{HTEG\_HYPHEN}(\\{b001},\39\|h){}$;\5
${}\}{}$\5
\2\&{break};\6
\4\&{case} \.{TAG}${}(\index{hyphen kind+\\{hyphen\_kind}}\\{hyphen\_kind},\39\\{b010}){}$:\1\6
\4${}\{{}$\5
\index{hyphen t+\&{hyphen\_t}}\&{hyphen\_t} \|h;\5
${}\index{HTEG HYPHEN+\.{HTEG\_HYPHEN}}\.{HTEG\_HYPHEN}(\\{b010},\39\|h){}$;\5
${}\}{}$\5
\2\&{break};\6
\4\&{case} \.{TAG}${}(\index{hyphen kind+\\{hyphen\_kind}}\\{hyphen\_kind},\39\\{b011}){}$:\1\6
\4${}\{{}$\5
\index{hyphen t+\&{hyphen\_t}}\&{hyphen\_t} \|h;\5
${}\index{HTEG HYPHEN+\.{HTEG\_HYPHEN}}\.{HTEG\_HYPHEN}(\\{b011},\39\|h){}$;\5
${}\}{}$\5
\2\&{break};\6
\4\&{case} \.{TAG}${}(\index{hyphen kind+\\{hyphen\_kind}}\\{hyphen\_kind},\39\\{b100}){}$:\1\6
\4${}\{{}$\5
\index{hyphen t+\&{hyphen\_t}}\&{hyphen\_t} \|h;\5
${}\index{HTEG HYPHEN+\.{HTEG\_HYPHEN}}\.{HTEG\_HYPHEN}(\\{b100},\39\|h){}$;\5
${}\}{}$\5
\2\&{break};\6
\4\&{case} \.{TAG}${}(\index{hyphen kind+\\{hyphen\_kind}}\\{hyphen\_kind},\39\\{b101}){}$:\1\6
\4${}\{{}$\5
\index{hyphen t+\&{hyphen\_t}}\&{hyphen\_t} \|h;\5
${}\index{HTEG HYPHEN+\.{HTEG\_HYPHEN}}\.{HTEG\_HYPHEN}(\\{b101},\39\|h){}$;\5
${}\}{}$\5
\2\&{break};\6
\4\&{case} \.{TAG}${}(\index{hyphen kind+\\{hyphen\_kind}}\\{hyphen\_kind},\39\\{b110}){}$:\1\6
\4${}\{{}$\5
\index{hyphen t+\&{hyphen\_t}}\&{hyphen\_t} \|h;\5
${}\index{HTEG HYPHEN+\.{HTEG\_HYPHEN}}\.{HTEG\_HYPHEN}(\\{b110},\39\|h){}$;\5
${}\}{}$\5
\2\&{break};\6
\4\&{case} \.{TAG}${}(\index{hyphen kind+\\{hyphen\_kind}}\\{hyphen\_kind},\39\\{b111}){}$:\1\6
\4${}\{{}$\5
\index{hyphen t+\&{hyphen\_t}}\&{hyphen\_t} \|h;\5
${}\index{HTEG HYPHEN+\.{HTEG\_HYPHEN}}\.{HTEG\_HYPHEN}(\\{b111},\39\|h){}$;\5
${}\}{}$\5
\2\&{break};
\Y
\fi

\M{415}


\subsection{Paragraphs}\index{paragraph}
\noindent
\Y\B\4\X384:skip macros\X${}\mathrel+\E{}$\6
\8\#\&{define} \index{HTEG PAR+\.{HTEG\_PAR}}\.{HTEG\_PAR}(\|I)\1\6
\4${}\{{}$\5
\index{list t+\&{list\_t}}\&{list\_t} \|l;\5
${}\index{hteg list+\\{hteg\_list}}\\{hteg\_list}({\AND}\|l){}$;\5
${}\}{}$\2\7
\&{if} ${}((\|I)\AND\\{b010}){}$\5
\1${}\{{}$\5
\index{list t+\&{list\_t}}\&{list\_t} \|l;\5
${}\index{hteg param list node+\\{hteg\_param\_list\_node}}\\{hteg\_param\_list\_node}({\AND}\|l){}$;\5
${}\}{}$\2\6
\&{else}\1\5
\index{HTEG REF+\.{HTEG\_REF}}\.{HTEG\_REF}(\index{param kind+\\{param\_kind}}\\{param\_kind});\2\6
\&{if} ${}((\|I)\AND\\{b100}){}$\5
\1${}\{{}$\5
\index{xdimen t+\&{xdimen\_t}}\&{xdimen\_t} \|x;\5
${}\index{hteg xdimen node+\\{hteg\_xdimen\_node}}\\{hteg\_xdimen\_node}({\AND}\|x){}$;\5
${}\}{}$\2\6
\&{else}\1\5
\index{HTEG REF+\.{HTEG\_REF}}\.{HTEG\_REF}(\index{xdimen kind+\\{xdimen\_kind}}\\{xdimen\_kind});\2
\Y
\fi

\M{416}

\Y\B\4\X390:cases to skip content\X${}\mathrel+\E{}$\6
\4\&{case} \.{TAG}${}(\index{par kind+\\{par\_kind}}\\{par\_kind},\39\\{b000}){}$:\5
\index{HTEG PAR+\.{HTEG\_PAR}}\.{HTEG\_PAR}(\\{b000});\5
\&{break};\6
\4\&{case} \.{TAG}${}(\index{par kind+\\{par\_kind}}\\{par\_kind},\39\\{b010}){}$:\5
\index{HTEG PAR+\.{HTEG\_PAR}}\.{HTEG\_PAR}(\\{b010});\5
\&{break};\6
\4\&{case} \.{TAG}${}(\index{par kind+\\{par\_kind}}\\{par\_kind},\39\\{b100}){}$:\5
\index{HTEG PAR+\.{HTEG\_PAR}}\.{HTEG\_PAR}(\\{b100});\5
\&{break};\6
\4\&{case} \.{TAG}${}(\index{par kind+\\{par\_kind}}\\{par\_kind},\39\\{b110}){}$:\5
\index{HTEG PAR+\.{HTEG\_PAR}}\.{HTEG\_PAR}(\\{b110});\5
\&{break};
\Y
\fi

\M{417}


\subsection{Mathematics}\index{mathematics}\index{displayed formula}
\noindent
\Y\B\4\X384:skip macros\X${}\mathrel+\E{}$\6
\8\#\&{define} \index{HTEG MATH+\.{HTEG\_MATH}}\.{HTEG\_MATH}(\|I) \6
\&{if} ${}((\|I)\AND\\{b001}){}$\1\5
\index{hteg hbox node+\\{hteg\_hbox\_node}}\\{hteg\_hbox\_node}(\,);\2\1\6
\4${}\{{}$\5
\index{list t+\&{list\_t}}\&{list\_t} \|l;\5
${}\index{hteg list+\\{hteg\_list}}\\{hteg\_list}({\AND}\|l){}$;\5
${}\}{}$\2\6
\&{if} ${}((\|I)\AND\\{b010}){}$\1\5
\index{hteg hbox node+\\{hteg\_hbox\_node}}\\{hteg\_hbox\_node}(\,);\2\6
\&{if} ${}((\|I)\AND\\{b100}){}$\5
\1${}\{{}$\5
\index{list t+\&{list\_t}}\&{list\_t} \|l;\5
${}\index{hteg param list node+\\{hteg\_param\_list\_node}}\\{hteg\_param\_list\_node}({\AND}\|l){}$;\5
${}\}{}$\5
\2\&{else}\1\5
\index{HTEG REF+\.{HTEG\_REF}}\.{HTEG\_REF}(\index{param kind+\\{param\_kind}}\\{param\_kind});\2
\Y
\fi

\M{418}

\Y\B\4\X390:cases to skip content\X${}\mathrel+\E{}$\6
\4\&{case} \.{TAG}${}(\index{math kind+\\{math\_kind}}\\{math\_kind},\39\\{b000}){}$:\5
\index{HTEG MATH+\.{HTEG\_MATH}}\.{HTEG\_MATH}(\\{b000});\5
\&{break};\6
\4\&{case} \.{TAG}${}(\index{math kind+\\{math\_kind}}\\{math\_kind},\39\\{b001}){}$:\5
\index{HTEG MATH+\.{HTEG\_MATH}}\.{HTEG\_MATH}(\\{b001});\5
\&{break};\6
\4\&{case} \.{TAG}${}(\index{math kind+\\{math\_kind}}\\{math\_kind},\39\\{b010}){}$:\5
\index{HTEG MATH+\.{HTEG\_MATH}}\.{HTEG\_MATH}(\\{b010});\5
\&{break};\6
\4\&{case} \.{TAG}${}(\index{math kind+\\{math\_kind}}\\{math\_kind},\39\\{b100}){}$:\5
\index{HTEG MATH+\.{HTEG\_MATH}}\.{HTEG\_MATH}(\\{b100});\5
\&{break};\6
\4\&{case} \.{TAG}${}(\index{math kind+\\{math\_kind}}\\{math\_kind},\39\\{b101}){}$:\5
\index{HTEG MATH+\.{HTEG\_MATH}}\.{HTEG\_MATH}(\\{b101});\5
\&{break};\6
\4\&{case} \.{TAG}${}(\index{math kind+\\{math\_kind}}\\{math\_kind},\39\\{b110}){}$:\5
\index{HTEG MATH+\.{HTEG\_MATH}}\.{HTEG\_MATH}(\\{b110});\5
\&{break};\6
\4\&{case} \.{TAG}${}(\index{math kind+\\{math\_kind}}\\{math\_kind},\39\\{b011}){}$:\5
\&{case} \.{TAG}${}(\index{math kind+\\{math\_kind}}\\{math\_kind},\39\\{b111}){}$:\5
\&{break};
\Y
\fi

\M{419}

\subsection{Images}\index{image}
\noindent
\Y\B\4\X384:skip macros\X${}\mathrel+\E{}$\6
\8\#\&{define} $\index{HTEG IMAGE+\.{HTEG\_IMAGE}}\.{HTEG\_IMAGE}(\|I,\39\|X){}$\6
\&{if} ${}(\|I\AND\\{b001}){}$\5
\1${}\{{}$\5
${}\index{HTEG STRETCH+\.{HTEG\_STRETCH}}\.{HTEG\_STRETCH}((\|X).\|m);{}$\6
${}\index{HTEG STRETCH+\.{HTEG\_STRETCH}}\.{HTEG\_STRETCH}((\|X).\|p){}$;\5
${}\}{}$\2\6
\&{else}\5
\1${}\{{}$\5
${}(\|X).\|p.\|f\K(\|X).\|m.\|f\K\T{0.0};{}$\6
${}(\|X).\|p.\|o\K(\|X).\|m.\|o\K\index{normal o+\\{normal\_o}}\\{normal\_o}{}$;\5
${}\}{}$\2\6
\&{if} ${}(\|I\AND\\{b010}){}$\5
\1${}\{{}$\5
${}\index{HTEG32+\.{HTEG32}}\.{HTEG32}((\|X).\|h);{}$\6
${}\index{HTEG32+\.{HTEG32}}\.{HTEG32}((\|X).\|w){}$;\5
${}\}{}$\2\6
\&{else}\1\5
${}(\|X).\|w\K(\|X).\|h\K\T{0};{}$\2\6
${}\index{HTEG16+\.{HTEG16}}\.{HTEG16}((\|X).\|n);$
\Y
\fi

\M{420}

\Y\B\4\X390:cases to skip content\X${}\mathrel+\E{}$\6
\4\&{case} \.{TAG}${}(\index{image kind+\\{image\_kind}}\\{image\_kind},\39\\{b100}){}$:\5
\1${}\{{}$\5
\index{image t+\&{image\_t}}\&{image\_t} \|x;\5
${}\index{HTEG IMAGE+\.{HTEG\_IMAGE}}\.{HTEG\_IMAGE}(\\{b100},\39\|x){}$;\5
${}\}{}$\5
\2\&{break};\6
\4\&{case} \.{TAG}${}(\index{image kind+\\{image\_kind}}\\{image\_kind},\39\\{b101}){}$:\5
\1${}\{{}$\5
\index{image t+\&{image\_t}}\&{image\_t} \|x;\5
${}\index{HTEG IMAGE+\.{HTEG\_IMAGE}}\.{HTEG\_IMAGE}(\\{b101},\39\|x){}$;\5
${}\}{}$\5
\2\&{break};\6
\4\&{case} \.{TAG}${}(\index{image kind+\\{image\_kind}}\\{image\_kind},\39\\{b110}){}$:\5
\1${}\{{}$\5
\index{image t+\&{image\_t}}\&{image\_t} \|x;\5
${}\index{HTEG IMAGE+\.{HTEG\_IMAGE}}\.{HTEG\_IMAGE}(\\{b110},\39\|x){}$;\5
${}\}{}$\5
\2\&{break};\6
\4\&{case} \.{TAG}${}(\index{image kind+\\{image\_kind}}\\{image\_kind},\39\\{b111}){}$:\5
\1${}\{{}$\5
\index{image t+\&{image\_t}}\&{image\_t} \|x;\5
${}\index{HTEG IMAGE+\.{HTEG\_IMAGE}}\.{HTEG\_IMAGE}(\\{b111},\39\|x){}$;\5
${}\}{}$\5
\2\&{break};
\Y
\fi

\M{421}

\subsection{Plain Lists, Texts, and Parameter Lists}\index{list}
\noindent
\Y\B\4\X380:skip functions\X${}\mathrel+\E{}$\6
\&{static} \&{void} \index{hteg size boundary+\\{hteg\_size\_boundary}}\\{hteg\_size\_boundary}(\index{info t+\&{info\_t}}\&{info\_t} \index{info+\\{info}}\\{info})\1\1\2\2\1\6
\4${}\{{}$\5
\&{uint32\_t} \|n;\7
\&{if} ${}(\index{info+\\{info}}\\{info}<\T{2}){}$\1\5
\&{return};\2\6
\index{HTEG8+\.{HTEG8}}\.{HTEG8}(\|n);\6
\&{if} ${}(\|n-\T{1}\I\T{\^100}-\index{info+\\{info}}\\{info}){}$\1\5
${}\.{QUIT}(\.{"List\ size\ boundary\ }\)\.{byte\ 0x\%x\ does\ not\ m}\)\.{atch\ info\ value\ \%d\ a}\)\.{t\ "}\.{SIZE\_F},\39\|n,\39\index{info+\\{info}}\\{info},\39\index{hpos+\\{hpos}}\\{hpos}-\index{hstart+\\{hstart}}\\{hstart});{}$\2\6
\4${}\}{}$\2\7
\&{static} \&{uint32\_t} \index{hteg list size+\\{hteg\_list\_size}}\\{hteg\_list\_size}(\index{info t+\&{info\_t}}\&{info\_t} \index{info+\\{info}}\\{info})\1\1\2\2\1\6
\4${}\{{}$\5
\&{uint32\_t} \|n;\7
\&{if} ${}(\index{info+\\{info}}\\{info}\E\T{1}){}$\1\5
\&{return} \T{0};\2\6
\&{else} \&{if} ${}(\index{info+\\{info}}\\{info}\E\T{2}){}$\1\5
\index{HTEG8+\.{HTEG8}}\.{HTEG8}(\|n);\2\6
\&{else} \&{if} ${}(\index{info+\\{info}}\\{info}\E\T{3}){}$\1\5
\index{HTEG16+\.{HTEG16}}\.{HTEG16}(\|n);\2\6
\&{else} \&{if} ${}(\index{info+\\{info}}\\{info}\E\T{4}){}$\1\5
\index{HTEG24+\.{HTEG24}}\.{HTEG24}(\|n);\2\6
\&{else} \&{if} ${}(\index{info+\\{info}}\\{info}\E\T{5}){}$\1\5
\index{HTEG32+\.{HTEG32}}\.{HTEG32}(\|n);\2\6
\&{else}\1\5
${}\.{QUIT}(\.{"List\ info\ \%d\ must\ b}\)\.{e\ 1,\ 2,\ 3,\ 4,\ or\ 5"},\39\index{info+\\{info}}\\{info});{}$\2\6
\&{return} \|n;\6
\4${}\}{}$\2\6
\&{static} \&{void} \index{hteg list+\\{hteg\_list}}\\{hteg\_list}(\index{list t+\&{list\_t}}\&{list\_t} ${}{*}\|l){}$\1\1 $\{$ \X382:skip the end byte \|z\X\,\5
\&{if} ${}(\index{KIND+\.{KIND}}\.{KIND}(\|z)\I\index{list kind+\\{list\_kind}}\\{list\_kind}\W\index{KIND+\.{KIND}}\.{KIND}(\|z)\I\index{text kind+\\{text\_kind}}\\{text\_kind}\W\3{-1}\index{KIND+\.{KIND}}\.{KIND}(\|z)\I\index{param kind+\\{param\_kind}}\\{param\_kind}{}$)\6
\1${}\{{}$\5
${}\index{hpos+\\{hpos}}\\{hpos}\PP;{}$\6
${}\|l\MG\|p\K\index{hpos+\\{hpos}}\\{hpos}-\index{hstart+\\{hstart}}\\{hstart}{}$;\5
${}\|l\MG\|s\K\T{0}{}$;\5
${}\|l\MG\|k\K\index{list kind+\\{list\_kind}}\\{list\_kind}{}$;\5
${}\}{}$\2\6
\&{else}\5
\1${}\{{}$\5
\&{uint32\_t} \|s;\7
${}\|l\MG\|k\K\index{KIND+\.{KIND}}\.{KIND}(\|z);{}$\6
${}\|l\MG\|s\K\index{hteg list size+\\{hteg\_list\_size}}\\{hteg\_list\_size}(\index{INFO+\.{INFO}}\.{INFO}(\|z));{}$\6
\index{hteg size boundary+\\{hteg\_size\_boundary}}\\{hteg\_size\_boundary}(\index{INFO+\.{INFO}}\.{INFO}(\|z));\6
${}\index{hpos+\\{hpos}}\\{hpos}\K\index{hpos+\\{hpos}}\\{hpos}-\|l\MG\|s;{}$\6
${}\|l\MG\|p\K\index{hpos+\\{hpos}}\\{hpos}-\index{hstart+\\{hstart}}\\{hstart};{}$\6
\index{hteg size boundary+\\{hteg\_size\_boundary}}\\{hteg\_size\_boundary}(\index{INFO+\.{INFO}}\.{INFO}(\|z));\6
${}\|s\K\index{hteg list size+\\{hteg\_list\_size}}\\{hteg\_list\_size}(\index{INFO+\.{INFO}}\.{INFO}(\|z));{}$\6
\&{if} ${}(\|s\I\|l\MG\|s){}$\1\5
${}\.{QUIT}(\.{"List\ sizes\ at\ "}\.{SIZE\_F}\.{"\ and\ 0x\%x\ do\ not\ ma}\)\.{tch\ 0x\%x\ !=\ 0x\%x"},\39\index{hpos+\\{hpos}}\\{hpos}-\index{hstart+\\{hstart}}\\{hstart},\39\\{node\_pos}-\T{1},\39\|s,\39\|l\MG\|s);{}$\2\6
\X383:skip and check the start byte \|a\X\6
\4${}\}{}$\2\7
${}\}{}$\7
\&{static} \&{void} \index{hteg param list node+\\{hteg\_param\_list\_node}}\\{hteg\_param\_list\_node}(\index{list t+\&{list\_t}}\&{list\_t} ${}{*}\|l){}$\1\1\2\2\1\6
\4${}\{{}$\5
\&{if} ${}(\index{KIND+\.{KIND}}\.{KIND}({*}(\index{hpos+\\{hpos}}\\{hpos}-\T{1}))\I\index{param kind+\\{param\_kind}}\\{param\_kind}){}$\1\5
\&{return};\2\6
\index{hteg list+\\{hteg\_list}}\\{hteg\_list}(\|l);\6
\4${}\}{}$\2
\Y
\fi

\M{422}

\subsection{Adjustments}\index{adjustment}
\noindent
\Y\B\4\X390:cases to skip content\X${}\mathrel+\E{}$\6
\4\&{case} \.{TAG}${}(\\{adjust\_kind},\39\\{b001}){}$:\5
\1${}\{{}$\5
\index{list t+\&{list\_t}}\&{list\_t} \|l;\5
${}\index{hteg list+\\{hteg\_list}}\\{hteg\_list}({\AND}\|l){}$;\5
${}\}{}$\5
\2\&{break};
\Y
\fi

\M{423}

\subsection{Tables}\index{table}
\noindent
\Y\B\4\X384:skip macros\X${}\mathrel+\E{}$\6
\8\#\&{define} \index{HTEG TABLE+\.{HTEG\_TABLE}}\.{HTEG\_TABLE}(\|I)\1\1\2\2\1\6
\4${}\{{}$\5
\index{list t+\&{list\_t}}\&{list\_t} \|l;\5
${}\index{hteg list+\\{hteg\_list}}\\{hteg\_list}({\AND}\|l){}$;\5
${}\}{}$\2\1\7
\4${}\{{}$\5
\index{list t+\&{list\_t}}\&{list\_t} \|l;\5
${}\index{hteg list+\\{hteg\_list}}\\{hteg\_list}({\AND}\|l){}$;\5
${}\}{}$\2\6
\&{if} ${}((\|I)\AND\\{b100}){}$\5
\1${}\{{}$\5
\index{xdimen t+\&{xdimen\_t}}\&{xdimen\_t} \|x;\5
${}\index{hteg xdimen node+\\{hteg\_xdimen\_node}}\\{hteg\_xdimen\_node}({\AND}\|x){}$;\5
${}\}{}$\2\6
\&{else}\1\5
\index{HTEG REF+\.{HTEG\_REF}}\.{HTEG\_REF}(\index{xdimen kind+\\{xdimen\_kind}}\\{xdimen\_kind})\2
\Y
\fi

\M{424}

\Y\B\4\X390:cases to skip content\X${}\mathrel+\E{}$\6
\4\&{case} \.{TAG}${}(\index{table kind+\\{table\_kind}}\\{table\_kind},\39\\{b000}){}$:\5
\index{HTEG TABLE+\.{HTEG\_TABLE}}\.{HTEG\_TABLE}(\\{b000});\5
\&{break};\6
\4\&{case} \.{TAG}${}(\index{table kind+\\{table\_kind}}\\{table\_kind},\39\\{b001}){}$:\5
\index{HTEG TABLE+\.{HTEG\_TABLE}}\.{HTEG\_TABLE}(\\{b001});\5
\&{break};\6
\4\&{case} \.{TAG}${}(\index{table kind+\\{table\_kind}}\\{table\_kind},\39\\{b010}){}$:\5
\index{HTEG TABLE+\.{HTEG\_TABLE}}\.{HTEG\_TABLE}(\\{b010});\5
\&{break};\6
\4\&{case} \.{TAG}${}(\index{table kind+\\{table\_kind}}\\{table\_kind},\39\\{b011}){}$:\5
\index{HTEG TABLE+\.{HTEG\_TABLE}}\.{HTEG\_TABLE}(\\{b011});\5
\&{break};\6
\4\&{case} \.{TAG}${}(\index{table kind+\\{table\_kind}}\\{table\_kind},\39\\{b100}){}$:\5
\index{HTEG TABLE+\.{HTEG\_TABLE}}\.{HTEG\_TABLE}(\\{b100});\5
\&{break};\6
\4\&{case} \.{TAG}${}(\index{table kind+\\{table\_kind}}\\{table\_kind},\39\\{b101}){}$:\5
\index{HTEG TABLE+\.{HTEG\_TABLE}}\.{HTEG\_TABLE}(\\{b101});\5
\&{break};\6
\4\&{case} \.{TAG}${}(\index{table kind+\\{table\_kind}}\\{table\_kind},\39\\{b110}){}$:\5
\index{HTEG TABLE+\.{HTEG\_TABLE}}\.{HTEG\_TABLE}(\\{b110});\5
\&{break};\6
\4\&{case} \.{TAG}${}(\index{table kind+\\{table\_kind}}\\{table\_kind},\39\\{b111}){}$:\5
\index{HTEG TABLE+\.{HTEG\_TABLE}}\.{HTEG\_TABLE}(\\{b111});\5
\&{break};\7
\4\&{case} \.{TAG}${}(\index{item kind+\\{item\_kind}}\\{item\_kind},\39\\{b000}){}$:\5
\1${}\{{}$\5
\index{list t+\&{list\_t}}\&{list\_t} \|l;\5
${}\index{hteg list+\\{hteg\_list}}\\{hteg\_list}({\AND}\|l){}$;\5
${}\}{}$\5
\2\&{break};\6
\4\&{case} \.{TAG}${}(\index{item kind+\\{item\_kind}}\\{item\_kind},\39\\{b001}){}$:\5
\index{hteg content node+\\{hteg\_content\_node}}\\{hteg\_content\_node}(\,);\5
\&{break};\6
\4\&{case} \.{TAG}${}(\index{item kind+\\{item\_kind}}\\{item\_kind},\39\\{b010}){}$:\5
\index{hteg content node+\\{hteg\_content\_node}}\\{hteg\_content\_node}(\,);\5
\&{break};\6
\4\&{case} \.{TAG}${}(\index{item kind+\\{item\_kind}}\\{item\_kind},\39\\{b011}){}$:\5
\index{hteg content node+\\{hteg\_content\_node}}\\{hteg\_content\_node}(\,);\5
\&{break};\6
\4\&{case} \.{TAG}${}(\index{item kind+\\{item\_kind}}\\{item\_kind},\39\\{b100}){}$:\5
\index{hteg content node+\\{hteg\_content\_node}}\\{hteg\_content\_node}(\,);\5
\&{break};\6
\4\&{case} \.{TAG}${}(\index{item kind+\\{item\_kind}}\\{item\_kind},\39\\{b101}){}$:\5
\index{hteg content node+\\{hteg\_content\_node}}\\{hteg\_content\_node}(\,);\5
\&{break};\6
\4\&{case} \.{TAG}${}(\index{item kind+\\{item\_kind}}\\{item\_kind},\39\\{b110}){}$:\5
\index{hteg content node+\\{hteg\_content\_node}}\\{hteg\_content\_node}(\,);\5
\&{break};\6
\4\&{case} \.{TAG}${}(\index{item kind+\\{item\_kind}}\\{item\_kind},\39\\{b111}){}$:\1\6
\4${}\{{}$\5
\&{uint8\_t} \|n;\5
\index{HTEG8+\.{HTEG8}}\.{HTEG8}(\|n);\5
${}\}{}$\5
\2\index{hteg content node+\\{hteg\_content\_node}}\\{hteg\_content\_node}(\,);\5
\&{break};
\Y
\fi

\M{425}



\subsection{Stream Nodes}\index{stream}
\Y\B\4\X384:skip macros\X${}\mathrel+\E{}$\6
\8\#\&{define} \index{HTEG STREAM+\.{HTEG\_STREAM}}\.{HTEG\_STREAM}(\|I)\1\6
\4${}\{{}$\5
\index{list t+\&{list\_t}}\&{list\_t} \|l;\5
${}\index{hteg list+\\{hteg\_list}}\\{hteg\_list}({\AND}\|l){}$;\5
${}\}{}$\2\7
\&{if} ${}((\|I)\AND\\{b010}){}$\5
\1${}\{{}$\5
\index{list t+\&{list\_t}}\&{list\_t} \|l;\5
${}\index{hteg param list node+\\{hteg\_param\_list\_node}}\\{hteg\_param\_list\_node}({\AND}\|l){}$;\5
${}\}{}$\5
\2\&{else}\1\5
\index{HTEG REF+\.{HTEG\_REF}}\.{HTEG\_REF}(\index{param kind+\\{param\_kind}}\\{param\_kind});\2\6
\index{HTEG REF+\.{HTEG\_REF}}\.{HTEG\_REF}(\index{stream kind+\\{stream\_kind}}\\{stream\_kind});
\Y
\fi

\M{426}

\Y\B\4\X390:cases to skip content\X${}\mathrel+\E{}$\6
\4\&{case} \.{TAG}${}(\index{stream kind+\\{stream\_kind}}\\{stream\_kind},\39\\{b000}){}$:\5
\index{HTEG STREAM+\.{HTEG\_STREAM}}\.{HTEG\_STREAM}(\\{b000});\5
\&{break};\6
\4\&{case} \.{TAG}${}(\index{stream kind+\\{stream\_kind}}\\{stream\_kind},\39\\{b010}){}$:\5
\index{HTEG STREAM+\.{HTEG\_STREAM}}\.{HTEG\_STREAM}(\\{b010});\5
\&{break};
\Y
\fi

\M{427}



\subsection{References}\index{reference}
\noindent
\Y\B\4\X384:skip macros\X${}\mathrel+\E{}$\6
\8\#\&{define} \index{HTEG REF+\.{HTEG\_REF}}\.{HTEG\_REF}(\|K) \&{do}\5
\1${}\{{}$\5
\&{uint8\_t} \|n;\5
\index{HTEG8+\.{HTEG8}}\.{HTEG8}(\|n);\5
${}\}{}$\5
\2\&{while} (\\{false})
\Y
\fi

\M{428}

\Y\B\4\X390:cases to skip content\X${}\mathrel+\E{}$\6
\4\&{case} \.{TAG}${}(\index{penalty kind+\\{penalty\_kind}}\\{penalty\_kind},\39\T{0}){}$:\5
\index{HTEG REF+\.{HTEG\_REF}}\.{HTEG\_REF}(\index{penalty kind+\\{penalty\_kind}}\\{penalty\_kind});\5
\&{break};\6
\4\&{case} \.{TAG}${}(\index{kern kind+\\{kern\_kind}}\\{kern\_kind},\39\\{b000}){}$:\5
\index{HTEG REF+\.{HTEG\_REF}}\.{HTEG\_REF}(\index{dimen kind+\\{dimen\_kind}}\\{dimen\_kind});\5
\&{break};\6
\4\&{case} \.{TAG}${}(\index{kern kind+\\{kern\_kind}}\\{kern\_kind},\39\\{b100}){}$:\5
\index{HTEG REF+\.{HTEG\_REF}}\.{HTEG\_REF}(\index{dimen kind+\\{dimen\_kind}}\\{dimen\_kind});\5
\&{break};\6
\4\&{case} \.{TAG}${}(\index{kern kind+\\{kern\_kind}}\\{kern\_kind},\39\\{b001}){}$:\5
\index{HTEG REF+\.{HTEG\_REF}}\.{HTEG\_REF}(\index{xdimen kind+\\{xdimen\_kind}}\\{xdimen\_kind});\5
\&{break};\6
\4\&{case} \.{TAG}${}(\index{kern kind+\\{kern\_kind}}\\{kern\_kind},\39\\{b101}){}$:\5
\index{HTEG REF+\.{HTEG\_REF}}\.{HTEG\_REF}(\index{xdimen kind+\\{xdimen\_kind}}\\{xdimen\_kind});\5
\&{break};\6
\4\&{case} \.{TAG}${}(\index{ligature kind+\\{ligature\_kind}}\\{ligature\_kind},\39\T{0}){}$:\5
\index{HTEG REF+\.{HTEG\_REF}}\.{HTEG\_REF}(\index{ligature kind+\\{ligature\_kind}}\\{ligature\_kind});\5
\&{break};\6
\4\&{case} \.{TAG}${}(\index{hyphen kind+\\{hyphen\_kind}}\\{hyphen\_kind},\39\T{0}){}$:\5
\index{HTEG REF+\.{HTEG\_REF}}\.{HTEG\_REF}(\index{hyphen kind+\\{hyphen\_kind}}\\{hyphen\_kind});\5
\&{break};\6
\4\&{case} \.{TAG}${}(\index{glue kind+\\{glue\_kind}}\\{glue\_kind},\39\T{0}){}$:\5
\index{HTEG REF+\.{HTEG\_REF}}\.{HTEG\_REF}(\index{glue kind+\\{glue\_kind}}\\{glue\_kind});\5
\&{break};\6
\4\&{case} \.{TAG}${}(\index{language kind+\\{language\_kind}}\\{language\_kind},\39\T{0}){}$:\5
\index{HTEG REF+\.{HTEG\_REF}}\.{HTEG\_REF}(\index{language kind+\\{language\_kind}}\\{language\_kind});\5
\&{break};\6
\4\&{case} \.{TAG}${}(\index{rule kind+\\{rule\_kind}}\\{rule\_kind},\39\T{0}){}$:\5
\index{HTEG REF+\.{HTEG\_REF}}\.{HTEG\_REF}(\index{rule kind+\\{rule\_kind}}\\{rule\_kind});\5
\&{break};\6
\4\&{case} \.{TAG}${}(\index{image kind+\\{image\_kind}}\\{image\_kind},\39\T{0}){}$:\5
\index{HTEG REF+\.{HTEG\_REF}}\.{HTEG\_REF}(\index{image kind+\\{image\_kind}}\\{image\_kind});\5
\&{break};\6
\4\&{case} \.{TAG}${}(\index{leaders kind+\\{leaders\_kind}}\\{leaders\_kind},\39\T{0}){}$:\5
\index{HTEG REF+\.{HTEG\_REF}}\.{HTEG\_REF}(\index{leaders kind+\\{leaders\_kind}}\\{leaders\_kind});\5
\&{break};\6
\4\&{case} \.{TAG}${}(\index{baseline kind+\\{baseline\_kind}}\\{baseline\_kind},\39\T{0}){}$:\5
\index{HTEG REF+\.{HTEG\_REF}}\.{HTEG\_REF}(\index{baseline kind+\\{baseline\_kind}}\\{baseline\_kind});\5
\&{break};
\Y
\fi

\M{429}


\section{Code and Header Files}\index{code file}\index{header file}

\subsection{{\tt basetypes.h}}
To define basic types in a portable way, we create an include file.
The macro \index{ MSC VER+\.{\_MSC\_VER}}\.{\_MSC\_VER} (Microsoft Visual C Version) is defined only if
using the respective compiler.\index{false+\\{false}}\index{true+\\{true}}\index{bool+\&{bool}}
\Y\B\4\X429:\.{basetypes.h }\X${}\E{}$\6
\8\#\&{ifndef} \.{\_\_BASETYPES\_H\_\_}\6
\8\#\&{define} \.{\_\_BASETYPES\_H\_\_}\6
\8\#\&{include} \.{<stdlib.h>}\6
\8\#\&{include} \.{<stdio.h>}\6
\8\#\&{ifndef} \.{\_STDLIB\_H}\6
\8\#\&{define} \.{\_STDLIB\_H}\6
\8\#\&{endif}\6
\8\#\&{ifdef} \index{ MSC VER+\.{\_MSC\_VER}}\.{\_MSC\_VER}\6
\8\#\&{include} \.{<windows.h>}\6
\8\#\&{define} \&{uint8\_t} \index{UINT8+\.{UINT8}}\.{UINT8} \6
\8\#\&{define} \&{uint16\_t} \index{UINT16+\.{UINT16}}\.{UINT16} \6
\8\#\&{define} \&{uint32\_t} \index{UINT32+\.{UINT32}}\.{UINT32} \6
\8\#\&{define} \&{uint64\_t} \index{UINT64+\.{UINT64}}\.{UINT64} \6
\8\#\&{define} \&{int8\_t} \index{INT8+\.{INT8}}\.{INT8} \6
\8\#\&{define} \&{int16\_t} \index{INT16+\.{INT16}}\.{INT16} \6
\8\#\&{define} \&{int32\_t} \index{INT32+\.{INT32}}\.{INT32} \6
\8\#\&{define} \&{bool} \index{BOOL+\.{BOOL}}\.{BOOL} \6
\8\#\&{define} \\{true}\5${}(\T{0}\E\T{0}){}$\6
\8\#\&{define} \\{false}\5${}(\R\\{true}){}$\6
\8\#\&{define} \.{\_\_SIZEOF\_FLOAT\_\_}\5\T{4}\6
\8\#\&{define} \.{\_\_SIZEOF\_DOUBLE\_\_}\5\T{8}\6
\&{typedef} \&{float} \&{float32\_t};\6
\&{typedef} \&{double} \&{float64\_t};\6
\8\#\&{define} \index{INT32 MAX+\.{INT32\_MAX}}\.{INT32\_MAX}\5(\T{2147483647})\6
\8\#\&{define} \index{PRIx64+\\{PRIx64}}\\{PRIx64}\5\.{"I64x"}\6
\8\#\&{pragma} \index{warning+\\{warning}}\\{warning}(\index{disable+\\{disable}}\\{disable}:\T{4244}\hbox{ }\T{4996}\hbox{ }\T{4127})\6
\8\#\&{else}\6
\8\#\&{include} \.{<stdint.h>}\6
\8\#\&{include} \.{<stdbool.h>}\6
\8\#\&{include} \.{<inttypes.h>}\6
\&{typedef} \&{float} \&{float32\_t};\6
\&{typedef} \&{double} \&{float64\_t};\6
\8\#\&{ifdef} \index{WIN32+\.{WIN32}}\.{WIN32}\6
\8\#\&{include} \.{<io.h>}\6
\8\#\&{endif}\6
\8\#\&{endif}\6
\8\#\&{if} ${}\.{\_\_SIZEOF\_FLOAT\_\_}\I\T{4}{}$\6
\8\#\&{error} \vb{float32\ type\ must\ have\ size\ 4}\6
\8\#\&{endif}\6
\8\#\&{if} ${}\.{\_\_SIZEOF\_DOUBLE\_\_}\I\T{8}{}$\6
\8\#\&{error} \vb{float64\ type\ must\ have\ size\ 8}\6
\8\#\&{endif}\6
\8\#\&{endif}
\Y
\fi

\M{430}



\subsection{{\tt hformat.h}}
The \.{hformat.c} file contains variables and functions that are needed
in other compilation units. Together with the required type and macro
definitions, the necessary information is contained in the \.{hformat.h}
header file.


\Y\B\4\X430:write function declarations\X${}\E{}$\6
\8\#\&{define} \index{hwritec+\\{hwritec}}\\{hwritec}(\|c)\5${}\index{putc+\\{putc}}\\{putc}(\|c,\39\index{hout+\\{hout}}\\{hout}){}$\6
\8\#\&{define} ${}\index{hwritef+\\{hwritef}}\\{hwritef}(\,\ldots\,)\5\index{fprintf+\\{fprintf}}\\{fprintf}(\index{hout+\\{hout}}\\{hout},\39\.{\_\_VA\_ARGS\_\_}){}$\6
\&{extern} \&{void} \index{hwrite range+\\{hwrite\_range}}\\{hwrite\_range}(\&{void});\6
\&{extern} \&{void} \index{hwrite charcode+\\{hwrite\_charcode}}\\{hwrite\_charcode}(\&{uint32\_t} \|c);\6
\&{extern} \&{void} \index{hwrite ref node+\\{hwrite\_ref\_node}}\\{hwrite\_ref\_node}(\&{uint8\_t} \|k${},\39{}$\&{uint8\_t} \|n);\6
\&{extern} \&{void} \index{hwrite ref+\\{hwrite\_ref}}\\{hwrite\_ref}(\&{uint8\_t} \|n);\6
\&{extern} \&{void} \index{hsort ranges+\\{hsort\_ranges}}\\{hsort\_ranges}(\&{void});
\U439.\Y
\fi

\M{431}


\subsection{{\tt hget.h}}\index{hget.h+{\tt hget.h}}
The \.{hget.h} file contains function prototypes for all the functions
that read the short format.

\Y\B\4\X431:get function declarations\X${}\E{}$\6
\&{extern} \&{uint8\_t} \index{hget content node+\\{hget\_content\_node}}\\{hget\_content\_node}(\&{void});\6
\&{extern} \&{int} \index{txt length+\\{txt\_length}}\\{txt\_length};\6
\&{extern} \&{int} \index{hget txt+\\{hget\_txt}}\\{hget\_txt}(\&{void});\6
\&{extern} \&{uint32\_t} \index{hget utf8+\\{hget\_utf8}}\\{hget\_utf8}(\&{void});\6
\&{extern} \&{void} \index{hget def node+\\{hget\_def\_node}}\\{hget\_def\_node}(\index{ref t+\&{ref\_t}}\&{ref\_t} ${}{*}\index{df+\\{df}}\\{df});{}$\6
\&{extern} \&{void} \index{hget content section+\\{hget\_content\_section}}\\{hget\_content\_section}(\&{void});\6
\&{extern} \&{void} \index{hget content+\\{hget\_content}}\\{hget\_content}(\&{uint8\_t} \|a);\6
\&{extern} \&{void} \index{hget xdimen node+\\{hget\_xdimen\_node}}\\{hget\_xdimen\_node}(\index{xdimen t+\&{xdimen\_t}}\&{xdimen\_t} ${}{*}\|x);{}$\6
\&{extern} \&{float32\_t} \index{hget float32+\\{hget\_float32}}\\{hget\_float32}(\&{void});\6
\&{extern} \&{void} \index{hget list+\\{hget\_list}}\\{hget\_list}(\index{list t+\&{list\_t}}\&{list\_t} ${}{*}\|l);{}$\6
\&{extern} \&{void} \index{hget glue node+\\{hget\_glue\_node}}\\{hget\_glue\_node}(\&{void});\6
\&{extern} \&{void} \index{hget rule node+\\{hget\_rule\_node}}\\{hget\_rule\_node}(\&{void});\6
\&{extern} \&{void} \index{hget hbox node+\\{hget\_hbox\_node}}\\{hget\_hbox\_node}(\&{void});\6
\&{extern} \&{void} \index{hget vbox node+\\{hget\_vbox\_node}}\\{hget\_vbox\_node}(\&{void});\6
\&{extern} \&{void} \index{hget param list node+\\{hget\_param\_list\_node}}\\{hget\_param\_list\_node}(\index{list t+\&{list\_t}}\&{list\_t} ${}{*}\|l);{}$\6
\&{extern} \&{uint32\_t} \index{hget list size+\\{hget\_list\_size}}\\{hget\_list\_size}(\index{info t+\&{info\_t}}\&{info\_t} \index{info+\\{info}}\\{info});\6
\&{extern} \&{void} \index{hget size boundary+\\{hget\_size\_boundary}}\\{hget\_size\_boundary}(\index{info t+\&{info\_t}}\&{info\_t} \index{info+\\{info}}\\{info});\6
\&{extern} \&{void} \index{hget max definitions+\\{hget\_max\_definitions}}\\{hget\_max\_definitions}(\&{void});\6
\&{extern} \&{void} \index{hget font def+\\{hget\_font\_def}}\\{hget\_font\_def}(\index{info t+\&{info\_t}}\&{info\_t} \|i${},\39{}$\&{uint8\_t} \|f);
\U439.\Y
\fi

\M{432}

\Y\B\4\X432:\.{hget.h }\X${}\E{}$\6
\X35:get file macros\X\6
\X283:directory entry type\X\7
\&{extern} \index{entry t+\&{entry\_t}}\&{entry\_t} ${}{*}\index{dir+\\{dir}}\\{dir};{}$\6
\&{extern} \&{uint16\_t} \index{section no+\\{section\_no}}\\{section\_no}${},{}$ \index{max section no+\\{max\_section\_no}}\\{max\_section\_no};\6
\&{extern} \&{uint8\_t} ${}{*}\index{hpos+\\{hpos}}\\{hpos},{}$ ${}{*}\index{hstart+\\{hstart}}\\{hstart},{}$ ${}{*}\index{hend+\\{hend}}\\{hend};{}$\6
\&{extern} \&{void} \index{hget map+\\{hget\_map}}\\{hget\_map}(\&{void});\6
\&{extern} \&{void} \index{hget unmap+\\{hget\_unmap}}\\{hget\_unmap}(\&{void});\6
\&{extern} \&{void} \index{new directory+\\{new\_directory}}\\{new\_directory}(\&{uint32\_t} \index{entries+\\{entries}}\\{entries});\6
\&{extern} \&{void} \index{hset entry+\\{hset\_entry}}\\{hset\_entry}(\index{entry t+\&{entry\_t}}\&{entry\_t} ${}{*}\|e,\39{}$\&{uint16\_t} \|i${},\39{}$\&{uint32\_t} \index{size+\\{size}}\\{size}${},\39{}$\&{uint32\_t} \index{xsize+\\{xsize}}\\{xsize}${},\3{-1}\39{}$\&{char} ${}{*}\index{file name+\\{file\_name}}\\{file\_name});{}$\6
\&{extern} \&{void} \index{hget banner+\\{hget\_banner}}\\{hget\_banner}(\&{void});\6
\&{extern} \&{void} \index{hget section+\\{hget\_section}}\\{hget\_section}(\&{uint16\_t} \|n);\6
\&{extern} \&{void} \index{hget entry+\\{hget\_entry}}\\{hget\_entry}(\index{entry t+\&{entry\_t}}\&{entry\_t} ${}{*}\|e);{}$\6
\&{extern} \&{void} \index{hget directory+\\{hget\_directory}}\\{hget\_directory}(\&{void});\6
\&{extern} \&{void} \index{hclear dir+\\{hclear\_dir}}\\{hclear\_dir}(\&{void});\6
\&{extern} \&{bool} \index{hcheck banner+\\{hcheck\_banner}}\\{hcheck\_banner}(\&{char} ${}{*}\index{magic+\\{magic}}\\{magic});{}$\6
\&{extern} \&{void} \index{hget max definitions+\\{hget\_max\_definitions}}\\{hget\_max\_definitions}(\&{void});
\Y
\fi

\M{433}



\subsection{{\tt hget.c}}\index{hget.c+{\tt hget.c}}
\Y\B\4\X433:\.{hget.c }\X${}\E{}$\6
\8\#\&{include} \.{"basetypes.h"}\6
\8\#\&{include} \.{<string.h>}\6
\8\#\&{include} \.{<math.h>}\6
\8\#\&{include} \.{<zlib.h>}\6
\8\#\&{include} \.{<sys/types.h>}\6
\8\#\&{include} \.{<sys/stat.h>}\6
\8\#\&{include} \.{<fcntl.h>}\6
\8\#\&{include} \.{"error.h"}\6
\8\#\&{include} \.{"hformat.h"}\6
\8\#\&{include} \.{"hget.h"}\6
\&{uint8\_t} ${}{*}\index{hpos+\\{hpos}}\\{hpos}\K\NULL,{}$ ${}{*}\index{hstart+\\{hstart}}\\{hstart}\K\NULL,{}$ ${}{*}\index{hend+\\{hend}}\\{hend}\K\NULL;{}$\7
\X274:map functions\X\6
\X262:function to check the banner\X\6
\X284:directory functions\X\6
\X35:get file macros\X\6
\X263:get file functions\X
\Y
\fi

\M{434}

\subsection{{\tt hput.h}}\index{hput.h+{\tt hput.h}}
The \.{hput.h} file contains function prototypes for all the functions
that write the short format.


\Y\B\4\X434:\.{hput.h }\X${}\E{}$\6
\X271:put macros\X\6
\X11:hint macros\X\6
\X1:hint types\X\6
\X283:directory entry type\X\7
\&{extern} \index{entry t+\&{entry\_t}}\&{entry\_t} ${}{*}\index{dir+\\{dir}}\\{dir};{}$\6
\&{extern} \&{uint16\_t} \index{section no+\\{section\_no}}\\{section\_no}${},{}$ \index{max section no+\\{max\_section\_no}}\\{max\_section\_no};\6
\&{extern} \&{uint8\_t} ${}{*}\index{hpos+\\{hpos}}\\{hpos},{}$ ${}{*}\index{hstart+\\{hstart}}\\{hstart},{}$ ${}{*}\index{hend+\\{hend}}\\{hend};{}$\6
\&{extern} \&{int} \index{next range+\\{next\_range}}\\{next\_range};\6
\&{extern} \index{range pos t+\&{range\_pos\_t}}\&{range\_pos\_t} ${}{*}\index{range pos+\\{range\_pos}}\\{range\_pos};{}$\6
\&{extern} \&{int} ${}{*}\index{page on+\\{page\_on}}\\{page\_on};{}$\6
\&{extern} \&{FILE} ${}{*}\index{hout+\\{hout}}\\{hout};{}$\6
\&{extern} \&{void} \index{new directory+\\{new\_directory}}\\{new\_directory}(\&{uint32\_t} \index{entries+\\{entries}}\\{entries});\6
\&{extern} \&{void} \index{new output buffers+\\{new\_output\_buffers}}\\{new\_output\_buffers}(\&{void});\C{ declarations for the parser }\6
\&{extern} \&{void} \index{hput definitions start+\\{hput\_definitions\_start}}\\{hput\_definitions\_start}(\&{void});\6
\&{extern} \&{void} \index{hput definitions end+\\{hput\_definitions\_end}}\\{hput\_definitions\_end}(\&{void});\6
\&{extern} \&{void} \index{hput content start+\\{hput\_content\_start}}\\{hput\_content\_start}(\&{void});\6
\&{extern} \&{void} \index{hput content end+\\{hput\_content\_end}}\\{hput\_content\_end}(\&{void});\6
\&{extern} \&{void} \index{hput tags+\\{hput\_tags}}\\{hput\_tags}(\&{uint32\_t} \index{pos+\\{pos}}\\{pos}${},\39{}$\&{uint8\_t} \index{tag+\\{tag}}\\{tag});\6
\&{extern} \&{uint8\_t} \index{hput glyph+\\{hput\_glyph}}\\{hput\_glyph}(\index{glyph t+\&{glyph\_t}}\&{glyph\_t} ${}{*}\|g);{}$\6
\&{extern} \&{uint8\_t} \index{hput xdimen+\\{hput\_xdimen}}\\{hput\_xdimen}(\index{xdimen t+\&{xdimen\_t}}\&{xdimen\_t} ${}{*}\|x);{}$\6
\&{extern} \&{uint8\_t} \index{hput int+\\{hput\_int}}\\{hput\_int}(\&{int32\_t} \|p);\6
\&{extern} \&{uint8\_t} \index{hput language+\\{hput\_language}}\\{hput\_language}(\&{uint8\_t} \|n);\6
\&{extern} \&{uint8\_t} \index{hput rule+\\{hput\_rule}}\\{hput\_rule}(\index{rule t+\&{rule\_t}}\&{rule\_t} ${}{*}\|r);{}$\6
\&{extern} \&{uint8\_t} \index{hput glue+\\{hput\_glue}}\\{hput\_glue}(\index{glue t+\&{glue\_t}}\&{glue\_t} ${}{*}\|g);{}$\6
\&{extern} \&{uint8\_t} \index{hput list+\\{hput\_list}}\\{hput\_list}(\&{uint32\_t} \index{size pos+\\{size\_pos}}\\{size\_pos}${},\39{}$\index{list t+\&{list\_t}}\&{list\_t} ${}{*}\|y);{}$\6
\&{extern} \&{uint8\_t} \index{hsize bytes+\\{hsize\_bytes}}\\{hsize\_bytes}(\&{uint32\_t} \|n);\6
\&{extern} \&{void} \index{hput txt cc+\\{hput\_txt\_cc}}\\{hput\_txt\_cc}(\&{uint32\_t} \|c);\6
\&{extern} \&{void} \index{hput txt font+\\{hput\_txt\_font}}\\{hput\_txt\_font}(\&{uint8\_t} \|f);\6
\&{extern} \&{void} \index{hput txt global+\\{hput\_txt\_global}}\\{hput\_txt\_global}(\index{ref t+\&{ref\_t}}\&{ref\_t} ${}{*}\|d);{}$\6
\&{extern} \&{void} \index{hput txt local+\\{hput\_txt\_local}}\\{hput\_txt\_local}(\&{uint8\_t} \|n);\6
\&{extern} \index{info t+\&{info\_t}}\&{info\_t} \index{hput box dimen+\\{hput\_box\_dimen}}\\{hput\_box\_dimen}(\index{dimen t+\&{dimen\_t}}\&{dimen\_t} \|h${},\39{}$\index{dimen t+\&{dimen\_t}}\&{dimen\_t} \|d${},\39{}$\index{dimen t+\&{dimen\_t}}\&{dimen\_t} \|w);\6
\&{extern} \index{info t+\&{info\_t}}\&{info\_t} \index{hput box shift+\\{hput\_box\_shift}}\\{hput\_box\_shift}(\index{dimen t+\&{dimen\_t}}\&{dimen\_t} \|a);\6
\&{extern} \index{info t+\&{info\_t}}\&{info\_t} \index{hput box glue set+\\{hput\_box\_glue\_set}}\\{hput\_box\_glue\_set}(\&{int8\_t} \|s${},\39{}$\&{float32\_t} \|r${},\39{}$\index{order t+\&{order\_t}}\&{order\_t} \|o);\6
\&{extern} \&{void} \index{hput stretch+\\{hput\_stretch}}\\{hput\_stretch}(\index{stretch t+\&{stretch\_t}}\&{stretch\_t} ${}{*}\|s);{}$\6
\&{extern} \&{uint8\_t} \index{hput kern+\\{hput\_kern}}\\{hput\_kern}(\index{kern t+\&{kern\_t}}\&{kern\_t} ${}{*}\|k);{}$\6
\&{extern} \&{void} \index{hput utf8+\\{hput\_utf8}}\\{hput\_utf8}(\&{uint32\_t} \|c);\6
\&{extern} \&{uint8\_t} \index{hput ligature+\\{hput\_ligature}}\\{hput\_ligature}(\index{lig t+\&{lig\_t}}\&{lig\_t} ${}{*}\|l);{}$\6
\&{extern} \&{uint8\_t} \index{hput hyphen+\\{hput\_hyphen}}\\{hput\_hyphen}(\index{hyphen t+\&{hyphen\_t}}\&{hyphen\_t} ${}{*}\|h);{}$\6
\&{extern} \&{uint8\_t} \index{hput item+\\{hput\_item}}\\{hput\_item}(\&{uint32\_t} \|n);\6
\&{extern} \&{uint8\_t} \index{hput image+\\{hput\_image}}\\{hput\_image}(\index{image t+\&{image\_t}}\&{image\_t} ${}{*}\|x);{}$\6
\&{extern} \&{void} \index{hput string+\\{hput\_string}}\\{hput\_string}(\&{char} ${}{*}\index{str+\\{str}}\\{str});{}$\6
\&{extern} \&{void} \index{hput range+\\{hput\_range}}\\{hput\_range}(\&{uint8\_t} \index{pg+\\{pg}}\\{pg}${},\39{}$\&{bool} \index{on+\\{on}}\\{on});\6
\&{extern} \&{void} \index{hput max definitions+\\{hput\_max\_definitions}}\\{hput\_max\_definitions}(\&{void});\6
\&{extern} \&{uint8\_t} \index{hput dimen+\\{hput\_dimen}}\\{hput\_dimen}(\index{dimen t+\&{dimen\_t}}\&{dimen\_t} \|d);\6
\&{extern} \&{uint8\_t} \index{hput font head+\\{hput\_font\_head}}\\{hput\_font\_head}(\&{uint8\_t} \|f${},\39{}$\&{char} ${}{*}\|n,\39{}$\index{dimen t+\&{dimen\_t}}\&{dimen\_t} \|s${},\39{}$\&{uint16\_t} \|m${},\39{}$\&{uint16\_t} \|y);\6
\&{extern} \&{void} \index{hput range defs+\\{hput\_range\_defs}}\\{hput\_range\_defs}(\&{void});\C{ declarations for HiTeX }\6
\&{extern} \&{void} \index{hput xdimen node+\\{hput\_xdimen\_node}}\\{hput\_xdimen\_node}(\index{xdimen t+\&{xdimen\_t}}\&{xdimen\_t} ${}{*}\|x);{}$\6
\&{extern} \&{void} \index{hput directory+\\{hput\_directory}}\\{hput\_directory}(\&{void});\6
\&{extern} \&{void} \index{hput hint+\\{hput\_hint}}\\{hput\_hint}(\&{char} ${}{*}\index{str+\\{str}}\\{str});{}$\6
\&{extern} \&{void} \index{hput list size+\\{hput\_list\_size}}\\{hput\_list\_size}(\&{uint32\_t} \|n${},\39{}$\&{int} \|i);
\Y
\fi

\M{435}


\subsection{{\tt hput.c}}\label{writeshort}\index{hput.c+{\tt hput.c}}
\noindent
\Y\B\4\X435:\.{hput.c }\X${}\E{}$\6
\8\#\&{include} \.{"basetypes.h"}\6
\8\#\&{include} \.{<string.h>}\6
\8\#\&{include} \.{<ctype.h>}\6
\8\#\&{include} \.{<sys/stat.h>}\6
\8\#\&{include} \.{<zlib.h>}\6
\8\#\&{include} \.{"error.h"}\6
\8\#\&{include} \.{"hformat.h"}\6
\8\#\&{include} \.{"hput.h"}\6
\&{uint8\_t} ${}{*}\index{hpos+\\{hpos}}\\{hpos}\K\NULL,{}$ ${}{*}\index{hstart+\\{hstart}}\\{hstart}\K\NULL,{}$ ${}{*}\index{hend+\\{hend}}\\{hend}\K\NULL;{}$\6
\&{FILE} ${}{*}\index{hout+\\{hout}}\\{hout};{}$\6
\&{int} \index{version+\\{version}}\\{version}${}\K\T{1},{}$ \index{subversion+\\{subversion}}\\{subversion}${}\K\T{0};{}$\6
\&{bool} \index{option compress+\\{option\_compress}}\\{option\_compress}${}\K\\{false};{}$\6
\&{bool} \index{option global+\\{option\_global}}\\{option\_global}${}\K\\{false};{}$\6
\&{int} \index{next range+\\{next\_range}}\\{next\_range};\6
\index{range pos t+\&{range\_pos\_t}}\&{range\_pos\_t} ${}{*}\index{range pos+\\{range\_pos}}\\{range\_pos};{}$\6
\&{int} ${}{*}\index{page on+\\{page\_on}}\\{page\_on};{}$\6
\&{char} ${}{*}\index{stem name+\\{stem\_name}}\\{stem\_name}\K\NULL;{}$\6
\&{int} \index{stem length+\\{stem\_length}}\\{stem\_length}${}\K\T{0};{}$\7
\X284:directory functions\X\6
\X265:function to write the banner\X\6
\X12:put functions\X
\Y
\fi

\M{436}

\subsection{{\tt shrink.l}}\index{shrink.l+{\tt shrink.l}}\index{scanning}
The definitions for lex are collected in the file {\tt shrink.l}

\Y\B\4\X436:\.{shrink.l }\X${}\E{}$\6
\8\%\raise 1pt\hbox{$\,\{$}\6
\8\#\&{include} \.{"basetypes.h"}\6
\8\#\&{include} \.{<unistd.h>}\6
\8\#\&{include} \.{"error.h"}\6
\8\#\&{include} \.{"hformat.h"}\6
\8\#\&{include} \.{"hput.h"}\6
\X379:enable bison debugging\X\6
\8\#\&{include} \.{"shrink.tab.h"}\6
\X20:scanning macros\X\6
\X59:scanning functions\X\7
\&{int} \index{yywrap+\\{yywrap}}\\{yywrap}(\&{void})\1\1\2\2\1\6
\4${}\{{}$\5
\&{return} \T{1};\6
\4${}\}{}$\2\6
\8\#\&{ifdef} \index{ MSC VER+\.{\_MSC\_VER}}\.{\_MSC\_VER}\6
\8\#\&{pragma} \index{warning+\\{warning}}\\{warning} ( \index{disable+\\{disable}}\\{disable}: \T{4267} ) \6
\8\#\&{endif}\6
\8\%\raise 1pt\hbox{$\,\}$}\6
\8\%\index{option+\&{option}}\&{option} \index{yylineno+\\{yylineno}}\\{yylineno} \\{batch}\5\index{stack+\\{stack}}\\{stack}\6
\8\%\index{option+\&{option}}\&{option} \index{debug+\\{debug}}\\{debug}\6
\8\%\index{option+\&{option}}\&{option} \index{nounistd+\\{nounistd}}\\{nounistd} \index{nounput+\\{nounput}}\\{nounput} \index{noinput+\\{noinput}}\\{noinput}\5\index{noyy top state+\\{noyy\_top\_state}}\\{noyy\_top\_state}\6
\X21:scanning definitions\X\6
\8\%\%\6
\X3:scanning rules\X\6
${}\8\re{\vb{[a-z]+}}{}$\ac${}\.{QUIT}(\.{"Unexpected\ keyword\ }\)\.{'\%s'\ in\ line\ \%d"},\3{-1}\39\index{yytext+\\{yytext}}\\{yytext},\39\index{yylineno+\\{yylineno}}\\{yylineno});\eac{}$\7
${}\8\re{\vb{.}}{}$\ac${}\.{QUIT}(\.{"Unexpected\ characte}\)\.{r\ '\%c'\ (0x\%02X)\ in\ l}\)\.{ine\ \%d"},\3{-1}\39\index{yytext+\\{yytext}}\\{yytext}[\T{0}]>\.{'\ '}\?\index{yytext+\\{yytext}}\\{yytext}[\T{0}]:\.{'\ '},\39\index{yytext+\\{yytext}}\\{yytext}[\T{0}],\39\index{yylineno+\\{yylineno}}\\{yylineno});\eac{}$\7
\8\%\%
\Y
\fi

\M{437}



\subsection{{\tt shrink.y}}\index{shrink.y+{\tt shrink.y}}\index{parsing}

The grammar rules for bison are collected in the file  {\tt shrink.y}.
% for the option %token-table use the command line parameter -k


\Y\B\4\X437:\.{shrink.y }\X${}\E{}$\6
\8\%\raise 1pt\hbox{$\,\{$}\6
\8\#\&{include} \.{"basetypes.h"}\6
\8\#\&{include} \.{<string.h>}\6
\8\#\&{include} \.{<math.h>}\6
\8\#\&{include} \.{"error.h"}\6
\8\#\&{include} \.{"hformat.h"}\6
\8\#\&{include} \.{"hput.h"}\6
\&{char} ${}{*}{*}\index{hfont name+\\{hfont\_name}}\\{hfont\_name};{}$\7
\X313:definition checks\X\7
\&{extern} \&{void} \index{hset entry+\\{hset\_entry}}\\{hset\_entry}(\index{entry t+\&{entry\_t}}\&{entry\_t} ${}{*}\|e,\39{}$\&{uint16\_t} \|i${},\39{}$\&{uint32\_t} \index{size+\\{size}}\\{size}${},\39{}$\&{uint32\_t} \index{xsize+\\{xsize}}\\{xsize}${},\3{-1}\39{}$\&{char} ${}{*}\index{file name+\\{file\_name}}\\{file\_name});{}$\7
\X379:enable bison debugging\X\7
\&{extern} \&{int} \index{yylex+\\{yylex}}\\{yylex}(\&{void});\7
\X309:parsing functions\X\6
\8\%\raise 1pt\hbox{$\,\}$}\6
\hbox{{\label{union}\index{union}\index{parsing}}}\6
\8\%\&{union} $\{$ \&{uint32\_t} \|u;\5
\&{int32\_t} \|i;\5
\&{char} ${}{*}\|s{}$;\5
\&{float64\_t} \|f;\5
\index{glyph t+\&{glyph\_t}}\&{glyph\_t} \|c; \5
\index{dimen t+\&{dimen\_t}}\&{dimen\_t} \|d;\5
\index{stretch t+\&{stretch\_t}}\&{stretch\_t} \index{st+\\{st}}\\{st};\5
\index{xdimen t+\&{xdimen\_t}}\&{xdimen\_t} \index{xd+\\{xd}}\\{xd};\5
\index{kern t+\&{kern\_t}}\&{kern\_t} \index{kt+\\{kt}}\\{kt};\5
\index{rule t+\&{rule\_t}}\&{rule\_t} \|r;\5
\index{glue t+\&{glue\_t}}\&{glue\_t} \|g;\5
\5
\index{image t+\&{image\_t}}\&{image\_t} \|x;\5
\index{list t+\&{list\_t}}\&{list\_t} \|l;\5
\index{box t+\&{box\_t}}\&{box\_t} \|h;\5
\index{hyphen t+\&{hyphen\_t}}\&{hyphen\_t} \index{hy+\\{hy}}\\{hy};\5
\index{lig t+\&{lig\_t}}\&{lig\_t} \index{lg+\\{lg}}\\{lg};\5
\index{ref t+\&{ref\_t}}\&{ref\_t} \index{rf+\\{rf}}\\{rf};\5
\index{info t+\&{info\_t}}\&{info\_t} \index{info+\\{info}}\\{info};\5
\index{order t+\&{order\_t}}\&{order\_t} \|o;\5
\&{bool} \|b;\5
$\}{}$\hbox{{}}\6
\8\%\&{error\_verbose}\6
\8\%\index{start+\&{start}}\&{start} \index{hint+\nts{hint}}\nts{hint}\6
\hbox{}\X2:symbols\X\6
\hbox{}\6
\8\%\%\6
\X5:parsing rules\X\6
\8\%\%
\Y
\fi

\M{438}

\subsection{{\tt shrink.c}}\index{shrink.c+{\tt shrink.c}}

\.{shrink} is a \CEE/ program translating a \HINT/ file in long format into a \HINT/ file in short format.

\Y\B\4\X438:\.{shrink.c }\X${}\E{}$\6
\8\#\&{include} \.{"basetypes.h"}\6
\8\#\&{include} \.{<string.h>}\6
\8\#\&{include} \.{<ctype.h>}\6
\8\#\&{include} \.{<sys/stat.h>}\6
\8\#\&{include} \.{<zlib.h>}\6
\8\#\&{include} \.{"error.h"}\6
\8\#\&{include} \.{"hformat.h"}\6
\X1:hint types\X\6
\8\#\&{include} \.{"shrink.tab.h"}\7
\&{extern} \&{void} \\{yyset\_debug}(\&{int} \index{lex debug+\\{lex\_debug}}\\{lex\_debug});\6
\&{extern} \&{int} \index{yylineno+\\{yylineno}}\\{yylineno};\6
\&{extern} \&{FILE} ${}{*}\index{yyin+\\{yyin}}\\{yyin},{}$ ${}{*}\index{yyout+\\{yyout}}\\{yyout};{}$\6
\&{extern} \&{int} \index{yyparse+\\{yyparse}}\\{yyparse}(\&{void});\7
\X271:put macros\X\6
\X252:common variables\X\6
\X262:function to check the banner\X\6
\X283:directory entry type\X\6
\X284:directory functions\X\6
\X265:function to write the banner\X\6
\X12:put functions\X\7
\&{int} \index{main+\\{main}}\\{main}(\&{int} \index{argc+\\{argc}}\\{argc}${},\39{}$\&{char} ${}{*}\index{argv+\\{argv}}\\{argv}[\,]){}$\1\1\2\2\1\6
\4${}\{{}$\5
\X365:local variables in \\{main}\X\6
${}\index{in ext+\\{in\_ext}}\\{in\_ext}\K\.{".HINT"};{}$\6
${}\index{out ext+\\{out\_ext}}\\{out\_ext}\K\.{".hnt"};{}$\6
\X366:process the command line\X\6
\&{if} ${}(\index{debugflags+\\{debugflags}}\\{debugflags}\AND\index{DBGFLEX+\.{DBGFLEX}}\.{DBGFLEX}){}$\1\5
\\{yyset\_debug}(\T{1});\2\6
\&{else}\1\5
\\{yyset\_debug}(\T{0});\2\6
\8\#\&{if} \index{YYDEBUG+\.{YYDEBUG}}\.{YYDEBUG}\6
\&{if} ${}(\index{debugflags+\\{debugflags}}\\{debugflags}\AND\index{DBGBISON+\.{DBGBISON}}\.{DBGBISON}){}$\1\5
${}\index{yydebug+\\{yydebug}}\\{yydebug}\K\T{1};{}$\2\6
\&{else}\1\5
${}\index{yydebug+\\{yydebug}}\\{yydebug}\K\T{0};{}$\2\6
\8\#\&{endif}\6
\X369:open the log file\X\6
\X370:open the input file\X\6
\X371:open the output file\X\6
${}\index{yyin+\\{yyin}}\\{yyin}\K\index{hin+\\{hin}}\\{hin};{}$\6
${}\index{yyout+\\{yyout}}\\{yyout}\K\index{hlog+\\{hlog}}\\{hlog};{}$\6
\X264:read the banner\X\6
\index{hcheck banner+\\{hcheck\_banner}}\\{hcheck\_banner}(\.{"HINT"});\6
${}\index{yylineno+\\{yylineno}}\\{yylineno}\PP;{}$\6
${}\.{DBG}(\index{DBGBISON+\.{DBGBISON}}\.{DBGBISON}\OR\index{DBGFLEX+\.{DBGFLEX}}\.{DBGFLEX},\39\.{"Parsing\ Input\\n"});{}$\6
\index{yyparse+\\{yyparse}}\\{yyparse}(\,);\6
\index{hput directory+\\{hput\_directory}}\\{hput\_directory}(\,);\6
\index{hput hint+\\{hput\_hint}}\\{hput\_hint}(\.{"shrink"});\6
\X374:close the output file\X\6
\X373:close the input file\X\6
\X375:close the log file\X\6
\&{return} \T{0};\6
\4\index{explain usage+\\{explain\_usage}}\\{explain\_usage}:\5
\X361:explain usage\X\6
\&{return} \T{1};\6
\4${}\}{}$\2
\Y
\fi

\M{439}



\subsection{{\tt stretch.c}}\index{stretch.c+{\tt stretch.c}}
\index{stretch+\.{stretch}}\.{stretch} is a \CEE/ program translating a \HINT/ file in short
format into a \HINT/ file in long format.

\Y\B\4\X439:\.{stretch.c }\X${}\E{}$\6
\8\#\&{include} \.{"basetypes.h"}\6
\8\#\&{include} \.{<math.h>}\6
\8\#\&{include} \.{<string.h>}\6
\8\#\&{include} \.{<ctype.h>}\6
\8\#\&{include} \.{<zlib.h>}\6
\8\#\&{include} \.{<sys/stat.h>}\6
\8\#\&{include} \.{<fcntl.h>}\6
\8\#\&{include} \.{"error.h"}\6
\8\#\&{include} \.{"hformat.h"}\6
\X1:hint types\X\6
\X252:common variables\X\6
\X274:map functions\X\6
\X262:function to check the banner\X\6
\X265:function to write the banner\X\6
\X283:directory entry type\X\6
\X284:directory functions\X\6
\X35:get file macros\X\6
\X263:get file functions\X\6
\X430:write function declarations\X \X431:get function declarations\X\6
\X19:write functions\X\6
\X313:definition checks\X\6
\X17:get macros\X\6
\X16:get functions\X\7
\&{int} \index{main+\\{main}}\\{main}(\&{int} \index{argc+\\{argc}}\\{argc}${},\39{}$\&{char} ${}{*}\index{argv+\\{argv}}\\{argv}[\,]){}$\1\1\2\2\1\6
\4${}\{{}$\5
\X365:local variables in \\{main}\X\6
${}\index{in ext+\\{in\_ext}}\\{in\_ext}\K\.{".hnt"};{}$\6
${}\index{out ext+\\{out\_ext}}\\{out\_ext}\K\.{".HINT"};{}$\6
\X366:process the command line\X\6
\X369:open the log file\X\6
\X371:open the output file\X\6
\X372:determine the \\{stem\_name} from the output \\{file\_name}\X\6
\index{hget map+\\{hget\_map}}\\{hget\_map}(\,);\6
\index{hget banner+\\{hget\_banner}}\\{hget\_banner}(\,);\6
\index{hcheck banner+\\{hcheck\_banner}}\\{hcheck\_banner}(\.{"hint"});\6
${}\index{hput banner+\\{hput\_banner}}\\{hput\_banner}(\.{"HINT"},\39\.{"stretch"});{}$\6
\index{hget directory+\\{hget\_directory}}\\{hget\_directory}(\,);\6
\index{hwrite directory+\\{hwrite\_directory}}\\{hwrite\_directory}(\,);\6
\index{hget definition section+\\{hget\_definition\_section}}\\{hget\_definition\_section}(\,);\6
\index{hwrite content section+\\{hwrite\_content\_section}}\\{hwrite\_content\_section}(\,);\6
\index{hwrite aux files+\\{hwrite\_aux\_files}}\\{hwrite\_aux\_files}(\,);\6
\index{hget unmap+\\{hget\_unmap}}\\{hget\_unmap}(\,);\6
\X374:close the output file\X\6
\X375:close the log file\X\6
\&{return} \T{0};\6
\4\index{explain usage+\\{explain\_usage}}\\{explain\_usage}:\5
\X361:explain usage\X\6
\&{return} \T{1};\6
\4${}\}{}$\2
\Y
\fi

\M{440}

\subsection{{\tt hteg.h}}\index{hteg.h+{\tt hteg.h}}
\noindent
\Y\B\4\X440:skip function declarations\X${}\E{}$\6
\&{static} \&{void} \index{hteg content node+\\{hteg\_content\_node}}\\{hteg\_content\_node}(\&{void});\6
\&{static} \&{void} \index{hteg content+\\{hteg\_content}}\\{hteg\_content}(\&{uint8\_t} \|z);\6
\&{static} \&{void} \index{hteg xdimen node+\\{hteg\_xdimen\_node}}\\{hteg\_xdimen\_node}(\index{xdimen t+\&{xdimen\_t}}\&{xdimen\_t} ${}{*}\|x);{}$\6
\&{static} \&{void} \index{hteg list+\\{hteg\_list}}\\{hteg\_list}(\index{list t+\&{list\_t}}\&{list\_t} ${}{*}\|l);{}$\6
\&{static} \&{void} \index{hteg param list node+\\{hteg\_param\_list\_node}}\\{hteg\_param\_list\_node}(\index{list t+\&{list\_t}}\&{list\_t} ${}{*}\|l);{}$\6
\&{static} \&{float32\_t} \index{hteg float32+\\{hteg\_float32}}\\{hteg\_float32}(\&{void});\6
\&{static} \&{void} \index{hteg rule node+\\{hteg\_rule\_node}}\\{hteg\_rule\_node}(\&{void});\6
\&{static} \&{void} \index{hteg hbox node+\\{hteg\_hbox\_node}}\\{hteg\_hbox\_node}(\&{void});\6
\&{static} \&{void} \index{hteg vbox node+\\{hteg\_vbox\_node}}\\{hteg\_vbox\_node}(\&{void});\6
\&{static} \&{void} \index{hteg glue node+\\{hteg\_glue\_node}}\\{hteg\_glue\_node}(\&{void});
\U441.\Y
\fi

\M{441}

\subsection{{\tt skip.c}}\label{skip}\index{skip.c+{\tt skip.c}}
\.{skip} is a \CEE/ program reading the content section of a \HINT/ file in short format
backwards.

\Y\B\4\X441:\.{skip.c }\X${}\E{}$\6
\8\#\&{include} \.{"basetypes.h"}\6
\8\#\&{include} \.{<string.h>}\6
\8\#\&{include} \.{<zlib.h>}\6
\8\#\&{include} \.{<sys/stat.h>}\6
\8\#\&{include} \.{<fcntl.h>}\6
\8\#\&{include} \.{"error.h"}\6
\8\#\&{include} \.{"hformat.h"}\6
\X1:hint types\X\6
\X252:common variables\X\6
\X274:map functions\X\6
\X262:function to check the banner\X\6
\X283:directory entry type\X\6
\X284:directory functions\X\6
\X35:get file macros\X\6
\X263:get file functions\X\6
\X384:skip macros\X\6
\X440:skip function declarations\X\6
\X380:skip functions\X\7
\&{int} \index{main+\\{main}}\\{main}(\&{int} \index{argc+\\{argc}}\\{argc}${},\39{}$\&{char} ${}{*}\index{argv+\\{argv}}\\{argv}[\,]){}$\1\1\2\2\1\6
\4${}\{{}$\5
\X365:local variables in \\{main}\X\6
${}\index{in ext+\\{in\_ext}}\\{in\_ext}\K\.{".hnt"};{}$\6
${}\index{out ext+\\{out\_ext}}\\{out\_ext}\K\.{".bak"};{}$\6
\X366:process the command line\X\6
\X369:open the log file\X\6
\index{hget map+\\{hget\_map}}\\{hget\_map}(\,);\6
\index{hget banner+\\{hget\_banner}}\\{hget\_banner}(\,);\6
\index{hcheck banner+\\{hcheck\_banner}}\\{hcheck\_banner}(\.{"hint"});\6
\index{hget directory+\\{hget\_directory}}\\{hget\_directory}(\,);\6
\index{hskip content section+\\{hskip\_content\_section}}\\{hskip\_content\_section}(\,);\6
\index{hget unmap+\\{hget\_unmap}}\\{hget\_unmap}(\,);\6
\X375:close the log file\X\6
\&{return} \T{0};\6
\4\index{explain usage+\\{explain\_usage}}\\{explain\_usage}:\5
\X361:explain usage\X\6
\&{return} \T{1};\6
\4${}\}{}$\2
\Y
\fi

\M{442}

\thecodeindex


\plainsection{Cross Reference of Code}
\crosssections



\plainsection{References}

{\baselineskip=11pt
\def\bfblrm{\small\rm}%
\def\bblem{\small\it}%
\bibliography{hint}
\bibliographystyle{plain}
}

\plainsection{Index}
{
\def\_{{\tt \UL}} % underline in a string
\catcode`\_=\active \let_=\_ % underline is a letter
\input format.ind
}

\write\cont{} % ensure that the contents file isn't empty
%  \write\cont{\catcode `\noexpand\12\relax}   % \makeatother
\closeout\cont% the contents information has been fully gathered
\Y
\fi


\inx
\fin
\end
