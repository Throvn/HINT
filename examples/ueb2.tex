% ueb1.tex 
% 99-10-20

\documentclass[12pt]{article}
\usepackage{german}
%\usepackage{makeidx}
%\usepackage{bussproofs}
%\usepackage{amsmath}
\def\binom#1#2{\left({#1\atop #2}\right)}

%\usepackage{amssymb}
%\usepackage{amsfonts}
%\usepackage{pc}
\setlength{\textwidth}{16cm}
\setlength{\textheight}{23cm}
\setlength{\topmargin}{0cm}
\setlength{\oddsidemargin}{0cm}
\setlength{\evensidemargin}{0cm}

\def\N{\mathbb{N}}
\def\data{\textsc{Data}}
\def\expr{\textsc{Expr}}
\def\factiter{\texttt{fact{-}iter}}
\def\name{\langle name\rangle}
\def\natseg{\hbox{\sf nat{-}seg}}
\def\squaresum{\texttt{square{-}sum}}
\def\rmel{\texttt{rm{-}el}}
\def\val{\textsc{Val}}
\def\var{\textsc{Var}}
\def\vs{v\!s}

\input size.tex
\textwidth=320pt
\textheight=400pt

%\tracingall
\begin{document}
\emergencystretch=16pt
\thispagestyle{empty}

\noindent
\hbox{\parindent 0pt\vbox{\advance\hsize -4em\raggedright
Mathematisches Institut der Universit\"at M\"unchen\par
H.~Schwichtenberg, M.~Ruckert}\hfil\vbox{\hsize=4em\raggedleft
WS 2001\par
Blatt 2}}

\vskip 1cm plus 1cm minus 1cm
\begin{center}
\textbf{\"Ubungen zum Ferienkurs Nichtnumerisches Programmieren (Scheme)}
\end{center}
\vskip 1cm plus 1cm minus 1cm
\noindent
\textbf{Aufgabe 5.}
F\"ur die durch geschachtelte Rekursion definierte Funktion
\begin{verbatim}
(define (ackermann x y)
  (cond ((= y 0) 0)
        ((= x 0) (* 2 y))
        ((= y 1) 2)
        (else (ackermann 
                (- x 1)
                (ackermann x (- y 1))))))
\end{verbatim}
berechne man {\tt (ackermann 1 10)}, {\tt (ackermann 2 4)} und
{\tt (ackermann 3 3)}.  Man verfolge den Ablauf der Rechnung mittels
{\tt (trace ackermann)}.  Ferner gebe man \"ubliche mathematische 
Definitionen f\"ur die Funktionen $f_i(y):=f(i,y)$ f\"ur $i=0,1,2$.


\bigskip\noindent\textbf{Aufgabe 6.}
Die Zahlen $\binom{n}{k}$ hei{\ss}en Bi\-no\-mial\-koeffi\-zien\-ten.
F\"ur $1\le k\le n$ gilt
%
$$\binom{n}{k}=\binom{n-1}{k-1}+\binom{n-1}{k}.$$

Schreiben Sie eine Funktion {\tt (binom n k)} zur Berechnung
der Binomialkoeffizienten.




\bigskip\noindent\textbf{Aufgabe 7.}
Man definiere eine Funktion {\tt (integral f a b n)},
die eine $n$-te N"aherung $$ 1/n\sum_{i=0}^{n-1}\cdot f(a+(b-a)i/n)$$
des Integrals  "uber $f$ von $a$ bis $b$ berechnet.

Experimentieren sie mit rationalen Funktionen $f$, etwa
{\tt (define (f x) (/ 1 x))} und beobachten Sie die Unterschiede
zwischen
\begin{verse}
 {\tt (integral f 1 2 100)} und \\
 {\tt (integral f 1.0 2.0 100)}
\end{verse}
 bez"uglich Ergebniss, Genauigkeit
und Laufzeit.


\bigskip\noindent\textbf{Aufgabe 8.}
Funktionen mit Parametern lassen sich in Scheme leicht
realisieren. Betrachten wir z.B. die Funktion $g_r: x \mapsto rx^2+1$.
Schreiben Sie, unter Verwendung von \verb/lambda/, eine Funktion \verb/g/,
so da"s der Aufruf von \verb/(g 3)/ als Wert die
entsprechende Funktion $g_3$ liefert. Versuchen Sie auch 
sich genau klar zu machen, was und wie
 der Ausdruck {\tt (integral (g (sqrt 2)) 1 2 100)} berechnet.






\end{document}
