\input size.tex
\input ifhint.tex

\def\makeatletter{\catcode`\@=11\relax}
\def\makeatother{\catcode`\@=12\relax}

\makeatletter
  \message{\the\language\ number}
  \let\l@german=\lang@german
 % \language=\lang@german
  \showhyphens{Umgehungsstrasse}
\makeatother

\input german.sty
\selectlanguage{german}

\parskip=0pt
\parindent=0pt
\hangindent=4.5\baselineskip
\hangafter=-4
\hbox to 0pt{\vbox to 0pt{\ifhint\HINTimage=A.png height 48pt\else A\fi \vss}\hss}  n\-fangs\-buch\-sta\-ben wurden in fr"uheren Zeiten
oft als Initialen ausgeschm"uckt. Dabei wurde von den Schreibern ein hohes Ma"s  an Kunstfertigkeit
erwartet. Nicht immer bestand die Initiale einfach nur aus dem kunstfertig gezeichneten Buchstaben.
Oft wurden kleine Szenen gemalt, die den nachfolgenden Text illustrierten.

\parskip 6pt plus 3pt minus 3pt
\parindent=4.5\baselineskip
In der heutigen Zeit wird f"ur die Initialen meist nur ein Grossbuchstabe aus einer
verf"ugbaren Schmuckschrift oder einfach nur ein vergr"o"serter Buchstabe aus der auch
im Text verwendeten Hauptschrift genommen.


\vfill
\eject
\end
